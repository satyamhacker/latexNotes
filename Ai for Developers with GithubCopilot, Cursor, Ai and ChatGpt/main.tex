\documentclass[a4paper, 12pt]{article}

% Packages
\usepackage{geometry}
\usepackage{xcolor}
\usepackage{listings}
\usepackage{enumitem}
\usepackage{graphicx}
\usepackage{fancyhdr}
\usepackage{titlesec}
\usepackage{array}
\usepackage{booktabs}

% Layout
\geometry{a4paper, margin=1in}
\setlength{\parindent}{0pt}
\setlength{\parskip}{1em}

% Colors
\definecolor{sectionblue}{RGB}{0, 0, 255}
\definecolor{importantred}{RGB}{255, 0, 0}
\definecolor{keyconceptgreen}{RGB}{0, 128, 0}
\definecolor{codegray}{RGB}{240, 240, 240}

% Section formatting
\titleformat{\section}
  {\normalfont\Large\bfseries\color{sectionblue}}
  {\thesection}{1em}{}
\titleformat{\subsection}
  {\normalfont\large\bfseries\color{sectionblue}}
  {\thesubsection}{1em}{}

% Header and footer
\pagestyle{fancy}
\fancyhf{}
\rhead{Installing and Setting up GitHub Copilot}
\lhead{\leftmark}
\rfoot{Page \thepage}

% Document start
\begin{document}

% Title
\title{\textbf{\color{sectionblue}udemy Ai for Devleopers with Github copilot, Cursor Ai & chatGpt course...}}
\author{Your Name}
\date{\today}
\maketitle

% Introduction
\section{Introduction}
This document provides a detailed guide on installing and setting up \textbf{\color{keyconceptgreen}GitHub Copilot}, a powerful AI-powered code completion tool that enhances your coding experience across various editors and IDEs. \textbf{\color{importantred}Important steps} for installation and setup are highlighted in red, and \textbf{\color{keyconceptgreen}key concepts} are emphasized in green for clarity.

% Section 1
\section{Extension Tab se Install Karna}
This section explains how to install GitHub Copilot in Visual Studio Code (VS Code).

\subsection{Step-by-Step Process (Visual Studio Code ke example ke saath)}
\begin{itemize}
    \item Open karo apna \textbf{VS Code}.
    \item Left sidebar mein jao \textbf{Extensions Tab} pe (ya shortcut \texttt{Ctrl+Shift+X} dabao).
    \item Search karo \textbf{\color{importantred}"GitHub Copilot"}.
    \item Install karo \textbf{\color{importantred}"GitHub Copilot"} extension.
    \item Phir search karo \textbf{\color{importantred}"GitHub Copilot Chat"} aur isko bhi install karo.
\end{itemize}
Check: \textbf{\color{keyconceptgreen}Ab aapka Copilot aur Copilot Chat ready hai use karne ke liye.}

% Section 2
\section{Copilot Availability in Different IDEs/Editors}
GitHub Copilot is available for \textbf{\color{keyconceptgreen}alag-alag editors aur platforms}. Below is a table summarizing the supported platforms:

\begin{table}[h!]
  \centering
  \begin{tabular}{|>{\centering}p{4cm}|>{\centering}p{2.5cm}|>{\centering}p{6cm}|}
    \hline
    \textbf{Platform} & \textbf{Support} & \textbf{Setup Steps} \\
    \hline
    Visual Studio Code & Supported & Install extensions directly from marketplace. \\
    \hline
    Visual Studio (2022 aur aage ke versions) & Supported & Visual Studio Extension Manager se install karo. \\
    \hline
    JetBrains IDEs (Jaise: IntelliJ IDEA, PyCharm, WebStorm) & Supported & JetBrains Marketplace se "GitHub Copilot" plugin install karo. \\
    \hline
    Neovim & Supported & Plugin manager ke through GitHub Copilot ka plugin install karo. \\
    \hline
    Web Browser (CodeSpaces, GitHub.dev) & Supported & Browser ke andar hi extensions activate hote hain. \\
    \hline
    Android Studio & \textbf{Abhi Direct Official Support nahi hai} & JetBrains Plugin work kar sakta hai kuch workaround ke saath, but officially Android Studio ke liye Copilot fully supported nahi hai. \\
    \hline
  \end{tabular}
  \caption{Copilot Availability Across Platforms}
\end{table}

% Section 3
\section{Detailed Installation Steps for Different Editors}
This section provides detailed steps for installing GitHub Copilot in various editors.

\subsection{Visual Studio Code}
\begin{itemize}
    \item Extension Marketplace khol ke install karo: \textbf{\color{importantred}"GitHub Copilot"} aur \textbf{\color{importantred}"GitHub Copilot Chat"}.
    \item GitHub Account se login karo jab prompt aaye.
    \item Setting mein jaake customize kar sakte ho (jaise: suggestions ka frequency, chat shortcuts, etc.).
\end{itemize}

\subsection{Visual Studio (Windows ke liye)}
\begin{itemize}
    \item Visual Studio ko open karo.
    \item Extension Manager (\texttt{Extensions} > \texttt{Manage Extensions}) mein jao.
    \item Search karo \textbf{\color{importantred}"GitHub Copilot"}.
    \item Install karo aur Visual Studio ko restart karo.
    \item GitHub Account se login karo.
\end{itemize}

\subsection{JetBrains IDEs (IntelliJ, PyCharm, etc.)}
\begin{itemize}
    \item IDE open karo.
    \item \texttt{Settings/Preferences} > \texttt{Plugins} > \texttt{Marketplace} mein jao.
    \item Search karo \textbf{\color{importantred}"GitHub Copilot"}.
    \item Install karo aur IDE restart karo.
    \item GitHub login process complete karo.
\end{itemize}

\subsection{Web Browser (CodeSpaces, github.dev)}
\begin{itemize}
    \item Browser mein jab aap \textbf{CodeSpaces} ya \textbf{github.dev} open karte ho,
    \item Agar Copilot enabled hai, to directly extensions ka support milta hai.
    \item \textbf{\color{keyconceptgreen}Manual installation ki zarurat nahi hoti.}
\end{itemize}

% Section 4
\section{Tips for Best Usage}
To maximize your experience with GitHub Copilot:
\begin{itemize}
    \item Hammesha apne extensions ko \textbf{\color{importantred}updated} rakho.
    \item Correct GitHub account (jispe Copilot subscription active hai) se login karo.
    \item Settings mein apne hisaab se Copilot behavior (jaise suggestions ki frequency) tweak kar sakte ho.
    \item Copilot Chat ka use \textbf{\color{keyconceptgreen}complex queries aur code explanation} ke liye karo.
\end{itemize}

% Conclusion
\section{Final Summary}
This document summarized the key steps for setting up GitHub Copilot:
\begin{itemize}
    \item \textbf{\color{importantred}Install karo GitHub Copilot aur Copilot Chat} apne editor ke hisaab se.
    \item \textbf{\color{keyconceptgreen}Har editor ke liye alag setup steps hain}, wo dhyan se follow karo.
    \item \textbf{\color{importantred}Galat setup se productivity affect hoti hai}, isliye sahi installation important hai.
    \item \textbf{\color{keyconceptgreen}Copilot aur Chat dono milke coding experience next-level bana dete hain!}
\end{itemize}

===============================
===============================
\hrule


\setlength{\parindent}{0pt}
\setlength{\parskip}{1em}

% Colors
\definecolor{sectionblue}{RGB}{0, 0, 255}
\definecolor{importantred}{RGB}{255, 0, 0}
\definecolor{keyconceptgreen}{RGB}{0, 128, 0}
\definecolor{codegray}{RGB}{240, 240, 240}
\definecolor{keywordblue}{RGB}{0, 0, 255}
\definecolor{stringred}{RGB}{255, 0, 0}
\definecolor{commentgreen}{RGB}{0, 128, 0}
\definecolor{functionorange}{RGB}{255, 165, 0}
\definecolor{variablepurple}{RGB}{128, 0, 128}
\definecolor{violet}{RGB}{128, 0, 128}

% Section formatting
\titleformat{\section}
  {\normalfont\Large\bfseries\color{sectionblue}}
  {\thesection}{1em}{}
\titleformat{\subsection}
  {\normalfont\large\bfseries\color{sectionblue}}
  {\thesubsection}{1em}{}

% Custom Kotlin language definition for listings
\lstdefinelanguage{Kotlin}{
  keywords={val, var, fun, return, if, else, while, for, when, class, object, interface, package, import},
  keywordstyle=\color{keywordblue},
  comment=[l]{//},
  commentstyle=\color{commentgreen},
  string=[b]{"},
  stringstyle=\color{stringred},
  identifierstyle=\color{variablepurple},
  functionstyle=\color{functionorange},
  morecomment=[s]{/*}{*/},
  morestring=[b]',
  morestring=[b]"""
}

% Code listing setup
\lstset{
  backgroundcolor=\color{codegray},
  basicstyle=\ttfamily\footnotesize,
  breaklines=true,
  frame=single,
  numbers=left,
  numberstyle=\tiny\color{gray},
  showstringspaces=false,
  tabsize=2
}

% Header and footer
\pagestyle{fancy}
\fancyhf{}
\rhead{Mastering Prompts with GitHub Copilot}
\lhead{\leftmark}
\rfoot{Page \thepage}

% Document start
\begin{document}

% Title
\begin{center}
\section*{\textbf{\LARGE \textcolor{violet}{Mastering Prompts using Comments with GitHub Copilot’s AI}}}
\end{center}
\vspace{1em}

% Introduction
\section{Introduction}
Prompt likhne ka matlab sirf input dena nahi hai — \textbf{\color{importantred}agar aap sahi comment likho}, toh GitHub Copilot ussi ke hisaab se \textbf{\color{keyconceptgreen}accurate code generate karta hai}. This document explains how to use comments as prompts and leverage Copilot's inline chat feature.

% Section 1
\section{Kaise kaam karta hai?}
Copilot aapke likhe gaye \textbf{\color{keyconceptgreen}comments ko prompt} ki tarah treat karta hai, aur uske base pe \textbf{\color{importantred}smart code suggestions} deta hai.

% Section 2
\section{Step-by-Step: Comment-based Prompting}

\subsection{Ek meaningful comment likho}
\begin{lstlisting}[language=Python, caption={Example Python Comment}]
# function that takes two parameters a and b, adds them and returns the result
\end{lstlisting}

\subsection{Cursor ko next line pe rakho}
Copilot automatically suggestion dega:
\begin{lstlisting}[language=Python, caption={Generated Python Code}]
def add(a, b):
    return a + b
\end{lstlisting}
Check: \textbf{\color{keyconceptgreen}Ye comment hi prompt ban gaya AI ke liye.}

% Section 3
\section{Example Prompts (Different Languages)}

\subsection{Python}
\begin{lstlisting}[language=Python, caption={Python Prompt}]
# generate a fibonacci series of n numbers
\end{lstlisting}

\subsection{JavaScript}
\begin{lstlisting}[language=JavaScript, caption={JavaScript Prompt}]
// create a function to check if a number is prime
\end{lstlisting}

\subsection{Kotlin (for Android)}
\begin{lstlisting}[language=Kotlin, caption={Kotlin Prompt}]
// function to validate email format using regex
\end{lstlisting}
Check: \textbf{\color{keyconceptgreen}Jitna zyada clear and specific comment likhoge, utni zyada useful suggestion milegi.}

% Section 4
\section{Agar vague comment doge}
\begin{lstlisting}[language=Python, caption={Vague Comment}]
# do something
\end{lstlisting}
Copilot confused hoga aur \textbf{\color{importantred}random ya generic code} dega.

% Section 5
\section{Best Practices}
\begin{itemize}
    \item Comment mein \textbf{\color{importantred}task clearly explain karo}.
    \item Agar possible ho to \textbf{\color{keyconceptgreen}inputs aur output} define karo.
    \item Use karo “how”, “what”, ya “return” jaise words.
\end{itemize}

% Section 6
\section{Using the Inline Chat Feature (Copilot Chat inside Editor)}
GitHub Copilot ke paas ek aur powerful tool hai: \textbf{\color{keyconceptgreen}Editor Inline Chat}. Iska use aap code ko contextually modify karne ke liye kar sakte ho.

% Section 7
\section{Step-by-Step: Inline Chat ka Use}

\subsection{Function ya code block select karo}
(Jis part mein change chahiye)

\subsection{Right-click karo}
Choose: \textbf{\color{importantred}GitHub Copilot - Inline Chat}

\subsection{Prompt likho}
\begin{lstlisting}[language={}, caption={Inline Chat Prompt}]
Change this function to also return the difference between a and b.
\end{lstlisting}
Copilot will:
\begin{itemize}
    \item Understand selected code
    \item Suggest changes \textbf{\color{keyconceptgreen}only to that block}
\end{itemize}

% Section 8
\section{Why Select Code First?}
Agar aap pura code select nahi karte, toh Copilot pura file ya nearby context mein change suggest karega. \textbf{\color{importantred}Isliye select karna zaroori hai} taaki sirf wahi block modify ho jaye.

% Section 9
\section{Special Trick: Use \#filename in Prompt}
Agar aap inline chat mein ye likhte ho:
\begin{lstlisting}[language={}, caption={Prompt with Filename}]
#main.py — Convert this function to async
\end{lstlisting}
Copilot samjhega ki aap \textbf{\color{keyconceptgreen}main.py ke context} mein kaam kar rahe ho, aur accordingly suggestions dega.

% Section 10
\section{Tips for Effective Inline Chat Prompts}
\begin{table}[h!]
  \centering
  \begin{tabular}{|>{\centering}p{5cm}|>{\centering}p{5cm}|}
    \hline
    \textbf{Weak Prompt} & \textbf{Better Prompt} \\
    \hline
    make this better & optimize this function for large inputs \\
    \hline
    change it & convert this function to use async/await \\
    \hline
    fix it & fix the off-by-one error in this loop \\
    \hline
  \end{tabular}
  \caption{Weak vs. Better Inline Chat Prompts}
\end{table}

% Section 11
\section{Agar Inline Chat ya Comments use nahi karte}
\begin{itemize}
    \item \textbf{\color{importantred}Manual changes karne padenge}
    \item Copilot ki full power use nahi ho paayegi
    \item Repetitive code editing mein time waste hoga
\end{itemize}

% Conclusion
\section{Summary}
This document summarized the key points for mastering GitHub Copilot:
\begin{itemize}
    \item \textbf{\color{importantred}Comments = Prompts} — Clear likho, relevant likho.
    \item Use \textbf{\color{keyconceptgreen}Inline Chat} jab selected code block ko change karna ho.
    \item Prompt mein \textbf{\color{keyconceptgreen}\#filename} likhne se Copilot context aur better samajhta hai.
    \item Ye tools \textbf{\color{importantred}productivity boost} karte hain aur debugging/editing easy bana dete hain.
\end{itemize}

===============================
===============================
\hrule


\setlength{\parindent}{0pt}
\setlength{\parskip}{1em}

% Colors
\definecolor{sectionblue}{RGB}{0, 0, 255}
\definecolor{importantred}{RGB}{255, 0, 0}
\definecolor{keyconceptgreen}{RGB}{0, 128, 0}
\definecolor{codegray}{RGB}{240, 240, 240}
\definecolor{keywordblue}{RGB}{0, 0, 255}
\definecolor{stringred}{RGB}{255, 0, 0}
\definecolor{commentgreen}{RGB}{0, 128, 0}
\definecolor{functionorange}{RGB}{255, 165, 0}
\definecolor{variablepurple}{RGB}{128, 0, 128}
\definecolor{violet}{RGB}{128, 0, 128}

% Section formatting
\titleformat{\section}
  {\normalfont\Large\bfseries\color{sectionblue}}
  {\thesection}{1em}{}
\titleformat{\subsection}
  {\normalfont\large\bfseries\color{sectionblue}}
  {\thesubsection}{1em}{}

% Custom Kotlin language definition for listings (included for consistency, though not used here)
\lstdefinelanguage{Kotlin}{
  keywords={val, var, fun, return, if, else, while, for, when, class, object, interface, package, import},
  keywordstyle=\color{keywordblue},
  comment=[l]{//},
  commentstyle=\color{commentgreen},
  string=[b]{"},
  stringstyle=\color{stringred},
  identifierstyle=\color{variablepurple},
  functionstyle=\color{functionorange},
  morecomment=[s]{/*}{*/},
  morestring=[b]',
  morestring=[b]"""
}

% Code listing setup
\lstset{
  backgroundcolor=\color{codegray},
  basicstyle=\ttfamily\footnotesize,
  breaklines=true,
  frame=single,
  numbers=left,
  numberstyle=\tiny\color{gray},
  showstringspaces=false,
  tabsize=2
}

% Header and footer
\pagestyle{fancy}
\fancyhf{}
\rhead{Inline Chat in GitHub Copilot}
\lhead{\leftmark}
\rfoot{Page \thepage}

% Document start
\begin{document}

% Title
\begin{center}
\section*{\textbf{\LARGE \textcolor{violet}{�� Using Inline Chat Feature in GitHub Copilot}}}
\end{center}
\vspace{1em}

% Section 1
\section{�� Kaise use kare Inline Chat}
\begin{itemize}
    \item Jab aapko \textbf{\color{keyconceptgreen}sirf ek particular function} me code changes chahiye, to:
    \begin{itemize}
        \item \textbf{\color{importantred}Poora function select karo} (na ki sirf ek line), taki changes sirf usi function me ho.
        \item \textbf{\color{importantred}Right-click} karo aur \textbf{\color{importantred}"Copilot: Edit with Inline Chat"} option choose karo.
    \end{itemize}
    \item Ab ek \textbf{\color{keyconceptgreen}Inline prompt} open hoga.
    \begin{itemize}
        \item Yahan pe \textbf{\color{importantred}apna prompt likho} — jo changes chahiye woh batao.
        \item Copilot sirf \textbf{\color{keyconceptgreen}selected code ke andar} hi changes karega.
    \end{itemize}
\end{itemize}

% Section 2
\section{�� Note}
Agar aap \textbf{\color{keyconceptgreen}Inline Chat} ke prompt me \textbf{\color{importantred}\texttt{\# filename}} mention karte ho, \\
to Copilot \textbf{\color{keyconceptgreen}ussi file ke context} ke hisaab se code generate karega. \\
Example: Agar aap \texttt{\# utils.js} likhte ho, to Copilot sochta hai ki yeh code \texttt{utils.js} file ke according hona chahiye.

% Section 3
\section{�� Kab GitHub Copilot ka "Chat" aur "Inline Chat" use karna chahiye?}
\begin{table}[h!]
  \centering
  \begin{tabular}{|>{\centering}p{5cm}|>{\centering}p{4cm}|>{\centering}p{4cm}|}
    \hline
    \textbf{Situation} & \textbf{Use Chat Feature} & \textbf{Use Inline Chat Feature} \\
    \hline
    Jab aapko \textbf{\color{keyconceptgreen}general coding help}, \textbf{\color{keyconceptgreen}concept explain} karwana ho (jaise "what is debounce function?") & ✅ & ❌ \\
    \hline
    Jab aapko \textbf{\color{keyconceptgreen}poori file me improvement} ya \textbf{\color{keyconceptgreen}naya code generate} karwana ho & ✅ & ❌ \\
    \hline
    Jab aapko \textbf{\color{importantred}sirf ek specific chhoti jagah} (jaise ek function ya ek block of code) me \textbf{\color{importantred}change} karwana ho & ❌ & ✅ \\
    \hline
    Jab aapko \textbf{\color{keyconceptgreen}bigger refactoring} ideas chahiye (file ka structure change karna, best practices ke hisaab se) & ✅ & ❌ \\
    \hline
    Jab aapko ek chhoti \textbf{\color{importantred}specific cheez} fix ya modify karni ho \textbf{\color{importantred}within selected code} & ❌ & ✅ \\
    \hline
  \end{tabular}
  \caption{Chat vs. Inline Chat Usage}
\end{table}

% Section 4
\section{�� Example}

\subsection{Chat Feature ka Example}
Mujhe React me lazy loading ke concept samjhao aur ek example code do.

\subsection{Inline Chat Feature ka Example}
Ek function select karke prompt dena:
\begin{lstlisting}[language={}, caption={Inline Chat Prompt}]
Convert this function to use async/await instead of .then() chaining.
\end{lstlisting}

% Section 5
\section{�� Quick Tip}
\begin{itemize}
    \item \textbf{\color{keyconceptgreen}Chat} zyada useful hota hai jab aapko \textbf{\color{keyconceptgreen}idea generation} ya \textbf{\color{keyconceptgreen}general help} chahiye.
    \item \textbf{\color{keyconceptgreen}Inline Chat} best hota hai jab aapko \textbf{\color{keyconceptgreen}precision targeting} chahiye — sirf selected code me changes karne ke liye.
\end{itemize}


===============================
===============================
\hrule


\setlength{\parindent}{0pt}
\setlength{\parskip}{1em}

% Colors
\definecolor{sectionblue}{RGB}{0, 0, 255}
\definecolor{importantred}{RGB}{255, 0, 0}
\definecolor{keyconceptgreen}{RGB}{0, 128, 0}
\definecolor{codegray}{RGB}{240, 240, 240}
\definecolor{keywordblue}{RGB}{0, 0, 255}
\definecolor{stringred}{RGB}{255, 0, 0}
\definecolor{commentgreen}{RGB}{0, 128, 0}
\definecolor{functionorange}{RGB}{255, 165, 0}
\definecolor{variablepurple}{RGB}{128, 0, 128}
\definecolor{violet}{RGB}{128, 0, 128}

% Section formatting
\titleformat{\section}
  {\normalfont\Large\bfseries\color{sectionblue}}
  {\thesection}{1em}{}
\titleformat{\subsection}
  {\normalfont\large\bfseries\color{sectionblue}}
  {\thesubsection}{1em}{}

% Custom Kotlin language definition for listings (included for consistency, though not used here)
\lstdefinelanguage{Kotlin}{
  keywords={val, var, fun, return, if, else, while, for, when, class, object, interface, package, import},
  keywordstyle=\color{keywordblue},
  comment=[l]{//},
  commentstyle=\color{commentgreen},
  string=[b]{"},
  stringstyle=\color{stringred},
  identifierstyle=\color{variablepurple},
  functionstyle=\color{functionorange},
  morecomment=[s]{/*}{*/},
  morestring=[b]',
  morestring=[b]"""
}

% Code listing setup
\lstset{
  backgroundcolor=\color{codegray},
  basicstyle=\ttfamily\footnotesize,
  breaklines=true,
  frame=single,
  numbers=left,
  numberstyle=\tiny\color{gray},
  showstringspaces=false,
  tabsize=2
}

% Header and footer
\pagestyle{fancy}
\fancyhf{}
\rhead{Configuring GitHub Copilot}
\lhead{\leftmark}
\rfoot{Page \thepage}

% Document start
\begin{document}

% Title
\begin{center}
\section*{\textbf{\LARGE \textcolor{violet}{Configuring GitHub Copilot – Tips for Efficient Use (VS Code)}}}
\end{center}
\vspace{1em}

% Introduction
\section{Introduction}
Agar aap chahte ho ki GitHub Copilot \textbf{\color{keyconceptgreen}aapke coding style ko samjhe aur usi ke according suggestions de}, toh kuch \textbf{\color{importantred}important settings} ko configure karna zaroori hai.

% Section 1
\section{Step-by-Step: Open Copilot Settings in VS Code}
\begin{enumerate}
    \item Open karo \textbf{VS Code}.
    \item Go to: \texttt{File > Preferences > Settings} \\
          Ya shortcut: \texttt{Ctrl + ,}
    \item Search karo: \textbf{\color{importantred}"Copilot"} \\
          (Aapko GitHub Copilot se related saari settings dikhegi)
\end{enumerate}

% Section 2
\section{Enable "Use Instruction File" Option}

\subsection{Setting: Use instruction file}
Check: Ye setting GitHub Copilot ko \textbf{\color{keyconceptgreen}aapki likhi hui instructions follow karne ke liye bolti hai}, jisme aap define kar sakte ho:
\begin{itemize}
    \item Naming convention
    \item Function structure
    \item Preferred patterns
\end{itemize}

\subsection{File: .github/copilot-instructions.md}
Check: Ye file aapke project root me hoti hai, agar nahi hai to manually bana sakte ho.

\subsection{Isme likho instructions like}
\begin{lstlisting}[language={}, caption={Example copilot-instructions.md}]
Prefer snake_case variable names.  
Split long functions into smaller ones.  
Avoid using nested ternary operators.  
Use comments before complex logic.
\end{lstlisting}
Ab Copilot in instructions ko \textbf{\color{keyconceptgreen}base banake suggestions dega}.

% Section 3
\section{Example}

\subsection{Instruction diya}
\begin{lstlisting}[language={}, caption={Instruction Example}]
Use snake_case for variables.
\end{lstlisting}

\subsection{Copilot Suggestion}
\begin{lstlisting}[language=Python, caption={Generated Python Code}]
def get_user_data(user_id):
\end{lstlisting}
Check: \textbf{\color{keyconceptgreen}Aapki style ka automatically dhyan rakha jaata hai!}

% Section 4
\section{Enable "Temporal Context" Option}

\subsection{Setting: Enable temporal context}
Check: Is setting ka kaam hai: \\
\textbf{\color{keyconceptgreen}Copilot ko allow karna ki wo aapke last edited files ka context use kare} \\
aur uske base pe naye file mein suggestion de.

% Section 5
\section{What exactly it does?}
\begin{itemize}
    \item Recent file (jo aapne edit kiya hai) ka code context mein liya jaata hai.
    \item Us file ke functions ko \textbf{\color{keyconceptgreen}suggested file mein import karke} use karta hai.
    \item Helps in \textbf{\color{keyconceptgreen}cross-file suggestions} aur \textbf{\color{keyconceptgreen}project-level awareness}.
\end{itemize}

% Section 6
\section{Ye Feature Sirf Inline Chat ke liye hota hai kya?}
\textbf{\color{importantred}Nahi!} \\
`Temporal Context` feature sirf \textbf{inline chat tak limited nahi hai.} \\
Check: Ye feature:
\begin{itemize}
    \item \textbf{\color{keyconceptgreen}Inline Suggestions} (jo normal typing ke dauraan milti hain)
    \item \textbf{\color{keyconceptgreen}Inline Chat} (selected block par prompt dete ho)
\end{itemize}
dono mein kaam karta hai — especially jab aap ek bade project pe ho jahan kai files \textbf{\color{keyconceptgreen}interlinked} hoti hain.

% Section 7
\section{Quick Summary Table}
\begin{table}[h!]
  \centering
  \begin{tabular}{|>{\centering}p{3cm}|>{\centering}p{4cm}|>{\centering}p{4cm}|>{\centering}p{4cm}|}
    \hline
    \textbf{Setting} & \textbf{Purpose} & \textbf{File Used} & \textbf{Where it works} \\
    \hline
    Check: Use instruction file & Suggest code based on your coding style & .github/copilot-instructions.md & Inline Suggestions \& Chat \\
    \hline
    Check: Temporal context & Use recent file edits to influence code suggestions & No file, automatic & Inline Suggestions \& Chat \\
    \hline
  \end{tabular}
  \caption{Summary of Copilot Settings}
\end{table}

% Section 8
\section{Agar ye settings enable nahi ki}
\begin{itemize}
    \item Copilot aapke \textbf{\color{importantred}style} ko samjhe bina generic suggestion dega.
    \item Recent file context ka use nahi hoga → suggestions \textbf{\color{importantred}incomplete ya irrelevant} ho sakte hain.
    \item AI personalization ka faida nahi milega.
\end{itemize}

% Section 9
\section{Best Practices}
\begin{itemize}
    \item Har project ke liye ek \textbf{\color{importantred}.github/copilot-instructions.md} file zarur banao.
    \item Har new project start karte hi ye 2 settings enable karo:
        \begin{itemize}
            \item \textbf{\color{keyconceptgreen}Use Instruction File}
            \item \textbf{\color{keyconceptgreen}Temporal Context}
        \end{itemize}
    \item Ye settings aapko \textbf{\color{keyconceptgreen}project-level smartness} aur \textbf{\color{keyconceptgreen}consistent coding pattern} denge.
\end{itemize}

% Summary
\section{Summary}
This document summarized key tips for configuring GitHub Copilot in VS Code:
\begin{itemize}
    \item \textbf{\color{importantred}Use instruction file} to enforce coding style.
    \item \textbf{\color{keyconceptgreen}Temporal context} for cross-file suggestions.
    \item Create \textbf{\color{importantred}.github/copilot-instructions.md} for every project.
    \item Enable both settings for \textbf{\color{keyconceptgreen}smart and consistent} coding experience.
\end{itemize}

===============================
===============================
\hrule

\setlength{\parindent}{0pt}
\setlength{\parskip}{1em}

% Colors
\definecolor{sectionblue}{RGB}{0, 0, 255}
\definecolor{importantred}{RGB}{255, 0, 0}
\definecolor{keyconceptgreen}{RGB}{0, 128, 0}
\definecolor{codegray}{RGB}{240, 240, 240}
\definecolor{keywordblue}{RGB}{0, 0, 255}
\definecolor{stringred}{RGB}{255, 0, 0}
\definecolor{commentgreen}{RGB}{0, 128, 0}
\definecolor{functionorange}{RGB}{255, 165, 0}
\definecolor{variablepurple}{RGB}{128, 0, 128}
\definecolor{violet}{RGB}{128, 0, 128}

% Section formatting
\titleformat{\section}
  {\normalfont\Large\bfseries\color{sectionblue}}
  {\thesection}{1em}{}
\titleformat{\subsection}
  {\normalfont\large\bfseries\color{sectionblue}}
  {\thesubsection}{1em}{}

% Custom Kotlin language definition for listings (included for consistency, though not used here)
\lstdefinelanguage{Kotlin}{
  keywords={val, var, fun, return, if, else, while, for, when, class, object, interface, package, import},
  keywordstyle=\color{keywordblue},
  comment=[l]{//},
  commentstyle=\color{commentgreen},
  string=[b]{"},
  stringstyle=\color{stringred},
  identifierstyle=\color{variablepurple},
  functionstyle=\color{functionorange},
  morecomment=[s]{/*}{*/},
  morestring=[b]',
  morestring=[b]"""
}

% Code listing setup
\lstset{
  backgroundcolor=\color{codegray},
  basicstyle=\ttfamily\footnotesize,
  breaklines=true,
  frame=single,
  numbers=left,
  numberstyle=\tiny\color{gray},
  showstringspaces=false,
  tabsize=2
}

% Header and footer
\pagestyle{fancy}
\fancyhf{}
\rhead{Taking Advantage of Code Actions}
\lhead{\leftmark}
\rfoot{Page \thepage}

% Document start
\begin{document}

% Title
\begin{center}
\section*{\textbf{\LARGE \textcolor{violet}{Taking Advantage of Code Actions (GitHub Copilot)}}}
\end{center}
\vspace{1em}

% Introduction
\section{What are Code Actions?}
Jab aap \textbf{\color{keyconceptgreen}code ke kuch lines select karte ho} VS Code mein, toh ek \textbf{\color{importantred}small star icon} dikhta hai — ye \textbf{\color{keyconceptgreen}Code Action} ko trigger karne ka symbol hota hai.

% Section 1
\section{Step-by-Step: How to Use Code Actions}
\begin{enumerate}
    \item Code ke kuch lines \textbf{\color{importantred}select karo}.
    \item Dekho \textbf{\color{keyconceptgreen}ek star type ka icon} left side margin mein ya line ke end pe aayega.
    \item Click karo star icon pe.
\end{enumerate}
Aapko \textbf{\color{importantred}2 important options} milte hain:
\begin{table}[h!]
  \centering
  \begin{tabular}{|>{\centering}p{5cm}|>{\centering}p{7cm}|}
    \hline
    \textbf{Option} & \textbf{What it Does} \\
    \hline
    Modify using Copilot & Aapke selected code ko rewrite karta hai ya improve karta hai based on AI understanding. \\
    \hline
    Review using Copilot & Code ko analyze karta hai aur suggestions ya feedback deta hai (jaise performance improve karne ke liye tips). \\
    \hline
  \end{tabular}
  \caption{Code Action Options}
\end{table}

% Section 2
\section{Modify using Copilot}

\subsection{Kya karta hai?}
\begin{itemize}
    \item Aapke existing code ko \textbf{\color{keyconceptgreen}modify ya refactor} karta hai.
    \item Better readability, error fixing, ya optimization suggest karta hai.
\end{itemize}

\subsection{Kab Use Karein?}
Jab aap chahte ho ki aapka code \textbf{\color{keyconceptgreen}aur efficient, clean ya modern} ban jaye.

\subsection{Example}
Suppose aapne ye likha hai:
\begin{lstlisting}[language=Python, caption={Original Python Code}]
def add(a, b): return a+b
\end{lstlisting}
Select karo, Modify using Copilot choose karo → Suggestion aa sakta hai:
\begin{lstlisting}[language=Python, caption={Modified Python Code}]
def add(a, b):
    """Returns the sum of two numbers."""
    return a + b
\end{lstlisting}
Check: \textbf{\color{keyconceptgreen}More readable and documented}

% Section 3
\section{Review using Copilot}

\subsection{Kya karta hai?}
\begin{itemize}
    \item Aapke code ko \textbf{\color{keyconceptgreen}audit karta hai}.
    \item Performance, best practices, ya hidden bugs ka feedback deta hai.
\end{itemize}

\subsection{Kab Use Karein?}
\begin{itemize}
    \item Jab aapko apne likhe code ka \textbf{\color{keyconceptgreen}AI based review} chahiye.
    \item Especially jab code complex ho ya production ready karna ho.
\end{itemize}

\subsection{Example}
Suppose aapne likha:
\begin{lstlisting}[language=Python, caption={Original Python Code}]
def getData():
  pass
\end{lstlisting}
Select karo, Review using Copilot → Suggest karega:
\begin{itemize}
    \item Better naming: \texttt{get\_data()}
    \item Add docstring
    \item Raise \texttt{NotImplementedError} inside empty functions
\end{itemize}

% Section 4
\section{Sidebar Chat vs Inline Chat — Kab use karein?}
\begin{table}[h!]
  \centering
  \begin{tabular}{|>{\centering}p{3cm}|>{\centering}p{5cm}|>{\centering}p{5cm}|}
    \hline
    \textbf{Feature} & \textbf{Inline Chat} & \textbf{Sidebar Chat} \\
    \hline
    Context & Sirf selected code block pe focus karta hai & Pure file ya project-level broader context pe kaam karta hai \\
    \hline
    Speed & Quick fixes ya small modifications ke liye & Detailed discussions ya large feature additions ke liye \\
    \hline
    Example & Ek function ko async mein convert karna & Pura app ke structure pe suggestion lena (like: How to implement auth system?) \\
    \hline
    When to Use & Chhoti chhoti code improvements ke liye & Jab aapko detailed, research-level ya big feature guidance chahiye \\
    \hline
  \end{tabular}
  \caption{Inline Chat vs. Sidebar Chat}
\end{table}

% Section 5
\section{Quick Examples}

\subsection{Inline Chat Example}
\begin{lstlisting}[language={}, caption={Inline Chat Prompt}]
Select a function → Inline Chat → Prompt: "Convert this to async."
\end{lstlisting}
Check: \textbf{\color{keyconceptgreen}Sirf wahi function async mein convert hoga.}

\subsection{Sidebar Chat Example}
\begin{lstlisting}[language={}, caption={Sidebar Chat Prompt}]
Prompt: "How to implement Google Login in React Native app?"
\end{lstlisting}
Check: \textbf{\color{keyconceptgreen}Ye aapko pura structure aur libraries batayega.}

% Section 6
\section{How to Open Inline Chat in VS Code Terminal?}
Normally Inline Chat sirf Editor pe hota hai. \\
\textbf{\color{importantred}Direct Terminal mein Inline Chat open nahi hota.} \\
Check: Use: \texttt{Ctrl + I} (or right-click in editor > GitHub Copilot - Inline Chat) \\
Terminal se directly nahi, lekin \textbf{\color{keyconceptgreen}keyboard shortcut} ya editor ke shortcut se aap inline chat instantly trigger kar sakte ho.

% Section 7
\section{Agar Inline Chat aur Sidebar Chat ka galat use kiya}
\begin{itemize}
    \item Inline Chat mein bade complex features banwane ki koshish karoge toh \textbf{\color{importantred}context lost hoga}.
    \item Sidebar Chat mein chhoti chhoti cheeze poochoge toh \textbf{\color{importantred}time waste hoga}.
\end{itemize}

% Section 8
\section{Best Practices}
\begin{itemize}
    \item \textbf{\color{importantred}Chhote code improvements ke liye} → Inline Chat use karo.
    \item \textbf{\color{importantred}Large feature development ke liye} → Sidebar Chat use karo.
    \item Jab kuch specific lines ko refactor/modify karwana ho → \textbf{\color{keyconceptgreen}Code Actions} use karo.
\end{itemize}

% Summary
\section{Final Summary}
This document summarized key points for using Code Actions with GitHub Copilot:
\begin{itemize}
    \item Check: \textbf{\color{keyconceptgreen}Code Action = Quick AI help on selected code}
    \item Check: \textbf{\color{importantred}Modify using Copilot = Code ko improve/refactor karo}
    \item Check: \textbf{\color{keyconceptgreen}Review using Copilot = Code ka AI based review aur feedback lo}
    \item Check: \textbf{\color{keyconceptgreen}Inline Chat = Focused, small edits}
    \item Check: \textbf{\color{keyconceptgreen}Sidebar Chat = Big, full-featured help}
    \item Check: \textbf{\color{importantred}Terminal se direct Inline Chat nahi, editor shortcut se karo.}
\end{itemize}

===============================
===============================
\hrule


\setlength{\parindent}{0pt}
\setlength{\parskip}{1em}

% Colors
\definecolor{sectionblue}{RGB}{0, 0, 255}
\definecolor{importantred}{RGB}{255, 0, 0}
\definecolor{keyconceptgreen}{RGB}{0, 128, 0}
\definecolor{codegray}{RGB}{240, 240, 240}
\definecolor{keywordblue}{RGB}{0, 0, 255}
\definecolor{stringred}{RGB}{255, 0, 0}
\definecolor{commentgreen}{RGB}{0, 128, 0}
\definecolor{functionorange}{RGB}{255, 165, 0}
\definecolor{variablepurple}{RGB}{128, 0, 128}
\definecolor{violet}{RGB}{128, 0, 128}

% Section formatting
\titleformat{\section}
  {\normalfont\Large\bfseries\color{sectionblue}}
  {\thesection}{1em}{}
\titleformat{\subsection}
  {\normalfont\large\bfseries\color{sectionblue}}
  {\thesubsection}{1em}{}

% Custom Kotlin language definition for listings (included for consistency, though not used here)
\lstdefinelanguage{Kotlin}{
  keywords={val, var, fun, return, if, else, while, for, when, class, object, interface, package, import},
  keywordstyle=\color{keywordblue},
  comment=[l]{//},
  commentstyle=\color{commentgreen},
  string=[b]{"},
  stringstyle=\color{stringred},
  identifierstyle=\color{variablepurple},
  functionstyle=\color{functionorange},
  morecomment=[s]{/*}{*/},
  morestring=[b]',
  morestring=[b]"""
}

% Code listing setup
\lstset{
  backgroundcolor=\color{codegray},
  basicstyle=\ttfamily\footnotesize,
  breaklines=true,
  frame=single,
  numbers=left,
  numberstyle=\tiny\color{gray},
  showstringspaces=false,
  tabsize=2
}

% Header and footer
\pagestyle{fancy}
\fancyhf{}
\rhead{Multiple Edits with Copilot Editors}
\lhead{\leftmark}
\rfoot{Page \thepage}

% Document start
\begin{document}

% Title
\begin{center}
\section*{\textbf{\LARGE \textcolor{violet}{✨ Multiple Edits with Copilot Editors}}}
\end{center}
\vspace{1em}

% Section 1
\section{�� Kya hai Copilot Edits Mode?}
\begin{itemize}
    \item \textbf{\color{keyconceptgreen}Copilot Edit Mode} ek feature hai jo \textbf{\color{keyconceptgreen}GitHub Copilot Chat} me aata hai.
    \item Jab aap chat kholte ho, upar \textbf{\color{importantred}\texttt{+} button ke paas} ek option milta hai: \\
          \textbf{\color{importantred}"Copilot: Edit"}.
    \item Is mode ka use hota hai jab aap \textbf{\color{importantred}apne existing codebase me multiple jagah edits karwana chahte ho}, ek single prompt ke through.
\end{itemize}
Samajhne ke liye: Jaise ek command doge aur Copilot poore project me ya specific file me changes suggest karega.

% Section 2
\section{��️ Copilot Edits Mode Kya Karta Hai?}
\begin{itemize}
    \item Aap ek \textbf{\color{keyconceptgreen}natural language prompt} dete ho (jaise normal English me).
    \item Copilot \textbf{\color{keyconceptgreen}poori file}, \textbf{\color{keyconceptgreen}selected files}, ya \textbf{\color{keyconceptgreen}poore project (workspace)} ke context me dekhta hai.
    \item Fir \textbf{\color{keyconceptgreen}code updates, improvements, ya naya code add} karta hai jaha zarurat ho.
    \item Ye \textbf{\color{keyconceptgreen}inline suggestions} deta hai, jisko aap \textbf{\color{importantred}accept ya reject} kar sakte ho.
\end{itemize}

% Section 3
\section{�� Kab Use Karna Chahiye Copilot Edits Mode?}
\begin{table}[h!]
  \centering
  \begin{tabular}{|>{\centering}p{7cm}|>{\centering}p{5cm}|}
    \hline
    \textbf{Situation} & \textbf{Use Copilot Edits Mode} \\
    \hline
    Jab aapko \textbf{\color{importantred}ek type ka change multiple files} me karwana hai & ✅ \\
    \hline
    Jab aapko \textbf{\color{importantred}bohot bade codebase me ek improvement} karni hai (e.g., har function ko async banana) & ✅ \\
    \hline
    Jab aapko \textbf{\color{importantred}small refactoring} chahiye across project (e.g., add error handling) & ✅ \\
    \hline
    Jab aapko \textbf{\color{importantred}general edits ek hi file me} karne ho & ✅ \\
    \hline
    Jab aapko \textbf{\color{importantred}new feature add karna ho specific file me} & ✅ \\
    \hline
  \end{tabular}
  \caption{When to Use Copilot Edits Mode}
\end{table}

% Section 4
\section{�� Important Concepts: \texttt{\@workspace}, \texttt{\#filename}, and Common Decorators}

\subsection{1. \texttt{\@workspace}}
\begin{itemize}
    \item \textbf{Meaning:} \\
          Iska matlab hota hai \textbf{\color{keyconceptgreen}poore project (workspace)} ke context me kaam karna.
    \item \textbf{Use kab kare:} \\
          Jab aapko \textbf{\color{importantred}project ke kai files} me changes chahiye based on your prompt.
    \item \textbf{Example:} \\
          \texttt{\@workspace Create a login form inside \#login.html} \\
          Yani, Copilot pura workspace dekhega aur \texttt{login.html} file ke andar login form banayega.
\end{itemize}

\subsection{2. \texttt{\#filename}}
\begin{itemize}
    \item \textbf{Meaning:} \\
          Isse aap Copilot ko \textbf{\color{keyconceptgreen}specifically ek file} ka context dete ho.
    \item \textbf{Use kab kare:} \\
          Jab aapko sirf ek file me changes chahiye.
    \item \textbf{Example:} \\
          \texttt{\#dashboard.js Add a new function to fetch user analytics} \\
          Yani, Copilot sirf \texttt{dashboard.js} file ke liye kaam karega.
\end{itemize}

\subsection{3. Commonly Used Decorators in Copilot Edit Prompts}
\begin{table}[h!]
  \centering
  \begin{tabular}{|>{\centering}p{3cm}|>{\centering}p{5cm}|>{\centering}p{4cm}|}
    \hline
    \textbf{Decorator} & \textbf{Meaning} & \textbf{Example} \\
    \hline
    \texttt{\@workspace} & Project ke sare files ke context me kaam karega & \texttt{\@workspace Add input validation to all login forms} \\
    \hline
    \texttt{\#filename} & Specific file me changes karega & \texttt{\#userModel.js Add email verification method} \\
    \hline
    \texttt{\@openEditors} & Jo files currently open hai editor me, unke liye kaam karega & \texttt{\@openEditors Refactor all functions to arrow functions} \\
    \hline
    \texttt{\@activeFile} & Sirf currently active file me changes karega & \texttt{\@activeFile Improve error handling in API calls} \\
    \hline
  \end{tabular}
  \caption{Commonly Used Decorators}
\end{table}

% Section 5
\section{�� Kaise Likhe Achha Prompt for Copilot Edits?}
\begin{enumerate}
    \item \textbf{Clear and short likho:} \\
          \texttt{\@workspace Change all var to let and const}
    \item \textbf{Specify file agar specific ho:} \\
          \texttt{\#profile.html Add user avatar upload field}
    \item \textbf{Focus on one task at a time:} \\
          \texttt{\@openEditors Add console error logging wherever a fetch fails}
    \item \textbf{Use decorators properly:} \\
          Jaise project wide ho to \texttt{\@workspace}, \\
          sirf ek file ke liye ho to \texttt{\#filename}.
\end{enumerate}

% Section 6
\section{�� Quick Tips}
\begin{itemize}
    \item Agar bada project hai to \textbf{\color{keyconceptgreen}\@workspace} powerful hota hai, but dhyan rahe bohot saare unwanted changes ho sakte hai, to carefully review karo.
    \item Agar sirf ek file ya open file me chhote changes chahiye to \textbf{\color{keyconceptgreen}\@activeFile} ya \textbf{\color{keyconceptgreen}\#filename} perfect hai.
    \item Always \textbf{\color{importantred}read and review suggestions} before applying edits!
\end{itemize}

% Section 7
\section{�� Example Real-World Use Case}

\subsection{Case 1: Add Comment Headers to All API Files}
\begin{lstlisting}[language={}, caption={Workspace Prompt}]
@workspace Add function header comments to all API related files.
\end{lstlisting}

\subsection{Case 2: Modify Specific File Only 	
}
\begin{lstlisting}[language={}, caption={File-Specific Prompt}]
#authService.js Change password hashing algorithm to bcrypt.
\end{lstlisting}

\subsection{Case 3: Refactor All Open Files}
\begin{lstlisting}[language={}, caption={Open Editors Prompt}]
@openEditors Replace all `var` declarations with `let` or `const`.
\end{lstlisting}


===============================
===============================
\hrule




\setlength{\parindent}{0pt}
\setlength{\parskip}{1em}

% Colors
\definecolor{sectionblue}{RGB}{0, 0, 255}
\definecolor{importantred}{RGB}{255, 0, 0}
\definecolor{keyconceptgreen}{RGB}{0, 128, 0}
\definecolor{codegray}{RGB}{240, 240, 240}
\definecolor{keywordblue}{RGB}{0, 0, 255}
\definecolor{stringred}{RGB}{255, 0, 0}
\definecolor{commentgreen}{RGB}{0, 128, 0}
\definecolor{functionorange}{RGB}{255, 165, 0}
\definecolor{variablepurple}{RGB}{128, 0, 128}
\definecolor{violet}{RGB}{128, 0, 128}

% Section formatting
\titleformat{\section}
  {\normalfont\Large\bfseries\color{sectionblue}}
  {\thesection}{1em}{}
\titleformat{\subsection}
  {\normalfont\large\bfseries\color{sectionblue}}
  {\thesubsection}{1em}{}

% Custom Kotlin language definition for listings (included for consistency, though not used here)
\lstdefinelanguage{Kotlin}{
  keywords={val, var, fun, return, if, else, while, for, when, class, object, interface, package, import},
  keywordstyle=\color{keywordblue},
  comment=[l]{//},
  commentstyle=\color{commentgreen},
  string=[b]{"},
  stringstyle=\color{stringred},
  identifierstyle=\color{variablepurple},
  functionstyle=\color{functionorange},
  morecomment=[s]{/*}{*/},
  morestring=[b]',
  morestring=[b]"""
}

% Code listing setup
\lstset{
  backgroundcolor=\color{codegray},
  basicstyle=\ttfamily\footnotesize,
  breaklines=true,
  frame=single,
  numbers=left,
  numberstyle=\tiny\color{gray},
  showstringspaces=false,
  tabsize=2
}

% Header and footer
\pagestyle{fancy}
\fancyhf{}
\rhead{Copilot Extensions}
\lhead{\leftmark}
\rfoot{Page \thepage}

% Document start
\begin{document}

% Title
\begin{center}
\section*{\textbf{\LARGE \textcolor{violet}{GitHub Copilot Extensions -- Boost Your Copilot's IQ!}}}
\end{center}
\vspace{1em}

% Section 1
\section{What are Copilot Extensions?}
GitHub Copilot Extensions ka matlab hai: \\
Aise \textbf{\color{keyconceptgreen}official add-ons} jo GitHub Copilot ko \textbf{\color{keyconceptgreen}naye domains} mein \textbf{\color{keyconceptgreen}smart banate hain}, jaise Docker, Kubernetes, Python Testing, etc.

% Section 2
\section{Why Use Copilot Extensions?}
Normal Copilot sirf general coding suggestions deta hai. \\
Lekin Extensions se aap usko \textbf{\color{keyconceptgreen}domain-specific knowledge} de sakte ho, jisse wo:
\begin{itemize}
    \item Domain-specific files (like Dockerfile) generate kare
    \item \texttt{@} syntax ke through direct help de
    \item Better and accurate responses de in specific tech areas
\end{itemize}

% Section 3
\section{Example: Docker for GitHub Copilot}
Ye extension Copilot ko \textbf{\color{keyconceptgreen}Docker aur Docker Compose} ke baare mein sikhata hai.

\subsection{Aap kya kar sakte ho isse?}
\begin{itemize}
    \item Generate \texttt{Dockerfile}, \texttt{docker-compose.yml}
    \item Docker container setup
    \item Multi-stage builds
    \item Docker commands
\end{itemize}

\subsection{Copilot Chat mein likho}
\begin{lstlisting}[language={}, caption={Docker Extension Prompt}]
@docker create a Dockerfile for Node.js app with port 3000 exposed
\end{lstlisting}
Check: \textbf{\color{keyconceptgreen}Copilot smartly Dockerfile bana dega!}

% Section 4
\section{How to Add Copilot Extensions (These are NOT Regular VS Code Extensions)}

\subsection{Step-by-Step Guide}
\begin{enumerate}
    \item Open GitHub Copilot Chat panel in VS Code \\
          (Shortcut: \texttt{Ctrl+I} or sidebar icon)
    \item Top-right corner pe dekho: Settings icon → Click karo
    \item Waha milega: \textbf{\color{importantred}Extensions tab}
    \item Aapko list milegi available Copilot extensions ki, jaise: \\
          Docker, Kubernetes, Python Test Generator, YAML, Terraform, etc.
    \item Click on \textbf{\color{importantred}Install} button next to the extension you want.
    \item Bas ho gaya! Copilot ab uss domain mein \textbf{\color{keyconceptgreen}aur intelligent} ho gaya.
\end{enumerate}

% Section 5
\section{What Happens After Installation?}
\begin{itemize}
    \item Copilot aapke installed extensions ke hisaab se suggestions dena start karega.
    \item Chat mein \texttt{@extensionname} ka use karke direct prompt kar sakte ho.
\end{itemize}

% Section 6
\section{Example Prompts Using Installed Extensions}
\begin{table}[h!]
  \centering
  \begin{tabular}{|>{\centering}p{4cm}|>{\centering}p{8cm}|}
    \hline
    \textbf{Extension} & \textbf{Prompt} \\
    \hline
    \texttt{@docker} & \texttt{@docker generate a multi-stage Dockerfile for a Python Flask app} \\
    \hline
    \texttt{@kubernetes} & \texttt{@kubernetes create deployment yaml for my react app} \\
    \hline
    \texttt{@python-tests} & \texttt{@python-tests write pytest for this function} \\
    \hline
    \texttt{@terraform} & \texttt{@terraform write config for creating an EC2 instance} \\
    \hline
  \end{tabular}
  \caption{Example Prompts for Copilot Extensions}
\end{table}

% Section 7
\section{Important Notes}
\begin{itemize}
    \item Ye \textbf{\color{importantred}VS Code Marketplace extensions nahi hote} \\
          (na hi Extensions tab mein dikhte hain)
    \item Sirf \textbf{\color{importantred}GitHub Copilot Chat} ke settings mein available hote hain
    \item Extensions \textbf{\color{importantred}per-user basis pe install} hote hain (per project nahi)
\end{itemize}

% Section 8
\section{When to Use Copilot Extensions?}
\begin{table}[h!]
  \centering
  \begin{tabular}{|>{\centering}p{7cm}|>{\centering}p{5cm}|}
    \hline
    \textbf{Situation} & \textbf{Use Extension?} \\
    \hline
    Aapko Dockerfile banana hai? & Check: \texttt{@docker} \\
    \hline
    Infra-as-code likhna hai? & Check: \texttt{@terraform} \\
    \hline
    Kubernetes YAML banana hai? & Check: \texttt{@kubernetes} \\
    \hline
    Test cases likhwaane hain? & Check: \texttt{@python-tests} \\
    \hline
  \end{tabular}
  \caption{When to Use Copilot Extensions}
\end{table}

% Section 9
\section{Agar Extensions Install nahi kiye}
\begin{itemize}
    \item Copilot general suggestions dega, \textbf{\color{importantred}domain-specific deep knowledge nahi milegi}.
    \item \texttt{@docker}, \texttt{@kubernetes} jaise prompts samjhega hi nahi.
    \item \textbf{\color{importantred}Productivity kam aur manual search jyada}.
\end{itemize}

% Section 10
\section{Final Summary}
This document summarized key points for using GitHub Copilot Extensions:
\begin{itemize}
    \item Check: \textbf{\color{keyconceptgreen}Copilot Extensions = Domain-specific AI upgrade}
    \item Check: Add via Copilot Chat settings → Extensions tab
    \item Check: Use \texttt{@docker}, \texttt{@kubernetes}, etc. in prompts
    \item Check: Ye \textbf{\color{importantred}VS Code extensions nahi hote} — only Copilot-specific
    \item Check: Boost karta hai aapki \textbf{\color{keyconceptgreen}coding + DevOps + infra writing speed}
\end{itemize}

===============================
===============================
\hrule


\setlength{\parindent}{0pt}
\setlength{\parskip}{1em}

% Colors
\definecolor{sectionblue}{RGB}{0, 0, 255}
\definecolor{importantred}{RGB}{255, 0, 0}
\definecolor{keyconceptgreen}{RGB}{0, 128, 0}
\definecolor{codegray}{RGB}{240, 240, 240}
\definecolor{keywordblue}{RGB}{0, 0, 255}
\definecolor{stringred}{RGB}{255, 0, 0}
\definecolor{commentgreen}{RGB}{0, 128, 0}
\definecolor{functionorange}{RGB}{255, 165, 0}
\definecolor{variablepurple}{RGB}{128, 0, 128}
\definecolor{violet}{RGB}{128, 0, 128}

% Section formatting
\titleformat{\section}
  {\normalfont\Large\bfseries\color{sectionblue}}
  {\thesection}{1em}{}
\titleformat{\subsection}
  {\normalfont\large\bfseries\color{sectionblue}}
  {\thesubsection}{1em}{}

% Custom Kotlin language definition for listings (included for consistency, though not used here)
\lstdefinelanguage{Kotlin}{
  keywords={val, var, fun, return, if, else, while, for, when, class, object, interface, package, import},
  keywordstyle=\color{keywordblue},
  comment=[l]{//},
  commentstyle=\color{commentgreen},
  string=[b]{"},
  stringstyle=\color{stringred},
  identifierstyle=\color{variablepurple},
  functionstyle=\color{functionorange},
  morecomment=[s]{/*}{*/},
  morestring=[b]',
  morestring=[b]"""
}

% Code listing setup
\lstset{
  backgroundcolor=\color{codegray},
  basicstyle=\ttfamily\footnotesize,
  breaklines=true,
  frame=single,
  numbers=left,
  numberstyle=\tiny\color{gray},
  showstringspaces=false,
  tabsize=2
}

% Header and footer
\pagestyle{fancy}
\fancyhf{}
\rhead{Cursor AI Features}
\lhead{\leftmark}
\rfoot{Page \thepage}

% Document start
\begin{document}

% Title
\begin{center}
\section*{\textbf{\LARGE \textcolor{violet}{Introducing Cursor AI -- Smart Suggestions, Chat, and Composer Mode}}}
\end{center}
\vspace{1em}

% Section 1
\section{What is Cursor AI?}
\textbf{\color{keyconceptgreen}Cursor} ek modern \textbf{\color{keyconceptgreen}AI-first code editor} hai — jo GitHub Copilot jaise tools se \textbf{\color{keyconceptgreen}aur bhi smart aur context-aware} hai. \\
Cursor ka goal hai: \\
Let AI understand and change your entire codebase like a real pair programmer.

% Section 2
\section{Why to Use Cursor?}
\begin{table}[h!]
  \centering
  \begin{tabular}{|>{\centering}p{4cm}|>{\centering}p{8cm}|}
    \hline
    \textbf{Feature} & \textbf{Why It Matters} \\
    \hline
    Full project context & Cursor AI pura codebase samajhta hai — isolated function nahi \\
    \hline
    Smart inline suggestions & Regular autocomplete se zyada intelligent, based on your actual code structure \\
    \hline
    Built-in AI chat & ChatGPT jaisa interface with real code awareness \\
    \hline
    Composer mode & Multiple files edit kar sakta hai ek single prompt se! \\
    \hline
  \end{tabular}
  \caption{Why Use Cursor?}
\end{table}

% Section 3
\section{Cursor Chat -- The Smart Sidekick}

\subsection{What is Cursor Chat?}
Cursor editor ke right sidebar mein ek AI Chat hota hai, jo:
\begin{itemize}
    \item Aapke code ka \textbf{\color{keyconceptgreen}pura structure samajhta hai}
    \item Real file references ke saath answer deta hai
    \item Functions/Classes pe jump bhi kara sakta hai
    \item Errors fix, refactoring, aur explanations de sakta hai
\end{itemize}

\subsection{When to Use Cursor Chat?}
\begin{table}[h!]
  \centering
  \begin{tabular}{|>{\centering}p{5cm}|>{\centering}p{7cm}|}
    \hline
    \textbf{Use-Case} & \textbf{Prompt Example} \\
    \hline
    Error ka explanation & Why am I getting this null reference error? \\
    \hline
    Code understanding & Explain how the auth middleware works \\
    \hline
    Optimization & Optimize the database query in userController \\
    \hline
    Refactoring & Split this function into smaller parts \\
    \hline
  \end{tabular}
  \caption{When to Use Cursor Chat?}
\end{table}

% Section 4
\section{Cursor Composer Mode -- The Real Magic}

\subsection{What is Cursor’s Composer?}
Composer ek \textbf{\color{keyconceptgreen}AI assistant mode} hai jo: \\
Aapke \textbf{\color{keyconceptgreen}entire codebase mein changes suggest + implement} kar sakta hai, based on a single prompt. \\
Ye GitHub Copilot se alag isliye hai, kyunki:
\begin{itemize}
    \item Ye \textbf{\color{keyconceptgreen}multi-file editing} karta hai
    \item Pure project ka context samajhkar \textbf{\color{keyconceptgreen}intelligent, interconnected changes} karta hai
    \item AI ke saath milke \textbf{\color{keyconceptgreen}high-level feature development} mein madad karta hai
\end{itemize}

\subsection{How to Use Cursor Composer?}

\subsubsection{Step-by-Step:}
\begin{enumerate}
    \item Cursor editor open karo
    \item Left bottom mein click karo: \texttt{Ask AI} → Select: \textbf{\color{importantred}Use Composer}
    \item Type your prompt, jaise:
\begin{lstlisting}[language={}, caption={Composer Prompt}]
Add a password reset feature to the app.
\end{lstlisting}
    \item Cursor:
    \begin{itemize}
        \item Check karega kaunsi files change karni hongi (e.g. \texttt{routes.js}, \texttt{resetController.js}, \texttt{emailService.js})
        \item Aapko ek \textbf{\color{keyconceptgreen}list of proposed changes} dikhayega
        \item Aap accept/reject kar sakte ho
        \item Then apply all changes in one go!
    \end{itemize}
\end{enumerate}

\subsection{Example Prompts for Composer}
\begin{table}[h!]
  \centering
  \begin{tabular}{|>{\centering}p{5cm}|>{\centering}p{7cm}|}
    \hline
    \textbf{Prompt} & \textbf{What Composer Does} \\
    \hline
    Add user profile page with editable bio & Creates new route, controller, view \\
    \hline
    Migrate from Mongoose to Prisma & Refactor models and DB connections \\
    \hline
    Add forgot password functionality & Add route, controller, email sender code \\
    \hline
  \end{tabular}
  \caption{Example Prompts for Composer}
\end{table}

\subsection{When to Use Composer?}
\begin{table}[h!]
  \centering
  \begin{tabular}{|>{\centering}p{7cm}|>{\centering}p{5cm}|}
    \hline
    \textbf{Situation} & \textbf{Use Composer?} \\
    \hline
    Complex feature add karna hai & Check: \\
    \hline
    Multiple files mein interconnected changes chahiye & Check: \\
    \hline
    Manual refactor boring hai & Check: \\
    \hline
    Time save karna hai with AI superpowers & Check: \\
    \hline
  \end{tabular}
  \caption{When to Use Composer?}
\end{table}

\subsection{When NOT to Use Composer?}
\begin{itemize}
    \item Tiny, one-line changes → \textbf{\color{importantred}Inline Chat ya suggestion better}
    \item Code understanding / debugging → \textbf{\color{importantred}Use Sidebar AI Chat}
\end{itemize}

\subsection{Combo Use: Chat + Composer}
\begin{itemize}
    \item Chat se poochho: How should I implement forgot password?
    \item Fir Composer mode use karo: Add forgot password to the app
    \item AI + You = \textbf{\color{keyconceptgreen}Supercharged productivity}
\end{itemize}

% Section 5
\section{Final Summary}
\begin{table}[h!]
  \centering
  \begin{tabular}{|>{\centering}p{3cm}|>{\centering}p{5cm}|>{\centering}p{4cm}|}
    \hline
    \textbf{Feature} & \textbf{Use it For} & \textbf{Why it Rocks} \\
    \hline
    Cursor Chat & Understand, explain, debug code & Real-time, context-aware help \\
    \hline
    Cursor Composer & Multi-file edits, big feature changes & Saves hours of manual work \\
    \hline
    Inline Suggestions & Small quick changes & Copilot-style speed boost \\
    \hline
  \end{tabular}
  \caption{Summary of Cursor AI Features}
\end{table}

===============================
===============================
\hrule


\end{document}


