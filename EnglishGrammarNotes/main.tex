\documentclass[a4paper, 12pt]{article}

% Packages
\usepackage{geometry}
\usepackage{xcolor}
\usepackage{listings}
\usepackage{enumitem}
\usepackage{graphicx}
\usepackage{fancyhdr}
\usepackage{titlesec}

% Layout
\geometry{a4paper, margin=1in}
\setlength{\parindent}{0pt}
\setlength{\parskip}{1em}

% Colors
\definecolor{sectionblue}{RGB}{0, 0, 255}
\definecolor{importantred}{RGB}{255, 0, 0}
\definecolor{keyconceptgreen}{RGB}{0, 128, 0}
\definecolor{codegray}{RGB}{240, 240, 240}
\definecolor{keywordblue}{RGB}{0, 0, 255}
\definecolor{stringred}{RGB}{255, 0, 0}
\definecolor{commentgreen}{RGB}{0, 128, 0}
\definecolor{functionorange}{RGB}{255, 165, 0}
\definecolor{variablepurple}{RGB}{128, 0, 128}

% Section formatting
\titleformat{\section}
  {\normalfont\Large\bfseries\color{sectionblue}}
  {\thesection}{1em}{}
\titleformat{\subsection}
  {\normalfont\large\bfseries\color{sectionblue}}
  {\thesubsection}{1em}{}

% Custom Kotlin language definition for listings (not used here, but included per spec)
\lstdefinelanguage{Kotlin}{
  keywords={val, var, fun, return, if, else, while, for, when, class, object, interface, package, import},
  keywordstyle=\color{keywordblue},
  comment=[l]{//},
  commentstyle=\color{commentgreen},
  string=[b]{"},
  stringstyle=\color{stringred},
  identifierstyle=\color{variablepurple},
  functionstyle=\color{functionorange},
  morecomment=[s]{/*}{*/},
  morestring=[b]',
  morestring=[b]"""
}

% Code listing setup
\lstset{
  backgroundcolor=\color{codegray},
  basicstyle=\ttfamily\footnotesize,
  breaklines=true,
  frame=single,
  numbers=left,
  numberstyle=\tiny\color{gray},
  showstringspaces=false,
  tabsize=2,
  language=Kotlin % Default language (not used here but kept per spec)
}

% Header and footer
\pagestyle{fancy}
\fancyhf{}
\rhead{Verb General Rules Explained in Hinglish}
\lhead{\leftmark}
\rfoot{Page \thepage}

% Document start
\begin{document}

% Title
\title{\textbf{\color{sectionblue}Verb General Rules Explained in Hinglish}}
\date{March 25, 2025}
\maketitle

% Introduction
\section{Introduction}
This document explains the general rules of verbs in English with a Hinglish twist, making it easy to understand for beginners. We’ll cover \textbf{\color{importantred}past, present, and future tenses}} and highlight \textbf{\color{keyconceptgreen}how these rules apply}} to regular and irregular verbs, with examples and practical tips.

% Section 1
\section{Verb General Rules Explained in Hinglish}
Here are the basic rules for verbs explained in Hinglish:

\subsection{Past Tense}
\textbf{\color{importantred}Rule:}} Verb mein "ed" add karo.  
- Matlab jab kuch pehle hua ho, toh verb ke end mein "ed" laga do.  
- \textbf{Use:} Kal ya pehle ki baat bolni ho.  
- \textbf{Example:} "Main kal khela" → "I played yesterday."  
- \textbf{\color{keyconceptgreen}Kahan Use Karna:}} Jab action pura ho chuka ho. Jaise, "She walked to school."

\subsection{Present Tense}
\textbf{\color{importantred}Rule:}} Verb unchanged ya "s/es" add karo.  
- Agar abhi ki baat hai, toh verb jaisa hai waisa hi rakho (I/You/We/They ke liye). Lekin agar He/She/It hai, toh "s" ya "es" add karo.  
- \textbf{Use:} Roz ki baat ya abhi wali baat.  
- \textbf{Example:} "Main khelta hoon" → "I play." | "Woh khelti hai" → "She plays."  
- \textbf{\color{keyconceptgreen}Kahan Use Karna:}} Aadat ya sachai batani ho. Jaise, "He goes to work."

\subsection{Future Tense}
\textbf{\color{importantred}Rule:}} Verb ke pehle "will" lagao.  
- Jab baat aage ki ho, toh bas verb ke aage "will" daal do.  
- \textbf{Use:} Agli baat bolni ho.  
- \textbf{Example:} "Main kal khelunga" → "I will play tomorrow."  
- \textbf{\color{keyconceptgreen}Kahan Use Karna:}} Future plans ya predictions. Jaise, "It will rain."

% Section 2
\section{Problem: Har Verb Yeh Rule Kyun Nahi Follow Karta?}
Tu bola ki "Not every verb follow this rule" – yeh sahi hai! Iska reason hai ki English mein do tarah ke verbs hote hain:  
\begin{itemize}
    \item \textbf{\color{importantred}Regular Verbs:}} Yeh "ed", "s/es", ya "will" wale rule follow karte hain. Jaise "walk" → "walked", "play" → "plays".  
    \item \textbf{\color{importantred}Irregular Verbs:}} Yeh apna alag style rakhte hain, inka past ya present form badal jata hai bina "ed" ke. Jaise "go" → "went" (nahi "goed"), "eat" → "ate" (nahi "eated").  
\end{itemize}

\textbf{\color{keyconceptgreen}Kyun Aisa Hota Hai?}}  
Irregular verbs purane English se aaye hain aur inka apna pattern hai. Inko yaad karna padta hai, kyunki yeh rule se nahi chalte.

% Section 3
\section{General Rule Jo Zyadatar Verbs Pe Kaam Kare}
Agar tujhe confusion hai ki kaunsa verb regular hai ya irregular, toh yeh trick try kar:  
\begin{enumerate}
    \item \textbf{Past:} Agar verb regular lagta hai, toh "ed" add kar do. Nahi toh common irregular verbs (jaise go-went, eat-ate, see-saw) yaad rakho ya guess karo aur check kar lo.  
        - \textbf{Example:} "I danced" (regular) | "I saw" (irregular).  
    \item \textbf{Present:} He/She/It ke liye "s/es" add karo, baaki cases mein verb same rakho. Irregular ho ya regular, yeh mostly kaam karta hai.  
        - \textbf{Example:} "She runs" (regular) | "He eats" (irregular).  
    \item \textbf{Future:} Har verb ke saath "will" laga do – yeh sab pe kaam karta hai!  
        - \textbf{Example:} "I will jump" (regular) | "I will go" (irregular).  
\end{enumerate}

\textbf{\color{importantred}Note Keyword:}} \textit{Pattern} – Regular verbs ka pattern fixed hai, irregular ka alag hai, lekin "will" future mein sabko cover karta hai.

% Section 4
\section{Kahan Use Karna, Kahan Nahi}
\begin{itemize}
    \item \textbf{\color{keyconceptgreen}Use:}} Yeh rules daily speaking mein kaam aate hain jab tu time ke hisaab se baat karna chahta hai – kal kya kiya, aaj kya kar raha hai, kal kya karunga.  
    \item \textbf{\color{importantred}Nahi Use Karna:}} Agar verb irregular hai aur tu "ed" forcefully laga dega, toh galat ho jayega. Jaise "I goed" bolna galat hai, "I went" bolo.
\end{itemize}

% Conclusion
\section{Summary}
Past mein "ed", present mein "s/es" (He/She/It ke liye), aur future mein "will" basic rule hai. \textbf{\color{importantred}Regular verbs}} yeh follow karte hain, irregular nahi. Trick yeh hai ki \textbf{\color{keyconceptgreen}common irregular verbs yaad karo}} (jaise go-went, take-took), baaki cases mein yeh rule laga do. Speaking ke liye yeh kaafi hai!

===============================
===============================
\hrule

\begin{itemize}
  

\setlength{\parindent}{0pt}
\setlength{\parskip}{1em}

% Colors
\definecolor{sectionblue}{RGB}{0, 0, 255}
\definecolor{importantred}{RGB}{255, 0, 0}
\definecolor{keyconceptgreen}{RGB}{0, 128, 0}
\definecolor{codegray}{RGB}{240, 240, 240}
\definecolor{keywordblue}{RGB}{0, 0, 255}
\definecolor{stringred}{RGB}{255, 0, 0}
\definecolor{commentgreen}{RGB}{0, 128, 0}
\definecolor{functionorange}{RGB}{255, 165, 0}
\definecolor{variablepurple}{RGB}{128, 0, 128}

% Section formatting
\titleformat{\section}
  {\normalfont\Large\bfseries\color{sectionblue}}
  {\thesection}{1em}{}
\titleformat{\subsection}
  {\normalfont\large\bfseries\color{sectionblue}}
  {\thesubsection}{1em}{}

% Custom Kotlin language definition for listings (not used here, but included per spec)
\lstdefinelanguage{Kotlin}{
  keywords={val, var, fun, return, if, else, while, for, when, class, object, interface, package, import},
  keywordstyle=\color{keywordblue},
  comment=[l]{//},
  commentstyle=\color{commentgreen},
  string=[b]{"},
  stringstyle=\color{stringred},
  identifierstyle=\color{variablepurple},
  functionstyle=\color{functionorange},
  morecomment=[s]{/*}{*/},
  morestring=[b]',
  morestring=[b]"""
}

% Code listing setup
\lstset{
  backgroundcolor=\color{codegray},
  basicstyle=\ttfamily\footnotesize,
  breaklines=true,
  frame=single,
  numbers=left,
  numberstyle=\tiny\color{gray},
  showstringspaces=false,
  tabsize=2,
  language=Kotlin % Default language (not used here but kept per spec)
}

% Header and footer
\pagestyle{fancy}
\fancyhf{}
\rhead{Simple Present Tense in Hinglish}
\lhead{\leftmark}
\rfoot{Page \thepage}

% Document start
\begin{document}

% Title
\title{\textbf{\color{sectionblue}Simple Present Tense in Hinglish}}
\author{Your Name}
\date{March 25, 2025}
\maketitle

% Introduction
\section{Introduction}
Simple Present Tense ka matlab hai abhi ki baat ya roz ki baat jo hamesha hoti hai. Iska rule simple hai:  
\begin{itemize}
    \item \textbf{\color{importantred}First Person (I, We):}} Verb jaisa hai waisa hi rehta hai, kuch add nahi karte.  
    \item \textbf{\color{importantred}Second Person (You):}} Bhi verb same rehta hai.  
    \item \textbf{\color{importantred}Third Person (He, She, It):}} Verb mein "s" ya "es" add karte hain.  
\end{itemize}
\textbf{\color{keyconceptgreen}Formula:}} Subject + Verb (ya Verb + s/es, agar third person ho).

% Section 1
\section{Examples}
Yeh examples se samajh aayega kaise use karna hai:

\subsection{First Person (I, We)}
\begin{itemize}
    \item "Main khelta hoon" → "I play."  
    \item "Hum padhte hain" → "We read."  
    \item \textbf{\color{keyconceptgreen}Point:}} Verb mein kuch nahi jodna, bas simple rakho.
\end{itemize}

\subsection{Second Person (You)}
\begin{itemize}
    \item "Tu khata hai" → "You eat."  
    \item \textbf{\color{keyconceptgreen}Point:}} Yahan bhi verb same rehta hai, no change.
\end{itemize}

\subsection{Third Person (He, She, It)}
\begin{itemize}
    \item "Woh khelta hai" → "He plays." (s add kiya)  
    \item "Woh jati hai" → "She goes." (es add kiya, kyunki verb "go" hai)  
    \item "Yeh kaam karta hai" → "It works." (s add kiya)  
\end{itemize}
\textbf{\color{importantred}Note:}} Agar verb "o", "s", "sh", "ch", ya "x" pe end hota hai, toh "es" lagao (jaise go → goes, watch → watches). Baaki mein bas "s".

% Section 2
\section{Real Life Use}
Simple Present Tense ka real-world use samajh lo:

\subsection{Kahan Use Karna}
\begin{itemize}
    \item \textbf{\color{keyconceptgreen}Roz ki aadat batane ke liye:}} "I wake up at 6." (Main roz 6 baje uthta hoon.)  
    \item \textbf{\color{keyconceptgreen}Sachai ya facts ke liye:}} "The sun rises in the east." (Suraj purab mein ugta hai.)  
    \item \textbf{\color{keyconceptgreen}Feelings ya states ke liye:}} "She likes coffee." (Usko coffee pasand hai.)  
\end{itemize}

\subsection{Kahan Nahi Use Karna}
\begin{itemize}
    \item \textbf{\color{importantred}Abhi chal rahi cheez ke liye mat bolo:}} Jaise "I eat food now" galat hai, sahi hoga "I am eating food now" (Present Continuous).  
    \item \textbf{\color{importantred}Past ya future ki baat ke liye bhi nahi:}} "I play yesterday" galat hai, "I played yesterday" bolo.
\end{itemize}

% Section 3
\section{Speaking Mein Kaise Use Kare}
\begin{itemize}
    \item \textbf{\color{keyconceptgreen}Routine ya kisi ke baare mein:}} "Main school jata hoon" → "I go to school."  
    \item \textbf{\color{keyconceptgreen}Advice ya general baat:}} "You drink water daily" → "Tu roz paani peeta hai."
\end{itemize}

% Section 4
\section{Note Keyword: \textit{Habit}}
Yeh tense \textbf{\color{importantred}habits, facts, aur regular baaton}} ke liye perfect hai, lekin \textbf{\color{importantred}ongoing action}} ke liye nahi.

% Conclusion
\section{Summary}
Simple Present Tense bolne mein tab use hota hai jab tu roz ki baat, sachai, ya aadat batana chahta hai. \textbf{\color{importantred}First (I, We) aur second (You) person mein verb same rehta hai}}, third (He, She, It) mein "s" ya "es" add karo. Isko past ya abhi chal rahi cheez ke liye mat use karna. \textbf{\color{keyconceptgreen}Speaking ke liye yeh basic aur kaam ka tense hai!}

===============================
===============================
\hrule


\setlength{\parindent}{0pt}
\setlength{\parskip}{1em}

% Colors
\definecolor{sectionblue}{RGB}{0, 0, 255}
\definecolor{importantred}{RGB}{255, 0, 0}
\definecolor{keyconceptgreen}{RGB}{0, 128, 0}
\definecolor{codegray}{RGB}{240, 240, 240}
\definecolor{keywordblue}{RGB}{0, 0, 255}
\definecolor{stringred}{RGB}{255, 0, 0}
\definecolor{commentgreen}{RGB}{0, 128, 0}
\definecolor{functionorange}{RGB}{255, 165, 0}
\definecolor{variablepurple}{RGB}{128, 0, 128}

% Section formatting
\titleformat{\section}
  {\normalfont\Large\bfseries\color{sectionblue}}
  {\thesection}{1em}{}
\titleformat{\subsection}
  {\normalfont\large\bfseries\color{sectionblue}}
  {\thesubsection}{1em}{}

% Custom Kotlin language definition for listings (not used here, but included per spec)
\lstdefinelanguage{Kotlin}{
  keywords={val, var, fun, return, if, else, while, for, when, class, object, interface, package, import},
  keywordstyle=\color{keywordblue},
  comment=[l]{//},
  commentstyle=\color{commentgreen},
  string=[b]{"},
  stringstyle=\color{stringred},
  identifierstyle=\color{variablepurple},
  functionstyle=\color{functionorange},
  morecomment=[s]{/*}{*/},
  morestring=[b]',
  morestring=[b]"""
}

% Code listing setup
\lstset{
  backgroundcolor=\color{codegray},
  basicstyle=\ttfamily\footnotesize,
  breaklines=true,
  frame=single,
  numbers=left,
  numberstyle=\tiny\color{gray},
  showstringspaces=false,
  tabsize=2,
  language=Kotlin % Default language (not used here but kept per spec)
}

% Header and footer
\pagestyle{fancy}
\fancyhf{}
\rhead{Verb Conjugation Explained in Hinglish}
\lhead{\leftmark}
\rfoot{Page \thepage}

% Document start
\begin{document}

% Title
\title{\textbf{\color{sectionblue}Verb Conjugation Explained in Hinglish}}
\author{Your Name}
\date{March 25, 2025}
\maketitle

% Introduction
\section{Verb Conjugation Kya Hai?}
Verb conjugation ka matlab hai verb ko subject ke hisaab se change karna – jaise I, You, He, They ke saath alag-alag form use hota hai. Yeh \textbf{\color{importantred}"to be" verb (am, is, are, was, were)}} ke saath clearly dikhta hai.

% Section 1
\section{Singular aur Plural Kya Hai?}
\begin{itemize}
    \item \textbf{\color{importantred}Singular:}} Ek insaan ya cheez ke liye. Jaise "I", "He", "She", "It".  
    \item \textbf{\color{importantred}Plural:}} Ek se zyada ke liye. Jaise "We", "You" (group), "They".  
\end{itemize}
\textbf{\color{keyconceptgreen}Example:}}  
- Singular: "He runs." (Ek banda bhaagta hai)  
- Plural: "They run." (Sab bhaagte hain)

% Section 2
\section{Persons aur "To Be" Verb Ke Forms}
Yeh "to be" verb ke forms hain different persons ke liye:

\subsection{Present Tense}
\begin{itemize}
    \item \textbf{1st Person:}  
        \begin{itemize}
            \item Singular: "I am" (Main hoon) → "I am happy."  
            \item Plural: "We are" (Hum hain) → "We are friends."  
        \end{itemize}
    \item \textbf{2nd Person:}  
        \begin{itemize}
            \item Singular: "You are" (Tu hai) → "You are smart."  
            \item Plural: "You are" (Tum log ho) → "You are late." (Same form dono ke liye)  
        \end{itemize}
    \item \textbf{3rd Person:}  
        \begin{itemize}
            \item Singular: "He/She/It is" (Woh hai) → "He is tall." | "It is a dog."  
            \item Plural: "They are" (Woh log hain) → "They are here."  
        \end{itemize}
\end{itemize}

\subsection{Past Tense}
\begin{itemize}
    \item \textbf{1st Person:}  
        \begin{itemize}
            \item Singular: "I was" (Main tha) → "I was tired."  
            \item Plural: "We were" (Hum the) → "We were at home."  
        \end{itemize}
    \item \textbf{2nd Person:}  
        \begin{itemize}
            \item Singular: "You were" (Tu tha) → "You were late."  
            \item Plural: "You were" (Tum log the) → "You were busy." (Same form)  
        \end{itemize}
    \item \textbf{3rd Person:}  
        \begin{itemize}
            \item Singular: "He/She/It was" (Woh tha) → "He was sick."  
            \item Plural: "They were" (Woh log the) → "They were outside."  
        \end{itemize}
\end{itemize}

% Section 3
\section{Speaking Mein Kahan Use Karna}
\begin{itemize}
    \item \textbf{\color{keyconceptgreen}Present (am, is, are):}  
        \begin{itemize}
            \item Abhi ki haalat batane ke liye: "I am hungry." (Main bhookha hoon.)  
            \item Kisi ke baare mein sachai: "She is a teacher." (Woh teacher hai.)  
            \item Group ki baat: "We are students." (Hum students hain.)  
        \end{itemize}
    \item \textbf{\color{keyconceptgreen}Past (was, were):}  
        \begin{itemize}
            \item Pehle ki baat bolni ho: "I was at school." (Main school mein tha.)  
            \item Kisi ke saath kya hua: "They were happy." (Woh log khush the.)  
        \end{itemize}
\end{itemize}

% Section 4
\section{Kahan Nahi Use Karna}
\begin{itemize}
    \item \textbf{\color{importantred}Present ke liye Past mat bolo:}} "I was happy now" galat hai, "I am happy now" sahi hai.  
    \item \textbf{\color{importantred}Past ke liye Present nahi:}} "She is here yesterday" galat hai, "She was here yesterday" bolo.  
    \item \textbf{\color{importantred}Action chal raha ho toh "ing" ke saath use karo, sirf "am/is/are" mat bolo:}} "I am play" galat hai, "I am playing" sahi hai.
\end{itemize}

% Section 5
\section{Real Life Examples}
\begin{itemize}
    \item \textbf{\color{keyconceptgreen}Present:}} "You are my best friend." (Tu mera best friend hai.) – Daily baat ke liye.  
    \item \textbf{\color{keyconceptgreen}Past:}} "We were late yesterday." (Hum kal late the.) – Kal ki baat ke liye.
\end{itemize}

% Section 6
\section{Note Keyword: \textit{State}}
Yeh forms (am, is, are, was, were) \textbf{\color{importantred}haal ya state}} batate hain (jaise happy, tired), action ke liye "ing" ya dusra verb add karna padta hai.

% Conclusion
\section{Summary}
Verb conjugation mein "to be" verb ko person aur time ke hisaab se badalte hain. \textbf{\color{importantred}Present mein:}} I am, You are, He/She/It is, We/They are. \textbf{\color{importantred}Past mein:}} I/He/She/It was, We/You/They were. \textbf{\color{keyconceptgreen}Speaking mein yeh haal ya pehle ki baat batane ke liye kaam aata hai.} Time ka dhyan rakho, present ko past mein mat ghuma dena!

===============================
===============================
\hrule


\setlength{\parindent}{0pt}
\setlength{\parskip}{1em}

% Colors
\definecolor{sectionblue}{RGB}{0, 0, 255}
\definecolor{importantred}{RGB}{255, 0, 0}
\definecolor{keyconceptgreen}{RGB}{0, 128, 0}
\definecolor{codegray}{RGB}{240, 240, 240}
\definecolor{keywordblue}{RGB}{0, 0, 255}
\definecolor{stringred}{RGB}{255, 0, 0}
\definecolor{commentgreen}{RGB}{0, 128, 0}
\definecolor{functionorange}{RGB}{255, 165, 0}
\definecolor{variablepurple}{RGB}{128, 0, 128}

% Section formatting
\titleformat{\section}
  {\normalfont\Large\bfseries\color{sectionblue}}
  {\thesection}{1em}{}
\titleformat{\subsection}
  {\normalfont\large\bfseries\color{sectionblue}}
  {\thesubsection}{1em}{}

% Custom Kotlin language definition for listings (not used here, but included per spec)
\lstdefinelanguage{Kotlin}{
  keywords={val, var, fun, return, if, else, while, for, when, class, object, interface, package, import},
  keywordstyle=\color{keywordblue},
  comment=[l]{//},
  commentstyle=\color{commentgreen},
  string=[b]{"},
  stringstyle=\color{stringred},
  identifierstyle=\color{variablepurple},
  functionstyle=\color{functionorange},
  morecomment=[s]{/*}{*/},
  morestring=[b]',
  morestring=[b]"""
}

% Code listing setup
\lstset{
  backgroundcolor=\color{codegray},
  basicstyle=\ttfamily\footnotesize,
  breaklines=true,
  frame=single,
  numbers=left,
  numberstyle=\tiny\color{gray},
  showstringspaces=false,
  tabsize=2,
  language=Kotlin % Default language (not used here but kept per spec)
}

% Header and footer
\pagestyle{fancy}
\fancyhf{}
\rhead{Helping Verbs, Modals, Phrasals, and Articles in Hinglish}
\lhead{\leftmark}
\rfoot{Page \thepage}

% Document start
\begin{document}

% Title
\title{\textbf{\color{sectionblue}Helping Verbs, Modals, Phrasals, and Articles in Hinglish}}
\author{Your Name}
\date{March 25, 2025}
\maketitle

% Introduction (implicit in structure)
\section{Helping Verb: "To Be" (am, is, are, was, were)}
\begin{itemize}
    \item \textbf{\color{importantred}Kya Hai:}} Yeh haal batata hai ya dusre verb ki madad karta hai.  
    \item \textbf{\color{keyconceptgreen}Kab Use Karna:}  
        \begin{itemize}
            \item Abhi chal raha ho: "I am running." (Main bhaag raha hoon.)  
            \item State ke liye: "She is smart." (Woh smart hai.)  
        \end{itemize}
    \item \textbf{\color{importantred}Kab Nahi Use Karna:}} Simple baat ke liye: "I am go" galat, "I go" bolo.  
    \item \textbf{Example:} "They are here." (Woh yahan hain.)
\end{itemize}

\section{Helping Verb: "To Do" (do, does, did)}
\begin{itemize}
    \item \textbf{\color{importantred}Kya Hai:}} Yeh sawal ya nahi ke liye madad karta hai.  
        \begin{itemize}
            \item \textbf{Does} present mein He/She/It ke saath.  
            \item \textbf{Did} past ka form hai.  
        \end{itemize}
    \item \textbf{\color{keyconceptgreen}Kab Use Karna:} "Do you play?" (Kya tu khelta hai?) | "She doesn’t know." (Woh nahi janti.)  
    \item \textbf{\color{importantred}Kab Nahi Use Karna:}} Simple bolne mein: "I do like" galat, "I like" bolo.  
    \item \textbf{Example:} "Did he go?" (Kya woh gaya?)
\end{itemize}

\section{Helping Verb: "To Have" (have, has, had)}
\begin{itemize}
    \item \textbf{\color{importantred}Kya Hai:}} Yeh owning ya past perfect ke liye hai.  
        \begin{itemize}
            \item \textbf{Has} present mein He/She/It ke saath.  
            \item \textbf{Had} past ka form hai.  
        \end{itemize}
    \item \textbf{\color{keyconceptgreen}Kab Use Karna:} "I have a pen." (Mere paas pen hai.) | "She has eaten." (Woh kha chuki hai.)  
    \item \textbf{\color{importantred}Kab Nahi Use Karna:}} Abhi ke action mein: "I have running" galat, "I am running" bolo.  
    \item \textbf{Example:} "He had left." (Woh chala gaya tha.)
\end{itemize}

\section{Modal Verb: "Can" aur "Could"}
\begin{itemize}
    \item \textbf{\color{importantred}Can:} Ability ya permission present mein.  
        \begin{itemize}
            \item \textbf{Could:} Past of "can" ya polite request.  
        \end{itemize}
    \item \textbf{\color{keyconceptgreen}Kab Use Karna:}  
        \begin{itemize}
            \item "Can": "I can swim." (Main tair sakta hoon.)  
            \item "Could": "I could swim last year." (Main pichle saal tair sakta tha.) | "Could you help?" (Kya tu madad karega?)  
        \end{itemize}
    \item \textbf{\color{importantred}Kab Nahi Use Karna:}} Future ke liye "could" nahi: "I could go tomorrow" galat, "I can go" bolo.  
    \item \textbf{Example:} "She can sing." (Woh ga sakti hai.)
\end{itemize}

\section{Modal Verb: "Will" aur "Would"}
\begin{itemize}
    \item \textbf{\color{importantred}Will:} Future ke liye.  
        \begin{itemize}
            \item \textbf{Would:} Past of "will" ya polite baat.  
        \end{itemize}
    \item \textbf{\color{keyconceptgreen}Kab Use Karna:}  
        \begin{itemize}
            \item "Will": "I will call you." (Main tujhe call karunga.)  
            \item "Would": "He said he would come." (Usne kaha woh aayega.) | "Would you like tea?" (Kya tu chai lega?)  
        \end{itemize}
    \item \textbf{\color{importantred}Kab Nahi Use Karna:}} Past ke liye "will" nahi: "I will go yesterday" galat, "I went" bolo.  
    \item \textbf{Example:} "She would help." (Woh madad karti thi.)
\end{itemize}

\section{Modal Verb: "Shall" aur "Should"}
\begin{itemize}
    \item \textbf{\color{importantred}Shall:} Future ke liye (purana style).  
        \begin{itemize}
            \item \textbf{Should:} Advice ya duty ke liye "shall" ka soft form hai.  
        \end{itemize}
    \item \textbf{\color{keyconceptgreen}Kab Use Karna:}  
        \begin{itemize}
            \item "Shall": "We shall win." (Hum jeetenge.)  
            \item "Should": "You should sleep." (Tujhe sona chahiye.)  
        \end{itemize}
    \item \textbf{\color{importantred}Kab Nahi Use Karna:}} Casual mein "shall" kam bolo, "will" zyada chalta hai.  
    \item \textbf{Example:} "He should try." (Usko koshish karni chahiye.)
\end{itemize}

\section{Modal Verb: "Must, May, Might"}
\begin{itemize}
    \item \textbf{\color{importantred}Must:} Zarurat ke liye.  
        \begin{itemize}
            \item \textbf{May:} Permission ya chance present mein.  
            \item \textbf{Might:} "May" ka past ya kam chance.  
        \end{itemize}
    \item \textbf{\color{keyconceptgreen}Kab Use Karna:}  
        \begin{itemize}
            \item "Must": "You must finish." (Tujhe khatam karna hi hai.)  
            \item "May": "You may go." (Tu ja sakta hai.)  
            \item "Might": "It might rain." (Barish ho sakti hai.)  
        \end{itemize}
    \item \textbf{\color{importantred}Kab Nahi Use Karna:}} Past ke liye "must" nahi: "I must go yesterday" galat, "I had to go" bolo.  
    \item \textbf{Example:} "She may come." (Woh aa sakti hai.)
\end{itemize}

\section{Phrasal Verb: "Carry On" aur "Put Off"}
\begin{itemize}
    \item \textbf{\color{importantred}Carry On:} Continue karna.  
        \begin{itemize}
            \item \textbf{Put Off:} Taal dena (postpone).  
        \end{itemize}
    \item \textbf{\color{keyconceptgreen}Kab Use Karna:}  
        \begin{itemize}
            \item "Carry On": "Carry on studying." (Padhai jari rakho.)  
            \item "Put Off": "I put off the meeting." (Maine meeting taal di.)  
        \end{itemize}
    \item \textbf{\color{importantred}Kab Nahi Use Karna:}} Simple verb ke liye mat daal: "I carry on my work" galat, "I continue my work" bolo.  
    \item \textbf{Example:} "Don’t put off your plan." (Apna plan mat taal.)
\end{itemize}

\section{Articles: "A, An, The"}
\begin{itemize}
    \item \textbf{\color{importantred}A:} Ek cheez, consonant sound ke liye.  
        \begin{itemize}
            \item \textbf{An:} Ek cheez, vowel sound ke liye.  
            \item \textbf{The:} Specific cheez ke liye.  
        \end{itemize}
    \item \textbf{\color{keyconceptgreen}Kab Use Karna:}  
        \begin{itemize}
            \item "A": "A cat." (Ek billi.)  
            \item "An": "An orange." (Ek santara.)  
            \item "The": "The moon." (Woh chand.)  
        \end{itemize}
    \item \textbf{\color{importantred}Kab Nahi Use Karna:}  
        \begin{itemize}
            \item General baat mein "the" nahi: "The cats are cute" nahi, "Cats are cute" bolo.  
            \item "An" consonant ke saath nahi: "An dog" galat, "A dog" bolo.  
        \end{itemize}
    \item \textbf{Example:} "I saw a bird in the sky." (Maine ek chidiya aasman mein dekhi.)
\end{itemize}

\section{Note Keyword: \textit{Support}}
Helping verbs \textbf{\color{importantred}action ko support dete hain}}, modals zarurat ya mood batate hain, phrasal verbs bolne mein style dete hain, aur articles cheez ko pehchante hain.

\section{Summary}
\begin{itemize}
    \item \textbf{\color{keyconceptgreen}"To Be" haal ya ongoing ke liye, "To Do" sawal/nahi ke liye, "To Have" owning/past ke liye.}  
    \item \textbf{\color{keyconceptgreen}"Can" ability hai, "Could" uska past. "Will" future, "Would" uska past ya polite form.}  
    \item \textbf{\color{keyconceptgreen}"Shall" purana future, "Should" advice. "Must" zarurat, "May/Might" chance.}  
    \item \textbf{\color{keyconceptgreen}"Carry On" jari rakhna, "Put Off" taalna.}  
    \item \textbf{\color{keyconceptgreen}"A/An" ek ke liye, "The" specific ke liye.}  
\end{itemize}

===============================
===============================
\hrule


\setlength{\parindent}{0pt}
\setlength{\parskip}{1em}

% Colors
\definecolor{sectionblue}{RGB}{0, 0, 255}
\definecolor{importantred}{RGB}{255, 0, 0}
\definecolor{keyconceptgreen}{RGB}{0, 128, 0}
\definecolor{codegray}{RGB}{240, 240, 240}
\definecolor{keywordblue}{RGB}{0, 0, 255}
\definecolor{stringred}{RGB}{255, 0, 0}
\definecolor{commentgreen}{RGB}{0, 128, 0}
\definecolor{functionorange}{RGB}{255, 165, 0}
\definecolor{variablepurple}{RGB}{128, 0, 128}

% Section formatting
\titleformat{\section}
  {\normalfont\Large\bfseries\color{sectionblue}}
  {\thesection}{1em}{}
\titleformat{\subsection}
  {\normalfont\large\bfseries\color{sectionblue}}
  {\thesubsection}{1em}{}

% Custom Kotlin language definition for listings (not used here, but included per spec)
\lstdefinelanguage{Kotlin}{
  keywords={val, var, fun, return, if, else, while, for, when, class, object, interface, package, import},
  keywordstyle=\color{keywordblue},
  comment=[l]{//},
  commentstyle=\color{commentgreen},
  string=[b]{"},
  stringstyle=\color{stringred},
  identifierstyle=\color{variablepurple},
  functionstyle=\color{functionorange},
  morecomment=[s]{/*}{*/},
  morestring=[b]',
  morestring=[b]"""
}

% Code listing setup
\lstset{
  backgroundcolor=\color{codegray},
  basicstyle=\ttfamily\footnotesize,
  breaklines=true,
  frame=single,
  numbers=left,
  numberstyle=\tiny\color{gray},
  showstringspaces=false,
  tabsize=2,
  language=Kotlin % Default language (not used here but kept per spec)
}

% Header and footer
\pagestyle{fancy}
\fancyhf{}
\rhead{Common English Word Pairs in Hinglish}
\lhead{\leftmark}
\rfoot{Page \thepage}

% Document start
\begin{document}

% Title
\title{\textbf{\color{sectionblue}Common English Word Pairs in Hinglish}}
\author{Your Name}
\date{March 25, 2025}
\maketitle

% Section 1
\section{"Each" aur "Every"}
\begin{itemize}
    \item \textbf{\color{importantred}Each:} Har ek ko alag-alag dekhta hai.  
        \begin{itemize}
            \item \textbf{\color{importantred}Every:} Sab ko ek saath group mein.  
        \end{itemize}
    \item \textbf{\color{keyconceptgreen}Kab Use Karna:}  
        \begin{itemize}
            \item "Each": "Each student got a pen." (Har ek student ko pen mila.)  
            \item "Every": "Every student is smart." (Har student smart hai.)  
        \end{itemize}
    \item \textbf{\color{importantred}Kab Nahi Use Karna:}  
        \begin{itemize}
            \item "Each" group ke liye nahi: "Each students" galat, "Each student" bolo.  
            \item "Every" ke saath plural nahi: "Every students" galat, "Every student" sahi.  
        \end{itemize}
    \item \textbf{Example:} "Each boy runs." (Har ladka bhaagta hai.) | "Every day is fun." (Har din mazedaar hai.)
\end{itemize}

% Section 2
\section{"Either" aur "Neither"}
\begin{itemize}
    \item \textbf{\color{importantred}Either:} Do mein se koi ek (past ka nahi hai yeh).  
        \begin{itemize}
            \item \textbf{\color{importantred}Neither:} Do mein se koi bhi nahi.  
            \item \textbf{\color{importantred}Note:} Either ke saath "or", Neither ke saath "nor".  
        \end{itemize}
    \item \textbf{\color{keyconceptgreen}Kab Use Karna:}  
        \begin{itemize}
            \item "Either": "Either tea or coffee." (Chai ya coffee mein se koi ek.)  
            \item "Neither": "Neither tea nor coffee." (Na chai na coffee.)  
        \end{itemize}
    \item \textbf{\color{importantred}Kab Nahi Use Karna:}  
        \begin{itemize}
            \item "Either" teen cheezon ke liye nahi: "Either tea, coffee, juice" galat.  
            \item "Neither" positive mein nahi: "Neither is good" nahi, "Both are bad" bolo.  
        \end{itemize}
    \item \textbf{Example:} "Either you or I will go." (Ya tu ya main jaunga.) | "Neither he nor she came." (Na woh aaya na woh.)
\end{itemize}

% Section 3
\section{"A Few" aur "A Little"}
\begin{itemize}
    \item \textbf{\color{importantred}A Few:} Thodi si cheezein (countable).  
        \begin{itemize}
            \item \textbf{\color{importantred}A Little:} Thodi si miqdaar (uncountable).  
        \end{itemize}
    \item \textbf{\color{keyconceptgreen}Kab Use Karna:}  
        \begin{itemize}
            \item "A Few": "A few books." (Kuch kitabein.)  
            \item "A Little": "A little water." (Thoda paani.)  
        \end{itemize}
    \item \textbf{\color{importantred}Difference:} "A Few" ko gin sakte ho, "A Little" ko nahi.  
    \item \textbf{\color{importantred}Kab Nahi Use Karna:}  
        \begin{itemize}
            \item "A Few" uncountable ke liye nahi: "A few water" galat.  
            \item "A Little" countable ke liye nahi: "A little pens" galat.  
        \end{itemize}
    \item \textbf{Example:} "I have a few friends." (Mere kuch dost hain.) | "Add a little salt." (Thoda namak daal.)
\end{itemize}

% Section 4
\section{"Much" aur "Many"}
\begin{itemize}
    \item \textbf{\color{importantred}Much:} Zyada miqdaar (uncountable).  
        \begin{itemize}
            \item \textbf{\color{importantred}Many:} Zyada cheezein (countable).  
        \end{itemize}
    \item \textbf{\color{keyconceptgreen}Kab Use Karna:}  
        \begin{itemize}
            \item "Much": "Much time." (Bahut waqt.)  
            \item "Many": "Many cars." (Bahut gaadiyan.)  
        \end{itemize}
    \item \textbf{\color{importantred}Kab Nahi Use Karna:}  
        \begin{itemize}
            \item "Much" countable ke liye nahi: "Much people" galat, "Many people" bolo.  
            \item "Many" uncountable ke liye nahi: "Many water" galat.  
        \end{itemize}
    \item \textbf{Example:} "I don’t have much money." (Mere paas zyada paisa nahi.) | "She has many bags." (Uske paas bahut bags hain.)
\end{itemize}

% Section 5
\section{"Fewer" vs "Less"}
\begin{itemize}
    \item \textbf{\color{importantred}Fewer:} Kam cheezein (countable).  
        \begin{itemize}
            \item \textbf{\color{importantred}Less:} Kam miqdaar (uncountable).  
        \end{itemize}
    \item \textbf{\color{keyconceptgreen}Kab Use Karna:}  
        \begin{itemize}
            \item "Fewer": "Fewer students." (Kam students.)  
            \item "Less": "Less sugar." (Kam cheeni.)  
        \end{itemize}
    \item \textbf{\color{importantred}Kab Nahi Use Karna:}  
        \begin{itemize}
            \item "Fewer" uncountable ke liye nahi: "Fewer water" galat.  
            \item "Less" countable ke liye nahi: "Less books" galat.  
        \end{itemize}
    \item \textbf{Example:} "Fewer people came." (Kam log aaye.) | "Use less oil." (Kam tel use kar.)
\end{itemize}

% Section 6
\section{"Then" vs "Than"}
\begin{itemize}
    \item \textbf{\color{importantred}Then:} Time ke liye (pehle, uske baad).  
        \begin{itemize}
            \item \textbf{\color{importantred}Than:} Tulna ke liye (comparison).  
        \end{itemize}
    \item \textbf{\color{keyconceptgreen}Kab Use Karna:}  
        \begin{itemize}
            \item "Then": "I ate, then slept." (Maine khaya, phir soya.)  
            \item "Than": "She is taller than me." (Woh mujhse lambi hai.)  
        \end{itemize}
    \item \textbf{\color{importantred}Kab Nahi Use Karna:}  
        \begin{itemize}
            \item "Then" comparison mein nahi: "He is better then me" galat.  
            \item "Than" time ke liye nahi: "I went than came" galat.  
        \end{itemize}
    \item \textbf{Example:} "First study, then play." (Pehle padh, phir khel.) | "I run faster than you." (Main tujhse tez bhaagta hoon.)
\end{itemize}

% Section 7
\section{"There" vs "Their" (Pronunciation)}
\begin{itemize}
    \item \textbf{\color{importantred}There:} Wahan (place) – Bol: "dher".  
        \begin{itemize}
            \item \textbf{\color{importantred}Their:} Unka (possession) – Bol: "dheir".  
        \end{itemize}
    \item \textbf{\color{keyconceptgreen}Kab Use Karna:}  
        \begin{itemize}
            \item "There": "There is a dog." (Wahan ek kutta hai.)  
            \item "Their": "Their house is big." (Unka ghar bada hai.)  
        \end{itemize}
    \item \textbf{\color{importantred}Kab Nahi Use Karna:}  
        \begin{itemize}
            \item "There" owning ke liye nahi: "There car" galat.  
            \item "Their" jagah ke liye nahi: "Their is a park" galat.  
        \end{itemize}
    \item \textbf{Example:} "There are books." (Wahan kitabein hain.) | "Their names are cool." (Unke naam cool hain.)
\end{itemize}

% Section 8
\section{"Too" vs "Two"}
\begin{itemize}
    \item \textbf{\color{importantred}Too:} Zyada ya bhi (excess) – Bol: "tuu".  
        \begin{itemize}
            \item \textbf{\color{importantred}Two:} Do (number) – Bol: "tu".  
        \end{itemize}
    \item \textbf{\color{keyconceptgreen}Kab Use Karna:}  
        \begin{itemize}
            \item "Too": "It’s too hot." (Yeh bahut garam hai.) | "Me too." (Main bhi.)  
            \item "Two": "Two apples." (Do seb.)  
        \end{itemize}
    \item \textbf{\color{importantred}Kab Nahi Use Karna:}  
        \begin{itemize}
            \item "Too" number ke liye nahi: "Too books" galat.  
            \item "Two" excess ke liye nahi: "It’s two much" galat.  
        \end{itemize}
    \item \textbf{Example:} "Too loud." (Bahut tez.) | "I have two pens." (Mere paas do pen hain.)
\end{itemize}

% Section 9
\section{Note Keyword: \textit{Quantity}}
Yeh sab words \textbf{\color{importantred}quantity, choice, ya comparison}} ko clear karte hain, speaking mein sahi jagah use karo!

% Section 10
\section{Summary}
\begin{itemize}
    \item \textbf{\color{keyconceptgreen}"Each" har ek, "Every" sab group mein.}  
    \item \textbf{\color{keyconceptgreen}"Either" do mein ek, "Neither" do mein koi nahi.}  
    \item \textbf{\color{keyconceptgreen}"A Few" countable, "A Little" uncountable.}  
    \item \textbf{\color{keyconceptgreen}"Much" uncountable, "Many" countable.}  
    \item \textbf{\color{keyconceptgreen}"Fewer" countable kam, "Less" uncountable kam.}  
    \item \textbf{\color{keyconceptgreen}"Then" time, "Than" comparison.}  
    \item \textbf{\color{keyconceptgreen}"There" place, "Their" unka.}  
    \item \textbf{\color{keyconceptgreen}"Too" zyada, "Two" do.}  
\end{itemize}

===============================
===============================
\hrule


\setlength{\parindent}{0pt}
\setlength{\parskip}{1em}

% Colors
\definecolor{sectionblue}{RGB}{0, 0, 255}
\definecolor{importantred}{RGB}{255, 0, 0}
\definecolor{keyconceptgreen}{RGB}{0, 128, 0}
\definecolor{codegray}{RGB}{240, 240, 240}
\definecolor{keywordblue}{RGB}{0, 0, 255}
\definecolor{stringred}{RGB}{255, 0, 0}
\definecolor{commentgreen}{RGB}{0, 128, 0}
\definecolor{functionorange}{RGB}{255, 165, 0}
\definecolor{variablepurple}{RGB}{128, 0, 128}

% Section formatting
\titleformat{\section}
  {\normalfont\Large\bfseries\color{sectionblue}}
  {\thesection}{1em}{}
\titleformat{\subsection}
  {\normalfont\large\bfseries\color{sectionblue}}
  {\thesubsection}{1em}{}

% Custom Kotlin language definition for listings (not used here, but included per spec)
\lstdefinelanguage{Kotlin}{
  keywords={val, var, fun, return, if, else, while, for, when, class, object, interface, package, import},
  keywordstyle=\color{keywordblue},
  comment=[l]{//},
  commentstyle=\color{commentgreen},
  string=[b]{"},
  stringstyle=\color{stringred},
  identifierstyle=\color{variablepurple},
  functionstyle=\color{functionorange},
  morecomment=[s]{/*}{*/},
  morestring=[b]',
  morestring=[b]"""
}

% Code listing setup
\lstset{
  backgroundcolor=\color{codegray},
  basicstyle=\ttfamily\footnotesize,
  breaklines=true,
  frame=single,
  numbers=left,
  numberstyle=\tiny\color{gray},
  showstringspaces=false,
  tabsize=2,
  language=Kotlin % Default language (not used here but kept per spec)
}

% Header and footer
\pagestyle{fancy}
\fancyhf{}
\rhead{Punctuation in Hinglish}
\lhead{\leftmark}
\rfoot{Page \thepage}

% Document start
\begin{document}

% Title
\title{\textbf{\color{sectionblue}Punctuation in Hinglish}}
\author{Your Name}
\date{March 25, 2025}
\maketitle

% Section 1
\section{Period ya Full Stop (.)}
\begin{itemize}
    \item \textbf{\color{importantred}Kya Hai:} Sentence khatam karne ke liye.  
    \item \textbf{\color{keyconceptgreen}Kab Use Karna:} Baat puri ho jaye: "I am done." (Main khatam kar chuka.)  
    \item \textbf{\color{importantred}Kab Nahi Use Karna:} Beech mein ya sawal ke liye nahi: "What are you doing." galat, "?" bolo.  
    \item \textbf{Example:} "She sleeps." (Woh soti hai.)
\end{itemize}

% Section 2
\section{Exclamation Point (!)}
\begin{itemize}
    \item \textbf{\color{importantred}Kya Hai:} Josh, excitement, ya shock ke liye.  
    \item \textbf{\color{keyconceptgreen}Kab Use Karna:} Bada feeling dikhana ho: "Wow!" (Wah!)  
    \item \textbf{\color{importantred}Kab Nahi Use Karna:} Simple baat mein nahi: "I eat!" galat, "." use karo.  
    \item \textbf{Example:} "That’s amazing!" (Yeh gajab hai!)
\end{itemize}

% Section 3
\section{Comma (,)}
\begin{itemize}
    \item \textbf{\color{importantred}Kya Hai:} Chhota break ya list ke liye.  
    \item \textbf{\color{keyconceptgreen}Kab Use Karna:}  
        \begin{itemize}
            \item List: "Apples, oranges, bananas." (Seb, santare, kele.)  
            \item Break: "I ate, then slept." (Maine khaya, phir soya.)  
        \end{itemize}
    \item \textbf{\color{importantred}Kab Nahi Use Karna:} Puri baat alag karne ke liye nahi: "I am tired, I slept" galat, "." ya ";" bolo.  
    \item \textbf{Example:} "He runs, jumps, plays." (Woh bhaagta, koodta, khelta hai.)
\end{itemize}

% Section 4
\section{Semicolon (;)}
\begin{itemize}
    \item \textbf{\color{importantred}Kya Hai:} Do badi baaton ko jodne ke liye jo related hain.  
    \item \textbf{\color{keyconceptgreen}Kab Use Karna:} "I was late; she was early." (Main late tha; woh jaldi thi.)  
    \item \textbf{\color{importantred}Kab Nahi Use Karna:} Chhoti list ke liye nahi: "Apples; oranges" galat, "," bolo.  
    \item \textbf{Example:} "He likes tea; I like coffee." (Usko chai pasand hai; mujhe coffee.)
\end{itemize}

% Section 5
\section{Colon (:)}
\begin{itemize}
    \item \textbf{\color{importantred}Kya Hai:} Explanation ya list shuru karne ke liye.  
    \item \textbf{\color{keyconceptgreen}Kab Use Karna:}  
        \begin{itemize}
            \item List: "I need: pens, books, bags." (Mujhe chahiye: pen, kitabein, bag.)  
            \item Wajah: "He’s happy: he won." (Woh khush hai: woh jeeta.)  
        \end{itemize}
    \item \textbf{\color{importantred}Kab Nahi Use Karna:} Simple break ke liye nahi: "I ate: slept" galat, "," ya ";" bolo.  
    \item \textbf{Example:} "She has one goal: to win." (Uska ek goal hai: jeetna.)
\end{itemize}

% Section 6
\section{Apostrophe (’)}
\begin{itemize}
    \item \textbf{\color{importantred}Kya Hai:} Owning ya short form ke liye.  
    \item \textbf{\color{keyconceptgreen}Kab Use Karna:}  
        \begin{itemize}
            \item Owning: "Ravi’s book." (Ravi ki kitab.)  
            \item Short: "It’s raining." (It is ka short form.)  
        \end{itemize}
    \item \textbf{\color{importantred}Kab Nahi Use Karna:} Plural ke liye nahi: "Cats’" galat, "Cats" bolo jab owning nahi hai.  
    \item \textbf{Example:} "This is John’s pen." (Yeh John ka pen hai.)
\end{itemize}

% Section 7
\section{Quotation Marks (" ")}
\begin{itemize}
    \item \textbf{\color{importantred}Kya Hai:} Kisi ke bole hue words ya special cheez ke liye.  
    \item \textbf{\color{keyconceptgreen}Kab Use Karna:} "He said, ‘Hi!’" (Usne kaha, "Hi!")  
    \item \textbf{\color{importantred}Kab Nahi Use Karna:} Random baat ke liye nahi: "I eat" food galat.  
    \item \textbf{Example:} "She shouted, ‘Run!’" (Usne chillaya, "Bhaag!")
\end{itemize}

% Section 8
\section{Parentheses ( ) ya Brackets}
\begin{itemize}
    \item \textbf{\color{importantred}Kya Hai:} Extra info ke liye.  
    \item \textbf{\color{keyconceptgreen}Kab Use Karna:} "He’s tall (6 feet)." (Woh lamba hai (6 feet).)  
    \item \textbf{\color{importantred}Kab Nahi Use Karna:} Main baat ke liye nahi: "I (eat)" galat.  
    \item \textbf{Example:} "I study (at night)." (Main padhta hoon (raat ko).)
\end{itemize}

% Section 9
\section{Braces \{ \}}
\begin{itemize}
    \item \textbf{\color{importantred}Kya Hai:} Set ya group dikhane ke liye (kam use hota hai).  
    \item \textbf{\color{keyconceptgreen}Kab Use Karna:} "Options \{A, B, C\}." (Vikalp \{A, B, C\}.)  
    \item \textbf{\color{importantred}Kab Nahi Use Karna:} Normal baat mein nahi, zyada formal ya tech mein hota hai.  
    \item \textbf{Example:} "Choose \{red, blue\}." (Chun \{laal, neela\}.)
\end{itemize}

% Section 10
\section{"i.e.", "etc.", "e.g."}
\begin{itemize}
    \item \textbf{\color{importantred}i.e.:} Yani (meaning explain karne ke liye).  
        \begin{itemize}
            \item \textbf{\color{importantred}etc.:} Aur bhi (list khatam nahi).  
            \item \textbf{\color{importantred}e.g.:} Misal ke liye (example ke liye).  
        \end{itemize}
    \item \textbf{\color{keyconceptgreen}Kab Use Karna:}  
        \begin{itemize}
            \item "i.e.": "I love sweets, i.e., laddoos." (Mujhe mithai pasand hai, yani laddoo.)  
            \item "etc.": "Pens, books, etc." (Pen, kitabein, aur bhi.)  
            \item "e.g.": "Fruits, e.g., apple." (Phal, misal ke liye seb.)  
        \end{itemize}
    \item \textbf{\color{importantred}Kab Nahi Use Karna:}  
        \begin{itemize}
            \item "i.e." example ke liye nahi, "e.g." bolo.  
            \item "etc." jab list356 puri ho.  
        \end{itemize}
    \item \textbf{Example:} "I need tools, e.g., hammer." (Mujhe tools chahiye, jaise hammer.)
\end{itemize}

% Section 11
\section{Note Keyword: \textit{Break}}
Punctuation \textbf{\color{importantred}baat ko todne, jodne, ya clear karne ke liye hai}}, sahi jagah use karo!

% Section 12
\section{Summary}
\begin{itemize}
    \item \textbf{\color{keyconceptgreen}Full Stop:} Baat khatam.  
    \item \textbf{\color{keyconceptgreen}Exclamation:} Josh ke liye.  
    \item \textbf{\color{keyconceptgreen}Comma:} Chhota break ya list.  
    \item \textbf{\color{keyconceptgreen}Semicolon:} Badi related baatein jodna.  
    \item \textbf{\color{keyconceptgreen}Colon:} List ya wajah.  
    \item \textbf{\color{keyconceptgreen}Apostrophe:} Owning ya short form.  
    \item \textbf{\color{keyconceptgreen}Quotation:} Bolne wali baat.  
    \item \textbf{\color{keyconceptgreen}Parentheses/Braces:} Extra info.  
    \item \textbf{\color{keyconceptgreen}i.e./etc./e.g.:} Meaning, aur bhi, ya example.  
\end{itemize}
Speaking mein flow aur matlab ke liye inka dhyan rakho!

===============================
===============================
\hrule


\end{document}