\documentclass[a4paper]{article}
\usepackage[margin=25mm]{geometry}
\usepackage{xcolor}
\usepackage{tcolorbox}
\tcbuselibrary{listings, breakable}
\usepackage{listings}
\usepackage[parfill]{parskip}
\usepackage{enumitem}
\usepackage{titlesec}
\usepackage{fancyhdr}
\usepackage{caption}
\usepackage[sfdefault]{roboto}
\usepackage{mathptmx}
\usepackage{hyperref}

\usepackage{longtable}
\usepackage{array}

% Palette
\definecolor{HeadingBlue}{RGB}{0,119,187}
\definecolor{ImportantRed}{RGB}{227,27,35}
\definecolor{CodeBackground}{RGB}{247,247,247}
\definecolor{KeywordColor}{RGB}{85,55,123}
\definecolor{StringColor}{RGB}{180,130,0}
\definecolor{CommentColor}{RGB}{42,122,52}
\definecolor{FunctionColor}{RGB}{0,63,154}
\definecolor{NumberColor}{RGB}{93,173,93}

% Heading styles
\titleformat{\section}
  {\LARGE\sffamily\color{HeadingBlue}}
  {\thesection}
  {0em}
  {\centering}
\titlespacing*{\section}{0pt}{2em}{1em}

% Important text (red)
\newcommand\imp[1]{{\color{ImportantRed}#1}}

% Code box setup
\lstset{
  language=sh,
  basicstyle=\ttfamily,
  backgroundcolor=\color{CodeBackground},
  frame=tb,
  framerule=0pt,
  aboveskip=6pt,
  belowskip=6pt,
  showstringspaces=false,
  numbers=left,
  numberstyle=\footnotesize\color{gray},
  numbersep=10pt,
  numberblanklines=false,
  breaklines=true,
  prebreak={\mbox{\textcolor{gray}{$\hookrightarrow$}\space}},
  postbreak={\mbox{\space\textcolor{gray}{$\hookleftarrow$}}},
  keywordstyle=\color{KeywordColor},
  stringstyle=\color{StringColor},
  commentstyle=\color{CommentColor},
  identifierstyle=\color{FunctionColor},
  literate={*}{{\textcolor{KeywordColor}{*}}}1
    {=}{{\textcolor{KeywordColor}{=}}}1
    {|}{{\textcolor{KeywordColor}{|}}}1
    {>}{{\textcolor{KeywordColor}{>}}}1
    {<}{{\textcolor{KeywordColor}{<}}}1
    {-}{{\textcolor{KeywordColor}{-}}}1
    {/}{{\textcolor{KeywordColor}{/}}}1
    {(}{{\textcolor{KeywordColor}{(}}}1
    {)}{{\textcolor{KeywordColor}{)}}}1
    {[}{{\textcolor{KeywordColor}{[}}}1
    {]}{{\textcolor{KeywordColor}{]}}}1
    {+}{{\textcolor{KeywordColor}{+}}}1,
  escapeinside={*@}{@*}
}

\newtcblisting{codeblock}[1][sh]{
  listing only,
  listing options={language=#1,style=tcblatex},
  colback=CodeBackground,
  arc=3pt,
  boxrule=0.4pt,
  colframe=gray,
  size=fbox,
  left=4pt,
  right=4pt,
  top=4pt,
  bottom=4pt,
  breakable,
}

\pagestyle{fancy}
\fancyhf{}
\renewcommand{\headrulewidth}{0pt}
\fancyhead[L]{\footnotesize React Native Notes}
\fancyhead[R]{\footnotesize \thepage}

\begin{document}

===================================================================
\section*{React Native Setup \& Basics}
\rmfamily

React Native ek powerful framework hai jo JavaScript use karke ek hi codebase se dono (Android aur iOS) ke liye native mobile apps bana sakte ho. Setup karne ke liye \imp{Node.js} aur \imp{Java} (Android ke liye), \imp{Xcode} (iOS ke liye) chahiye hota hai.

Aap \imp{Android Studio} ya \imp{Xcode} aur \imp{expo} ya \imp{react-native-cli} use kar sakte ho project start karne ke liye.  
Pehle hi baat: \imp{npx react-native init ProjectName} se project create hota hai. Fir \texttt{cd ProjectName} se us project ke folder mein chale jao.

===================================================================
\section*{React Native Commands}
\subsection*{1. npx react-native init ProjectName}
\imp{Kya karta hai?}\\
Yeh ek naya React Native project create karta hai, projectName ke naam se. Uske andar source code, dependencies aur ek ready-to-run mobile app ka structure generate hota hai.

\imp{Kyun use karein?}\\
Kyunki koi project start karne ke liye ye command basic step hai. Without this, project ka structure nahi milega, app banayege kahan se?

\imp{Agar nahi karenge toh kya hoga?}\\
Project structure hi nhi milega, app bana hi nahi payenge.

\vspace{0.9em}
\subsection*{2. npx react-native start}
\imp{Kya karta hai?}\\
Yeh Metro bundler start karta hai – jo JavaScript code ko compile karke device/emulator pe run karne ke liye bundle banata hai.

\imp{Kyun use karein?}\\
App development ke time, Metro bundler ke bina, JS code changes device/emulator pe reflect nahi honge.

\vspace{0.9em}
\subsection*{3. npx react-native start --reset-cache}
\imp{Kya karta hai?}\\
Yeh Metro bundler ko restart karta hai \imp{cache ko clear kar ke}. Cache clear karne se bundler pura naya build karta hai, old cached files se effect nahi aata.

\imp{Kyun use karein?}\\
Agar app mein koi style change ya code change reflect nahi ho raha, errors aa rahe ho, bundler stuck ho gaya hai, ya hot reloading kaam nahi kar raha, toh iss command se bundler restart hoga aur fresh start ho jata hai.

\imp{Agar use na karein toh kya hoga?}\\
Old cached files ke karan app mein bugs ya unexpected behavior aa sakte hain, ya changes reflect nahi honge.

\imp{Kab use karein?}\\
App pe bug ya unexpected behavior dikhe, ya changes reflect nahi ho rahe ho, toh Metro bundler ko ye command dekar restart karein.

\imp{Example}\\
Open terminal, app ke root folder pe:

\begin{codeblock}
npx react-native start --reset-cache
\end{codeblock}

\vspace{0.9em}
\subsection*{4. npx react-native run-android}
\imp{Kya karta hai?}\\
Yeh command apni app ko Android device/emulator pe install aur run kar deta hai.

\imp{Kyun use karein?}\\
Apni app ko Android pe check karna hai, ya test karna hai toh ye command zaruri hai.

\imp{Agar use na karein toh kya hoga?}\\
App device pe nahi chal payegi.

\imp{Kab use karein?}\\
Jab app ka development kar rahe ho aur check karna chahte ho ki Android device/emulator mein kaise run ho raha hai.

\vspace{0.9em}
\subsection*{5. cd android}
\imp{Kya karta hai?}\\
Yeh command app ke \imp{android} folder mein chale jata hai.

\imp{Kyun use karein?}\\
Android ke native operations, like Gradle build commands chalaane ke liye.

\vspace{0.9em}
\subsection*{6. ./gradlew clean}
\imp{Kya karta hai?}\\
Yeh command Android project ke build folder ko clean kar deta hai, sab old build files delete ho jaate hain.

\imp{Kyun use karein?}\\
Agar app build karne mein errors aa rahe ho, ya build fail ho raha ho, ya strange behavior dikh raha ho, toh clean command se build cache clear hota hai aur naya build banta hai.

\imp{Agar use na karein toh kya hoga?}\\
Build errors ya cache related problems rahenge.

\imp{Kab use karein?}\\
Jab Android project mein koi build error ho, ya app run nahi ho rahi ho, ya app behavior unexpected ho, toh iss command se cache clean karein.

\imp{Example}\\
App ke root folder pe:

\begin{codeblock}
cd android
./gradlew clean
\end{codeblock}

Fir \texttt{cd ..} se wapas app ke root pe aao.

\vspace{0.9em}
\section*{Saare Commands Ek Sath: Common Flow}
- \imp{Create Project:}\\
  \texttt{npx react-native init MyApp}
- \imp{Go to Project Directory:}\\
  \texttt{cd MyApp}
- \imp{Start Metro Bundler:}\\
  \texttt{npx react-native start}\\
  (ya fir \texttt{npx react-native start --reset-cache} agar issues ho)
- \imp{Run App on Android:}\\
  \texttt{npx react-native run-android}
- \imp{Clean Android Build (if needed):}
  \begin{codeblock}
cd android
./gradlew clean
\end{codeblock}
  Fir wapas \texttt{cd ..} pe aao.

\vspace{0.5em}
\section*{Summary Table: Commands, Usages, Aur Effects}

\begin{center}
\small % Reduce font size to fit content
\begin{longtable}{|p{2.5cm}|p{3cm}|p{3cm}|p{3cm}|p{3cm}|}
\hline
\textbf{Command} & \textbf{Kya Karta Hai} & \textbf{Kyun Use Karein} & \textbf{Agar Na Karen Toh?} & \textbf{Example Use Case} \\
\hline
\endfirsthead

\multicolumn{5}{c}%
{{\bfseries Summary Table: Commands, Usages, Aur Effects - continued from previous page}} \\
\hline
\textbf{Command} & \textbf{Kya Karta Hai} & \textbf{Kyun Use Karein} & \textbf{Agar Na Karen Toh?} & \textbf{Example Use Case} \\
\hline
\endhead

\hline \multicolumn{5}{|r|}{{Continued on next page}} \\ \hline
\endfoot

\hline
\endlastfoot

npx react-native init MyApp & New project banata hai & App shuru karne ke liye & Project create hi nahi hoga & Sabse pehla step – app banana hai \\
\hline

npx react-native start & Metro bundler start karta hai & App development/testing ke liye & Changes reflect nahi honge & Normal development time \\
\hline

npx react-native start --reset-cache & Cache clear karke bundler restart karta hai & Jaldi naye changes, bugs resolve & Old issues theek nahi honge & Errors, unexpected behavior, changes reflect nahi ho \\
\hline

npx react-native run-android & App Android pe run karta hai & App check/run karne ke liye & App device pe run nahi hogi & Development, testing \\
\hline

cd android & Android folder mein enter karein & Android-specific operations ke liye & Android commands run nahi kar payenge & App build error/time \\
\hline

./gradlew clean & Android build cache clean karta hai & Clean build, errors resolve & Build errors rahenge & Build issues, app run nahi ho rahi \\
\hline

\end{longtable}
\end{center}

\vspace{1em}
\section*{Conclusion}
In commands ko samajhna zaruri hai kyuki har command ka apna purpose hai – kuch project setup, kuch app run, kuch clean build, kuch debugging. Agar kisi command ka sahi use na karo, ya use na karo, toh app mein bugs, build errors, ya unexpected behavior dikh sakta hai.  
Sahi tareeke se sab use karna sikho, toh kisi bhi problem se quickly recover kar payenge – aur app development fast, smooth aur professional hota hai!

=============================================================
\section*{ADB Basics – Kya Hai Aur Kaam Kaise Karta Hai?}
\imp{ADB (Android Debug Bridge)} ek command-line tool hai jo aapko Android devices (real ya emulator) se communicate karne mein help karta hai. Developer, tester, ya koi bhi techie apne computer se Android device control kar sakta hai, directly code ki help se. ADB ka use karke, app install/uninstall kar sakte ho, files copy kar sakte ho, device restart ya reboot kar sakte ho, logcat log dekh sakte ho, screen record, screenshot, aur bahut kuch kar sakte ho.

\imp{ADB ka main purpose:}\\
- \imp{Debugging:} App ke bugs ko find and fix karna\\
- \imp{Device Testing:} Apps test karna\\
- \imp{File Transfer:} File upload/download karna\\
- \imp{System Control:} Device reboot, shell access, logs capture karna\\
- \imp{Automation:} Automation testing ke liye adb command use hota hai

\imp{Kuch Common ADB Commands:}\\
- \texttt{adb devices} – Connected Android devices list karta hai\\
- \texttt{adb install app.apk} – Kisi bhi APK ko device pe install karta hai\\
- \texttt{adb logcat} – Device ka logcat (logs) real-time dekhta hai\\
- \texttt{adb reboot} – Device restart karta hai\\
- \texttt{adb shell} – Device pe shell access deta hai (jaise terminal)\\
- \texttt{adb kill-server} – ADB server restart karta hai (jab devices show nahi ho rahe ho)

\imp{Kisne use kiya chahiye?}\\
App developer, tester, ya jo Android pe kuch bhi advanced karna chahe, uske liye ADB must hai.

\vspace{1em}
\section*{React Native Commands – npx react-native start, npx react-native run-android. Kab Kon Sa Use Karein?}
\imp{npx react-native start}\\
- \imp{Kya karta hai?}\\
  Ye \imp{Metro bundler} start karta hai. Metro bundler, apne JavaScript code ko compile karke device/emulator ke liye ek bundle banata hai.\\
- \imp{Kyun use karein?}\\
  Development ke time pe, Metro bundler running ho chahiye nahi toh changes reflect nahi honge device pe.\\
- \imp{Kab use karein?}\\
  Jab app ka development kar rahe ho, kuch change kiya ho, toh jab bhi phone/emulator pe reload/refresh karenge, Metro bundler code ko refresh karega.\\
- \imp{Agar ye nahi chalaoge toh?}\\
  Changes reflect nahi honge. Bundler ke bina app run hi nahi hogi (live reload, error display bhi nahi hoga).

\imp{npx react-native run-android}\\
- \imp{Kya karta hai?}\\
  Ye command apni app ko \imp{Android device/emulator} pe build, install, aur run kar deta hai.\\
- \imp{Kyun use karein?}\\
  Apni app ko Android pe run karna hai, test karna hai, ya koi native part change kiya hai toh, ye command hi chalana padega.\\
- \imp{Kab use karein?}\\
  Jab app ka native side (jaise AndroidManifest.xml, new library, etc.) kuch change hua hai, ya platform-specific code likha hai.\\
- \imp{Agar ye command nahi chalaoge toh?}\\
  App device/emulator pe install hi nahi hogi, yani test ya run hi nahi kar payenge.

\vspace{1em}
\section*{Dono Commands ka Farq – Kya, Kyun, Kab?}
\begin{center}
\small % Reduce font size to fit content
\begin{longtable}{|p{2.5cm}|p{3.5cm}|p{3cm}|p{2.5cm}|p{3.5cm}|}
\hline
\textbf{Command} & \textbf{Kya Karta Hai} & \textbf{Kyun Important Hai?} & \textbf{Kab Use Karein?} & \textbf{Agar Na Chalaoge Toh?} \\
\hline
\endfirsthead

\multicolumn{5}{c}%
{{\bfseries Continued Table - Commands Summary}} \\
\hline
\textbf{Command} & \textbf{Kya Karta Hai} & \textbf{Kyun Important Hai?} & \textbf{Kab Use Karein?} & \textbf{Agar Na Chalaoge Toh?} \\
\hline
\endhead

\hline \multicolumn{5}{|r|}{{Continued on next page}} \\ \hline
\endfoot

\hline
\endlastfoot

npx react-native start & Metro bundler start karta hai & JS code changes reflect karne ke liye & Dev time, jab code change ho & No live reload, no error display, app naya code pe nahi chalega \\
\hline

npx react-native run-android & App ko device/emulator pe build-install-run & App ko test/run karna & Native code change hai, ya app install karna hai & App install nahi hogi, device pe nahi chalegi \\
\hline

\end{longtable}
\end{center}

\imp{Real Life Example:}\\
Suppose aapne app.js mein kuch change kiya. Sirf \texttt{npx react-native start} chalate raho, app reload karo, changes dikh jayenge.  
Lekin agar aapne AndroidManifest.xml ya koi native library add/remove ki hai, toh \texttt{npx react-native run-android} chalana padega, tabhi changes device pe apply honge.

\vspace{1em}
\section*{Dono Chalaney ka Saahi Process}
1. \imp{Metro bundler chalao:}\\
   \texttt{npx react-native start}\\
   (Isse bundler shuru ho jayega, app reload pe code changes reflect honge.)
2. \imp{Agle terminal se:}\\
   \texttt{npx react-native run-android}\\
   (Isse app device/emulator pe build, install aur run hogi.)

\imp{Kya Hoga Agar Sirf Run-Android Chalao?}\\
Windows mein, \texttt{npx react-native run-android} chalaane se automatically bundler bhi chalu ho jata hai.  
Lekin Linux pe, alag se bundler chalu karna padta hai.  
Best practice hai dono command chalao:  
Pehle \texttt{npx react-native start}, phir alag terminal pe \texttt{npx react-native run-android}.  
Isse live reload, error display, aur app run dono sahi se honge.

\vspace{1em}
\section*{Saara Doubt Clear – Final Tips}
- \imp{Jab bhi React Native ki development karo, Metro bundler run hona chahiye.}\\
  Metro bundler ke bina, JS code changes reflect nahi honge device pe.
- \imp{Jab bhi native code (Android/iOS) mein kuch change karo, naya library add karo, toh npx react-native run-android ya run-ios chalana padega.}\\
  Sirf Metro bundler ke bina native changes reflect nahi honge.
- \imp{Metro bundler kabhi bhi close na ho warna app kaam nahi karegi.}\\
  Hot Reload/Fast Refresh Metro bundler par depend karta hai.
- \imp{Kabhi error aaye, logcat se log dekhlo (adb logcat).}
- \imp{Kabhi device show na ho, toh adb devices karke device list dekho, ya adb kill-server phir adb start-server kar do.}

\vspace{1em}
\section*{Summary in One Line}
\imp{npx react-native start} – JS code changes reflect karne ke liye;\\
\imp{npx react-native run-android} – App ko build-install-run karne ke liye;\\
Agar native code change hua ho, toh dono chalao – bundler aur run-android.\\
ADB – Device ko control, debug, log read, file transfer, aur advanced automation ke liye use karo.

\imp{Ye doubt clear ho gaya hoga ki kab, kaun sa command use karna hai. Ab toh live reload, code change, native kaam, sab smooth ho jayega!}

=============================================================

****
\section*{React Native Project Config Files: Metro \& Babel – Ek Dum Beginner Level, Pure Hinglish, Sab Cheezein Clear (Notes Type)}
\rmfamily

****
\section*{Metro – Kya Hai?}
- \imp{Metro} React Native ka default \imp{JavaScript bundler} hai.
- \imp{Bundler ka kaam:} Aapke saare JS/JSX/TypeScript files ko \imp{eke JavaScript bundle} mein convert karta hai, jo device pe chalega.
- \imp{Live Reload/Fast Refresh:} Agar aap code change karo, toh Metro bundler instantly usme reflect karta hai, app apne aap refresh ho jati hai screen pe (hot reload).
- \imp{Aur bhi karte hai:} Assets (images, fonts, etc.) bundle karna, code optimize/minify karna, aur app ko production-ready banana.

\imp{Why Important?}\\
Metro ke bina React Native app chal hi nahi sakti. Metro bundler chalao, tabhi JS code device pe aayega aur changes dikhenge.

****
\section*{metro.config.js – What Is This File?}
- \imp{metro.config.js} ek configuration file hai jo Metro bundler ki \imp{custom tuning} karne ke liye hai.
- \imp{Isme kya hota hai?} Aap settings daal sakte ho — jaise kaunsi files bundle hogi, kaunsi asset files include hogi, ya custom transformer/plugins add kar sakte ho.
- \imp{Usually:} Ye file automatically ban jati hai, aur mostly empty ya default rehti hai, kyuki Metro khud decent settings use karta hai.
- \imp{Customize kab karein?} Jab aapko bundling ke rules change karne ho, ya third-party tools integrate karne ho, tab.

\imp{Example:}\\
Normally, ye file mein sirf default config import hoti hai:

\begin{codeblock}
const { getDefaultConfig } = require('@react-native/metro-config');
const config = getDefaultConfig(__dirname);
module.exports = config;
\end{codeblock}

Isse app default Metro config par chalegi.

****
\section*{metro.config.js – Kab Change Karein? Real-life Examples}

\subsection*{1. Extra Asset Files (Images, Fonts, etc.) Bundle Karna}
\imp{Scenario:} Hume \texttt{.svg}, \texttt{.ttf}, \texttt{.webp} files bhi bundle karvani hai, Metro by default inko nahi le raha.
\imp{Solution:}\\
\texttt{metro.config.js} mein extra extensions add karo:

\begin{codeblock}
const { getDefaultConfig } = require('@react-native/metro-config');
const config = getDefaultConfig(__dirname);
config.resolver.assetExts.push('svg', 'ttf', 'webp');
module.exports = config;
\end{codeblock}

Isse aapki app in extra files ko bhi bundle karegi.

****
\subsection*{2. Monorepo (Multi-folder/Shared Code) ka Scene}
\imp{Scenario:} Aapke project ke alawa ek \texttt{shared-lib} folder bhi hai, usme code hai jo app mein use karna hai.
\imp{Solution:}\\
Ek extra folder (\texttt{watchFolders}) Metro ko batao:

\begin{codeblock}
const { getDefaultConfig } = require('@react-native/metro-config');
const path = require('path');
const config = getDefaultConfig(__dirname);
config.watchFolders = [
  ...config.watchFolders,
  path.resolve(__dirname, '../shared-lib')
];
module.exports = config;
\end{codeblock}

Isse Metro apni wajah se wahan ka code bhi bundle karega.

****
\subsection*{3. Sentry (Third-part\subsection*{3. Sentry (Third-party Tool) Integration}
\imp{Scenario:} Sentry SDK (error monitoring) use karna hai.
\imp{Solution:}\\
Sentry ke wrapper ko config mein shamil karo:

\begin{codeblock}
const { getDefaultConfig } = require('@react-native/metro-config');
const { withSentryConfig } = require('@sentry/react-native/metro');
const config = getDefaultConfig(__dirname);
module.exports = withSentryConfig(config);
\end{codeblock}

Isse Sentry bhi bundling ke waqt apna kaam kar payega.

****
\subsection*{4. Custom Transformer/Plugin Add Karna}
\imp{Scenario:} Koi extra plugin ya custom transformer chahiye ho.
\imp{Solution:}\\
Config object mein transformer/plugin add karo.

****
\section*{Agar metro.config.js File Na Ho?}
- \imp{Agar file nahi hogi, toh Metro default config use karega.} App chalti rahegi.
- \imp{Agar custom karna ho, ya third-party tool ka requirement ho, toh hi file banana zaruri hai.}

****
\section*{Agar metro.config.js Mein Change Karo, Toh Kya Karein?}
- \imp{Agar aapne \texttt{metro.config.js} ya kisi bhi Metro config file mein kuch change kiya, toh Metro bundler ko restart karna padta hai.}
- \imp{Restart kaise karein?}
  - \imp{Pehle Metro bundler band karo} (terminal jis par \texttt{npx react-native start} chal raha tha, vahan Ctrl+C dabao).
  - \imp{Naya bundler chalao:}
\begin{bashblock}
npx react-native start --reset-cache
\end{bashblock}
  \imp{--reset-cache ka matlab, pura purana cache saaf ho jayega, aur naya build hoga jo apne changes ko reflect karega.}
- \imp{Bina restart kiye Metro, naye changes apply nahi honge!}

****
\section*{babel.config.js – Kya Hai? Kyun Zaruri Hai?}
- \imp{Babel} ek \imp{JavaScript transpiler} hai, jo modern JS (\texttt{const}, \texttt{async/await}, \texttt{import/export}, etc.) ko older devices/browsers ke liye compatible code mein convert karta hai.
- \imp{React Native} bhi Babel ka use karta hai, taaki aap latest JS likh sako, lekin device par wo code chal jaye.
- \imp{babel.config.js} React Native project ki \imp{Babel settings ki configuration file} hai.
- Yahan aap \imp{presets} (e.g. React Native preset), \imp{plugins} (jaise module aliases, experimental features) define kar sakte ho.

\imp{Default Example:}\\
App root par \texttt{babel.config.js}:

\begin{codeblock}
module.exports = {
  presets: ['module:metro-react-native-babel-preset']
};
\end{codeblock}

Isse app React Native ki zaruri transformations use karegi.

****
\section*{Kab babel.config.js Mein Change Karna Padega?}

\subsection*{1. Module Aliases (Short Import Paths)}
\imp{Scenario:} Aap chahte ho ki \texttt{@components/Button} import ho sake, istead of \texttt{../../components/Button}.
\imp{Solution:}\\
Plugin \texttt{babel-plugin-module-resolver} add karo, aur config karo:

\begin{codeblock}
module.exports = function(api) {
  api.cache(true);
  return {
    presets: ['module:metro-react-native-babel-preset'],
    plugins: [
      ['module-resolver', {
        root: ['./src'],
        alias: {
          '@components': './src/components',
          '@screens': './src/screens'
        }
      }]
    ]
  };
};
\end{codeblock}

Isse aap puri project mein \texttt{@components/Button} import kar sakte hain, aur path confusion nahi rahegi.

****
\subsection*{2. Experimental JS Features (Decorators, etc.)}
\imp{Scenario:} Aap chahte ho ki \texttt{@decorator} jaise ES proposals work karein.
\imp{Solution:}\\
Extra plugin add karo:

\begin{codeblock}
module.exports = {
  presets: ['module:metro-react-native-babel-preset'],
  plugins: [
    ['@babel/plugin-proposal-decorators', { legacy: true }],
    '@babel/plugin-proposal-class-properties'
  ]
};
\end{codeblock}

Isse app development mein experimental JS features use kar sakoge.

****
\subsection*{3. Monorepo Ya Custom Build Pipeline}
\imp{Scenario:} Monorepo project hai, ya build process modified karna hai.
\imp{Solution:}\\
Custom Babel config karo, aur extra plugins/presets daalo.

****
\section*{Agar babel.config.js File Na Ho?}
- \imp{Agar file nahi hogi, toh preset \texttt{package.json} ki \texttt{babel} key mein bhi ho sakta hai.}
- \imp{Lekin best practice hai root pe file banaye rakhna.}
- \imp{Agar file delete kar do, toh \texttt{react-native} preset dekhne lagta hai, lekin agar koi custom plugin, alias, experimental feature add karna ho, toh file banana padega.}

****
\section*{Agar babel.config.js Mein Change Karo, Toh Kya Karein?}
- \imp{Agar aapne \texttt{babel.config.js} mein change kiya, toh Metro bundler ko restart karna zaruri hai.}\\
  \imp{Restart command:}
\begin{codeblock}
npx react-native start --reset-cache
\end{codeblock}
- \imp{Bina restart kiye Babel config ke changes reflect nahi honge!}

****
\section*{Summary Table: Metro aur Babel Files – Sab Kuch Ek Sath}

\small % Reduce font size to fit content
\begin{center}
\begin{longtable}{|p{2.5cm}|p{3cm}|p{3cm}|p{3cm}|p{3cm}|}
\hline
\textbf{File Name}         & \textbf{Kya Hai?}                       & \textbf{Kya Rakhna Chahiye?}                    & \textbf{Agar Na Ho?}                & \textbf{Agar Change Ho Toh?}                \\
\hline
\endfirsthead
\multicolumn{5}{c}{{\bfseries Summary Table: Metro aur Babel Files – Sab Kuch Ek Sath - continued from previous page}} \\
\hline
\textbf{File Name}         & \textbf{Kya Hai?}                       & \textbf{Kya Rakhna Chahiye?}                    & \textbf{Agar Na Ho?}                & \textbf{Agar Change Ho Toh?}                \\
\hline
\endhead
\hline
\multicolumn{5}{|r|}{{Continued on next page}} \\
\hline
\endfoot
\hline
\endlastfoot

metro.config.js   & Metro bundler configuration   & Default: kuch nahi, custom: code       & Metro default use karegi  & Metro restart (\texttt{start --reset-cache}) \\
\hline
babel.config.js   & Babel config                  & Preset: \texttt{'module:metro-react-native-babel-preset'} & Preset package.json me bhi ho sakta & Metro restart (\texttt{start --reset-cache}) \\
\hline
\end{longtable}
\end{center}
\vspace{1em}

****
\section*{Final Cheat Sheet (Saare Points: Metro \& Babel)}
- \imp{Metro} React Native ka bundler hai, code/assets ko device pe bundle karta hai.
- \imp{metro.config.js} mein changes tab karo jab custom bundling, assets, plugins, ya third-party integration chahiye.
- \imp{babel.config.js} mein changes tab karo jab advanced JS features, module aliases, ya experimental Babel plugins chahiye.
- \imp{Dono files ke changes ke baad Metro bundler restart zaruri hai.}\\
\begin{codeblock}
npx react-native start --reset-cache
\end{codeblock}
- \imp{Bina restart kiye naye changes apply nahi honge.}
- \imp{Agar dono files nahi bhi ho, toh React Native default config par chalta hai, lekin custom setting, hot reload, aliases, etc. ka scene hota hai, toh file banana padega.}
- \imp{Ye files project root (\texttt{package.json} ke paas) mein rahegi.}

****
\section*{Ek Dum Beginner-Friendly Example}
\imp{Scenario:}\\
Aapne babel.config.js mein module alias (\texttt{@components/Button}) add kiya, lekin app par error de raha hai.\\
\imp{Samajh:}\\
Aapne Metro bundler restart nahi kiya.\\
\imp{Solution:}\\
Metro band karo, aur ye command chalao:\\
\begin{codeblock}
npx react-native start --reset-cache
\end{codeblock}
Ab app reload karo, import kaam karega!\\
\imp{Yahi process sab file ke changes ke sath hai. Restart karo, changes reflect honge!}

****
\section*{Extra Advice (Aapke Notes Ke Liye)}
- \imp{Don’t touch metro.config.js/babel.config.js jab tak custom requirement na ho.}
- \imp{React Native project create karte hue ye files automatically ban jaati hain.}
- \imp{Third-party tools (jaise Sentry) ya advanced JS features chahiye ho, tab configure karo.}
- \imp{Har bar config change karne ke baad Metro restart zaruri hai.}
- \imp{Agar kuch samajh nahi aaye, toh Metro band karke naya bundler chalao.}

****
\section*{Conclusion}
\imp{Yeh dono config files React Native ke bundling aur JS transformations ko control karti hain.}\\
\imp{Aap beginner ho, toh default se chalao. Customize tab karo jab koi special requirement ho.}\\
\imp{Har bar config change ke baad bundler restart karo, warna changes reflect nahi honge!}\\
\imp{Ye sab padhne ke baad, aapko Metro aur Babel ki files ka full concept samajh a gaya hoga.}\\
\imp{Ab aap apni project ki structure, file aliases, experimental JS features, ya third-party tool integration kar sakte ho.}\\
\imp{Jab bhi doubt ho, just \texttt{npx react-native start --reset-cache} chala lo!}

****
\imp{Yehi sab cheezein sirf notes ke tarah likhni hai, toh aap yaani yehi sab samajh ke likh lo, aapko complete clarity ho jayegi React Native project config files ke bare mein!}\\
\imp{Aage agar aur topics ke notes chahiye, process chahiye, ya kisi file ka internal structure samajhna hai, toh bolo!}

=============================================================

\documentclass[a4paper]{article}
\usepackage[margin=25mm]{geometry}
\usepackage{xcolor}
\usepackage{tcolorbox}
\tcbuselibrary{listings, breakable}
\usepackage{listings}
\usepackage[parfill]{parskip}
\usepackage{enumitem}
\usepackage{titlesec}
\usepackage{fancyhdr}
\usepackage{caption}
\usepackage{mathptmx}
\usepackage{hyperref}
\usepackage{longtable}
\usepackage{array}

% Colors
\definecolor{HeadingBlue}{RGB}{0,119,187}
\definecolor{ImportantRed}{RGB}{227,27,35}
\definecolor{CodeBackground}{RGB}{247,247,247}
\definecolor{KeywordColor}{RGB}{85,55,123}
\definecolor{StringColor}{RGB}{180,130,0}
\definecolor{CommentColor}{RGB}{42,122,52}
\definecolor{FunctionColor}{RGB}{0,63,154}

% Heading style - left aligned, serif font, no centering
\titleformat{\section}
  {\LARGE\rmfamily\color{HeadingBlue}}
  {\thesection}
  {0em}
  {}
\titlespacing*{\section}{0pt}{2em}{1em}

% Important text (red)
\newcommand\imp[1]{{\color{ImportantRed}#1}}

% Listings styles
\lstdefinestyle{js}{
  language=javascript,
  basicstyle=\ttfamily\footnotesize,
  backgroundcolor=\color{CodeBackground},
  frame=tb,
  framerule=0pt,
  aboveskip=6pt,
  belowskip=6pt,
  showstringspaces=false,
  numbers=left,
  numberstyle=\tiny\color{gray},
  numbersep=10pt,
  numberblanklines=false,
  breaklines=true,
  prebreak={\mbox{\textcolor{gray}{$\hookrightarrow$}\space}},
  postbreak={\mbox{\space\textcolor{gray}{$\hookleftarrow$}}},
  keywordstyle=\color{KeywordColor},
  stringstyle=\color{StringColor},
  commentstyle=\color{CommentColor},
  identifierstyle=\color{FunctionColor},
}

\lstdefinestyle{bash}{
  language=sh,
  basicstyle=\ttfamily\footnotesize,
  backgroundcolor=\color{CodeBackground},
  frame=tb,
  framerule=0pt,
  showstringspaces=false,
  numbers=left,
  numberstyle=\tiny\color{gray},
  numbersep=10pt,
  keywordstyle=\color{KeywordColor},
  stringstyle=\color{StringColor},
  commentstyle=\color{CommentColor},
  identifierstyle=\color{FunctionColor},
}

% Code environments
\newtcblisting{codeblock}{
  listing only,
  listing options={style=js},
  colback=CodeBackground,
  arc=3pt,
  boxrule=0.4pt,
  colframe=gray,
  size=fbox,
  left=4pt,
  right=4pt,
  top=4pt,
  bottom=4pt,
  breakable,
}

\newtcblisting{bashblock}{
  listing only,
  listing options={style=bash},
  colback=CodeBackground,
  arc=3pt,
  boxrule=0.4pt,
  colframe=gray,
  size=fbox,
  left=4pt,
  right=4pt,
  top=4pt,
  bottom=4pt,
  breakable,
}

% Header/Footer
\pagestyle{fancy}
\fancyhf{}
\fancyhead[L]{\footnotesize React Native Notes}
\fancyhead[R]{\footnotesize \thepage}
\renewcommand{\headrulewidth}{0pt}

\begin{document}

=============================================================
\section*{npm Important Commands: Saral Detail Mein, Kab Kya Use Karna Hai, Har Cheez Clear}
***
\section*{npm list / npm ls}
- \imp{Kya hai?}  
  \texttt{npm list} ya \texttt{npm ls} command aapke current project (yani jis directory mein ho) ke saare installed packages aur unke dependencies ka \imp{tree structure} dikhata hai.[4][5]
- \imp{Kab use karein?}  
  Jab ye dekhna ho ki project mein konsa package, kaunse version mein install hai, aur wo kis package ke dependent hai.
- \imp{Basic Usage:}  
\begin{bashblock}
npm list
\end{bashblock}
Output mein saare packages dikhenge, tree structure mein (top-level + nested dependencies).[5]
- \imp{Tab use karo:}  
  Dependency issues, version conflicts, ya kisi package ki asli requirement samajhni ho.
- \imp{Agar dependencies zyada lambi tree dikhe toh:}  
\begin{bashblock}
npm list --depth=0
\end{bashblock}
Isse sirf top-level packages dikhenge, dependencies tree show nahi hoga.[4]
- \imp{Examples:}  
\begin{bashblock}
npm list
npm ls --depth=0
\end{bashblock}
- \imp{Ek specific package ki details:}  
\begin{bashblock}
npm list <pkg-name>
\end{bashblock}
Ye command batayega ki wo specific package install hai ya nahi, version kya hai, aur konsi file me konsi version available hai.[2]
***
\section*{npm list -g / npm ls -g}
- \imp{Kya hai?}  
  Ye command aapke globally (system-wide) install kare hue packages list karega, jo sirf is user ke liye use hote hain, kisi ek project ke liye nahi.[4]
- \imp{Kab use karein?}  
  Jab globally kon se packages install kare hain, ya kisi package ko globally uninstall karne se pehle dekhna ho.
- \imp{Basic Usage:}  
\begin{bashblock}
npm list -g
npm list -g --depth=0
\end{bashblock}
Isse sirf top-level globally installed packages dikhenge, dependencies tree nahi dikhega.[4]
- \imp{Example:}  
\begin{bashblock}
npm list -g
npm list -g --depth=0
\end{bashblock}
***
\section*{npm outdated}
- \imp{Kya hai?}  
  Ye command check karta hai ki aapke project mein koi package outdated (purana version) hai ya nahi, aur kaunse version available hain npm registry mein.[7]
- \imp{Kab use karein?}  
  Jab yaad aaye ki "kisi package ka update aata hai, kya update karna chahiye?" Tab use karo.
- \imp{Basic Usage:}  
\begin{bashblock}
npm outdated
\end{bashblock}
Output dikhayega ki konsi package ka current/target/latest version hai.
- \imp{Global packages mein bhi kar sakte ho:}  
\begin{bashblock}
npm outdated -g
\end{bashblock}
- \imp{Tab use karo:}  
  Project maintain karte hue, dependencies up-to-date rakhne ke liye, ya npm audit se pehle bhi.
***
\section*{npm audit}
- \imp{Kya hai?}  
  Ye aapke project ke dependencies ke security vulnerabilities check karta hai.[6]
- \imp{Kab use karein?}  
  Code ko ship karne se pehle, dependencies add karne ke baad, ya security ki chinta ho.
- \imp{Basic Usage:}  
\begin{bashblock}
npm audit
\end{bashblock}
Vulnerabilities (security threats) aur unki detail dikhayega.
- \imp{Agar fix karna ho toh:}  
\begin{bashblock}
npm audit fix
\end{bashblock}
Ye automatically safe version ko install karega, jitni vulnerabilities fix ho sakti hain.
- \imp{Tab use karo:}  
  Safety check, project onboarding, ya kisi bhi package install/update ke baad.
***
\section*{Related Commands (For Reference)}
\begin{center}
\small
\begin{tabular}{|p{3cm}|p{6cm}|p{5cm}|}
\hline
Command & Use Case & Example \\
\hline
npm init & Project banana, package.json banata hai & \texttt{npm init} \\
npm install & Package install karta hai & \texttt{npm install <package-name>} \\
npm uninstall & Package hata deta hai & \texttt{npm uninstall <package-name>} \\
npm update & Package update karta hai & \texttt{npm update <package-name>} \\
npm search & Package search karta hai & \texttt{npm search <keyword>} \\
npm doctor & Environment ki health check karta hai & \texttt{npm doctor} \\
\hline
\end{tabular}
\end{center}

***
\section*{Summary Table: npm List, Outdated, Audit – Sab Command, Sab Detail}
\small
\begin{center}
\begin{longtable}{|p{3.5cm}|p{5cm}|p{5cm}|p{5cm}|}
\hline
\textbf{Command} & \textbf{Kya Hai?} & \textbf{Kab Use Karein?} & \textbf{Example} \\
\hline
\endfirsthead
\multicolumn{4}{c}{{\bfseries Summary Table: npm List, Outdated, Audit – Continued}} \\
\hline
\textbf{Command} & \textbf{Kya Hai?} & \textbf{Kab Use Karein?} & \textbf{Example} \\
\hline
\endhead
\hline
\multicolumn{4}{|r|}{{Continued on next page}} \\
\hline
\endfoot
\hline
\endlastfoot
npm list / npm ls & Install packages dikhata hai & Package structure dekhna ho & \texttt{npm list} \\
npm list --depth=0 & Top-level packages only & Only main packages dekhna ho & \texttt{npm list --depth=0} \\
npm list -g & Globally install packages list & Global packages dekhna ho & \texttt{npm list -g} \\
npm list -g --depth=0 & Top-level global packages & Only main global packages dekhna ho & \texttt{npm list -g --depth=0} \\
npm list <package-name> & Specific package info & Kisi ek package ka detail chahiye & \texttt{npm list react} \\
npm outdated & Outdated packages list karta hai & Updates check karna & \texttt{npm outdated} \\
npm outdated -g & Outdated global packages & Global packages update check & \texttt{npm outdated -g} \\
npm audit & Security vulnerabilities check & Code ka security dekhna ho & \texttt{npm audit} \\
npm audit fix & Fix vulnerabilities & Vulnerabilities saaf karna ho & \texttt{npm audit fix} \\
\hline
\end{longtable}
\end{center}

***
\section*{Epic Notes (Aapke Liye Yad Rakhein)}
\begin{itemize}[itemsep=0.5em]
\item \imp{npm list} – Project/global packages (tree/detail).
\item \imp{npm outdated} – Kon sa package update kar sakte ho.
\item \imp{npm audit} – Security ka check, vulnerabilities.
\item \imp{npm audit fix} – Vulnerabilities fix karne ke liye.
\item \imp{Har baar dependency ya project ka structure, updates, ya security check karna hai, toh ye commands use karo.}
\item \imp{Globally use karne ke liye -g flag use karo.}
\item \imp{Ek package ka detail chahiye, toh package name specify karo.}
\item \imp{npm list output tree structure deta hai, --depth se tree ko shorten kar sakte ho.}
\end{itemize}

\vspace{0.5em}

***

\section*{Ek Dum Beginner Level – Final Cheat Sheet}
\begin{itemize}[itemsep=0.5em]
\item \imp{\texttt{npm list}}: Jo packages install hain, wo dekhne ke liye.
\item \imp{\texttt{npm list --depth=0}}: Sirf main packages dekhne ke liye.
\item \imp{\texttt{npm list -g}}: Global packages dekhne ke liye.
\item \imp{\texttt{npm list -g --depth=0}}: Sirf main global packages dekhne ke liye.
\item \imp{\texttt{npm list <pkg-name>}}: Ek package ka detail chahiye?
\item \imp{\texttt{npm outdated}}: Kaun sa package update karna chahiye?
\item \imp{\texttt{npm audit}}: Kaun sa package security risk hai?
\item \imp{\texttt{npm audit fix}}: Security risk fix karne ke liye.
\end{itemize}
\imp{Aaj se aapko koi bhi npm command ka doubt ho, seedha ye notes check karo, sab doubt clear ho jayega!}

\vspace{0.5em}
=============================================================

\documentclass[a4paper,12pt]{article}
\usepackage[left=25mm,right=25mm,top=25mm,bottom=25mm]{geometry}
\usepackage{setspace}
\usepackage{xcolor}
\usepackage{tcolorbox}
\tcbuselibrary{listings,breakable}
\usepackage{listings}
\usepackage{enumitem}
\usepackage{titlesec}
\usepackage{fancyhdr}
\usepackage{caption}
\usepackage{hyperref}
\usepackage{mathptmx}
\usepackage{array}
\usepackage{longtable}

% Define color palette
\definecolor{HeadingBlue}{RGB}{0,119,187}
\definecolor{ImportantRed}{RGB}{227,27,35}
\definecolor{CodeBackground}{RGB}{247,247,247}
\definecolor{KeywordColor}{RGB}{0,0,180}
\definecolor{StringColor}{RGB}{127,0,85}
\definecolor{CommentColor}{RGB}{63,127,95}
\definecolor{FunctionColor}{RGB}{0,0,139}

% Section heading style
\titleformat{\section}[block]
  {\normalfont\large\bfseries\color{HeadingBlue}\rmfamily}
  {}
  {0pt}
  {}

\titlespacing*{\section}{0pt}{3ex plus 1ex minus .2ex}{2ex plus .2ex}

% Paragraph spacing and indentation
\setlength{\parindent}{0pt}
\setlength{\parskip}{0.9em plus0.2em minus0.2em}

% List style with enumitem
\setlist[itemize]{noitemsep, topsep=0.5em, parsep=0pt, partopsep=0pt}

% Red color command for important text
\newcommand{\imp}[1]{\textcolor{ImportantRed}{#1}}

% Listings style for JSX
\lstdefinelanguage{jsx}{
  keywords={import,from,export,default,function,return,const,let,var,if,else,new,this,super,as,class,extends,constructor,static,async,await,try,catch,finally,throw,yield,break,continue,for,while,do,switch,case,default,typeof,instanceof,in},
  keywordstyle=\color{KeywordColor}\bfseries,
  ndkeywords={use,package,json,registry,install,require,module},
  ndkeywordstyle=\color{KeywordColor}\bfseries,
  sensitive=true,
  comment=[l]{//},
  morecomment=[s]{/*}{*/},
  commentstyle=\color{CommentColor}\itshape,
  stringstyle=\color{StringColor},
  morestring=[b]',
  morestring=[b]",
  morestring=[b]`,
  literate=%
   *{~}{{\texttildelow}}1
}

% Listings style for bash
\lstdefinelanguage{bash}{
  keywords={npm,cd,ls,run,install,start,update,audit,doctor},
  keywordstyle=\color{KeywordColor}\bfseries,
  sensitive=true,
  comment=[l]{//},
  morecomment=[s]{/*}{*/},
  commentstyle=\color{CommentColor}\itshape,
  stringstyle=\color{StringColor},
  basicstyle=\ttfamily\footnotesize,
  numberstyle=\tiny\color{gray},
  numbers=left,
  numbersep=5pt,
  xleftmargin=15pt,
  tabsize=2,
  breaklines=true,
  showstringspaces=false,
}

% Generic listings style
\lstset{
  basicstyle=\ttfamily\footnotesize,
  backgroundcolor=\color{CodeBackground},
  frame=single,
  framesep=5pt,
  rulecolor=\color{black!20},
  breaklines=true,
  numbers=left,
  numberstyle=\tiny\color{gray},
  numbersep=5pt,
  xleftmargin=15pt,
  tabsize=2,
  captionpos=b,
  keywordstyle=\color{KeywordColor},
  commentstyle=\color{CommentColor},
  stringstyle=\color{StringColor},
  showspaces=false,
  showstringspaces=false,
  showtabs=false,
  escapeinside={(*@}{@*)},
  belowskip=10pt,
  aboveskip=10pt,
}

% Code block environments
\newtcolorbox[auto counter, number within=section]{codeblock}[2][]{%
  enhanced,
  breakable,
  colback=CodeBackground,
  colframe=black!20,
  fonttitle=\bfseries,
  title=Code Block~\thetcbcounter,
  attach boxed title to top left={yshift=-2mm,xshift=5mm},
  boxed title style={colback=HeadingBlue!80, colframe=HeadingBlue!80},
  listing only,
  listing options={style=jsx, numbers=left, #1},
  left=6pt,
  right=6pt,
  top=6pt,
  bottom=6pt,
  boxrule=0.5pt,
  arc=4pt,
  #2
}

\newtcolorbox[auto counter, number within=section]{bashblock}[2][]{%
  enhanced,
  breakable,
  colback=CodeBackground,
  colframe=black!20,
  fonttitle=\bfseries,
  title=Bash Code~\thetcbcounter,
  attach boxed title to top left={yshift=-2mm,xshift=5mm},
  boxed title style={colback=HeadingBlue!80, colframe=HeadingBlue!80},
  listing only,
  listing options={style=bash, numbers=left, #1},
  left=6pt,
  right=6pt,
  top=6pt,
  bottom=6pt,
  boxrule=0.5pt,
  arc=4pt,
  #2
}

% Header Footer
\usepackage{fancyhdr}
\pagestyle{fancy}
\fancyhf{}
\fancyhead[L]{\small React Native Notes}
\fancyhead[R]{\small \thepage}
\renewcommand{\headrulewidth}{0.4pt}

\begin{document}

\texttt{==============================================================}

\vspace{1em}

{\color{HeadingBlue}\noindent\textbf{\Large SafeAreaView – Ek Dum Basic, Practical, Sab Doubt Clear}}

\vspace{1em}

\noindent
\textbf{SafeAreaView Kya Hai?}

\noindent
\imp{SafeAreaView} ek special wrapper component hai React Native me, jo aapke content ko \textit{status bar}, \textit{notch}, ya \textit{device ke rounded corners} se \imp{bacha ke/yaani apne aap padding laga kar} dikhata hai ki content \imp{visible} ya \imp{cut na ho} kisi bhi device par.[11][12]

\noindent
\textbf{Regular View} use karne par, aapke button, text, ya koi bhi UI element \imp{status bar ke peeche chhip sakta hai} ya \imp{notch screen pe cut ho sakta hai}—SafeAreaView use karne se ye dikkat nahi hoti.

\vspace{1em}

\noindent
\textbf{Kaisa Dikhta Hai? (Visual Example)}

\vspace{0.5em}

\noindent
\textbf{Picture ke taur pe samjho:}  

Mobile phones me upar ek thin white/grey bar hota hai—status bar (time, network, battery, etc.). Jab aap normal \textbf{View} use karte ho, aapka content is status bar ke \textbf{neeche se shuru nhi hota}—status bar ke niche se shuru hota hai (status bar ke peeche content aata hai), lekin \textbf{SafeAreaView} aapke content ko status bar \& notch ke \textbf{neeche se shuru kar deta hai}—content overlap/cut nahi hota.

\vspace{0.5em}

\begin{quote}
\textbf{No SafeAreaView:} \\
Status bar ke peeche se content shuru, notch/camera/sensor area me UI cut ho sakta hai.  

\textbf{With SafeAreaView:} \\
Content status bar \& notch ke \textbf{neeche se shuru hota hai}—UI visible, no overlap/cut.
\end{quote}

\vspace{1em}

\noindent
\textbf{Kab Use Karein?}

\vspace{0.5em}

\begin{itemize}
\item \imp{Jab aapka screen ke top/bottom pe UI (\textbf{jaise header, buttons, ya footer}) notch/status bar ke karan \imp{cut ho raha ho} ya \imp{chhip raha ho}.}
\item \imp{Sabse zyada zarurat hai iPhone X, iPhone 11, aur sabse naye iPhones ke liye (jisme notch, sensor area, ya rounded corners hota hai).}
\item \imp{Android pe agar aapko notch, punch-hole kam karne hai, toh aapko react-native-safe-area-context library use karni hogi.}
\end{itemize}

\vspace{1em}

\noindent
\textbf{Kyun Use Karein?}

\vspace{0.5em}

\begin{itemize}
\item \imp{UI ko cut hone, overlap hone ya invisible hone se bachata hai.}
\item \imp{Sab devices pe UI consistently visible rahega, notch/status bar se chupke na rahe.}
\item \imp{Manually padding calculate/lagane ki zarurat nahi hai, SafeAreaView khud se kar deta hai.}
\end{itemize}

\vspace{1em}

\noindent
\textbf{Basic Code Example (Simple Usage – iOS only)}

\begin{codeblock}
import { SafeAreaView, StyleSheet, Text } from 'react-native';
function App() {
  return (
    <SafeAreaView style={styles.container}>
      <Text>React Native SafeAreaView Example</Text>
    </SafeAreaView>
  );
}
const styles = StyleSheet.create({
  container: {
    flex: 1,
    alignItems: 'center',
    justifyContent: 'center',
  }
});
\end{codeblock}

\vspace{1em}

\noindent
\textbf{Cross-Platform (iOS + Android) – react-native-safe-area-context}

\vspace{0.5em}

\noindent
\imp{React Native ka default SafeAreaView sirf iOS pe kaam karta hai.}  

Android notches/punch-holes ke liye aapko ye library use karni hogi:

\begin{bashblock}
npm install react-native-safe-area-context
\end{bashblock}

Wrap app with provider:

\begin{codeblock}
import { SafeAreaProvider } from 'react-native-safe-area-context';
// ...
return (
  <SafeAreaProvider>
    <App />
  </SafeAreaProvider>
);
\end{codeblock}

Use SafeAreaView on every screen:

\begin{codeblock}
import { SafeAreaView as SafeAreaPlatform } from 'react-native-safe-area-context';
// ...
<SafeAreaPlatform style={{ flex: 1 }}>
  <Text>Cross-platform Safe Area!</Text>
</SafeAreaPlatform>
\end{codeblock}

Yeh approach iOS \& Android dono safe area (notch, punch-hole, status bar) handle karega.

\vspace{1em}

\noindent
\textbf{Agar SafeAreaView Na Use Karein Toh Kya Hota Hai?}

\vspace{0.5em}

\begin{itemize}
\item \imp{iOS me:} status bar ke peeche UI chhip jaati hai, notch area me UI overlap/cut ho sakta hai.
\item \imp{Android me:} by default, status bar ke neeche content shuru ho jata hai lekin notch/punch-hole phones pe UI cut ho sakta hai.
\item \imp{Manually padding laga kar bach sakte ho, lekin wo device pe device change hota hai—SafeAreaView sab automatic handle karta hai.}
\end{itemize}

\vspace{1em}

\noindent
\textbf{Real-Life Example – Kya Difference Dikhega?}

\vspace{0.5em}

\noindent
\textbf{Without SafeAreaView:}  

(iOS pe)  
App header ya koi bhi UI notch, status bar ya sensor area ke peeche chhupa rahega — user dikh nahi paayega.

\vspace{0.5em}

\noindent
\textbf{With SafeAreaView:}  

UI visible hoga aur sahi jagah dikhai dega, user ke liye accessible rahega.

\vspace{1em}

\noindent
\textbf{Summary Table}

\begin{center}
\begin{tabular}{|p{3.5cm}|p{3.5cm}|p{3.5cm}|p{3.5cm}|}
\hline
Type & iOS (Notch/Status Bar) & Android (Notch/Punch-hole) & Cross-Platform Solution \\
\hline
Default SafeAreaView & Works, content safe & Not working & Use react-native-safe-area-context \\
Without SafeAreaView & Content cut/overlap & Depends on device & Maybe content cut/pad manually \\
\hline
\end{tabular}
\end{center}

\vspace{1em}

\noindent
\textbf{Aapke Notes Ke Liye Chahiye – Ek Dum Basic}

\begin{itemize}
\item \imp{SafeAreaView ek wrapper hai, jo status bar/notch ke karan cut hone wale content ko bachata hai.}
\item \imp{iOS pe React Native ke built-in SafeAreaView se acha result milta hai, Android pe react-native-safe-area-context use karo.}
\item \imp{Kab use karein?} Jab aapko iOS ya Android notch/punch-hole/status bar ke karan UI cut/chhip raha ho.
\item \imp{Kyun use karein?} UI sahi dikhe, user experience better ho, manual padding ki zarurat na pade.
\item \imp{Code example dekh lo, samajh aa jayega.}
\end{itemize}

\vspace{1em}

\noindent
\textbf{Koi bhi issue/query ho, aap bas ek bar apna UI dekho—agar notch/status bar pe UI chhip raha hai ya cut ho raha hai, toh SafeAreaView ya react-native-safe-area-context use karo—sab sahi ho jayega. SafeAreaView ka use karke aapka app sab devices pe professional aur user-friendly dikhega!}

\texttt{==============================================================}

\vspace{2em}

{\color{HeadingBlue}\noindent\textbf{\Large Picker (@react-native-picker/picker \& Alternatives): Pura Topic Clear Hinglish Mein}}

\vspace{1em}

\noindent
\textbf{Picker Kya Hai?}

\noindent
\imp{Picker} ek UI component hai jo \textbf{dropdown} ki tarah kaam karta hai—user ko ek list se kuch select karne ki facility deta hai.[1][2]

\noindent
Web pe \texttt{select} tag jaisa, mobile app me popup/list dikhata hai jahan user ek option chunta hai.

\vspace{1em}

\noindent
\textbf{Kisliye Use Hota Hai?}

\noindent
Jab aapko user se \imp{kuch bhi select karwana ho}—jaise country, state, city, language, gender, ya koi bhi category.

Bina picker ke aapko button, modal, ya custom UI banana padta hai, lekin picker inbuilt, native, aur easy solution hai.

\vspace{1em}

\noindent
\textbf{React Native Picker Types:}

\begin{itemize}
\item \imp{@react-native-picker/picker}: Official replacement, iOS, Android, Windows, macOS pe kaam karta hai.[3]
\item \imp{react-native-picker-select}: Ek cross-platform picker jo native feel deta hai.[4]
\item \imp{react-native-dropdown-picker}: Flat list dropdown, customizations & animations.[5]
\item \imp{react-native-picker-modal-view}: Modal picker, bada list, search, indexing.[6]
\end{itemize}

\noindent
\textbf{Note:}

React Native ka built-in \texttt{<Picker>} deprecated ho gaya hai, ab \imp{@react-native-picker/picker} use karo.[1]

\vspace{1em}

\noindent
\textbf{Kya Problem Solve Karta Hai?}

\begin{itemize}
\item Basic selection ke liye bina extra code ke easy picker.
\item Form support jaise gender, language, country select karwana.
\item Native platform specific look and feel.
\item Dynamic list enable karta hai.
\item Easy integration.
\end{itemize}

\vspace{1em}

\noindent
\textbf{Platform-wise Difference (iOS vs Android):}

\begin{itemize}
\item iOS: Wheel style picker.
\item Android: Dropdown/spinner with dialog popup.
\item Modal based pickers (react-native-picker-select) dono platform pe same UI dete hain.
\item Native picker styling tough hoti hai, isliye custom dropdown jaise react-native-dropdown-picker popular hain.[6][5]
\end{itemize}

\vspace{1em}

\noindent
\textbf{Code Example (Basic Picker):}

\vspace{0.5em}

\begin{bashblock}
npm install @react-native-picker/picker
\end{bashblock}

\vspace{0.5em}

\begin{codeblock}
import React, { useState } from 'react';
import { View, Text, StyleSheet } from 'react-native';
import { Picker } from '@react-native-picker/picker';

function App() {
  const [selectedLanguage, setSelectedLanguage] = useState('JavaScript');

  return (
    <View style={styles.container}>
      <Picker
        selectedValue={selectedLanguage}
        style={{ height: 50, width: 150 }}
        onValueChange={(itemValue) => setSelectedLanguage(itemValue)}
      >
        <Picker.Item label="JavaScript" value="JavaScript" />
        <Picker.Item label="React Native" value="React" />
        <Picker.Item label="Java" value="Java" />
        <Picker.Item label="Python" value="Python" />
      </Picker>
      <Text>Selected: {selectedLanguage}</Text>
    </View>
  );
}

const styles = StyleSheet.create({
  container: {
    flex: 1,
    justifyContent: 'center',
    alignItems: 'center',
  },
});

export default App;
\end{codeblock}

\vspace{1em}

\noindent
\textbf{Kuch Common Props \& Features:}

\begin{center}
\begin{tabular}{|p{4cm}|p{8cm}|}
\hline
Prop/Feature & Kya Karta Hai \\
\hline
selectedValue & Selected option set karta hai \\
onValueChange & Option change par callback run karta hai \\
style & Picker styling karne ke liye \\
enabled & Enable/Disable karta hai \\
mode & Android par dialog ya dropdown specify karta hai \\
prompt & Android dialog me title set karta hai \\
\hline
\end{tabular}
\end{center}

\vspace{1em}

\noindent
\textbf{External Alternatives (Advanced/Custom UI ke liye):}

\begin{itemize}
\item react-native-picker-select: Modal based, same UI ios/android dono pe.
\item react-native-dropdown-picker: Dropdown list with animations & customizations.
\item react-native-picker-modal-view: Large lists ke liye, search, indexing ke saath.
\end{itemize}

\vspace{1em}

\noindent
\textbf{Kab Use Karein?}

\begin{itemize}
\item Simple list selection ke liye: @react-native-picker/picker.
\item Same UI modal style ke liye: react-native-picker-select.
\item Custom dropdown/animations chahiye to: react-native-dropdown-picker.
\item Bade lists, search, indexing ke liye: react-native-picker-modal-view.
\end{itemize}

\vspace{1em}

\noindent
\textbf{Saare Doubt Clear – Cheat Sheet:}

\begin{itemize}
\item Picker ek dropdown/list select component hai.
\item Purana \texttt{<Picker>} deprecated hai, ab community package use karo.
\item Platform aur use case ke mutabiq alternatives use karo.
\item Install, import kariye, code mein use kariye.
\end{itemize}

\vspace{1em}

\noindent
\textbf{Aapke Notes Ke Liye:}

\begin{itemize}
\item Picker dropdown selection ki tarah kaam karta hai.
\item Use forms me option select karwane ke liye karo.
\item Built-in React Native picker ab nai hai, community packages best hain.
\item Agar same UI chahiye multiple platforms pe, external libraries use karo.
\end{itemize}

\noindent
\textbf{Aaj se picker ke baare mein koi bhi doubt nahi rahega!}

\texttt{==============================================================}

\documentclass[a4paper,12pt]{article}

% Geometry with comfortable margins
\usepackage[left=25mm,right=25mm,top=25mm,bottom=25mm]{geometry}

% Fonts
\usepackage{tgtermes} % TeX Gyre Termes (Times-like serif for body)
\usepackage[T1]{fontenc}
\usepackage[utf8]{inputenc}
\usepackage{sectsty} % for section title font control

% Color definitions
\usepackage{xcolor}
\definecolor{headingblue}{RGB}{0,70,132} % strong blue for headings
\definecolor{importantred}{RGB}{204,0,0} % bright red for important text
\definecolor{codebg}{RGB}{245,245,245} % very light gray background for code
\definecolor{keywordcolor}{RGB}{0,0,180} % dark blue for keywords
\definecolor{stringcolor}{RGB}{163,21,21} % dark red for strings
\definecolor{commentcolor}{RGB}{0,128,0} % green for comments
\definecolor{functioncolor}{RGB}{127,0,85} % purple for function names
\definecolor{linenumbercolor}{gray}{0.6} % subtle gray for line numbers

% Listings for code blocks with multi-color syntax highlighting
\usepackage{listings}
\lstdefinelanguage{jsx}{
  keywords={import,function,let,const,var,if,else,return,for,while,do,switch,case,break,continue,class,new,try,catch,finally,throw,async,await,void,static,public,private,protected,yield,default},
  sensitive=true,
  morecomment=[l]{//},
  morecomment=[s]{/*}{*/},
  morestring=[b]",
  morestring=[b]',
}
\lstset{
  language=jsx,
  backgroundcolor=\color{codebg},
  keywordstyle=\color{keywordcolor}\bfseries,
  stringstyle=\color{stringcolor},
  commentstyle=\color{commentcolor}\itshape,
  identifierstyle=\color{black},
  basicstyle=\ttfamily\footnotesize,
  numbers=left,
  numberstyle=\color{linenumbercolor}\scriptsize,
  stepnumber=1,
  numbersep=8pt,
  frame=none,
  tabsize=4,
  showstringspaces=false,
  breaklines=true,
  breakatwhitespace=true,
  captionpos=b,
  xleftmargin=10pt,
  xrightmargin=10pt,
  framexleftmargin=5pt,
  framexrightmargin=5pt,
  rulecolor=\color{codebg},
  upquote=true,
  literate={~}{{\textasciitilde}}1
}

% tcolorbox for code blocks with rounded corners and light background
\usepackage[many]{tcolorbox}
\tcbset{
  colback=codebg,
  colframe=codebg,
  boxrule=0pt,
  arc=3mm,
  auto outer arc,
  leftrule=0pt,
  rightrule=0pt,
  toprule=0pt,
  bottomrule=0pt,
  left=6pt,
  right=6pt,
  top=6pt,
  bottom=6pt,
  enhanced,
}

% Section title formatting with colors and size
\usepackage{titlesec}
\titleformat{\section}
  {\color{headingblue}\LARGE\bfseries}
  {}
  {0pt}
  {}
\titlespacing*{\section}{0pt}{4ex plus 1ex minus .2ex}{1.5ex plus .2ex}

% For horizontal separator lines text exactly as given
\usepackage{titling}

% Header and footer with fancyhdr
\usepackage{fancyhdr}
\pagestyle{fancy}
\fancyhf{}
\fancyhead[L]{React Native Notes}
\fancyhead[R]{\thepage}
\renewcommand{\headrulewidth}{0.4pt}
\renewcommand{\footrulewidth}{0pt}

% Lists with enumitem for spacing control
\usepackage{enumitem}
\setlist{itemsep=0.9em, parsep=0pt, topsep=0.9em}

% Hyperlinks without colors
\usepackage[hidelinks]{hyperref}

% Custom command for important red text
\newcommand{\important}[1]{\textcolor{importantred}{#1}}

% Preserve monospace inline code styling
\newcommand{\inlinecode}[1]{\texttt{#1}}

% To preserve the exact separator line visually
\newcommand{\separatorline}{\texttt{=============================================================}}

\begin{document}

\section*{\color{headingblue}\LARGE\bfseries Toggle Element Inspector (Dev Menu): Sab Kuch Clear Hinglish Mein}
\noindent \texttt{text===================================================================}

\vspace{1.5em}

\section*{\color{headingblue}\LARGE\bfseries Toggle Element Inspector Kya Hai?}
\noindent \important{Toggle Element Inspector} React Native ka ek \important{in-built debugging tool} hai, jo \important{UI elements ko real-time mobile screen par inspect} karne mein help karta hai.[1][2]

\important{Element inspector ko toggle karte hai}, matlab \important{chalu kar sakte ho}, \important{band kar sakte ho}—jaise aap browser (Chrome, Firefox) mein F12 ya Inspect kar sakte ho, waise hi React Native mein bhi aap apne app ke UI components (buttons, text, lists, images, etc.) ko point and click karke dekh sakte ho ki kya UI kahan hai, uska size kya hai, kis position par hai, aur uska source code/tag kya hai.

\vspace{1.5em}
\section*{\color{headingblue}\LARGE\bfseries Dev Menu Kya Hai?}
\important{Dev Menu} ek popup menu hai jo React Native development ke waqt \important{device kaafi bhaari hila ke (shake karke)} ya \important{keyboard shortcut} se open hota hai.[1]

\important{Is menu mein debug karne ke liye kuch basic shortcuts hote hain—jaise:}  

\begin{itemize}
\item \important{Reload} (App ko dobara load karega)  
\item \important{Show Element Inspector} (UI elements inspect karega)  
\item \important{Open JS Debugger} (JavaScript debug karega)  
\item \important{Toggle Performance Monitor} (App ki performance dikhata hai)  
\item \important{Record Screenshot} (Screen capture)  
\item \important{Toggle Inspector} (UI elements identify karega)  
\end{itemize}

\important{In sabhi mein "Toggle Element Inspector" ya "Show Element Inspector" option hai.}

\vspace{1.5em}
\section*{\color{headingblue}\LARGE\bfseries Kisliye Use Karte Hain?}
\begin{itemize}
\item \important{UI Debugging:} Jab aapko apna app ka UI check karna ho, kisi button, text, ya list ki position/size/code dekhna ho, ya koi UI element clickable hai ya nahi—ye sab inspect kiya ja sakta hai.
\item \important{Layout Issues:} Jab aapki screen pe koi element kisi jagah par nahi dikh raha, ya UI overlapping ho rahi hai, toh inspect karke usko trace kar sakte ho.
\item \important{Developer Feedback:} Aap apne project team ko screen par click karke dikha sakte ho ki konsa element kya hai, wo kahan hai, kaise banaya hai, uska code kahan hai.
\item \important{Performance Debugging:} Performance monitor ke sath mila ke app ke memory, FPS (frames per second), aur UI rendering stats bhi dekh sakte ho.
\end{itemize}

\important{Browser ki DevTools jaisa hi hai, lekin directly mobile screen par kaam karta hai.}

\vspace{1.5em}
\section*{\color{headingblue}\LARGE\bfseries Kaise Use Karte Hain?}

\subsection*{\color{headingblue}\Large\bfseries Dev Menu Open Karne Ka Tarika}
\begin{itemize}
\item \important{Android emulator par:} \inlinecode{Ctrl + M} (Windows/Linux), \inlinecode{Cmd + M} (Mac)
\item \important{iOS simulator par:} \inlinecode{Cmd + D} ya \inlinecode{Ctrl + Cmd + Z}, ya device shake karo
\item \important{Expo Apps:} Terminal mein \inlinecode{m} daba do (app run karne wala terminal)
\item \important{Physical device (Jaladi, Real Phone):} Device ko shake karo
\end{itemize}

\important{Menu khul jayega. Usme "Show Element Inspector" ya "Toggle Inspector" select karo.}

\subsection*{\color{headingblue}\Large\bfseries Element Inspector Chalu Par Kaise Use Kare?}
\begin{itemize}
\item \important{Tap karo:} Screen par kisi bhi UI element par tap karo—wo highlight ho jayega, aur uski position, size display ho jayegi (jaise border dikhega, text/code highlight hoga).
\item \important{Multi-touch:} Zaada details ke liye, do finger/tap kar ke zoom in/out, pan kar sakte ho.
\item \important{Inspector band karne ke liye:} Waapas dev menu open karo, waapas "Hide/Toggle Element Inspector" select karo.
\end{itemize}

\vspace{1.5em}
\section*{\color{headingblue}\LARGE\bfseries Kya Problem Solve Karta Hai?}
\begin{itemize}
\item \important{UI/GUI Debugging:} Koi button, text, ya view kahan hai, kaise dikh raha hai—bina console log ke directly screen par dikh jayega.
\item \important{Layout/Position Issues:} Agar ek view nahi dikh raha, ya margin/padding galat hai, toh inspector se dekh kar thik kar sakte ho.
\item \important{Performance Issues:} Performance monitor bhi chalu kar sakte ho, FPS, memory, etc. monitor kar sakte ho.
\item \important{Fast Debugging:} Error debugging kaise hogi? Jab koi specific popup, alert, ya dialog screen par nahi dikh raha, inspector use karne se screen par tap karke pata chal jayega ki element hai bhi ya nahi.
\end{itemize}

\vspace{1.5em}
\section*{\color{headingblue}\LARGE\bfseries Kab Use Karte Hain?}
\begin{itemize}
\item \important{UI Development:} Jab naya screen banana ho, ya koi view/banner/footer/modal add karna ho, toh element inspector ka use karo.
\item \important{Bug Fixing:} Koi UI element click nahi ho raha, ya position galat hai, toh inspector chala ke dekho ki kya chal raha hai.
\item \important{Review/Testing:} Jab apne app ko kisi aur ko review karana ho, ya UI behavior ka test karna ho, toh inspector chalaya ja sakta hai.
\item \important{Performance Tuning:} Jab app slow ho, ya UI stutter ho, toh performance monitor and inspector sath mein use kar sakte ho.
\end{itemize}

\vspace{1.5em}
\section*{\color{headingblue}\LARGE\bfseries Real-Life Example (Koi UI Element Ka Position/Galati Check Karna)}
\textbf{Scenario:}  
Aapki screen par ek important button dikh nahi raha.  

\textbf{Kya Karein?}
\begin{enumerate}
\item \important{Dev menu open karo} (device shake karo ya shortcut dabaao).
\item \important{Toggle Element Inspector chalu karo}.
\item \important{App screen par tap-tap karo}—button highlight ho jayega, to dekho ki element ka tag kya hai, position kya hai.
\item \important{Agar border dikh raha hai par button visible nahi hai, toh uski styling/position/wrapping check karoge, direct code mein dekho ki kahan mistake hue hai.}
\item \important{Thik kar ke wapas app reload karo.}
\end{enumerate}

\vspace{1.5em}
\section*{\color{headingblue}\LARGE\bfseries Saare Doubts Clear – Cheat Sheet}
\begin{tabular}{|p{5cm}|p{5cm}|p{4cm}|p{3.5cm}|}
\hline
\textbf{Command/Option} & \textbf{Kya Karta Hai?} & \textbf{Kab Use Karein?} & \textbf{Shortcut (Emulator)} \\
\hline
Show/Toggle Element Inspector & UI elements ko inspect karta hai & UI, layout, positioning check karene ho & Dev Menu > Show Inspector \\
\hline
Performance Monitor & App ki performance dikhata hai & Slow/laggy UI, FPS, memory check karene ho & Dev Menu > Toggle Perf Monitor \\
\hline
Open JS Debugger & JavaScript debugger kholega & Console, error, log check karene ho & Dev Menu > Open JS Debugger \\
\hline
Reload & App dobara load karega & Code change karene ho, refresh karna ho & Dev Menu > Reload \\
\hline
\end{tabular}

\vspace{1.5em}
\section*{\color{headingblue}\LARGE\bfseries Conclusion \& Sab Kuch Notes Ke Liye}
\begin{itemize}
\item \important{Toggle Element Inspector} ek debugging tool hai jo UI elements ko mobile screen par tap-tap karke inspect karta hai.
\item \important{Dev Menu} se open hota hai—device shake karke, ya keyboard shortcut se.
\item \important{Kisliye use karen?} UI debug karne, layout issues find karne, element position/size cross-check karne.
\item \important{Kaise use karen?} Menu open karo, inspector chalu karo, screen par tap karo, code/UI/position check karo.
\item \important{Kis problem mein help karta hai?} UI chhupne, click na hone, layout issue, element visibility, etc.
\item \important{Ye hamesha development environment (dev builds) mein hi use ho sakta hai, production build (release) mein nahi chalta.}
\end{itemize}
\important{Aap har baar UI development karte waqt, aur kisi UI bug fix karte waqt, dev menu se inspector chala kar directly dekho ki aapka element kya kar raha hai—yani real-time, screen-inspector use kar sakte ho, jisse debugging fast aur easy hai!}

\important{Aaj se aapka element inspector ka bhi koi doubt nahi rahega—Chhupne, gaya, click nahi ho raha, ya position galat hai—bas dev menu chalu karo, inspector chalu karo, tap-tap kar ke sab samajh jao!}

\vspace{2em}

\separatorline

\vspace{2em}

\section*{\color{headingblue}\LARGE\bfseries navigation.goBack() – Sab Kuch Detalil Mein, Saare Doubt Clear}
\noindent \texttt{text===================================================================}

\vspace{1.5em}

\section*{\color{headingblue}\LARGE\bfseries navigation.goBack() Kya Hai?}
\important{navigation.goBack()} ek \important{programmatic navigation method} hai, jo \important{React Navigation} (jaise createStackNavigator, createBottomTabNavigator, vagera) ke navigation object me milta hai.[1][3]

\important{Iska use aap ek screen se wapas pichhle screen par jaane ke liye karte ho, jaise mobile apps me back button kaam karta hai.}

\vspace{1.5em}
\section*{\color{headingblue}\LARGE\bfseries Kyun Use Karte Hain?}
\begin{itemize}
\item \important{Back kaam karna:} Jab aap apne app me screen par screen navigate karte ho, toh back button ya back gesture chahiye hota hai—\important{programmatically} bhi back karvana ho, toh \inlinecode{navigation.goBack()} use karo.
\item \important{Keyboard ke alawa:} Jab user custom button click kare, ya aap chahate ho ki kisi specific event pe user back ho jaye, toh use karo.
\item \important{Android hardware back button bhi isi ko call karta hai} — matlab, app ka internal back aur hardware back button dono ek hi kaam karte hain.[5][1]
\item \important{Modal, popup, ya temporary screen close karna:} Kisi extra popup/dialog/action sheet screen close karna ho, toh bhi use karte hain.
\end{itemize}

\vspace{1.5em}
\section*{\color{headingblue}\LARGE\bfseries Kaise Use Karte Hain?}
Aapko \important{navigation object} milta hai screen component ko prop ya hook ke through.  

\textbf{Example:}

\begin{tcolorbox}[sharp corners,boxrule=0.5pt,colback=codebg]
\begin{lstlisting}[language=jsx, numbers=left]
import { Button, View, Text } from 'react-native';
import { useNavigation } from '@react-navigation/native';
function DetailsScreen() {
  const navigation = useNavigation();
  return (
    <View style={{ flex: 1, alignItems: 'center', justifyContent: 'center' }}>
      <Text>Details Screen</Text>
      <Button
        title="Go Back"
        onPress={() => navigation.goBack()}
      />
    </View>
  );
}
\end{lstlisting}
\end{tcolorbox}

Yaha \important{Go Back} button par click karte hi user pichhle screen par chala jayega.[3][6]

\vspace{1.5em}
\section*{\color{headingblue}\LARGE\bfseries Kab Use Karna Hai?}
\begin{itemize}
\item \important{Jab aap custom back button bana rahe ho} — header me, ya screen par, ya footer me.
\item \important{Jab aap kisi popup/modal screen ko programmatically band karna chahate ho.}
\item \important{Jab aap custom logic me user ko pichhle screen par bhejna chahte ho} — jaise kuch validation ho gaya, ya data save ho gaya, etc.
\item \important{Jab aap Android ke hardware back button ka custom behavior chahiye ho} — toh aap \important{BackHandler} API ke sath iss function ko call kar sakte ho.[9]
\end{itemize}

\vspace{1.5em}
\section*{\color{headingblue}\LARGE\bfseries Kya Problem Solve Karta Hai?}
\begin{itemize}
\item \important{UI consistency:} App me sab jagah back ka behavior ek jaisa rahe — header ya footer me custom button ho, ya device ka hardware back button, sab ek hi screen par leke jaye.
\item \important{User experience:} User apne mobile phone ke \important{default back button} jaisa experience hi app me chahiye, usko rokenge nahi.
\item \important{Flexibility:} Aap custom button ko bhi ek screen se dusre screen par transition me use kar sakte ho, bina kisi problem ke.
\end{itemize}

\vspace{1.5em}
\section*{\color{headingblue}\LARGE\bfseries Advanced \& Common Doubts}
\subsection*{\color{headingblue}\Large\bfseries GoBack() Kab Kaam Nahin Karega?}
\begin{itemize}
\item \important{Agar navigation stack me sirf ek hi screen hai} (matlab aap root/home pe ho) toh \inlinecode{navigation.goBack()} se kuch nahi hoga, kyonki back karne ko kuch bacha hi nahi.[1]
\item \important{Agar aap Tab Navigator ke screen par ho, aur Tab ke under Stack Navigator nahi hai}, toh goBack se kuch nahi hoga.
\item \important{Agar aap navigation stack me nested navigators ka complex setup ho}, toh \inlinecode{navigation.canGoBack()} check karo, aur fir goBack chalao.[3]
\end{itemize}

\subsection*{\color{headingblue}\Large\bfseries Ek Hi Bar Me Multiple Screen Back Karna Ho Toh?}
\begin{itemize}
\item \inlinecode{navigation.popTo(‘RouteName’)}: Kisi specific screen par jaane ke liye.
\item \inlinecode{navigation.popToTop()} : Stack ke pehle screen par return karne ke liye (jaise home).[1]
\item \inlinecode{navigation.navigate(‘Home’)} : Seedhe home screen par navigate karna, lekin stack history clear nahi hoti.
\end{itemize}

\vspace{1.5em}
\section*{\color{headingblue}\LARGE\bfseries Code Example (Custom Logic \& Multiple Backs)}
\begin{tcolorbox}[sharp corners,boxrule=0.5pt,colback=codebg]
\begin{lstlisting}[language=jsx, numbers=left]
import { useNavigation } from '@react-navigation/native';
function ProfileScreen({ route }) {
  const navigation = useNavigation();
  return (
    <View style={{ flex: 1, alignItems: 'center', justifyContent: 'center' }}>
      <Text>Profile Screen</Text>
      <Button title="Go back" onPress={() => navigation.goBack()} />
      <Button
        title="Go back to Home"
        onPress={() => navigation.popTo('Home')}
      />
      <Button
        title="Go to first screen in stack"
        onPress={() => navigation.popToTop()}
      />
    </View>
  );
}
\end{lstlisting}
\end{tcolorbox}

Yaha, \important{Go back} pe ek screen pichhe jayega, \important{Go back to Home} pe direct Home screen, aur \important{Go to first screen in stack} pe stack ka sabse first screen dikhega.[5][1]

\vspace{1.5em}
\section*{\color{headingblue}\LARGE\bfseries Saare Doubt Clear – Cheat Sheet}
\begin{tabular}{|p{6cm}|p{6cm}|p{6cm}|}
\hline
\textbf{Command/Example} & \textbf{Kya Karta Hai?} & \textbf{Kab Use Karein?} \\
\hline
navigation.goBack() & Pichhle screen par le jata hai & Custom back button, modal close, hardware back \\
\hline
navigation.canGoBack() & Check karta hai pichhe ja sakte ho & GoBack use karne se pehle check karo \\
\hline
navigation.popTo('RouteName') & Kisi specific screen par jaata hai & Multiscreen back, jaise dashboard/detail flow \\
\hline
navigation.popToTop() & Stack ke pehle screen par jaata hai & Jaise login se home, ya kisi flow me shuru se vapasi \\
\hline
navigation.navigate('Home') & Direct home screen par le jata hai, par history clear nahi hoti & Tab navigation, main screen par jaana ho \\
\hline
\end{tabular}

\vspace{1.5em}
\section*{\color{headingblue}\LARGE\bfseries Conclusion – Ek Dum Beginner Level}
\begin{itemize}
\item \important{navigation.goBack()} se screen se screen pichhe aate ho, jaise mobile phone me back button.
\item Header par default back button hota hai, lekin aap custom button pe bhi isse bhej sakte ho.
\item Android hardware back button bhi yehi function call karta hai.
\item Root screen pe goBack() se kuch nahi hoga.
\item Ek hi bar me multiple screen back karna ho, toh popTo/popToTop use karo.
\item GoBack() se app ki user experience consistent aur predictable rahegi.
\end{itemize}
\important{Aap jab bhi ek screen se dusri screen par navigate karenge, aur code se back karna ho—bus navigation object ka goBack() use kar lena—sab khud samajh me aa jayega!}

\vspace{2em}

\separatorline


=============================================================

\documentclass[a4paper,12pt]{article}

% Geometry with comfortable margins
\usepackage[left=25mm,right=25mm,top=25mm,bottom=25mm]{geometry}

% Fonts
\usepackage{tgtermes} % TeX Gyre Termes (Times-like serif for body)
\usepackage[T1]{fontenc}
\usepackage[utf8]{inputenc}
\usepackage{sectsty} % for section title font control

% Color definitions
\usepackage{xcolor}
\definecolor{headingblue}{RGB}{0,70,132} % strong blue for headings
\definecolor{importantred}{RGB}{204,0,0} % bright red for important text
\definecolor{codebg}{RGB}{245,245,245} % very light gray background for code
\definecolor{keywordcolor}{RGB}{0,0,180} % dark blue for keywords
\definecolor{stringcolor}{RGB}{163,21,21} % dark red for strings
\definecolor{commentcolor}{RGB}{0,128,0} % green for comments
\definecolor{functioncolor}{RGB}{127,0,85} % purple for function names
\definecolor{linenumbercolor}{gray}{0.6} % subtle gray for line numbers

% Listings for code blocks with multi-color syntax highlighting
\usepackage{listings}
\lstdefinelanguage{jsx}{
  keywords={import,function,let,const,var,if,else,return,for,while,do,switch,case,break,continue,class,new,try,catch,finally,throw,async,await,void,static,public,private,protected,yield,default},
  sensitive=true,
  morecomment=[l]{//},
  morecomment=[s]{/*}{*/},
  morestring=[b]",
  morestring=[b]',
}
\lstset{
  language=jsx,
  backgroundcolor=\color{codebg},
  keywordstyle=\color{keywordcolor}\bfseries,
  stringstyle=\color{stringcolor},
  commentstyle=\color{commentcolor}\itshape,
  identifierstyle=\color{black},
  basicstyle=\ttfamily\footnotesize,
  numbers=left,
  numberstyle=\color{linenumbercolor}\scriptsize,
  stepnumber=1,
  numbersep=8pt,
  frame=none,
  tabsize=4,
  showstringspaces=false,
  breaklines=true,
  breakatwhitespace=true,
  captionpos=b,
  xleftmargin=10pt,
  xrightmargin=10pt,
  framexleftmargin=5pt,
  framexrightmargin=5pt,
  rulecolor=\color{codebg},
  upquote=true,
  literate={~}{{\textasciitilde}}1
}

% tcolorbox for code blocks with rounded corners and light background
\usepackage[many]{tcolorbox}
\tcbset{
  colback=codebg,
  colframe=codebg,
  boxrule=0pt,
  arc=3mm,
  auto outer arc,
  leftrule=0pt,
  rightrule=0pt,
  toprule=0pt,
  bottomrule=0pt,
  left=6pt,
  right=6pt,
  top=6pt,
  bottom=6pt,
  enhanced,
}

% Section title formatting with colors and size
\usepackage{titlesec}
\titleformat{\section}
  {\color{headingblue}\LARGE\bfseries}
  {}
  {0pt}
  {}
\titlespacing*{\section}{0pt}{4ex plus 1ex minus .2ex}{1.5ex plus .2ex}

% For horizontal separator lines text exactly as given
\usepackage{titling}

% Header and footer with fancyhdr
\usepackage{fancyhdr}
\pagestyle{fancy}
\fancyhf{}
\fancyhead[L]{React Native Notes}
\fancyhead[R]{\thepage}
\renewcommand{\headrulewidth}{0.4pt}
\renewcommand{\footrulewidth}{0pt}

% Lists with enumitem for spacing control
\usepackage{enumitem}
\setlist{itemsep=0.9em, parsep=0pt, topsep=0.9em}

% Hyperlinks without colors
\usepackage[hidelinks]{hyperref}

% Custom command for important red text
\newcommand{\important}[1]{\textcolor{importantred}{#1}}

% Preserve monospace inline code styling
\newcommand{\inlinecode}[1]{\texttt{#1}}

% To preserve the exact separator line visually
\newcommand{\separatorline}{\texttt{=============================================================}}

\begin{document}

\section*{\color{headingblue}\LARGE\bfseries React Native Basic UI \& Flexbox: Fundamentals, Responsive Layout, Multi-Screen Support – Sab Kuch Detail Mein, Simple Examples Ke Sath, Sarre Doubt Clear}
\noindent \texttt{text===================================================================}

\vspace{1.5em}

\section*{\color{headingblue}\LARGE\bfseries Flexbox Fundamentals}
\important{Flexbox} React Native me UI layout karne ka sabse powerful system hai. Isme aap \important{flex}, \important{justifyContent}, \important{alignItems} jaise properties use karte ho.[1][2]

\begin{itemize}
\item \important{flex:} Yeh batata hai ki ek View kis proportion mein baki Views ke sath space share karega.
\begin{itemize}
\item \important{flex: 1} se View apne parent ka poora space leta hai.
\item \important{flex: 2} se doosre View \important{flex: 1} se double space lega.
\item Bahut saare child views ko equal space dene ke liye sab ka \important{flex: 1} de do.
\end{itemize}
\item \important{flexDirection:} Yeh batata hai ki children \important{row} (left-to-right) ya \important{column} (top-to-bottom) mein arrange honge.
\item \important{justifyContent:} Yeh main axis (row ya column) par content ko kahan position karega—\important{flex-start, center, flex-end, space-between, space-around, space-evenly}.
\item \important{alignItems:} Yeh cross axis (row ka vertical, column ka horizontal) par content ko kahan position karega—\important{flex-start, center, flex-end, stretch}.
\end{itemize}

\important{Kab use karein?}  

Jab aapko UI ko dynamically, responsive, aur consistent banana ho, har device pe consistent dikhe—flexbox hi best hai.  

\important{Kab nahi karein?}  

Agar kisi element ka size exactly fixed hai, jaise icon ya avatar—tab fixed width/height use karo.

\vspace{1.5em}
\section*{\color{headingblue}\LARGE\bfseries Simple Example – Flexbox Use}

\begin{tcolorbox}[sharp corners,boxrule=0.5pt,colback=codebg]
\begin{lstlisting}[language=jsx, numbers=left]
import { StyleSheet, View } from 'react-native';
export default function App() {
  return (
    <View style={styles.container}>
      <View style={styles.red} />
      <View style={styles.green} />
      <View style={styles.blue} />
    </View>
  );
}
const styles = StyleSheet.create({
  container: {
    flex: 1,
    flexDirection: 'row',   // Left-to-right
    justifyContent: 'space-evenly',  // Equal space between items
    alignItems: 'center',   // Vertical center
  },
  red: { width: 50, height: 50, backgroundColor: 'red' },
  green: { width: 50, height: 50, backgroundColor: 'green' },
  blue: { width: 50, height: 50, backgroundColor: 'blue' },
});
\end{lstlisting}
\end{tcolorbox}

Isme, teen boxes horizontal flex row mein hai, sabme equal space, vertical center par.[2][3]

\vspace{1.5em}
\section*{\color{headingblue}\LARGE\bfseries Responsive UI – Percentage-Based Sizing}

\important{Hard-coded width/height se bachna chahiye}—kyunki alag-alag screen sizes pe element chhote ya bade ho sakte hain, UI ka design bigad sakta hai.[4][5]

\begin{itemize}
\item \important{Percentage (\% ) Use Karein:}
\end{itemize}

\begin{tcolorbox}[sharp corners,boxrule=0.5pt,colback=codebg]
\begin{lstlisting}[language=jsx,numbers=none]
<View style={{ width: '80%', height: '50%' }}>
  <Text>Responsive Box</Text>
</View>
\end{lstlisting}
\end{tcolorbox}

Isse element screen ke according resize ho jayega, har device pe proportional rehga.[5][4]

\begin{itemize}
\item \important{Dimensions API:}  
Screen width/height nikal ke dynamic size calculate karo:
\end{itemize}

\begin{tcolorbox}[sharp corners,boxrule=0.5pt,colback=codebg]
\begin{lstlisting}[language=jsx,numbers=none]
import { Dimensions } from 'react-native';
const { width, height } = Dimensions.get('window');
...
<View style={{ width: width * 0.6, height: height / 6 }} />
\end{lstlisting}
\end{tcolorbox}

\important{Isse aap pixel-perfect responsive sizing kar sakte ho, big screen (tablet, TV) par bhi.}[6][4]

\begin{itemize}
\item \important{react-native-responsive-screen:}  
Ye library direct responsive percentage deti hai, easy hai:
\end{itemize}

\begin{tcolorbox}[sharp corners,boxrule=0.5pt,colback=codebg]
\begin{lstlisting}[language=jsx,numbers=none]
import { widthPercentageToDP as wp, heightPercentageToDP as hp } from 'react-native-responsive-screen';
<View style={{ width: wp('80%'), height: hp('30%') }} />
\end{lstlisting}
\end{tcolorbox}

\vspace{1.5em}
\section*{\color{headingblue}\LARGE\bfseries Kab Use Karein, Kab Nahi?}

\begin{tabular}{|p{5cm}|p{6cm}|p{6cm}|}
\hline
\textbf{Use Case} & \textbf{Solution} & \textbf{Example/Nahi Karein Toh} \\
\hline
Parent container & flex: 1 & Poore screen ko cover karega \\
Equal space (row/column) & flexDirection, flex: 1 & List item, card grid \\
Center content & justifyContent, alignItems & Home screen, logo, buttons \\
Proportional resize & Percentage, Dimensions API & Buttons, cards, containers (avoid fixed w/h) \\
Exakt fix chahiye (icon) & Fixed width/height & Icons, avatars (rare cases) \\
\hline
\end{tabular}

\vspace{1.5em}
\section*{\color{headingblue}\LARGE\bfseries Responsive, Multi-Screen (Tablet, Android TV, Phones)}
\begin{itemize}
\item Flexbox se aapka layout har screen size par proportional, stretch, ya center ho jata hai—tablet, TV, phone sab par.
\item Percentage-based sizing se elements sab devices par sahi scale hote hain—na jyada chhote, na jyada bade.
\item Dimensions API se pixel ratio, padding, margins bhi calculate kar sakte ho.
\item \important{Platform-specific styling:}  
Kuch UI components chahiye ho specific devices ke liye, tab Platform.select use karo:
\end{itemize}

\begin{tcolorbox}[sharp corners,boxrule=0.5pt,colback=codebg]
\begin{lstlisting}[language=jsx,numbers=none]
import { Platform } from 'react-native';
const styles = StyleSheet.create({
  box: {
    ...Platform.select({
      ios: { backgroundColor: '#007AFF' },
      android: { backgroundColor: '#3DDC84' },
    }),
  },
});
\end{lstlisting}
\end{tcolorbox}

\begin{itemize}
\item \important{For TV support:}  
Layouts thode bade chahiye, fonts/text/large buttons—use percentage, flex, Dimensions API for everything.[7]
\item \important{ScrollView:}  
Jab content scroll ho sakta ho, ya zyada content ho, toh ScrollView use karo—static height avoid karo.
\end{itemize}

\vspace{1.5em}
\section*{\color{headingblue}\LARGE\bfseries Cheat Sheet – Saare Points}
\begin{itemize}
\item Flexbox ka use karo har jagah—flex, justifyContent, alignItems.
\item Hard-coded width/height avoid karo—apne app ko responsive banao.
\item Percentage (\% ) and Dimensions API use karo—sab devices par UI proportional rehga.
\item Platform.select se platform-specific styling karo.
\item Tablet, TV, phone—sab par UI fit hona chahiye, flexbox aur percentage se ho jayega.
\item Exakt fix size icon, avatar etc. ke liye hi hard-coded width/height use karo.
\end{itemize}

\vspace{1.5em}
\section*{\color{headingblue}\LARGE\bfseries Ek Dum Simple Example – Multi-Screen Responsive UI}

\begin{tcolorbox}[sharp corners,boxrule=0.5pt,colback=codebg]
\begin{lstlisting}[language=jsx, numbers=left]
import { StyleSheet, View, Text, Dimensions } from 'react-native';
export default function App() {
  const { width, height } = Dimensions.get('window');
  return (
    <View style={styles.container}>
      <View style={[styles.box, { width: width * 0.8, height: height * 0.3 }]}>
        <Text style={[styles.text, { fontSize: width * 0.05 }]}>Hello, Responsive!</Text>
      </View>
    </View>
  );
}
const styles = StyleSheet.create({
  container: {
    flex: 1,
    justifyContent: 'center',
    alignItems: 'center',
  },
  box: {
    backgroundColor: 'lightblue',
    justifyContent: 'center',
    alignItems: 'center',
  },
  text: { fontWeight: 'bold' }
});
\end{lstlisting}
\end{tcolorbox}

Yahan box ka size screen ke according hai, text size bhi proportional—sab devices par acche se dikhega.[4][6]

\vspace{1.5em}
\section*{\color{headingblue}\LARGE\bfseries Conclusion – Aapke Notes Ke Liye}
\begin{itemize}
\item Flexbox se UI flexible, responsive, aur consistent rakho.
\item Hard-coded width/height se bacho—percentage aur Dimensions API use karo.
\item Platform.select se platform/device specific styling karo.
\item Tablet, TV, phone—sab par UI fit ho jayega.
\item Exakt icon, avatar ke liye hi hard-coded width/height use karo.
\item ScrollView use karo jab content zyada ho.
\item Practice karo, design karo, screen size change karke test karo—sab automatic fit ho jayega.
\end{itemize}

\important{Aaj se aapka koi bhi doubt flexbox, responsive UI, multi-screen support ka nahi rahega—yahi rules follow karo, app sab devices par mast dikhegi!}

\vspace{2em}

\separatorline

\vspace{2em}

\section*{\color{headingblue}\LARGE\bfseries React Native CLI vs Expo — Kya Hai?}
\noindent 

\begin{itemize}
\item \important{React Native CLI:} Ye hai “bare bones” project. Isme app banate waqt \important{Android Studio, Xcode, aur saare native tools install} karne padte hain. \important{Full native code access} hai — matlab aap apni marzi ka Java/Kotlin/Swift/Obective-C code react native ke sath use kar sakte ho, kisi bhi native library ko directly integrate kar sakte ho.  
\important{Best hai advanced, highly customized, aur heavy apps ke liye} — jaise gaming, video editing, customized camera, ya koi bahut unique native integration chahiye ho.[4][5]
\item \important{Expo:} Ye hai “managed workflow”. Aapko \important{bas ek command} se project banake, \important{expo start} ya \important{eas build} kar sakte ho. \important{Android Studio/Xcode set up nahi karna padta}. \important{Native code access limited} — aap sirf wahi native features use kar sakte ho jo Expo SDK provide karta hai. \important{Testing ke liye Expo Go app} hain — QR code scan karke real device par chal jata hai, \important{simulator/emulator set up nahi chahiye}.[5][6]
\end{itemize}

\vspace{1.5em}
\section*{\color{headingblue}\LARGE\bfseries Kab Kisko Use Karna Chahiye?}

\subsection*{\color{headingblue}\Large\bfseries Expo Use Karein:}
\begin{itemize}
\item \important{Jaldi app banana ho:} Demo, prototype, ya POC app jaldi chahiye, toh Expo best hai. Setup bahot fast, testing bhi fast.
\item \important{Beginner ho:} Agar aapko native development nahi aata, ya set-up me pareshani nahi leni, toh Expo use karo. \important{Saari native complexity Expo khud handle karta hai}.
\item \important{Simple features chahiye:} Agar aapko camera, notification, maps, GPS, storage, biometric, aur aise basic mobile features chahiye, toh Expo pe sab mil jata hai.
\item \important{OTA updates:} Expo apps ko store pe publish kiye bina over-the-air (OTA) update kar sakte ho — jaise website hota hai waisa, app store se naye update aane ka wait nahi karna.
\item \important{Kisi aur ko test app bhejna ho:} Expo Go QR code share karo, app chali jayegi kisi bhi mobile pe.
\item \important{Internal tools ya small business app:} Office apps, attendance, field reporting — jaha native integration chahiye nahi, Expo best hai.
\end{itemize}

\textbf{Example:}  

Aapko ek “News App” banana hai, image, text, maps, notification chahiye — sab Expo se 5 minute me chal jata hai.

\vspace{1.5em}
\subsection*{\color{headingblue}\Large\bfseries React Native CLI Use Karein:}
\begin{itemize}
\item \important{Big, complex, ya production-grade app bana ho:} Gaming, video editing, live streaming, ya koi app jisme advanced native code, background processing, custom hardware, ya third-party native libraries chahiye, toh CLI hi chahiye.
\item \important{Native code customize karna ho:} Agar aapko apna custom splash screen, custom camera, custom notification, ya kisi device ke hardware ko directly access karna ho, toh CLI mein wahi milega, Expo pe mushkil.
\item \important{App size \& performance matter kare:} Expo apps thoda bulky ho sakte hain, CLI se aap sirf wahi libraries include kar sakte jo aapko chahiye, size kam ho jata hai.
\item \important{Enterprise, public-facing, ya consumer-facing app:} Facebook, Instagram, Uber, Twitter — sabka code custom native integration wala hai, isliye CLI use karte hain.
\item \important{Team me native devs ho:} Agar team me android/iOS devs hain, jo java/kotlin/swift/objective-c likh sakte hain, toh CLI ka best use hota hai.
\end{itemize}

\textbf{Example:}  

Aapko ek Uber, Instagram, Snapchat jaisa app banana hai, customized native features, advanced camera, gesture recognition, ya kuch aisa unique ho, toh CLI hi use karo.

\vspace{1.5em}
\section*{\color{headingblue}\LARGE\bfseries Kab Nahi Karna Chahiye?}

\begin{itemize}
\item \important{Expo use na karein agar:}  
Aapko native code me direct tweak karna ho, aapko koi native library add karni ho jo Expo support nahi karta, ya aapko app size and performance optimize karna ho.
\item \important{CLI use na karein agar:}  
Aapka requirement simple hai, aapko jaldi start kar ke jaldi demo banana hai, ya aapko app store deployment and updates me pareshani karni hai.
\end{itemize}

\vspace{1.5em}
\section*{\color{headingblue}\LARGE\bfseries Expo vs CLI — Cheet Sheet}

\begin{tabular}{|p{5cm}|p{6cm}|p{6cm}|}
\hline
\textbf{Feature/Use Case} & \textbf{Expo} & \textbf{React Native CLI} \\
\hline
Setup & Bas ek command, sab automatic & Xcode, Android Studio, sab khud hi karo \\
Native Access & Limited (Expo SDK tak) & Full (kuch bhi native kar sakte ho) \\
Build & Expo server pe, ya eas se & Khud apni machine pe, manual \\
App Size & Thoda bada (optimize bhi hota hai) & Chota (bas jo chahiye wahi milega) \\
Development Speed & Bahot fast & Slow (set-up, testing, build sab me time lagta hai) \\
Best For & Beginners, MVP, demo, internal apps & Big apps, custom native, gamer, enterprise apps \\
OTA Updates & Hai & Nahi (third-party tools se ho sakta hai) \\
Backward Compatibility & Expo SDK pe depend, kabhi-kabhi hiccup ho sakta hai & Directly react-native dependency, chalta rehta hai \\
Cross-Platform & Android, iOS, Web & Android, iOS (web alag se karna padta hai) \\
Ease of Testing & Expo Go app se, bas QR code & Simulator/Emulator, ya manual device pe test karo \\
\hline
\end{tabular}

\vspace{1.5em}
\section*{\color{headingblue}\LARGE\bfseries Popular Myths — Sab Clear}
\begin{itemize}
\item Expo apps slow nahi hoti ab: Pehle Expo apps ka size bada hota tha, ab sab optimize ho gaya hai.
\item Expo se native code ka access nahi: Ab \important{Expo development build} se aap native code add kar sakte ho, lekin thoda complex hota hai.
\item Expo ke liye app store publish nahi ho sakta: Galat! Expo se directly published ho sakta hai, process easy hai.
\item CLI wali apps store pe upload nahi hoti: Galat! CLI apps bhi store pe directly chal jati hain, lekin manual setup karna padta hai.
\end{itemize}

\vspace{1.5em}
\section*{\color{headingblue}\LARGE\bfseries Example — Kaun Sa Kaisa Hai?}
\begin{itemize}
\item Small Project: News App, Notes App, Attendance App — \important{Expo}.
\item Big Project: Uber, Instagram, Custom Camera App, Streaming App — \important{CLI}.
\item Agar pata nahi toh kya use karein: Pehle \important{Expo} se start karo, agar kuch feature Expo pe nahi mil raha, toh \important{Expo development build} ya \important{CLI} pe switch karo.
\end{itemize}

\vspace{1.5em}
\section*{\color{headingblue}\LARGE\bfseries Summary — Ek Dum Shabdo Me}
\begin{itemize}
\item React Native CLI = Sab kuch customize, sab kuch control, lekin set-up me mehnat zyada, time lagega, big/complex apps ke liye, professional teams ke liye.
\item Expo = Jaldi, aasani se, beginners ke liye, simple apps, internal tools, demo, jaldi market me app dalna hai.
\item Agar pata nahi toh Expo pe start karo, bad me agar kuch zarurat pade toh CLI pe shift ho jaao!
\end{itemize}

\important{Agar aapka koi specific use case ho, toh batao, main recommend kar dunga Kya use karna chahiye aur kya examples ho sakte hain!}  

\important{Ab aap notes banani chaho toh saamne se copy-paste kar lo, aap apne cheet sheet me likh lo — sab clear hai!}




=============================================================

\documentclass[a4paper,12pt]{article}

% Geometry with comfortable margins
\usepackage[left=25mm,right=25mm,top=25mm,bottom=25mm]{geometry}

% Fonts
\usepackage{tgtermes} % TeX Gyre Termes (Times-like serif for body)
\usepackage[T1]{fontenc}
\usepackage[utf8]{inputenc}
\usepackage{sectsty} % for section title font control

% Color definitions
\usepackage{xcolor}
\definecolor{headingblue}{RGB}{0,70,132} % strong blue for headings
\definecolor{importantred}{RGB}{204,0,0} % bright red for important text
\definecolor{codebg}{RGB}{245,245,245} % very light gray background for code
\definecolor{keywordcolor}{RGB}{0,0,180} % dark blue for keywords
\definecolor{stringcolor}{RGB}{163,21,21} % dark red for strings
\definecolor{commentcolor}{RGB}{0,128,0} % green for comments
\definecolor{functioncolor}{RGB}{127,0,85} % purple for function names
\definecolor{linenumbercolor}{gray}{0.6} % subtle gray for line numbers

% Listings for code blocks with multi-color syntax highlighting
\usepackage{listings}
\lstdefinelanguage{gradle}{
  keywords={implementation,debugImplementation},
  sensitive=true,
  morecomment=[l]{//},
  morestring=[b]",
}
\lstdefinelanguage{ruby}{
  keywords={use_frameworks!,use_flipper!,FlipperConfiguration,enabled},
  sensitive=true,
  morecomment=[l]{#},
  morestring=[b]",
}
\lstdefinelanguage{json}{
  morestring=[b]",
  morecomment=[l]{//},
  sensitive=true,
  morekeywords={true,false,null},
  alsoletter={:,[],{}},
}
\lstdefinelanguage{js}{
  keywords={import,from,as,export,const,function,return,if,else},
  sensitive=true,
  morecomment=[l]{//},
  morecomment=[s]{/*}{*/},
  morestring=[b]",
  morestring=[b]',
}
\lstset{
  basicstyle=\ttfamily\footnotesize,
  backgroundcolor=\color{codebg},
  keywordstyle=\color{keywordcolor}\bfseries,
  stringstyle=\color{stringcolor},
  commentstyle=\color{commentcolor}\itshape,
  identifierstyle=\color{black},
  numbers=left,
  numberstyle=\color{linenumbercolor}\scriptsize,
  stepnumber=1,
  numbersep=8pt,
  frame=none,
  tabsize=2,
  showstringspaces=false,
  breaklines=true,
  breakatwhitespace=true,
  captionpos=b,
  xleftmargin=10pt,
  xrightmargin=10pt,
  framexleftmargin=5pt,
  framexrightmargin=5pt,
  rulecolor=\color{codebg},
  upquote=true,
}

% tcolorbox for code blocks with rounded corners and light background
\usepackage[many]{tcolorbox}
\tcbset{
  colback=codebg,
  colframe=codebg,
  boxrule=0pt,
  arc=3mm,
  auto outer arc,
  leftrule=0pt,
  rightrule=0pt,
  toprule=0pt,
  bottomrule=0pt,
  left=6pt,
  right=6pt,
  top=6pt,
  bottom=6pt,
  enhanced,
}

% Section title formatting with colors and size
\usepackage{titlesec}
\titleformat{\section}
  {\color{headingblue}\LARGE\bfseries}
  {}
  {0pt}
  {}
\titlespacing*{\section}{0pt}{4ex plus 1ex minus .2ex}{1.5ex plus .2ex}

% Header and footer with fancyhdr
\usepackage{fancyhdr}
\pagestyle{fancy}
\fancyhf{}
\fancyhead[L]{React Native Notes}
\fancyhead[R]{\thepage}
\renewcommand{\headrulewidth}{0.4pt}
\renewcommand{\footrulewidth}{0pt}

% Lists with enumitem for spacing control
\usepackage{enumitem}
\setlist{itemsep=0.9em, parsep=0pt, topsep=0.9em}

% Hyperlinks without colors
\usepackage[hidelinks]{hyperref}

% Custom command for important red text
\newcommand{\important}[1]{\textcolor{importantred}{#1}}

% Preserve monospace inline code styling
\newcommand{\inlinecode}[1]{\texttt{#1}}

% To preserve the exact separator line visually
\newcommand{\separatorline}{\texttt{=============================================================}}

\begin{document}

\section*{\color{headingblue}\LARGE\bfseries React Native Debug Tools: Flipper, React Native Debugger – Sab Kuch Hinglish Mein, Step by Step, Features, Use Cases, Cheat Sheet (Updated as of September 2025)}
\separatorline

\vspace{1.5em}

\section*{\color{headingblue}\LARGE\bfseries 1. Flipper Kya Hai? (What is Flipper?)}

\important{Flipper} ek desktop app hai jo \important{Meta (Facebook)} ne banaya hai. \important{React Native, iOS, aur Android apps ko debug karne} ka kaam aata hai. Flipper pe aap \important{apni app ko visually inspect} kar sakte ho, \important{network calls, native logs, UI layout, images, storage, crashes, aur performance} sab kuch ek hi screen pe dekh sakte ho. Ye \important{plugin-based} hai, matlab aap alag-alag debugging features ko alag-alag plugins se add kar sakte ho.

\important{React Native 0.62+} se direct support hai (0.69+ ke liye latest versions best hain). \important{Expo} me bhi Flipper use kar sakte ho, lekin Expo SDK 50+ pe Expo DevTools plugins prefer karo kyunki remote JS debugging deprecated hai. Thoda extra setup chahiye Flipper ke liye.

\textbf{Note:} Flipper desktop app ka latest version \inlinecode{~0.273.0} hai (npm pe react-native-flipper 0.273.0 as of Nov 2024), lekin agar React DevTools issues aa rahe hain (jaise v4 limitations), toh desktop app v0.239.0 use karo. RN 0.75+ pe compatible hai, Hermes/New Architecture ke saath.

\vspace{1.5em}
\section*{\color{headingblue}\LARGE\bfseries 2. React Native Debugger Kya Hai?}

\important{React Native Debugger (RnD)} bhi ek desktop app hai, specially \important{JavaScript code ko debug} karne ke liye—jaise Chrome DevTools, lekin React Native optimized. \important{JavaScript console, network, redux, component state/props} inspect kar sakte ho. \important{Ye sirf JavaScript debug karta hai, Native side kaam nahi dikhata}.

\important{Important Update (2025):} Ye old remote debugger pe based hai aur Hermes, JSI, New Architecture ko support nahi karta (RN 0.73+ se remote JS debugging deprecated hai). Expo SDK 50+ pe Expo DevTools plugins use karo instead. Agar RN 0.62+ use kar rahe ho, toh v0.11+ RnD version le lo, lekin latest RN ke liye built-in React Native DevTools better hai.

\vspace{1.5em}
\section*{\color{headingblue}\LARGE\bfseries 3. Flipper vs React Native Debugger – Kab Kya Use Karein?}

\begin{tabular}{|p{4cm}|p{5cm}|p{5cm}|p{5cm}|}
\hline
\textbf{Tool} & \textbf{Kya Dekh Sakte Ho?} & \textbf{Best For} & \textbf{Limitations} \\
\hline
\important{Flipper} & Network, logs, layout, images, storage, crash, performance, plugins (Redux, AsyncStorage, etc.) & \important{Complete app debug}, \important{native + JavaScript}, \important{multi-device support}, Hermes/New Arch compatible & \important{Breakpoint debugging nahi}, \important{plugin setup me thoda mehnat}, Windows pe issues ho sakte hain (v0.239.0 use karo) \\
\hline
\important{React Native Debugger} & JavaScript, Redux, network, console, state/props & \important{JavaScript debugging}, \important{redux/state/props inspect} (old RN versions ke liye) & \important{Native logs/layout/performance nahi}, \important{Hermes/JSI/New Arch nahi support karta}, deprecated features pe depend \\
\hline
\end{tabular}

\vspace{0.5em}
\noindent
\important{Flipper} use karo jab \important{pure app ka native + JavaScript debug} chahiye ho, specially modern RN pe.  

\important{React Native Debugger} use karo jab \important{sirf JS code pe galti dhondhni ho} aur old setup hai (naye RN pe avoid karo).

\vspace{1.5em}
\section*{\color{headingblue}\LARGE\bfseries 4. Flipper – Setup, Configuration – Step by Step}

\subsection*{\color{headingblue}\large\bfseries A. React Native CLI Project (0.62 or Above) pe Flipper}

\begin{itemize}
\item \important{Flipper Desktop App} install karo (fbflipper.com se download, latest ~0.273.0, lekin agar RN debug issues toh v0.239.0 le lo).
\item \important{Android:}
\begin{enumerate}
  \item \inlinecode{android/gradle.properties} me \inlinecode{FLIPPER_VERSION=0.273.0} set karo (latest match karo).
  \item \inlinecode{android/app/build.gradle} me dependencies section me ye add karo (missing tha pehle):  
\end{enumerate}
\end{itemize}

\begin{tcolorbox}[sharp corners,boxrule=0.5pt,colback=codebg]
\begin{lstlisting}[language=gradle,numbers=left]
debugImplementation 'com.facebook.flipper:flipper:0.273.0'
debugImplementation 'com.facebook.flipper:flipper-network-plugin:0.273.0'
debugImplementation 'com.facebook.flipper:flipper-fresco-plugin:0.273.0'  // Agar images use kar rahe ho
\end{lstlisting}
\end{tcolorbox}

\begin{itemize}
\begin{enumerate}
  \setcounter{enumi}{2}
  \item \inlinecode{cd android \&\& ./gradlew clean} chalao (build clean karne ke liye).
  \item App ko \inlinecode{yarn android} ya \inlinecode{npm run android} se chalao, Flipper khud detect kar lega.
\end{enumerate}
\item \important{iOS:}
\begin{enumerate}
  \item Agar RN 0.69+ hai: \inlinecode{ios/Podfile} me ye add karo:  
\end{enumerate}
\end{itemize}

\begin{tcolorbox}[sharp corners,boxrule=0.5pt,colback=codebg]
\begin{lstlisting}[language=ruby,numbers=left]
:flipper_configuration => FlipperConfiguration.enabled(["Debug"], { 'Flipper' => '0.273.0' }),  // Latest version use karo, example 0.190.0 bhi diya hai docs me lekin 0.273.0 try
\end{lstlisting}
\end{tcolorbox}

\begin{itemize}
\begin{enumerate}
  \setcounter{enumi}{1}
  \item (Note: \inlinecode{use_frameworks!} enabled ho toh Flipper kaam nahi karega, disable karo.)
  \item Agar RN <0.69: \inlinecode{ios/Podfile} me \inlinecode{use_flipper!({ 'Flipper' => '0.273.0' })} add karo.
  \item \inlinecode{cd ios \&\& pod install --repo-update} chalao.
  \item App ko \inlinecode{yarn ios} ya \inlinecode{npm run ios} se chalao, Flipper connect ho jayega.
\end{enumerate}
\end{itemize}

\begin{itemize}
\item \important{Flipper open karo:} App automatically show ho jayegi. Agar nahi dikhti, Flipper restart karo ya device/emulator check karo.
\end{itemize}

\textbf{Advanced/Troubleshooting:}  
\begin{itemize}
\item Windows pe Issue: Latest v0.273.0 me .exe miss ho sakta hai, v0.239.0 use karo ya manual setup.
\item Hermes/New Arch: Compatible hai, koi extra step nahi.
\item react-native-flipper Package: Agar manual integration chahiye (jaise plugins ke liye), \inlinecode{yarn add react-native-flipper} karo aur app code me initialize karo (optional for basic).
\end{itemize}

\subsection*{\color{headingblue}\large\bfseries B. Expo Project pe Flipper}

\begin{itemize}
\item Expo SDK 50+ Update: Remote debugging deprecated hai, isliye Expo DevTools plugins use karo as alternative. Lekin Flipper integrate karne ke liye \inlinecode{expo-build-properties} plugin use karo (bare workflow ya custom dev client pe).
\begin{enumerate}
  \item \inlinecode{npx expo install expo-build-properties} chalao.
  \item \inlinecode{app.json} me plugins section add karo:
\end{enumerate}
\end{itemize}

\begin{tcolorbox}[sharp corners,boxrule=0.5pt,colback=codebg]
\begin{lstlisting}[language=json,numbers=left]
{
  "plugins": [
    ["expo-build-properties", {
      "ios": { "flipper": true },
      "android": { "flipper": true }
    }]
  ]
}
\end{lstlisting}
\end{tcolorbox}

\begin{itemize}
\begin{enumerate}
  \setcounter{enumi}{2}
  \item \inlinecode{npx expo prebuild} chalao (project configure karne ke liye).
  \item Agar bare workflow: \inlinecode{yarn add react-native-flipper react-devtools-core} (optional for JS side).
\end{enumerate}
\item Expo SDK <50: Old way use karo: \inlinecode{npx expo install react-native-flipper react-devtools-core}.
\item Expo Go pe Flipper kaam nahi karega—\important{custom dev client ya EAS build} chahiye (\inlinecode{npx expo run:android/ios}).
\item Note: Expo DevTools plugins Flipper jaise features deta hai, GitHub pe check karo migration ke liye.
\end{itemize}

\vspace{1.5em}
\section*{\color{headingblue}\LARGE\bfseries 5. Flipper – Main Features, Kaise Use Karein}

\subsection*{\color{headingblue}\large\bfseries A. Network Inspector}

\begin{itemize}
\item Network tab open karo, sab network requests dikhenge—fetch, axios, API, kuch bhi (websockets bhi support).
\item URL, method, headers, body, response, status, timing—sab inspect kar sakte ho, curl export bhi kar sakte ho.
\item Use case: API fail ho raha hai, kya response aa raha hai, kya request ja raha hai—sab yahan dikh jayega.
\item Example: Jab aap app me login karte ho, login API ka request, response—Network tab me real-time dekh sakte ho, error codes filter karo.
\end{itemize}

\subsection*{\color{headingblue}\large\bfseries B. Layout Inspector}

\begin{itemize}
\item Layout tab me UI component tree dikh jata hai (native views aur React components).
\item Component select karo: uska props, state, style, bounds—sab inspect kar sakte ho, highlight karo device pe.
\item Use case: Button nahi dikh raha, layout bigad raha hai—component select karo, style ya props me galti dikh jayegi.
\end{itemize}

\subsection*{\color{headingblue}\large\bfseries C. Logs}

\begin{itemize}
\item Logs tab me JavaScript aur native logs dono dikhte hain (console.log, warnings, errors).
\item Filter laga ke search kar sakte ho (level, tag, regex se).
\item Use case: App crash ho raha hai, kon sa error/console.log/exception aaya hai—sab yahan mil jayega, timestamps ke saath.
\end{itemize}

\subsection*{\color{headingblue}\large\bfseries D. Crash Reporter}

\begin{itemize}
\item Crash tab me app crashes ka report dikhta hai.
\item Stack trace, crash reason, device details, threads sab milta hai (Android/iOS both).
\item Use case: Unexpected crash—full report export karo BugSnag ya Sentry me.
\end{itemize}

\subsection*{\color{headingblue}\large\bfseries E. Images}

\begin{itemize}
\item Images tab me app me load ho rahi images dikh jati hain (loaded aur cached).
\item Cache, size, source, format—sab inspect kar sakte ho, zoom karo.
\item Use case: Image blurry ya load fail—source check karo.
\end{itemize}

\subsection*{\color{headingblue}\large\bfseries F. Shared Preferences/Database}

\begin{itemize}
\item Shared Preferences/Database tab me local storage/data ke values dekh sakte ho (AsyncStorage, SQLite, Keychain).
\item Edit bhi kar sakte ho real-time.
\item Use case: Local data save kiya, retrieve nahi ho raha—yahan value dekh/edit lo.
\end{itemize}

\subsection*{\color{headingblue}\large\bfseries G. Metro Logs}

\begin{itemize}
\item Metro bundler ke logs bhi Flipper me dikh sakte hain (bundling errors).
\item Build/meta errors yahan milte hain, filter karo.
\end{itemize}

\subsection*{\color{headingblue}\large\bfseries H. React DevTools}

\begin{itemize}
\item React DevTools plugin enable karo (built-in ya install), React component tree, state, props, hooks inspect kar sakte ho.
\item Use case: State/props me galti hai, component update nahi ho raha—yahan see/edit karo (v5 support in latest).
\end{itemize}

\subsection*{\color{headingblue}\large\bfseries I. Plugins}

\begin{itemize}
\item Plugin Manager se extra plugins install kar sakte ho (Redux, AsyncStorage, Performance, GraphQL, etc.).
\item Plugins install karte ho: Flipper me Plugin Manager open karo, search karo (jaise 'redux-debugger'), install karo, app ko restart/rebuild karo.
\item Third-party plugins: Example: flipper-plugin-redux-debugger (Redux actions/state inspect), flipper-plugin-async-storage (AsyncStorage values)—ye install karne padte hain, app me code add karo jaise:
\end{itemize}

\begin{tcolorbox}[sharp corners,boxrule=0.5pt,colback=codebg]
\begin{lstlisting}[language=js,numbers=left]
import { addPlugin } from 'react-native-flipper';
// For Redux: addPlugin({ name: 'redux', ... });
\end{lstlisting}
\end{tcolorbox}

(Npm se install: \inlinecode{yarn add flipper-plugin-redux-debugger}).

\vspace{1.5em}
\section*{\color{headingblue}\LARGE\bfseries 6. Flipper – Kaise Use Karein, Step by Step}

\begin{enumerate}[itemsep=0.9em]
\item Flipper desktop app install aur open karo.
\item App run karo (\inlinecode{yarn android}/\inlinecode{npm run ios} ya Expo ke liye \inlinecode{npx expo run}).
\item Flipper me app ka icon show hoga—click karo, device select karo.
\item Network, Layout, Logs, Images, etc. tabs pe click karo—real-time inspect karo, filters lagaao.
\item Plugin kisi feature ke liye chahiye ho toh Plugin Manager se install karo—like Redux (app me code add), AsyncStorage, Performance—phir app rebuild.
\item App crash ho toh Crash Reporter tab pe report mil jayegi—export karo.
\item React DevTools plugin enable karo—component tree, state/props dekh/edit lo.
\item Multi-device: Ek hi Flipper me multiple emulators/devices connect kar sakte ho.
\item Export/Share: Inspections export karo reports ke liye.
\end{enumerate}

\vspace{1.5em}
\section*{\color{headingblue}\LARGE\bfseries 7. Flipper Agar Use Nahi Karein Toh Kya Hoga?}

\begin{itemize}
\item Network calls, logs, layout, database, images, crash reports, performance—sab manual check karna padta hai (time waste).
\item JS side pe React Native Debugger ya Chrome DevTools (old remote debug) kuch dekha ja sakta hai, lekin limited.
\item Native side pe Android Studio (logcat, profiler), Xcode (console, instruments) me jana padta hai—time consuming, no unified visual inspection.
\item UI layout bigad raha hai—built-in UI Inspector use karo, ya console.log daalna padta hai.
\item Real-time, multi-device, visual debugging ke liye Flipper sabse best tool hai, specially 2025 me modern RN ke saath.
\end{itemize}

\vspace{1.5em}
\section*{\color{headingblue}\LARGE\bfseries 8. Cheat Sheet – Flipper Features ka Sahi Use}

\begin{tabular}{|p{5cm}|p{5cm}|p{7cm}|}
\hline
\textbf{Problem} & \textbf{Flipper Tab/Plugin} & \textbf{Kaise Use Karein} \\
\hline
API kaam nahi kar raha & Network & Request, response, status, error inspect; curl export karo \\
UI dikh nahi raha/overlap & Layout & Component tree, props, style, bounds inspect; device pe highlight \\
Crash ho gaya & Crash Reporter & Crash log, reason, stacktrace, threads dekh; report export \\
Image load nahi ho rahi & Images & Cache, source, size, format inspect; zoom karo \\
Local data save/retrieve me dikkat & Shared Preferences/Database & Value, storage inspect/edit; real-time changes \\
State/props me galti & React DevTools & Component tree, state, props, hooks inspect/edit \\
Logs me error nahi dikh raha & Logs & Filter laga ke console.log, native logs dekh; search regex \\
Redux/AsyncStorage inspect karna hai & Plugins (redux-debugger, async-storage) & Plugin Manager se install, app me code add, actions/state dekh \\
Performance slow & Performance & CPU, memory, FPS inspect; bottlenecks find karo \\
\hline
\end{tabular}

\vspace{1.5em}
\section*{\color{headingblue}\LARGE\bfseries 9. Example – Flipper ka Practical Use}

\textbf{Scenario:}  

Aapka app pe login API fail ho raha hai, UI bhi thoda bigad raha hai, aur kuch native logs me error dikh raha hai, plus Redux state update nahi ho raha.  

\textbf{Kaam aapka:}  
\begin{itemize}
\item Flipper desktop app open karo.
\item App run karo, Flipper me connect ho jayegi (agar nahi toh version check ya rebuild).
\item Network tab me login API ka request/response dekh lo—kya error aa raha hai (status code, body), headers check, timing dekh.
\item Layout tab me UI components select karo—kaun sa button, text, view galat hai, props/style inspect, bounds adjust ideas lo.
\item Logs tab me console.log, native error dekh lo—filter laga ke exception search, timestamps match karo.
\item Agar kuch local data me dikkat hai toh Shared Preferences/Database tab me value dekh/edit lo.
\item Agar Redux/AsyncStorage ka pareshani hai, toh Plugin Manager se redux-debugger install karo, app me code add (import addPlugin), phir actions/state inspect.
\item Crash ho toh Crash Reporter se stack trace le lo.
\item Performance check karo agar slow hai.
\end{itemize}

Sab kuch ek hi jagah, real-time, visually—bina kisi extra tool ke! Save/export sab kar sakte ho team ke saath share ke liye.

\vspace{1.5em}
\section*{\color{headingblue}\LARGE\bfseries 10. Summary – Ek Dum Beginner Level Cheat Sheet}

\begin{itemize}
\item Flipper = All-in-one React Native, iOS, Android ka debug tool—network, logs, layout, images, crash, database, plugins, sab kuch ek hi jagah (modern RN ke liye best).
\item React Native Debugger = Only JavaScript debugging (console, redux, state/props, network)—native side kaam nahi dikhata, old RN ke liye (naye pe avoid).
\item Flipper setup: Flipper desktop install (v0.273.0), app ko debug mode me chalao (with gradle/Podfile updates), Flipper me app show ho jayegi.
\item Expo pe bhi Flipper chalta hai, thoda extra setup—expo-build-properties add karo, prebuild karo (SDK 50+ pe DevTools plugins prefer).
\item Sabse best features: Network (requests inspect), Layout (UI tree), Logs (filters), Images (cache), Crash Reporter (stacks), Plugins (Redux etc.).
\item Agar Flipper use nahi kare toh: Sab manual, time consuming, no visual inspection—network ke liye Chrome DevTools, logs ke liye logcat/Xcode, UI ke liye inspector, Redux/AsyncStorage ke liye console/alerts.
\item Flipper plugin system: Extra features (Redux, AsyncStorage, Performance, etc.) ke liye Plugin Manager se install karo, app rebuild.
\item Practical use: Sab kuch real-time, visually, ek hi jagah debug karo—bina pareshani ke! Troubleshooting: Version match karo, clean build, Windows pe old version.
\end{itemize}

\important{Aaj se aapka koi bhi doubt debug tools ka nahi rahega—Flipper, React Native Debugger, kaunsi features kiska use hai, kab kaise karna hai—sab updated aur complete aa chuka hai!}

=============================================================

\documentclass[a4paper,12pt]{article}

% Geometry with comfortable margins
\usepackage[left=25mm,right=25mm,top=25mm,bottom=25mm]{geometry}

% Fonts
\usepackage{tgtermes} % TeX Gyre Termes (Times-like serif for body)
\usepackage[T1]{fontenc}
\usepackage[utf8]{inputenc}
\usepackage{sectsty} % for section title font control

% Color definitions
\usepackage{xcolor}
\definecolor{headingblue}{RGB}{0,70,132} % strong blue for headings
\definecolor{importantred}{RGB}{204,0,0} % bright red for important text
\definecolor{codebg}{RGB}{245,245,245} % very light gray background for code
\definecolor{keywordcolor}{RGB}{0,0,180} % dark blue for keywords
\definecolor{stringcolor}{RGB}{163,21,21} % dark red for strings
\definecolor{commentcolor}{RGB}{0,128,0} % green for comments
\definecolor{functioncolor}{RGB}{127,0,85} % purple for function names
\definecolor{linenumbercolor}{gray}{0.6} % subtle gray for line numbers

% Listings for code with multi-color syntax highlighting
\usepackage{listings}
\lstdefinelanguage{jsx}{
  keywords={import,function,let,const,var,if,else,return,for,while,do,switch,case,break,continue,class,new,try,catch,finally,throw,async,await,void,static,public,private,protected,yield,default},
  sensitive=true,
  morecomment=[l]{//},
  morecomment=[s]{/*}{*/},
  morestring=[b]",
  morestring=[b]',
}
\lstset{
  language=jsx,
  backgroundcolor=\color{codebg},
  keywordstyle=\color{keywordcolor}\bfseries,
  stringstyle=\color{stringcolor},
  commentstyle=\color{commentcolor}\itshape,
  identifierstyle=\color{black},
  basicstyle=\ttfamily\footnotesize,
  numbers=left,
  numberstyle=\color{linenumbercolor}\scriptsize,
  stepnumber=1,
  numbersep=8pt,
  frame=none,
  tabsize=4,
  showstringspaces=false,
  breaklines=true,
  breakatwhitespace=true,
  captionpos=b,
  xleftmargin=10pt,
  xrightmargin=10pt,
  framexleftmargin=5pt,
  framexrightmargin=5pt,
  rulecolor=\color{codebg},
  upquote=true,
}

% tcolorbox for code blocks with rounded corners and light background
\usepackage[many]{tcolorbox}
\tcbset{
  colback=codebg,
  colframe=codebg,
  boxrule=0pt,
  arc=3mm,
  auto outer arc,
  leftrule=0pt,
  rightrule=0pt,
  toprule=0pt,
  bottomrule=0pt,
  left=6pt,
  right=6pt,
  top=6pt,
  bottom=6pt,
  enhanced,
}

% Section title formatting with colors and size
\usepackage{titlesec}
\titleformat{\section}
  {\color{headingblue}\LARGE\bfseries}
  {}
  {0pt}
  {}
\titlespacing*{\section}{0pt}{4ex plus 1ex minus .2ex}{1.5ex plus .2ex}

% Header and footer with fancyhdr
\usepackage{fancyhdr}
\pagestyle{fancy}
\fancyhf{}
\fancyhead[L]{React Native Notes}
\fancyhead[R]{\thepage}
\renewcommand{\headrulewidth}{0.4pt}
\renewcommand{\footrulewidth}{0pt}

% Lists with enumitem for spacing control
\usepackage{enumitem}
\setlist{itemsep=0.9em, parsep=0pt, topsep=0.9em}

% Hyperlinks without colors
\usepackage[hidelinks]{hyperref}

% Custom command for important red text
\newcommand{\important}[1]{\textcolor{importantred}{#1}}

% Preserve monospace inline code styling
\newcommand{\inlinecode}[1]{\texttt{#1}}

% To preserve the exact separator line visually
\newcommand{\separatorline}{\texttt{=============================================================}}

\begin{document}

\separatorline

\vspace{1.5em}

Bilkul, bhai! \important{Hinglish mein, step-by-step, sari cheezein clear karta hoon}—devtools, Flipper, stack traces copy, UI inspect, breakpoint, sab. \important{Beginner’s guide}, practical examples, step-by-step, cheating sheet types. \important{Apna notes banane ke liye direct copy-paste kar sakte ho.}

\vspace{1.5em}

\section*{\color{headingblue}\LARGE\bfseries 1. Copying Errors/Stack Traces, Red Screen — Kya Hai, Kya Dekhna Hai?}

\begin{itemize}
\item \important{Red Screen (Error Screen):} Jab app me koi \important{JavaScript error} aata hai, toh app ki screen red ho jati hai, aur error message show hota hai. \important{Isme stack trace bhi dikhata hai}—kaunsi file, kaunsi line, kaun function error aaya hai.[1]
\item \important{Ismein kaam ka info—error message, file name, line number, function ka naam, etc.} hota hai.
\end{itemize}

\vspace{1.5em}

\subsection*{\color{headingblue}\large\bfseries A. Copy Karne Ka Kya Fayda?}

\begin{itemize}
\item \important{Stack trace} ya \important{error message} copy karke \important{team chat, GitHub issue, Google, ya support forums} me paste kar sakte ho, jisse log aapko help kar paayein.
\item \important{Agar screen ka screenshot lete ho, toh bina type kiye hui info mil jayegi.}
\end{itemize}

\vspace{1.5em}

\subsection*{\color{headingblue}\large\bfseries B. Kaise Copy Karein?}

\begin{itemize}
\item \important{Red Screen:}  
\important{Ek dum direct copy nahi ho sakta} (sorry, no copy button on device screen!).[2]  
\important{Kya karein?}  
\important{Terminal ya console} (jahan se app chala rahe ho, Metro bundler, ya VS Code terminal) me \important{error log show hota hai}. Usee command line me \important{select karo, copy karo, paste karo} (Ctrl+C, Ctrl+V ya right-click).[2]
\item \important{DevTools/Chrome/Flipper:}  
\important{React Native Debugger ya Chrome DevTools} me \important{Console} tab me error log show hota hai.  
\important{Error ke text ko select karo, right-click > Copy ya Ctrl+C} karo—paste karo jahan bhi chahiye.[3]
\item \important{Flipper:}  
\important{Logs tab} me bhi console errors show hote hain. \important{Select karo, copy karo, paste karo}.  
\important{Crash tab} me crash ka complete log milta hai, usee bhi copy kar sakte ho.
\end{itemize}

\important{Summary:}  
Red screen ya app pe error aa raha hai? \important{Terminal, console, ya Flipper/DevTools me select karo, copy karo, share karo.}

\vspace{1.5em}

\section*{\color{headingblue}\LARGE\bfseries 2. DevTools vs Flipper — Kab, Kisko, Kaise Use Karein?}

\begin{tabular}{|p{4cm}|p{5cm}|p{5cm}|p{6cm}|}
\hline
\textbf{Tool} & \textbf{Kya Dekh Sakte Ho?} & \textbf{Kab Use Karein?} & \textbf{Kaise Use Karein?} \\
\hline
\important{DevTools/Chrome} & \important{JavaScript, console, network, state/props} & \important{Sirf JavaScript/JSX ka bug find karna ho, React DevTools kaam chahiye ho} & \important{App run karo, device/emulator me “Debug JS Remotely” select karo, Chrome DevTools open hota hai} \\
\hline
\important{Flipper} & \important{Network, logs, layout, images, storage, crash, native logs, plugins (Redux, AsyncStorage, etc.)} & \important{Pura app debug karna hai—network, layout, crash, native logs, Redux, etc. all-in-one tool} & \important{Flipper desktop install karo, app debug mode me chalao, Flipper me connect kar lo} \\
\hline
\end{tabular}

\vspace{1em}

\begin{itemize}
\item \important{DevTools:}  
Sirf \important{JavaScript side} ka debug hai. \important{Console, network, component tree, state/props, breakpoints} (line-by-line) dekhna ho toh DevTools best hai.
\item \important{Flipper:}  
\important{Sab kuch dekhta hai}—JavaScript + Native logs, UI layout, network, storage, images, crash, Redux, AsyncStorage, etc. \important{Ek hi jagah sab kuch}!
\end{itemize}

\important{Use Case Samajho:}  
\begin{itemize}
\item \important{Ek line pe code chalate chalate error aaya, debug karna hai}—\important{DevTools}.
\item \important{App pe UI bahar jaa raha hai, network call fail ho raha hai, native crash ho raha hai—sab kuch dekhe, ek tool me}—\important{Flipper}.
\end{itemize}

\vspace{1.5em}

\section*{\color{headingblue}\LARGE\bfseries 3. UI Debugging (Inspector, Border Debug, Flipper Layout)}

\subsection*{\color{headingblue}\large\bfseries A. Show Inspector (UI Layout Debug)}

\begin{itemize}
\item Device/Emulator me \inlinecode{CTRL+M} (Android), \inlinecode{CMD+D} (iOS) dabao, “Show Inspector” select karo.
\item UI overlay aa jayega—components par tap karo, \important{props, style, dimensions} dekho.
\item Use karein jab:  
UI bigad raha hai, view ghis raha hai, component dikh nahi raha, margin-padding galat hai.
\end{itemize}

\subsection*{\color{headingblue}\large\bfseries B. Border Debug (UI Layout Debug)}

\begin{itemize}
\item App me kisi View ko yellow/green/red border dekhna hai, toh StyleSheet me \inlinecode{borderWidth: 1, borderColor: 'red'} daal do.
\item Debug ho jayega ki kaun sa View kaha tk hai.
\item Sab components ko alag alag screen areas me dekhne ke liye use karo.
\end{itemize}

\subsection*{\color{headingblue}\large\bfseries C. Flipper Layout (UI Layout Debug)}

\begin{itemize}
\item Flipper open karo, Layout tab open karo.
\item Component tree milta hai—kaun sa View, kaun se children, props, style, sab dikh jayega.
\item Select karo—props, state, style inspect kar sakte ho.
\item Example: Button ka height badhana hai—layout tab me select karo, height change karo, live UI change hoga.
\end{itemize}

\vspace{1.5em}

\section*{\color{headingblue}\LARGE\bfseries 4. Breakpoint — Kaise Lagate Hain React Native Me?}

\subsection*{\color{headingblue}\large\bfseries A. Step-by-Step (React Native Debugger/React DevTools/VS Code)}

\begin{itemize}
\item App ko debug mode me chalao:
\begin{itemize}
\item Terminal: \inlinecode{npx react-native run-android} ya \inlinecode{run-ios}
\item Device/Emulator me \inlinecode{CTRL+M} (Android), \inlinecode{CMD+D} (iOS), “Debug JS Remotely” select karo.
\end{itemize}
\item Ab aapke paas 3 options:
\begin{itemize}
\item Chrome DevTools: Open Chrome DevTools (https://localhost:8081/debugger-ui), Sources tab me file select karo, line number par click karo (breakpoint add ho jayega).
\item React Native Debugger: Desktop app, Sources tab me file open karo, line number par click karo (green dot aayega).
\item VS Code: VS Code me App.js ya koi file open karo, left gutter me click karo (red dot aayega), “Start Debugging” select karo.
\end{itemize}
\item Breakpoint hit hoga: Jaise hi woh line execute hone ko aayegi, code ruk jayega (pause ho jayega).
\item Ab aap variables ka value, call stack, sab dekh sakte ho, step-by-step chal sakte ho, value change kar sakte ho.
\end{itemize}

\subsection*{\color{headingblue}\large\bfseries B. Practical Example – Button Press pe Breakpoint}

\begin{tcolorbox}[sharp corners,boxrule=0.5pt,colback=codebg]
\begin{lstlisting}[language=jsx,numbers=left]
import React from 'react';
import {SafeAreaView, StyleSheet, Button, View, Text} from 'react-native';

function App() {
  const onButtonPress = () => {
    console.log('Button pressed!'); // Yahan breakpoint daalo!
  };

  return (
    <SafeAreaView style={styles.container}>
      <View style={styles.buttonContainer}>
        <Button title="Press me!" onPress={onButtonPress} />
        <Text>Check breakpoint on button press</Text>
      </View>
    </SafeAreaView>
  );
}

const styles = StyleSheet.create({
  container: {
    flex: 1,
    justifyContent: 'center',
    alignItems: 'center',
    backgroundColor: 'white',
  },
  buttonContainer: {
    padding: 16,
  },
});
\end{lstlisting}
\end{tcolorbox}

\textbf{Steps:}
\begin{enumerate}[itemsep=0.9em]
\item App ko debug mode me chalao.
\item VS Code/React Native Debugger/Chrome Sources tab me App.js open karo.
\item \inlinecode{onButtonPress} function ke console line ke pehle breakpoint lagao (line number par click karo—red dot aayega).
\item App me button dabao.
\item Breakpoint hit hoga, execution ruk jayega, aap step, watch, variables, etc. dekh sakte ho.
\end{enumerate}

\subsection*{\color{headingblue}\large\bfseries C. Agar Breakpoint Hit Nahi Ho Raha?}

\begin{itemize}
\item Remote Debugging enable karo?  
Device me \inlinecode{CTRL+M/CMD+D}, “Debug JS Remotely” select karo (refresh bhi kar lo).
\item Metro bundler chala hoon? (\inlinecode{npx react-native start})
\item Debugger App pe app connect ho gaya hai?  
(Chrome DevTools open hai? React Native Debugger open hai? VS Code debugger connect hoon?)
\item Breakpoint line execute ho raha hai?  
(Agar function call hi nahi ho raha, toh breakpoint hit nahi hoga.)
\item Agar Expo ho toh—debugging me thoda extra setup chahiye, wahan bhi DevTools/React Native Debugger chalta hai.
\end{itemize}

\vspace{1.5em}

\section*{\color{headingblue}\LARGE\bfseries 5. Summary Cheet Sheet (Hinglish)}

\begin{tabular}{|p{3.5cm}|p{3.5cm}|p{3cm}|p{3cm}|p{4.5cm}|}
\hline
\textbf{Problem} & \textbf{DevTools/Chrome} & \textbf{Flipper} & \textbf{Where to Copy?} & \textbf{Breakpoint?} & \textbf{UI Debug?} \\
\hline
JS error, Red Screen & Console me error dikhata hai & Logs tab me error dikhata hai & Terminal, console, ya Flipper/DevTools & Sources tab me line number par click karo & Inspector/Show Inspector (device me) \\
\hline
Network API fail & Network tab me dikh jata hai & Network tab me dikh jata hai & Select karo, copy karo, paste karo & — & — \\
\hline
UI layout bigad gaya & Nahi dekhta & Layout tab me dekho, props/style & — & — & Flipper Layout, Show Inspector, border debug \\
\hline
Crash ho gaya & Nahi dekhta & Crash tab me dikh jata hai & Copy karo, share karo & — & — \\
\hline
State/props kharab & State/props inspect karo & State/props inspect karo & — & JS line pe breakpoint lao & Flipper Layout/Show Inspector \\
\hline
\end{tabular}

\vspace{1.5em}

\section*{\color{headingblue}\LARGE\bfseries 6. Ek Dum Beginner ke Liye Final Point}

\begin{itemize}
\item Red screen/error?  
\important{Terminal, console, Flipper, ya DevTools} me error copy karo, share karo.
\item UI bigad raha hai?  
\important{Show Inspector, Flipper Layout tab, border debug} use karo.
\item Network fail ho raha hai?  
\important{Flipper/DevTools Network tab} me dekh lo request/response.
\item Crash ho gaya?  
\important{Flipper Crash tab} me stack trace dekh lo, crash reason copy karo.
\item Line-by-line debug chahiye?  
\important{Debug JS Remotely} chalao, \important{Chrome DevTools/React Native Debugger/VS Code} me \important{line par breakpoint daalo}.
\item Sab kuch ek hi jagah dekho?  
\important{Flipper} sabse best hai—network, layout, logs, crash, UI, sab mil jata hai.
\end{itemize}

\vspace{1.5em}

\section*{\color{headingblue}\LARGE\bfseries 7. Example – Full Debug Flow in Hinglish}

\textbf{Scenario:}  
Aapke app me “Submit” button dabane par API call fail ho raha hai, UI bhi kuch load na ho raha, aur crash bhi ho raha hai ek do baar.

\textbf{Kaam aapka:}  
\begin{enumerate}[itemsep=0.9em]
\item App debug mode me chalao.
\item Flipper open karo, Layout tab me UI check karo (kaun sa View chhupa hua hai).
\item Network tab me API ka request/response dekh lo—kya error aa raha hai, kya request aa raha hai.
\item Logs tab me error logs dekh lo—konsa exception aaya hai.
\item Crash tab me crash stack trace dekh lo, screenshot le lo.
\item Agar JavaScript logic me galti lag rahi hai, toh Chrome DevTools/React Native Debugger/VSCode me line par breakpoint lagao, step-by-step debug karo.
\item Sab errors/logs/crash traces/API calls copy kar lo, team ko bhej do.
\end{enumerate}

\important{Sabse basic, ek dum beginner ka flow follow karo, sab cheezein clear ho jayegi!}

\vspace{1.5em}

\important{Aaj se aapka koi bhi doubt debug, UI inspecting, breakpoint, red screen, etc. ka nahi rahega—sab kuch Hinglish, step-by-step, cheet sheet types, ek dum clear aur beginner ke liye dala gaya!}  
\important{Aap apne projects ko isi tarah debug karo, notes banate raho, confidence badhao!}  
\important{Agar aur kuch doubt ho, toh seedha poocho—mai bana ke dunga!}

\vspace{1.5em}

\separatorline

\vspace{1.5em}

\section*{\color{headingblue}\LARGE\bfseries adb reverse tcp:8081 tcp:8081 – Kya Hai, Kab Use Karein, Kyun?}

\subsection*{\color{headingblue}\large\bfseries Command: adb reverse tcp:8081 tcp:8081}

\important{adb reverse} ek Android Debug Bridge (ADB) command hai jo aapko \important{port forwarding} karne mein help karta hai.  
\important{adb reverse tcp:8081 tcp:8081} ka matlab hai:  
\important{Aapke laptop/pc ka port 8081} (jo Metro bundler/react dev server typically use karta hai), \important{usko aapke Android device ke port 8081 pe expose} karna—iska matlab device pe “localhost:8081” likhoge toh woh seedha aapke laptop ke react dev server pe connect ho jaayega, WiFi se alag.[1][2]

\vspace{1.5em}

\subsection*{\color{headingblue}\large\bfseries Kyun Use Karein?}

\important{React Native (Android) development me jab aap real device pe app chalate ho, metro bundler (react dev server) port 8081 pe expect karta hai ki app usko server ke taur pe connect karega.}  
Lekin, \important{Android device me “localhost” likhoge toh woh device khud ka localhost hi pehchanta hai}—React Native ka dev server chala hua hai aapke laptop pe, device pe nahi, isliye server nahi milta.

\important{adb reverse command se, Android device se “localhost:8081” pe request aaye, toh woh aapke laptop pe running react dev server pe forward ho jayega}—hot reload, fast refresh, error messages, sab chalne lagta hai \important{bina WiFi/internet connectivity ke}.

\important{Ye sirf USB se connected device pe hi kaam karta hai} (adb tool ka kaam hai device se USB ke through connect karna).

\vspace{1.5em}

\subsection*{\color{headingblue}\large\bfseries Kab Use Karein?}

\begin{itemize}
\item Real Android device (not emulator) pe React Native app test karna ho.
\item App ka development server (Metro bundler) laptop pe chala ho, device pe nahi.
\item App se network request (like bundle, error, reload, etc.) laptop ke dev server tak directly pahunchana ho.
\item WiFi pe trust nahi hai ya WiFi proxy/network issue hai.
\end{itemize}

\important{Ye command, pehle app ko start karne se pehle, ya metro bundler chalaane ke pehle, ya emulator/device connect karne ke baad chalao.}

\vspace{1.5em}

\subsection*{\color{headingblue}\large\bfseries Agar Nahi Karein Toh Kya Hoga?}

\begin{itemize}
\item App Android device pe chalege, lekin hot reload, fast refresh, error reporting, reloading sab kaam nahi karega.
\item App error ayega – “Could not connect to development server” (server available nahi mila).
\item Development karne me dhakke khane padenge, har baar code change karke app uninstall-install karna padega.
\end{itemize}

\vspace{1.5em}

\subsection*{\color{headingblue}\large\bfseries Step-by-Step Example: Kaise Karein?}

\begin{itemize}
\item Laptop pe terminal open karo.
\item Android device USB se connect karo.
\item ADB tool install ho chuka hona chahiye (Android Studio se ya standalone).
\item \inlinecode{adb devices} likho—device show ho raha hai tabhi kaam karega.
\item \inlinecode{adb reverse tcp:8081 tcp:8081} likho.
\item React Native app run karo, metro bundler chalao (\inlinecode{npx react-native start}).
\item App Android device pe chalao (\inlinecode{npx react-native run-android}).
\item Ab app me hot reload/reload/galti, sab show hoga, network request direct laptop ke server pe jayega.
\end{itemize}

\vspace{1.5em}

\separatorline

\vspace{1.5em}

\section*{\color{headingblue}\LARGE\bfseries Linux dig Command – Kya Hai, Kab Use Karein, Example}

\subsection*{\color{headingblue}\large\bfseries dig: Domain Information Groper}

\important{dig} ek command hai jo \important{DNS (Domain Name System) query} ke liye use hota hai.  
Isse aap \important{domain ka IP, MX record, NS record, ya kisi bhi DNS record} ka detail directly command line se nikal sakte ho.

\vspace{1.5em}

\subsection*{\color{headingblue}\large\bfseries Kyun Use Karein?}

\begin{itemize}
\item Agar website ka IP pata karna ho.
\item Mail server (MX record), DNS server (NS record), ya koi aur DNS record dekhna ho.
\item Network troubleshoot karna ho, DNS resolution thik hai ya nahi.
\end{itemize}

\vspace{1.5em}

\subsection*{\color{headingblue}\large\bfseries Basic Syntax}

\begin{tcolorbox}[sharp corners,boxrule=0.5pt,colback=codebg]
\begin{lstlisting}[language=bash,numbers=none]
dig example.com
\end{lstlisting}
\end{tcolorbox}

\important{example.com ki jagah apna domain daalo.}

\vspace{1.5em}

\subsection*{\color{headingblue}\large\bfseries Examples with Output: Simple, A Record, MX Record}

\subsubsection*{\color{headingblue}\normalsize 1. Simple Example (A Record)}

\begin{tcolorbox}[sharp corners,boxrule=0.5pt,colback=codebg]
\begin{lstlisting}[language=bash,numbers=none]
dig google.com
\end{lstlisting}
\end{tcolorbox}

\textbf{Output:}

\begin{verbatim}
; <<>> DiG 9.16.1-Ubuntu <<>> google.com
;; ->>HEADER<<- opcode: QUERY, status: NOERROR, id: 12345
;; flags: qr rd ra; QUERY: 1, ANSWER: 1
;; ANSWER SECTION:
google.com.     300 IN  A   142.250.192.46
\end{verbatim}

\important{Yahan google.com ka IP address nikal jayega.}[3]

\vspace{1.5em}

\subsubsection*{\color{headingblue}\normalsize 2. MX Record (Mail Exchange Server)}

\begin{tcolorbox}[sharp corners,boxrule=0.5pt,colback=codebg]
\begin{lstlisting}[language=bash,numbers=none]
dig MX google.com
\end{lstlisting}
\end{tcolorbox}

\textbf{Output:}

\begin{verbatim}
google.com.     300 IN  MX  20 smtp.google.com.
\end{verbatim}

\important{Google ke mail server ka record niklega.}

\vspace{1.5em}

\subsubsection*{\color{headingblue}\normalsize 3. Specific DNS Server se Query}

\begin{tcolorbox}[sharp corners,boxrule=0.5pt,colback=codebg]
\begin{lstlisting}[language=bash,numbers=none]
dig @8.8.8.8 google.com
\end{lstlisting}
\end{tcolorbox}

\important{Yaha Google ke public DNS (8.8.8.8) se Google ka record puchenge.}[4]

\vspace{1.5em}

\subsection*{\color{headingblue}\large\bfseries Kab Use Karein?}

\begin{itemize}
\item Domain ka IP pata karna ho.
\item Mail server, DNS server record check karna ho.
\item DNS resolution me galti ho rahi ho, toh direct dig se check karo, browser pe toggal naa karo.
\item Network admin ho, ya app developer ho, toh ye command kaafi kaam aata hai.
\end{itemize}

\vspace{1.5em}

\subsection*{\color{headingblue}\large\bfseries Agar Nahi Chalanoge Toh Kya Hoga?}

\begin{itemize}
\item Domain ka IP ya DNS record manually pata karna padega, browser pe site khol kar.
\item Network/galti troubleshoot karna ho toh naseeb nahi, ping se sirf IP milta hai, detail nahi.
\item Professional/detailed troubleshooting me help nahi milta.
\end{itemize}

\vspace{1.5em}

\section*{\color{headingblue}\LARGE\bfseries Summary Table – Cheat Sheet}

\begin{tabular}{|p{5cm}|p{5cm}|p{6cm}|p{6cm}|}
\hline
\textbf{Command} & \textbf{Kya Hai?} & \textbf{Kab Use Karein?} & \textbf{If Not Used, What?} \\
\hline
adb reverse tcp:8081 tcp:8081 & Android device port forwarding & Real device pe React Native dev, metro se connect nahi ho & Hot reload, error, fast refresh, reload will not work \\
dig example.com & DNS lookup, IP pata karna & Domain ka IP, MX, NS, etc. record nikalna ho & Manual IP check, DNS troubleshooting tough \\
\hline
\end{tabular}

\vspace{1.5em}

\separatorline

=============================================================

\documentclass[a4paper,12pt]{article}

% Geometry with comfortable margins
\usepackage[left=25mm,right=25mm,top=25mm,bottom=25mm]{geometry}

% Fonts
\usepackage{tgtermes} % TeX Gyre Termes (Times-like serif for body)
\usepackage[T1]{fontenc}
\usepackage[utf8]{inputenc}
\usepackage{sectsty} % for section title font control

% Color definitions
\usepackage{xcolor}
\definecolor{headingblue}{RGB}{0,70,132} % strong blue for headings
\definecolor{importantred}{RGB}{204,0,0} % bright red for important text
\definecolor{codebg}{RGB}{245,245,245} % very light gray background for code
\definecolor{keywordcolor}{RGB}{0,0,180} % dark blue for keywords
\definecolor{stringcolor}{RGB}{163,21,21} % dark red for strings
\definecolor{commentcolor}{RGB}{0,128,0} % green for comments
\definecolor{functioncolor}{RGB}{127,0,85} % purple for function names
\definecolor{linenumbercolor}{gray}{0.6} % subtle gray for line numbers

% Listings for code blocks with multi-color syntax highlighting
\usepackage{listings}
\lstdefinelanguage{jsx}{
  keywords={import,function,let,const,var,if,else,return,for,while,do,switch,case,break,continue,class,new,try,catch,finally,throw,async,await,void,static,public,private,protected,yield,default},
  sensitive=true,
  morecomment=[l]{//},
  morecomment=[s]{/*}{*/},
  morestring=[b]",
  morestring=[b]',
}
\lstset{
  language=jsx,
  backgroundcolor=\color{codebg},
  keywordstyle=\color{keywordcolor}\bfseries,
  stringstyle=\color{stringcolor},
  commentstyle=\color{commentcolor}\itshape,
  identifierstyle=\color{black},
  basicstyle=\ttfamily\footnotesize,
  numbers=left,
  numberstyle=\color{linenumbercolor}\scriptsize,
  stepnumber=1,
  numbersep=8pt,
  frame=none,
  tabsize=4,
  showstringspaces=false,
  breaklines=true,
  breakatwhitespace=true,
  captionpos=b,
  xleftmargin=10pt,
  xrightmargin=10pt,
  framexleftmargin=5pt,
  framexrightmargin=5pt,
  rulecolor=\color{codebg},
  upquote=true,
}

% tcolorbox for code blocks with rounded corners and light background
\usepackage[many]{tcolorbox}
\tcbset{
  colback=codebg,
  colframe=codebg,
  boxrule=0pt,
  arc=3mm,
  auto outer arc,
  leftrule=0pt,
  rightrule=0pt,
  toprule=0pt,
  bottomrule=0pt,
  left=6pt,
  right=6pt,
  top=6pt,
  bottom=6pt,
  enhanced,
}

% Section title formatting with colors and size
\usepackage{titlesec}
\titleformat{\section}
  {\color{headingblue}\LARGE\bfseries}
  {}
  {0pt}
  {}
\titlespacing*{\section}{0pt}{4ex plus 1ex minus .2ex}{1.5ex plus .2ex}

% Header and footer with fancyhdr
\usepackage{fancyhdr}
\pagestyle{fancy}
\fancyhf{}
\fancyhead[L]{React Native Notes}
\fancyhead[R]{\thepage}
\renewcommand{\headrulewidth}{0.4pt}
\renewcommand{\footrulewidth}{0pt}

% Lists with enumitem for spacing control
\usepackage{enumitem}
\setlist{itemsep=0.9em, parsep=0pt, topsep=0.9em}

% Hyperlinks without colors
\usepackage[hidelinks]{hyperref}

% Custom command for important red text
\newcommand{\important}[1]{\textcolor{importantred}{#1}}

% Preserve monospace inline code styling
\newcommand{\inlinecode}[1]{\texttt{#1}}

% To preserve the exact separator line visibly
\newcommand{\separatorline}{\texttt{=============================================================}}

\begin{document}

\separatorline

\vspace{1.5em}

\section*{\color{headingblue}\LARGE\bfseries Section: Networking, State, App Behavior —}

\important{Fast Refresh vs Hot Reloading, Networking/Axios/Fetch, Offline Handling, SSL Pinning —}  

\important{Sab Kuch Beginner Friendly, Hinglish, Step-by-Step, With Simple Examples}

\vspace{1.5em}

\section*{\color{headingblue}\LARGE\bfseries Fast Refresh vs Hot Reloading: Difference \& Use Cases}

\subsection*{\color{headingblue}\large\bfseries Fast Refresh:}
\begin{itemize}
\item \important{Kya hai:}  
\important{Fast Refresh} React Native ka modern feature hai.  
Jab aap \important{component} (like \inlinecode{App.js}, \inlinecode{Button.js}, etc.) me \important{code change} karte ho,  
\important{React Native automatically uss component ko update bina poore app ko refresh kiye dikhata hai}  
(app ka state bhi raheta hai—jaise form me jo likha hai, woh chala jayega bina refresh ke).[1][2]
\item \important{Kab use karein:}  
Small UI changes, state preservation chahiye ho (jaise form filling, progress track, etc.),  
\important{Fast Refresh} use karo. \important{Yeh development fast, smooth, aur productive bana deta hai—bas code change karo, instantly screen update ho jayega}.[3][1]
\item \important{Agar Fast Refresh chalana hai toh:}  
App ko “Debug JS Remotely” karne ke liye kya bhi band mat karo.  
\important{Jab bhi code change karo, app khud update ho jayega—no need to shake, no explicit reload.}
\item \important{Agar nahi chalao toh:}  
Every time code change karo, \important{app poora restart hoga}, \important{state loss hoga} (jo aapne demo me log in kiya tha, form bhara tha, woh sab ghis jayega).[4][5]
\end{itemize}

\subsection*{\color{headingblue}\large\bfseries Hot Reloading:}
\begin{itemize}
\item \important{Kya hai:}  
\important{Hot Reloading} bhi React Native ka purana method tha.  
\important{Isme jab bhi code change karo, poora app reload hota tha—state bhi chala jata tha}  
(jaise browser refresh jaisa).[5][4]
\item \important{Kab use karein:}  
\important{Nowadays, React Native Fast Refresh pe mature ho gaya hai—Hot Reloading ki jarurat nahi.}
\item \important{Hot Reloading chalana ho, toh:}  
\important{Metro bundler manual restart karo (\inlinecode{npx react-native start}), ya device me “Reload” dabao.}
\item \important{Agar Hot Reloading chal na mil raha toh:}  
App poora restart hoga har bar—\important{time zyada lagega, productivity affect hogi}.
\end{itemize}

\vspace{1.5em}

\section*{\color{headingblue}\LARGE\bfseries Summary Table: Fast Refresh vs Hot Reloading}

\begin{tabular}{|p{5.3cm}|p{5.1cm}|p{5cm}|}
\hline
\textbf{Feature} & \textbf{Fast Refresh} & \textbf{Hot Reloading} \\
\hline
\important{State Preserved?} & Haan (form, progress sab rahta hai) & Nahi (state reset ho jata hai) \\
\hline
\important{App Reload?} & Nahi (sirf updated part refresh ho jata hai) & Haan (poore app ka restart hota hai) \\
\hline
\important{Best For} & Development, UI tweaking, debugging state & Not recommended now, old way \\
\hline
\important{How to Use} & Just change code, see live update & Reload app, shake device, or reload \\
\hline
\end{tabular}

\separatorline

\vspace{1.5em}

\section*{\color{headingblue}\LARGE\bfseries Networking \& APIs, Offline Handling, SSL Pinning – Hinglish Mein, Step by Step, With Examples}

\vspace{1.5em}

\section*{\color{headingblue}\LARGE\bfseries Axios \& Fetch – Kya Hai, Kab/Kyun Use Karein?}

\subsection*{\color{headingblue}\large\bfseries Axios/Fetch Basics}

\important{Axios/Fetch} dono \important{API call karne ke liye use hote hain}—apna app backend ke sath communicate kar sakta hai, data le sakta hai, bhej sakta hai.  
\important{Axios thoda zyada flexible, features bahut (interceptors, retry, easy error handling)}  
\important{Fetch} (built-in JavaScript utility), \important{basics ke liye bada simple hai.}  
\important{Axios} recommend kiya jata hai kyuki \important{interceptors, retry, easy config} sab kuch milta hai.

\vspace{1.5em}

\section*{\color{headingblue}\LARGE\bfseries Interceptors – Kya Hai, Kyun/Kab, Kaise?}

\subsection*{\color{headingblue}\large\bfseries Interceptors Kya Hai?}

\important{Interceptors} ek \important{middleware} ki tarah kaam karte hain, jo \important{API call se pehle ya baad mein automatically code chalata hai}.  
\begin{itemize}
\item \important{Request Interceptor:} API call se pehle chalega (jaise auth token lagana, headers modify karna).
\item \important{Response Interceptor:} API se response milne ke baad chalega (jaise global error handle karna, token expire hone par logout karna).
\end{itemize}

\important{Kab/Kyun Use Karein?}  
\begin{itemize}
\item Globally token daalna ho har request me.
\item Global error handling chahiye ho.
\item Server se aaya response change karna ho.
\item JWT token expire hone par automatically naya token mangwana ho.
\end{itemize}

\important{Agar Nahi Use Karein Toh?}  
\begin{itemize}
\item Har API call me manually token/headers daalna padega.
\item Response/error handling har jagah likhni hogi (duplication).
\item JWT token expire ho gaya toh har jagah manually login pe redirect karna padega.
\end{itemize}

\subsection*{\color{headingblue}\large\bfseries Example (Axios Interceptor)}

\begin{tcolorbox}[sharp corners,boxrule=0.5pt,colback=codebg]
\begin{lstlisting}[language=jsx]
import axios from 'axios';

const api = axios.create({
  baseURL: 'https://api.example.com',
});

// Request Interceptor: Har request se pehle token header me daal do
api.interceptors.request.use(
  (config) => {
    const token = getAuthToken(); // Fetch token from storage
    if (token) config.headers.Authorization = `Bearer ${token}`;
    return config;
  },
  (error) => {
    return Promise.reject(error);
  }
);

// Response Interceptor: Error handle karo, token refresh karo
api.interceptors.response.use(
  (response) => response.data, // Success route ka data direct de do
  async (error) => {
    if (error.response?.status === 401) {
      // Token expired? Refresh karo ya login pe redirect karo
      logoutOrRefreshToken();
    }
    return Promise.reject(error);
  }
);

export default api;
\end{lstlisting}
\end{tcolorbox}

Bina interceptor ke code repeatedly likhna padta hai, code badalne me bhi pareshani hoti hai.

\vspace{1.5em}

\section*{\color{headingblue}\LARGE\bfseries Retry Mechanism – Kya Hai, Kyun/Kab, Kaise?}

\subsection*{\color{headingblue}\large\bfseries Retry Kya Hai?}

\important{API fail hua toh ek do baar wait karo, phir dobara try karo.}  
\important{Network slow ho, timeout ho, server issue ho toh retry se app seamless rahega, user ko error nahi dikhega.}

\important{Kab/Kyun Use Karein?}  
\begin{itemize}
\item Network glitch, timeout, server error ho toh retry karo.
\item Mission critical API call ho (payment, important data fetch).
\item Aapko user experience smooth rakhna hai, app auto-recover kare.
\end{itemize}

\important{Agar Nahi Use Karein Toh?}  
\begin{itemize}
\item User ko manual retry karne ko bolna padega.
\item App jyada crash/error prone lagega.
\end{itemize}

\subsection*{\color{headingblue}\large\bfseries Example (Axios-retry Library)}

\begin{tcolorbox}[sharp corners,boxrule=0.5pt,colback=codebg]
\begin{lstlisting}[language=jsx]
import axios from 'axios';
import axiosRetry from 'axios-retry';

axiosRetry(axios, {
  retries: 3,
  retryDelay: axiosRetry.exponentialDelay,
  retryCondition: (error) => {
    // Network error ya 5XX server error ho toh retry karo
    return (
      axiosRetry.isNetworkOrIdempotentRequestError(error) ||
      error.response?.status >= 500
    );
  },
});

axios.get('https://api.example.com/data').then(res => console.log(res));
\end{lstlisting}
\end{tcolorbox}

Every request jo config me set kiye ho, usko ek ya jyada baar retry karega.

\vspace{1.5em}

\section*{\color{headingblue}\LARGE\bfseries Offline Handling – NetInfo, Caching – Kya Hai, Kyun/Kab, Kaise?}

\subsection*{\color{headingblue}\large\bfseries NetInfo – Kya Hai, Kyun/Kab}

\important{NetInfo} ek library hai, jo \important{network connectivity check} karta hai—online, offline, 2G, 3G, 4G, jo bhi ho.  

\important{Kab Use Karein?}  
\begin{itemize}
\item App me offline/online status dikhana ho.
\item API call se pehle check karna ho device online hai ya nahi.
\item Offline ho toh user ko message dikhao, online hua toh sync karo.
\end{itemize}

\important{Agar Nahi Use Karein Toh?}  
\begin{itemize}
\item User ko nahi pata chalega internet chala gaya hai.
\item API calls fail hogi, errors aayengi, user experience kharab hoga.
\end{itemize}

\subsection*{\color{headingblue}\large\bfseries Example (NetInfo Usage)}

\begin{tcolorbox}[sharp corners,boxrule=0.5pt,colback=codebg]
\begin{lstlisting}[language=jsx]
import NetInfo from '@react-native-community/netinfo';

// Real-time network status check
const unsubscribe = NetInfo.addEventListener(state => {
  console.log('Connection type', state.type);
  console.log('Is connected?', state.isConnected);
});

// API call se pehle check karo
const checkNetwork = async () => {
  const state = await NetInfo.fetch();
  if (state.isConnected) {
    // Online: API call karo
    fetchData();
  } else {
    // Offline: Cache me data dikhao ya offline message dikhao
    showOfflineMessage();
  }
};
\end{lstlisting}
\end{tcolorbox}

\subsection*{\color{headingblue}\large\bfseries Caching – Kya Hai, Kyun/Kab}

\important{Caching} matlab \important{API se latest data leke device pe save karo}, jisse offline time pe local data show kar sako.  

\important{Kab Use Karein?}  
\begin{itemize}
\item App offline chalna chahiye, ya slow connection pe
\item Bache hue data dikhana ho (news, articles, etc.)
\item Redux, AsyncStorage, Realm, SQLite, ya MMKV jaise libraries se cache karo.
\end{itemize}

\important{Agar Nahi Use Karein Toh?}  
\begin{itemize}
\item Offline pe kuch nahi dikhega.
\item User experience kharab hoga.
\end{itemize}

\subsection*{\color{headingblue}\large\bfseries Example (Caching Data)}

\begin{tcolorbox}[sharp corners,boxrule=0.5pt,colback=codebg]
\begin{lstlisting}[language=jsx]
import AsyncStorage from '@react-native-async-storage/async-storage';

const cacheKey = 'users-data';

const getCachedData = async () => {
  const cached = await AsyncStorage.getItem(cacheKey);
  if (cached) return JSON.parse(cached);
  return null;
};

const fetchAndCacheData = async () => {
  try {
    const data = await api.get('/users');
    await AsyncStorage.setItem(cacheKey, JSON.stringify(data));
    return data;
  } catch (error) {
    const cached = await getCachedData();
    if (cached) {
      console.log('Showing cached data');
      return cached; // Use if no internet
    }
    throw error;
  }
};
\end{lstlisting}
\end{tcolorbox}

\vspace{1.5em}

\section*{\color{headingblue}\LARGE\bfseries SSL Pinning Basics – Kya Hai, Kyun/Kab, Kaise?}

\subsection*{\color{headingblue}\large\bfseries SSL Pinning Kya Hai?}

\important{SSL Pinning} app ko \important{sirf apne server ke certificate ya public key} pe accept karne ko force karta hai.  
\important{Man-in-the-middle attacks} (hacker apne device pe fake server banayega, data intercept karega) se bachata hai.

\important{Kab/Kyun Use Karein?}  
\begin{itemize}
\item App me security high chahiye ho (banking, payment, enterprise apps).
\item Public WiFi, shared networks, kisi bhi risky condition me security chahiye ho.
\end{itemize}

\important{Agar Nahi Use Karein Toh?}  
\begin{itemize}
\item Anyone can intercept communication between app and server.
\item User data at risk.
\end{itemize}

\subsection*{\color{headingblue}\large\bfseries How to Implement (Library Example: react-native-ssl-pinning)}

\textbf{Step 1: Add library}  
\begin{tcolorbox}[sharp corners,boxrule=0.5pt,colback=codebg]
\begin{lstlisting}[language=bash,numbers=none]
yarn add react-native-ssl-pinning
\end{lstlisting}
\end{tcolorbox}

\textbf{Step 2: Configure}  
\begin{tcolorbox}[sharp corners,boxrule=0.5pt,colback=codebg]
\begin{lstlisting}[language=jsx]
import { initialize } from 'react-native-ssl-pinning';

initialize({
  yourdomain.com: {
    includeSubdomains: true,
    publicKeyHashes: [
      'SHA1_OR_SHA256_OF_YOUR_CERT', // Replace with actual key
      'BACKUP_KEY' // Optional backup key
    ]
  }
});
\end{lstlisting}
\end{tcolorbox}

Isse app sirf pin kiye hue certificate pe hi server se baat karega.

\vspace{1.5em}

\section*{\color{headingblue}\LARGE\bfseries Summary Table – Sab Kuch Ek Sath}

\begin{tabular}{|p{3.4cm}|p{5.2cm}|p{4.5cm}|p{4.7cm}|p{3.5cm}|}
\hline
\textbf{Concept} & \textbf{Kya Hai} & \textbf{Kab Use Karein} & \textbf{Agar Nahi Use Karein Toh?} & \textbf{Example} \\
\hline
Axios/Fetch & API call karne ke liye & Data fetch, send, update, delete & Manual HTTP calls, boilerplate code & \inlinecode{axios.get('/')} , \inlinecode{fetch('/')} \\
\hline
Interceptors & Global pre/post API hook & Token, error, response handling & Manual handling, code repeat, maintenance tough & See above \\
\hline
Retry & Failed call ko dobara try karo & Network/server issues pe seamless experience & User ko manual retry karna padega & See above \\
\hline
NetInfo & Network connectivity check & Offline/online msg, API call se pehle check & User ko error dikhega, experience kharab & See above \\
\hline
Caching & Data local me save karo & Offline pe bhi data dikhao, performance boost & Offline pe kuch nahi dikhega & \inlinecode{AsyncStorage}, \inlinecode{MMKV} \\
\hline
SSL Pinning & Secure connection (trusted cert only) & High security apps & MITM attacks, security breach & See above \\
\hline
\end{tabular}

\vspace{1.5em}

\section*{\color{headingblue}\LARGE\bfseries Final Cheat Sheet – Ek Dum Basic}

\begin{itemize}
\item \important{Axios/Fetch} – Server se data bhejna/lena.
\item \important{Interceptors} – Global token, error, response handling.
\item \important{Retry} – API fail ho toh auto-retry karo, user ko error na dikhao.
\item \important{NetInfo} – Online/offline pata chalao, offline pe message dikhao.
\item \important{Caching} – Offline pe puraana data dikhao, user experience fast banaye.
\item \important{SSL Pinning} – High security, sirf trusted server se connect karo.
\end{itemize}



=============================================================

\documentclass[a4paper,12pt]{article}

% Geometry with comfortable margins
\usepackage[left=25mm,right=25mm,top=25mm,bottom=25mm]{geometry}

% Fonts
\usepackage{tgtermes} % TeX Gyre Termes (Times-like serif for body)
\usepackage[T1]{fontenc}
\usepackage[utf8]{inputenc}
\usepackage{sectsty} % for section title font control

% Color definitions
\usepackage{xcolor}
\definecolor{headingblue}{RGB}{0,70,132} % strong blue for headings
\definecolor{importantred}{RGB}{204,0,0} % bright red for important text
\definecolor{codebg}{RGB}{245,245,245} % very light gray background for code
\definecolor{keywordcolor}{RGB}{0,0,180} % dark blue for keywords
\definecolor{stringcolor}{RGB}{163,21,21} % dark red for strings
\definecolor{commentcolor}{RGB}{0,128,0} % green for comments
\definecolor{functioncolor}{RGB}{127,0,85} % purple for function names
\definecolor{linenumbercolor}{gray}{0.6} % subtle gray for line numbers

% Listings for code blocks with multi-color syntax highlighting
\usepackage{listings}
\lstdefinelanguage{jsx}{
  keywords={import,function,let,const,var,if,else,return,for,while,do,switch,case,break,continue,class,new,try,catch,finally,throw,async,await,void,static,public,private,protected,yield,default},
  sensitive=true,
  morecomment=[l]{//},
  morecomment=[s]{/*}{*/},
  morestring=[b]",
  morestring=[b]',
}
\lstset{
  language=jsx,
  backgroundcolor=\color{codebg},
  keywordstyle=\color{keywordcolor}\bfseries,
  stringstyle=\color{stringcolor},
  commentstyle=\color{commentcolor}\itshape,
  identifierstyle=\color{black},
  basicstyle=\ttfamily\footnotesize,
  numbers=left,
  numberstyle=\color{linenumbercolor}\scriptsize,
  stepnumber=1,
  numbersep=8pt,
  frame=none,
  tabsize=4,
  showstringspaces=false,
  breaklines=true,
  breakatwhitespace=true,
  captionpos=b,
  xleftmargin=10pt,
  xrightmargin=10pt,
  framexleftmargin=5pt,
  framexrightmargin=5pt,
  rulecolor=\color{codebg},
  upquote=true,
}

% tcolorbox for code blocks with rounded corners and light background
\usepackage[many]{tcolorbox}
\tcbset{
  colback=codebg,
  colframe=codebg,
  boxrule=0pt,
  arc=3mm,
  auto outer arc,
  leftrule=0pt,
  rightrule=0pt,
  toprule=0pt,
  bottomrule=0pt,
  left=6pt,
  right=6pt,
  top=6pt,
  bottom=6pt,
  enhanced,
}

% Section title formatting with colors and size
\usepackage{titlesec}
\titleformat{\section}
  {\color{headingblue}\LARGE\bfseries}
  {}
  {0pt}
  {}
\titlespacing*{\section}{0pt}{4ex plus 1ex minus .2ex}{1.5ex plus .2ex}

% Header and footer with fancyhdr
\usepackage{fancyhdr}
\pagestyle{fancy}
\fancyhf{}
\fancyhead[L]{React Native Notes}
\fancyhead[R]{\thepage}
\renewcommand{\headrulewidth}{0.4pt}
\renewcommand{\footrulewidth}{0pt}

% Lists with enumitem for spacing control
\usepackage{enumitem}
\setlist{itemsep=0.9em, parsep=0pt, topsep=0.9em}

% Hyperlinks without colors
\usepackage[hidelinks]{hyperref}

% Custom command for important red text
\newcommand{\important}[1]{\textcolor{importantred}{#1}}

% Preserve monospace inline code styling
\newcommand{\inlinecode}[1]{\texttt{#1}}

% To preserve the exact separator line visibly
\newcommand{\separatorline}{\texttt{=============================================================}}

\begin{document}

\separatorline

\vspace{1.5em}

\section*{\color{headingblue}\LARGE\bfseries SSL Pinning Basics – Ek Dum Beginner Level, Hinglish Mein, Sab Doubts Clear}

\vspace{1.5em}

\section*{\color{headingblue}\LARGE\bfseries SSL Pinning Kya Hai?}

\important{SSL Pinning} ek security technique hai, jisme app \important{apne server ka SSL certificate ya public key} apne app me “pin” (hardcode ya store) karta hai. Jab bhi app server se connect karega, server ka certificate check hoga – \important{agar woh pin kiye hue certificate/key se match nahi hua, toh app connection hi reject kar dega}.  
Ye \important{man-in-the-middle attack} (hacker jo fake server bana kar user ka data intercept kare) se protect karta hai.[1][2]

\vspace{1.5em}

\section*{\color{headingblue}\LARGE\bfseries Actual Key/Pin Kahan Se Milegi?}

\begin{itemize}
\item Domain ka SSL certificate extract karna hai.
\item Website SSL Labs SSL Test (ssllabs.com) par jaakar, apne domain ko test karo – result me certificate ka “Pin SHA256” mil jayega. Ye public key ka hash (hash matlab, ek unique string jo certificate se nikala gaya hai).[1]
\item Actual pin format:  
Example: \inlinecode{sha256/ABC123xyz...}
\item Backup key:  
Sabhi sites do ya zyada intermediate certificates use karte hain. Un dono ka SHA256 hash extract karo – ek primary, ek backup pin banate ho.[1]
\item Ek link: \href{https://www.ssllabs.com/ssltest/analyze.html}{SSL Labs SSL Test} – yahan apne domain ka certificate details aur SHA256 pin mil jayega.
\end{itemize}

\vspace{1.5em}

\section*{\color{headingblue}\LARGE\bfseries Kaun Si File Me Likhen?}

\begin{itemize}
\item JavaScript/TypeScript file me hi – sabse easy hai react-native-ssl-public-key-pinning jaise library use karna.
\item Ek alag file banate hain (example: \inlinecode{sslPinning.js} ya \inlinecode{sslPinning.ts}), jisme configuration rakhte hain.
\item Environment variables me daalo:  
Pin ko seedha code me hardcode na karo, react-native-config jaise library me \inlinecode{.env} file me likho, security ke liye.[1]
\item React Native ke main entry file (App.js ya main.ts) me, sabse pehle SSL pinning initialize karo, taaki app shuru hoti hi pinning chalu ho.
\end{itemize}

\vspace{1.5em}

\section*{\color{headingblue}\LARGE\bfseries Example – Actual Code Kaisa Hoga?}

\begin{tcolorbox}[sharp corners,boxrule=0.5pt,colback=codebg]
\begin{lstlisting}[language=jsx]
import { initializeSslPinning } from 'react-native-ssl-public-key-pinning';

const initializePin = async () => {
  // Primary pin (SSLLabs se SHA256 key)
  const primaryPin = 'sha256/ABC123xyz...';

  // Backup pin (dusre intermediate certificate ka SHA256 key)
  const backupPin = 'sha256/DEF456abc...';

  const config = {
    'yourdomain.com': {
      includeSubdomains: true,  // Subdomains pe bhi pinning apply hoga
      publicKeyHashes: [primaryPin, backupPin], // Dono pins daalo
      expirationDate: '2026-01-01', // Certificate expiry date (optional)
    },
  };

  await initializeSslPinning(config);
  console.log('SSL Pinning initialized');
};

export { initializePin };
\end{lstlisting}
\end{tcolorbox}

App.js me use karo:
\begin{tcolorbox}[sharp corners,boxrule=0.5pt,colback=codebg]
\begin{lstlisting}[language=jsx]
import { initializePin } from './sslPinning';
// App ke sabse pehle runtime me initialize karo
initializePin().catch(e => console.error(e));
\end{lstlisting}
\end{tcolorbox}

\vspace{1.5em}

\section*{\color{headingblue}\LARGE\bfseries Backup Key Ka Use Kya Hai?}

\begin{itemize}
\item Backup key/pin hoti hai, taaki agar ek certificate expire ho gaya ya renew ho gaya, toh dusra certificate (backup pin wala) valid ho, app kaam karta rahe.
\item Don’t use only one pin – agar expire ho gaya ya hacker ne intercept kar liya toh app chutki bhi nahi bajayega.
\item Always provide at least two pins (one primary, one backup).[2][1]
\end{itemize}

\vspace{1.5em}

\section*{\color{headingblue}\LARGE\bfseries Kaisa Error Aayega Agar Pinning Fail Ho Gayi?}

\begin{itemize}
\item App api calls nahi kar payega.
\item Error dikhayega – server connect ho jaayega, lekin app request block kar dega (Error: SSL pinning failed).
\item User ko message dikh sakta hai: “Server security issue, please update app.”
\item Logs me bhi error dikhega – “Certificate pin check failed.”
\end{itemize}

\vspace{1.5em}

\section*{\color{headingblue}\LARGE\bfseries Native Code Me Kaise Hota Hai?}

\begin{itemize}
\item Android:  
\inlinecode{res/raw} folder me \inlinecode{.cer} files (certificates) daalo, code me reference do, ya public key hash daalo.
\item iOS:  
\inlinecode{TrustKit} ya similar library, \inlinecode{AppDelegate.m} me configuration dalo, certificate ya public key hash provide karo.
\item React Native libraries (like \inlinecode{react-native-ssl-public-key-pinning}) use karte ho toh native configuration nahi karna padta, sab JavaScript me hota hai.[3][1]
\end{itemize}

\vspace{1.5em}

\section*{\color{headingblue}\LARGE\bfseries Practical Steps – Ek Dum Basic}

\begin{enumerate}
\item SSLLabs.com pe jao, domain daalo.
\item PIN SHA256 nikalo (main certificate aur intermediate certificate ka).
\item React Native library install karo:  
\inlinecode{yarn add react-native-ssl-public-key-pinning}
\item Ek file banao (\inlinecode{sslPinning.js}), config daalo (upar example dekho).
\item App.js me sabse pehle initialize karo.
\item API calls test karo – SSL pinning kam karegi, invalid certificate par reject hogi.
\end{enumerate}

\vspace{1.5em}

\section*{\color{headingblue}\LARGE\bfseries Agar Nahi Use Karein Toh Kya Hoga?}

\begin{itemize}
\item Man-in-the-middle attack ka risk.
\item App kisi bhi server se connect ho sakta hai (same domain name, different certificate? Chal jayega!) – security breach!
\item Enterprise, payment, healthcare, banking apps me SSL pinning zaruri hai.
\end{itemize}

\vspace{1.5em}

\section*{\color{headingblue}\LARGE\bfseries Summary Cheat Sheet (Ek Dum Basic)}

\begin{tabular}{|p{3.6cm}|p{5.3cm}|p{5cm}|p{4cm}|p{3cm}|p{4.3cm}|}
\hline
\textbf{Concept} & \textbf{Kya Hai} & \textbf{Actual Key Kahan Se Milega} & \textbf{Kaun Si File Me?} & \textbf{Backup Key?} & \textbf{Agar Nahi Use Karein Toh?} \\
\hline
SSL Pinning & Server ka cert/app me pin karo & SSLLabs.com, domain ka PIN SHA256 dekho & JavaScript file/App.js & Haan, zaruri hai & MITM attack, security risk \\
\hline
Library & \inlinecode{react-native-ssl-public-key-pinning} & — & — & — & Native code, certificates, hassle \\
\hline
Native & Android/iOS ke trust manager use karo & SSLLabs se key nikalo, \inlinecode{.cer} file banake raw me daalo (Android), iOS me TrustKit & Native code folder/file & Haan, zaruri hai & MITM attack, security risk \\
\hline
\end{tabular}

\vspace{1.5em}

\section*{\color{headingblue}\LARGE\bfseries Ek Dum Beginner Level Example – Simple Steps}

\begin{enumerate}
\item Go to ssllabs.com
\item Enter your domain, get PIN SHA256 (main \& backup)
\item Add library  
\inlinecode{yarn add react-native-ssl-public-key-pinning}
\item Make a config file  
\inlinecode{sslPinning.js} (see above)
\item Initialize in App.js  
\inlinecode{initializePin().catch(e => console.error(e));}
\item Test app – pinning work karegi, invalid certificate par app api call block karegi.
\end{enumerate}

\vspace{1.5em}

\textbf{Agar abhi bhi koi doubt ho, toh apna domain batao, main SSLLabs ka screenshot ya steps detail me dikha dunga, ya code bana ke dunga!}  
\textbf{Ye sab apni notes me likh lo, SSL pinning kabhi bhi future projects me jarurat padegi, sab clear ho jayega.}

\separatorline


=============================================================

\documentclass[a4paper,12pt]{article}

% Geometry with comfortable margins
\usepackage[left=25mm,right=25mm,top=25mm,bottom=25mm]{geometry}

% Fonts
\usepackage{tgtermes} % TeX Gyre Termes (Times-like serif for body)
\usepackage[T1]{fontenc}
\usepackage[utf8]{inputenc}
\usepackage{sectsty} % for section title font control

% Color definitions
\usepackage{xcolor}
\definecolor{headingblue}{RGB}{0,70,132} % strong blue for headings
\definecolor{importantred}{RGB}{204,0,0} % bright red for important text
\definecolor{codebg}{RGB}{245,245,245} % very light gray background for code
\definecolor{keywordcolor}{RGB}{0,0,180} % dark blue for keywords
\definecolor{stringcolor}{RGB}{163,21,21} % dark red for strings
\definecolor{commentcolor}{RGB}{0,128,0} % green for comments
\definecolor{functioncolor}{RGB}{127,0,85} % purple for function names
\definecolor{linenumbercolor}{gray}{0.6} % subtle gray for line numbers

% Listings for code blocks with multi-color syntax highlighting
\usepackage{listings}
\lstdefinelanguage{jsx}{
  keywords={import,function,let,const,var,if,else,return,for,while,do,switch,case,break,continue,class,new,try,catch,finally,throw,async,await,void,static,public,private,protected,yield,default},
  sensitive=true,
  morecomment=[l]{//},
  morecomment=[s]{/*}{*/},
  morestring=[b]",
  morestring=[b]',
}
\lstset{
  language=jsx,
  backgroundcolor=\color{codebg},
  keywordstyle=\color{keywordcolor}\bfseries,
  stringstyle=\color{stringcolor},
  commentstyle=\color{commentcolor}\itshape,
  identifierstyle=\color{black},
  basicstyle=\ttfamily\footnotesize,
  numbers=left,
  numberstyle=\color{linenumbercolor}\scriptsize,
  stepnumber=1,
  numbersep=8pt,
  frame=none,
  tabsize=4,
  showstringspaces=false,
  breaklines=true,
  breakatwhitespace=true,
  captionpos=b,
  xleftmargin=10pt,
  xrightmargin=10pt,
  framexleftmargin=5pt,
  framexrightmargin=5pt,
  rulecolor=\color{codebg},
  upquote=true,
}

% tcolorbox for code blocks with rounded corners and light background
\usepackage[many]{tcolorbox}
\tcbset{
  colback=codebg,
  colframe=codebg,
  boxrule=0pt,
  arc=3mm,
  auto outer arc,
  leftrule=0pt,
  rightrule=0pt,
  toprule=0pt,
  bottomrule=0pt,
  left=6pt,
  right=6pt,
  top=6pt,
  bottom=6pt,
  enhanced,
}

% Section title formatting with colors and size
\usepackage{titlesec}
\titleformat{\section}
  {\color{headingblue}\LARGE\bfseries}
  {}
  {0pt}
  {}
\titlespacing*{\section}{0pt}{4ex plus 1ex minus .2ex}{1.5ex plus .2ex}

% Header and footer with fancyhdr
\usepackage{fancyhdr}
\pagestyle{fancy}
\fancyhf{}
\fancyhead[L]{React Native Notes}
\fancyhead[R]{\thepage}
\renewcommand{\headrulewidth}{0.4pt}
\renewcommand{\footrulewidth}{0pt}

% Lists with enumitem for spacing control
\usepackage{enumitem}
\setlist{itemsep=0.9em, parsep=0pt, topsep=0.9em}

% Hyperlinks without colors
\usepackage[hidelinks]{hyperref}

% Custom command for important red text
\newcommand{\important}[1]{\textcolor{importantred}{#1}}

% Preserve monospace inline code styling
\newcommand{\inlinecode}[1]{\texttt{#1}}

% To preserve the exact separator line visibly
\newcommand{\separatorline}{\texttt{=============================================================}}

\begin{document}

\separatorline

\vspace{1.5em}

\section*{\color{headingblue}\LARGE\bfseries App Lifecycle in React Native – Hinglish Explanation, Use Cases, Real-Life Example, Sab Doubt Clear}

\vspace{1.5em}

\section*{\color{headingblue}\LARGE\bfseries App Lifecycle Kya Hai?}

\important{App lifecycle} ka matlab hai – app kab \important{chalti hai} (foreground), kab \important{background me jati hai} (user dosre app pe switch karta hai), aur kab \important{completely band hoti hai} (kill, memory clear, etc.).  
React Native \important{AppState} API aapko bata deta hai ki \important{app abhi active hai, background me hai, ya inactive hai} (jaise incoming call, notification, sleep, etc.).[1][2]

\important{Possible states (React Native AppState):}  
\begin{itemize}
\item \important{active}: App foreground me chala raha hai.
\item \important{background}: App background me hai, user dosre app pe hai.
\item \important{inactive}: App thoda transition me hai (jaise incoming call, notification, etc.).[1]
\end{itemize}

\vspace{1.5em}

\section*{\color{headingblue}\LARGE\bfseries Kyun Mujhe Pata Hona Chahiye?}

\begin{itemize}
\item Background job karne ho, jaise data sync, notification, analytics, etc.
\item Resource save karna ho—jaise background me video/audio pause karo, animation band karo, CPU/memory optimize karo.
\item App open hote hi kuch karna ho—jaise data refresh, user status update.
\item App background me chale toh kuch important data save karo—jaise form data, timer, etc.
\item Security—jaise user screen lock hua, ya app background pe chali gayi, toh session/logout message dikhayo.
\end{itemize}

\vspace{1.5em}

\section*{\color{headingblue}\LARGE\bfseries Kahan Use Hota Hai?}

\begin{itemize}
\item Online/offline sync—jab app background se foreground me aaye, fresh data load karo.
\item Audio/video app—background me audio/video pause/stop karo.
\item Chat apps—user online/offline status update, last seen, notifications.
\item Caching/data save—app background me jate hi data save karo.
\item Analytics—jab user app use karta hai, kab karta hai, ye data collect karo.
\end{itemize}

\vspace{1.5em}

\section*{\color{headingblue}\LARGE\bfseries AppState API – Kaise Use Karein?}

\important{AppState.currentState} – current state batata hai (active, background, inactive)  
\important{AppState.addEventListener('change', handler)} – state change pe alert deta hai

\vspace{1em}

\textbf{Simple Example:}

\begin{tcolorbox}[sharp corners,boxrule=0.5pt,colback=codebg]
\begin{lstlisting}[language=jsx]
import React, { useEffect, useState } from 'react';
import { AppState, Text, View } from 'react-native';

function App() {
  const [appState, setAppState] = useState(AppState.currentState);

  useEffect(() => {
    const subscription = AppState.addEventListener('change', nextAppState => {
      // Jab state change hoga (active, background, inactive)
      setAppState(nextAppState);
      // Real-life use: background me jate hi data save karo, ya active hote hi refresh karo
      if (nextAppState === 'background') {
        saveData();
      } else if (nextAppState === 'active') {
        refreshData();
      }
    });

    // Cleanup
    return () => {
      subscription.remove();
    };
  }, []);

  function saveData() {
    // Data save karo AsyncStorage me, ya API me sync karo
    console.log('App background me gayi, data saving...');
  }

  function refreshData() {
    // API se naya data mangao, ya local data se update karo
    console.log('App foreground me aayi, refreshing data...');
  }

  return (
    <View style={{ flex: 1, justifyContent: 'center', alignItems: 'center' }}>
      <Text>App current state: {appState}</Text>
    </View>
  );
}

export default App;
\end{lstlisting}
\end{tcolorbox}

\important{Output:}  
App open karo, screen pe dikhega “Active”  
App background me bhejo, dikhega “Background”  
Incoming call ya notification pe dikhega “Inactive”

\vspace{1.5em}

\section*{\color{headingblue}\LARGE\bfseries Real-Life Example – Jab User App Chhode}

\begin{itemize}
\item Background me jate hi:  
User ne app se bana di, ya home button dabaya, toh app background me jati hai.  
\important{Tab aapka data save hona chahiye} (jaise chat draft, form, timer, etc.) – AppState.addEventListener me check karo, state ‘background’ hua toh save karo.[2]
\item Active me aate hi:  
User app wapas laata hai, active dikhega.  
\important{Tab naya data fetch/refresh karo} – notification, pending messages, jo bhi show karna ho, refresh karo.[2]
\item Inactive:  
Notification aati hai, incoming call, ya user screen lock karta hai—state ‘inactive’ ho jata hai.  
\important{Tab animation/audio/video pause karo}, ya important operations band karo.
\end{itemize}

\vspace{1.5em}

\section*{\color{headingblue}\LARGE\bfseries Agar Nahi Use Karen Toh Kya Hoga?}

\begin{itemize}
\item App background me jate hi data loss ho sakta hai.
\item Active me aate hi stale/old data dikhega, update nahi hoga.
\item Resources (CPU, memory, battery) zyada use honge.
\item Offline/online status, notifications, analytics, sab me pareshani hogi.
\item User experience kharab hoga—app thodi si pehle ki thi, wapas aate hi user ko fresh dikhana chahiye.
\end{itemize}

\vspace{1.5em}

\section*{\color{headingblue}\LARGE\bfseries Common Doubts Clear}

\begin{itemize}
\item Ye sab native (Android/iOS) lifecycle hooks se related hai.  
AppState API React Native me simple tareeke se expose karta hai, \important{aapko native code nahi likhna padega}.[2]
\item Expo apps me bhi chalta hai, sab devices pe ek hi tarah se.
\item Agar aap native developer ho, toh ye aapko zyada samajh aa sakta hai – lekin React Native me AppState kaam fix karta hai.
\item Expo, CLI, sab set-up me AppState kaam karta hai.
\item Android/iOS me Android’s Activity / iOS’s AppDelegate lifecycle events se related hai, lekin React Native me AppState se mil jata hai.
\end{itemize}

\vspace{1.5em}

\section*{\color{headingblue}\LARGE\bfseries Cheat Sheet – App Lifecycle in Hinglish}

\begin{tabular}{|p{4cm}|p{7cm}|p{5cm}|p{5cm}|}
\hline
\textbf{State} & \textbf{Meaning} & \textbf{Kab Use Karein?} & \textbf{Example Use Case} \\
\hline
\important{active} & App foreground me hai (user use kar raha hai) & Data refresh, online status, notifications & App open hote hi API call karo \\
\hline
\important{background} & App background me hai (user dosre app pe hai) & Data save, animation/video pause, sync & Chat bg me jaye, draft save karo \\
\hline
\important{inactive} & Transition state (notification, call, sleep, etc.) & Animation/video pause, resource save & Call aate hi audio/video pause \\
\hline
\end{tabular}

\vspace{1.5em}

\section*{\color{headingblue}\LARGE\bfseries Ek Dum Beginner Friendly Summary}

\begin{itemize}
\item App lifecycle React Native me “AppState” API se track hota hai.
\item Ye important hai kyuki jab user app chhoda, ya wapas laaya, aap apni state, data, resources, notifications, sab handle kar sakte ho.
\item Har use case me app ka sahi behavior chahiye hota hai.
\item AppState.currentState – current state check karo.
\item AppState.addEventListener – state change pe event listen karo, aur as per state apna logic likho.
\item Agar ye nahi lagayenge, toh data loss, stale data, resource leak, aur pareshani hogi.
\item Iska use karke apps professional, smooth, aur user-friendly bante hain.
\end{itemize}

\vspace{1.5em}

\important{Aap jab bhi app me koi important kaam hai jo app background/foreground transition pe karna hai, AppState API use karo.}  
\important{Ye sab code apne notes me rakh lo, future projects me bhulane ki zarurat nahi padegi!}

\separatorline


=============================================================



=============================================================


\end{document}



\end{document}


\end{document}
