\documentclass[a4paper]{article}
\usepackage[margin=2cm]{geometry}
\usepackage[utf8]{inputenc}
\usepackage{xcolor}
\usepackage{tcolorbox}
\tcbuselibrary{skins,breakable}
\usepackage{listings}
\usepackage{array, colortbl}
\usepackage{enumitem}
\usepackage{titlesec}
\usepackage{parskip}

% Colors define karte hain, sab vibrant
\definecolor{headingblue}{RGB}{0,102,204}
\definecolor{examplegreen}{RGB}{0,153,76}
\definecolor{warningred}{RGB}{204,0,0}
\definecolor{codeblue}{RGB}{173,216,230}
\definecolor{tablerowgreen}{RGB}{144,238,144}
\definecolor{tableheaderblue}{RGB}{0,102,204}
\definecolor{tableheadergreen}{RGB}{0,153,76}
\definecolor{tableheaderyellow}{RGB}{255,204,0}
\definecolor{violet}{RGB}{128,0,128}

% Section ka style set karo
\titleformat{\section}{\Large\bfseries\color{headingblue}}{}{0em}{}

% Code blocks ke liye settings
\lstset{
  language=Python,
  backgroundcolor=\color{codeblue},
  basicstyle=\ttfamily\small,
  frame=single,
  breaklines=true,
  keywordstyle=\color{blue},
  stringstyle=\color{purple},
  commentstyle=\color{gray},
  showstringspaces=false
}

% Example aur note ke boxes
\newtcolorbox{examplebox}[1]{
  colback=examplegreen!10,
  colframe=examplegreen!50!black,
  title=#1,
  breakable,
  enhanced
}
\newtcolorbox{notebox}{
  colback=warningred!5!white,
  colframe=warningred!75!black,
  title=Important Baatein,
  breakable,
  enhanced
}

\begin{document}

% Title page banate hain
\begin{titlepage}
  \centering
  \vspace*{\fill}
  {\Huge\bfseries\color{warningred} Figma Notes}\par
  \vspace{1cm}
  {\Large\color{violet} Figma ke Basics ko Samajho}\par
  \vspace*{\fill}
\end{titlepage}

% Topic 1: Figma kya hai, kyun use karo
\section*{\textbf{\LARGE \textcolor{violet}{Figma Kya Hai aur Kyun Use Karein?}}}
\textcolor{red}{Figma ek cloud-based design tool hai jo UI/UX, prototyping, aur team collaboration ke liye top-class hai.} Ye browser mein chalta hai, toh kahi bhi kaam kar sakta hai bina software download kiye.

\textbf{\textcolor{warningred}{Kyun Figma Jab Aur Tools Hain?}}
\begin{itemize}
  \item \textbf{\textcolor{red}{Cloud-Based}}: \textcolor{examplegreen}{Kisi bhi device se apna project khol, bas internet chahiye. No installation ka jhanjhat.}
  \item \textbf{\textcolor{red}{Collaborative}}: \textcolor{examplegreen}{Designers, developers, aur clients ek saath kaam kar sakte hain, live updates aur comments ke saath.} Example: Ek app ka design team ek file mein banaye, sabko changes dikhe.
  \item \textbf{\textcolor{red}{Design, Prototype, Code}}: \textcolor{examplegreen}{Ek hi tool mein sundar designs, interactive prototypes, aur developer-friendly code (CSS, iOS, Android) bana sakta hai.} Example: Button design kiya, uska hover effect prototype kiya, aur CSS export kar diya.
  \item \textbf{\textcolor{red}{Plugins}}: \textcolor{examplegreen}{Plugins jaise Unsplash (images), Chart (graphs), ya Auto Layout (responsive design) kaam ko aur aasan karte hain.} Example: “Content Reel” plugin se realistic user data daal sakta hai.
\end{itemize}

\begin{examplebox}{\textcolor{violet}{Real-Life Example}}
\textcolor{red}{Soch, ek startup ek e-commerce app bana raha hai.} Designer wireframes banata hai, UX wala prototype banata hai, aur developer button ka CSS nikaalta hai—sab ek Figma file mein. \textcolor{examplegreen}{Client shared link se prototype dekhta hai aur feedback daalta hai.} Team “FigJam” plugin se brainstorming karta hai, sab ek jagah sorted!
\end{examplebox}

% Topic 2: Canvas aur Frames
\section*{\textbf{\LARGE \textcolor{violet}{Canvas aur Frames}}}
\textcolor{warningred}{Canvas tera infinite workspace hai jahan tu designs banata hai.} Frames containers hain jo design elements (buttons, text, images) ke liye boundary set karte hain, jaise mobile ya desktop screen.

\textbf{\textcolor{red}{Frame Kaise Banaye}:}
\begin{enumerate}
  \item \textcolor{examplegreen}{Toolbar se Frame Tool chun (shortcut: \texttt{F}).}
  \item \textcolor{examplegreen}{Right panel mein preset size select kar (jaise iPhone 14: 1170x2532px) ya canvas pe drag karke custom frame bana.}
  \item Frame mein elements (text, shapes, images) daal sakta hai.
\end{enumerate}

\textbf{\textcolor{red}{Frame Customize Kaise Kare (Right Panel)}:}
\begin{itemize}
  \item \textbf{\textcolor{warningred}{Shape}}: \textcolor{examplegreen}{Frame ko oval ya koi aur shape mein badal sakta hai right panel se.}
  \item \textbf{\textcolor{warningred}{Corner Radius}}: \textcolor{examplegreen}{Corners ko round karne ke liye radius set kar, jaise 20px soft look ke liye.}
  \item \textbf{\textcolor{warningred}{Constraints}}: \textcolor{examplegreen}{Elements ko pin kar taaki sirf height ya width badle. Example: Button ka width frame ke saath scale ho, lekin height fixed rahe.}
\end{itemize}

\textbf{\textcolor{red}{Kab Use Karna}:}
\textcolor{examplegreen}{Frames tab use kar jab specific device ke liye design kar raha ho, jaise mobile app ka login screen ya website homepage.} Ye ensure karta hai ki design real-world screen pe fit ho.

\begin{examplebox}{\textcolor{violet}{Frame Example}}
\textcolor{red}{Ek tablet app ke liye frame banaya (1024x768px).} Button ke constraints set kiye taaki width frame ke saath badle, lekin height 48px fixed rahe. \textcolor{examplegreen}{Frame ko oval shape diya aur 30px corner radius set kiya.} Resize karne pe button ka width adjust hua, lekin height wahi raha, design consistent raha.
\end{examplebox}

% Topic 3: Layout Grid
\section*{\textbf{\LARGE \textcolor{violet}{Layout Grid Design Tab Mein}}}
\textcolor{warningred}{Layout grid ek column-row system hai jo Design tab mein milta hai (\texttt{View > Show Layout Grid} ya \texttt{Ctrl+’}).} Ye elements ko perfectly align karne mein madad karta hai.

\textbf{\textcolor{red}{Kab Use Karna}:}
\begin{itemize}
  \item \textcolor{examplegreen}{Elements ko consistent spacing ke liye use kar, jaise 8px grid mobile designs ke liye.}
  \item \textcolor{examplegreen}{Responsive layouts ke liye columns use kar, jaise 12-column grid website ke liye.}
\end{itemize}

\begin{examplebox}{\textcolor{violet}{Grid Example}}
\textcolor{red}{Website homepage ke liye 1440px frame banaya.} 12-column grid on kiya, 30px gutters ke saath. \textcolor{examplegreen}{Hero image pehle 6 columns mein aur text agle 6 mein daala, taaki spacing aur balance perfect rahe.}
\end{examplebox}

% Topic 4: Selection aur Grouping
\section*{\textbf{\LARGE \textcolor{violet}{Selection aur Grouping Layers}}}
\textcolor{warningred}{Kisi element (button, text, image) ko edit karne ke liye pehle select karna padta hai.} Grouping se multiple elements ek unit ban jaate hain.

\textbf{\textcolor{red}{Kaise Kare}:}
\begin{itemize}
  \item \textcolor{examplegreen}{Ek element pe click kar ya \texttt{Shift} daba ke multiple elements select kar.}
  \item \textcolor{examplegreen}{Group karne ke liye right-click se Group chun ya \texttt{Ctrl+G} use kar.}
\end{itemize}

\textbf{\textcolor{red}{Kab Use Karna}:}
\textcolor{examplegreen}{Related elements jaise navbar ke logo, links, aur buttons ko group kar taaki move ya edit karna aasan ho.} Example: App ke “Sign Up” button aur text ko group karo.

\begin{examplebox}{\textcolor{violet}{Grouping Example}}
\textcolor{red}{Food delivery app mein restaurant card banaya jisme image, name, rating, aur button tha.} Sabko select karke group kiya (\texttt{Ctrl+G}), naam rakha “Restaurant Card.” \textcolor{examplegreen}{Ab card ko kahin bhi move karo, sab aligned rahta hai, time bachta hai.}
\end{examplebox}

% Topic 5: Layers
\section*{\textbf{\LARGE \textcolor{violet}{Layers}}}
\textcolor{warningred}{Figma mein har element (text, shapes, images) ek layer hai, jo Layers Panel (left side) mein list hota hai.} Layers ek stack ki tarah kaam karte hain.

\textbf{\textcolor{red}{Layers Duplicate Kaise Kare}:}
\begin{enumerate}
  \item \textcolor{examplegreen}{Layer select kar Canvas ya Layers Panel mein.}
  \item \textcolor{examplegreen}{Right-click se Duplicate chun ya \texttt{Ctrl+D} use kar.}
\end{enumerate}

\textbf{\textcolor{red}{Kab Use Karna}:}
\textcolor{examplegreen}{Similar elements ke liye layers duplicate kar, jaise multiple buttons ya cards.} Example: Ek button bana, usko duplicate karke color change kar diya.

\begin{examplebox}{\textcolor{violet}{Layer Example}}
\textcolor{red}{Travel app mein flight search card banaya.} Return flight ke liye usko duplicate kiya (\texttt{Ctrl+D}), text thoda change kiya. \textcolor{examplegreen}{Layers ke naam “Flight Card 1” aur “Flight Card 2” rakhe taaki sab organized rahe.}
\end{examplebox}

% Topic 6: Text
\section*{\textbf{\LARGE \textcolor{violet}{Text}}}
\textcolor{warningred}{Text se headings, labels, ya buttons banaye jaate hain.} Properties right panel mein adjust kar sakta hai.

\textbf{\textcolor{red}{Text Kaise Add Kare}:}
\begin{enumerate}
  \item \textcolor{examplegreen}{Text Tool chun (\texttt{T}).}
  \item \textcolor{examplegreen}{Canvas pe click karke type kar ya drag karke text box bana.}
  \item \textcolor{examplegreen}{Right panel mein font, size, color, line spacing set kar.}
\end{enumerate}

\textbf{\textcolor{red}{Kab Use Karna}:}
\textcolor{examplegreen}{Headings, body text, ya button labels ke liye text use kar, brand ke style se match karke.} Example: Website ke “About Us” page ke liye bold heading aur readable paragraph.

\begin{examplebox}{\textcolor{violet}{Text Example}}
\textcolor{red}{Fitness app ke liye heading “Join Our Classes” 32px bold font mein daala.} Description “Explore yoga, Zumba, and more” 16px regular font mein. \textcolor{examplegreen}{Heading ko vibrant color diya aur description ko neutral taaki readable ho.}
\end{examplebox}

% Topic 7: Shapes
\section*{\textbf{\LARGE \textcolor{violet}{Shapes}}}
\textcolor{warningred}{Shapes UI elements ke building blocks hain, jaise buttons, icons, ya backgrounds.}

\textbf{\textcolor{red}{Shapes Kaise Banaye}:}
\begin{enumerate}
  \item \textcolor{examplegreen}{Shape tool chun: Rectangle (\texttt{R}), Ellipse (\texttt{O}), ya Polygon.}
  \item \textcolor{examplegreen}{Canvas pe drag karke shape bana.}
  \item \textcolor{examplegreen}{Right panel mein size, corner radius, fill color, border adjust kar.}
\end{enumerate}

\textbf{\textcolor{red}{Kab Use Karna}:}
\textcolor{examplegreen}{Buttons, icons, ya decorative elements ke liye shapes use kar.} Example: Rounded button ya circular profile picture.

\begin{examplebox}{\textcolor{violet}{Shapes Example}}
\textcolor{red}{Banking app ke liye rectangle (12px corner radius) se card background banaya.} Ellipse se “Balance” icon banaya. \textcolor{examplegreen}{Ellipse ko white border diya taaki contrast dikhe, design pro laga.}
\end{examplebox}

% Beginners ke liye tips
\section*{\textbf{\LARGE \textcolor{violet}{Naye Logon ke Liye Tips}}}
\textcolor{warningred}{In tips se Figma mein kaam jaldi aur professional banega:}
\begin{itemize}
  \item \textcolor{examplegreen}{Hamesha layers, frames, aur groups ka naam rakho (jaise “Homepage Frame”).} Isse team ke saath kaam aasan hota hai.
  \item \textcolor{examplegreen}{Layout grid on karo (\texttt{View > Show Layout Grid} ya \texttt{Ctrl+’}) taaki alignment perfect ho.}
  \item \textcolor{examplegreen}{Shortcuts seekho: \texttt{F} (Frame), \texttt{T} (Text), \texttt{R} (Rectangle), \texttt{Ctrl+G} (Group), \texttt{Ctrl+D} (Duplicate).}
  \item \textcolor{examplegreen}{Practice ke liye simple app screen redesign karo, jaise music player.}
\end{itemize}

\begin{table}[h]
  \centering
  \begin{tabular}{|m{4cm}|m{4cm}|m{4cm}|}
    \hline
    \cellcolor{tableheaderblue}{\color{white}\textbf{Tip}} & 
    \cellcolor{tableheadergreen}{\color{white}\textbf{Kya Hai}} & 
    \cellcolor{tableheaderyellow}{\color{black}\textbf{Fayda}} \\
    \hline
    \rowcolor{codeblue}
    Layer Naming & Frames, groups ka naam do & Team ke saath kaam aasan \\
    \hline
    \rowcolor{tablerowgreen}
    Layout Grid & Elements align karo & Design precise rahta hai \\
    \hline
    \rowcolor{codeblue}
    Shortcuts & \texttt{F}, \texttt{T}, \texttt{R} use karo & Kaam jaldi hota hai \\
    \hline
  \end{tabular}
  \caption{\textcolor{violet}{Beginner Tips}}
\end{table}

% Conclusion
\begin{notebox}
\textbf{\textcolor{warningred}{Important Baatein}:}
\begin{itemize}
  \item \textcolor{examplegreen}{Figma ka cloud aur collaboration feature design ko ekdum aasan aur fast banata hai.}
  \item \textcolor{examplegreen}{Frames aur layout grid se designs real devices ke liye perfect banta hai.}
  \item \textcolor{examplegreen}{Layers, grouping, text, aur shapes se organized aur sundar designs banaye.}
  \item \textcolor{examplegreen}{Shortcuts aur naming se kaam tez aur professional ho jata hai.}
\end{itemize}
\end{notebox}



===============================
\hrule

\documentclass[a4paper]{article}
\usepackage[margin=2cm]{geometry}
\usepackage[utf8]{inputenc}
\usepackage{xcolor}
\usepackage{tcolorbox}
\tcbuselibrary{skins,breakable}
\usepackage{listings}
\usepackage{array, colortbl}
\usepackage{enumitem}
\usepackage{titlesec}
\usepackage{parskip}

% Colors define karte hain, vibrant vibe ke liye
\definecolor{headingblue}{RGB}{0,102,204}
\definecolor{examplegreen}{RGB}{0,153,76}
\definecolor{warningred}{RGB}{204,0,0}
\definecolor{codeblue}{RGB}{173,216,230}
\definecolor{tablerowgreen}{RGB}{144,238,144}
\definecolor{tableheaderblue}{RGB}{0,102,204}
\definecolor{tableheadergreen}{RGB}{0,153,76}
\definecolor{tableheaderyellow}{RGB}{255,204,0}
\definecolor{violet}{RGB}{128,0,128}

% Section ka style set karo
\titleformat{\section}{\Large\bfseries\color{headingblue}}{}{0em}{}

% Code blocks ke liye settings
\lstset{
  language=Python,
  backgroundcolor=\color{codeblue},
  basicstyle=\ttfamily\small,
  frame=single,
  breaklines=true,
  keywordstyle=\color{blue},
  stringstyle=\color{purple},
  commentstyle=\color{gray},
  showstringspaces=false
}

% Example aur note ke boxes
\newtcolorbox{examplebox}[1]{
  colback=examplegreen!10,
  colframe=examplegreen!50!black,
  title=#1,
  breakable,
  enhanced
}
\newtcolorbox{notebox}{
  colback=warningred!5!white,
  colframe=warningred!75!black,
  title=Important Baatein,
  breakable,
  enhanced
}

\begin{document}

% Topic: Place Images and Fill Options
\section*{\textbf{\LARGE \textcolor{violet}{Images Place Karna aur Fill Options}}}
\textcolor{warningred}{Figma mein images add karna aur unko customize karna ek important skill hai.} Isme tu undraw.co jaise websites se images le sakta hai, aur Figma ke Design tab mein alignment, constraints, aur fill options ke saath kaam kar sakta hai. Chalo, sab detail mein dekhte hain!

\textbf{\textcolor{red}{undraw.co Kya Hai aur Kab Use Karna?}} \\
\textcolor{examplegreen}{undraw.co ek free website hai jo open-source illustrations deti hai, jo UI/UX designs, presentations, ya mockups ke liye perfect hain.} Ye colorful, customizable SVG images deta hai jo Figma mein easily use ho sakte hain. \\
\textbf{\textcolor{red}{Features aur Kab Use Karna}:}
\begin{itemize}
  \item \textcolor{examplegreen}{Hazaar se zyada illustrations, jaise characters, objects, aur scenes, jo free mein download kar sakta hai.}
  \item \textcolor{examplegreen}{Colors customize kar sakta hai undraw.co pe ya Figma mein import karne ke baad.}
  \item \textcolor{examplegreen}{Use kar jab quick, professional-looking visuals chahiye, jaise app ke placeholder images, website banners, ya onboarding screens.}
  \item \textcolor{examplegreen}{Example: Ek e-learning app ke liye “studying” illustration use kar sakta hai homepage pe.}
\end{itemize}

\textbf{\textcolor{red}{Figma Mein undraw.co Images Kaise Use Kare}:}
\begin{enumerate}
  \item \textcolor{examplegreen}{undraw.co pe ja, apni pasand ka illustration chun, aur SVG ya PNG format mein download kar.}
  \item \textcolor{examplegreen}{Figma mein, hamburger menu (top-left) pe click kar, phir \texttt{File > Place Image} select kar.}
  \item \textcolor{examplegreen}{Downloaded image chun aur frame ya canvas pe place kar. Image ek layer ban jayega.}
\end{enumerate}

\begin{examplebox}{\textcolor{violet}{Real-Life Example}}
\textcolor{red}{Soch, tu ek fitness app design kar raha hai.} undraw.co se “yoga” illustration SVG mein download kiya. Figma mein \texttt{File > Place Image} se usko homepage ke frame (375x812px) mein daala. \textcolor{examplegreen}{Right panel se colors change kiye taaki app ke brand colors se match ho, aur image ko top-center align kiya.} Ye design ko ekdum professional aur attractive bana deta hai!
\end{examplebox}

\textbf{\textcolor{red}{Design Tab Mein Image Customization}:} \\
\textcolor{warningred}{Image place karne ke baad, Design tab (right panel) mein yeh options use kar sakta hai:}

\begin{itemize}
  \item \textbf{\textcolor{warningred}{Alignment}}: \textcolor{examplegreen}{Image ko frame ke andar left, right, center, top, bottom, ya kisi bhi position pe align kar sakta hai.} \\
  \textbf{\textcolor{red}{Kaise Kare}}: Design tab mein alignment icons (top of right panel) se chun. Example: Ek banner image ko top-left align kar taaki focus upar rahe. \\
  \textbf{\textcolor{red}{Kab Use Karna}}: Jab image ka position frame ke specific area mein chahiye, jaise profile picture center mein ya background image top pe.

  \item \textbf{\textcolor{warningred}{Constraints}}: \textcolor{examplegreen}{Constraints decide karte hain ki frame resize hone pe image kaise behave karega.} \\
  \textbf{\textcolor{red}{Kaise Kare}}: Design tab mein Constraints section se options chun (Left, Right, Top, Bottom, Center, Scale). Example: Ek button image ko “Center” aur “Scale” set kar taaki frame resize hone pe image stretch ho. \\
  \textbf{\textcolor{red}{Kab Use Karna}}: Responsive designs ke liye, jab image ka position ya size frame ke saath adjust hona chahiye. Jaise, mobile se desktop layout pe switch karte waqt.[](https://help.figma.com/hc/en-us/articles/360039957734-Apply-constraints-to-define-how-layers-resize)

  \item \textbf{\textcolor{warningred}{Fill Options}}: \textcolor{examplegreen}{Ye decide karta hai ki image frame ya shape ke andar kaise fit hogi.} Design tab ke Fill section mein yeh options milte hain:
  \begin{itemize}
    \item \textbf{\textcolor{red}{Fill}}: \textcolor{examplegreen}{Image pura frame bhar deta hai, agar frame ka size image se alag hai toh stretch ya crop ho sakta hai.} Use kar jab pura frame cover karna ho, jaise background images ke liye.[](https://webdesign.tutsplus.com/figma-image-fill-settings--cms-35470t)
    \item \textbf{\textcolor{red}{Fit}}: \textcolor{examplegreen}{Pura image dikhta hai, frame ke andar adjust hota hai, lekin blank space (padding) ho sakta hai.} Use kar jab image ka koi part crop nahi karna, jaise profile pictures ke liye.[](https://help.figma.com/hc/en-us/articles/360041098433-Adjust-the-properties-of-an-image)
    \item \textbf{\textcolor{red}{Crop}}: \textcolor{examplegreen}{Image ke specific part ko chun sakta hai, jaise zoom karke ya position adjust karke.} Use kar jab sirf image ka ek hissa dikhana ho, jaise close-up shots ke liye. Shortcut: Image select karke top toolbar se Crop button ya \texttt{Option + double-click} use kar.[](https://www.figma.com/best-practices/working-with-images-in-figma/)
  \end{itemize}
\end{itemize}

\begin{examplebox}{\textcolor{violet}{Fill Options Example}}
\textcolor{red}{Ek social media app ke liye profile picture frame banaya (100x100px).} undraw.co se “avatar” illustration daala. \textcolor{examplegreen}{Fit option use kiya taaki pura image dikhe, aur center align kiya.} Phir ek banner ke liye “Crop” use kiya taaki image ka top part focus mein rahe. Constraints set kiye “Left and Top” taaki frame resize hone pe image wahi rahe. Design ekdum clean aur responsive laga!
\end{examplebox}

% Table for Image Options
\begin{table}[h]
  \centering
  \begin{tabular}{|m{4cm}|m{4cm}|m{4cm}|}
    \hline
    \cellcolor{tableheaderblue}{\color{white}\textbf{Feature}} & 
    \cellcolor{tableheadergreen}{\color{white}\textbf{Kya Hai}} & 
    \cellcolor{tableheaderyellow}{\color{black}\textbf{Kab Use Karna}} \\
    \hline
    \rowcolor{codeblue}
    Alignment & Image ka position set karo & Specific area focus ke liye \\
    \hline
    \rowcolor{tablerowgreen}
    Constraints & Resize behavior control karo & Responsive designs ke liye \\
    \hline
    \rowcolor{codeblue}
    Fill & Pura frame bharo & Background images ke liye \\
    \hline
    \rowcolor{tablerowgreen}
    Fit & Pura image dikhao & Profile pictures ke liye \\
    \hline
    \rowcolor{codeblue}
    Crop & Specific part chuno & Close-up shots ke liye \\
    \hline
  \end{tabular}
  \caption{\textcolor{violet}{Image Options Overview}}
\end{table}

% Conclusion
\begin{notebox}
\textbf{\textcolor{warningred}{Important Baatein}:}
\begin{itemize}
  \item \textcolor{examplegreen}{undraw.co se free illustrations le aur Figma mein \texttt{File > Place Image} se add kar.}
  \item \textcolor{examplegreen}{Alignment aur constraints se image ka position aur resize behavior set kar taaki design responsive ho.}
  \item \textcolor{examplegreen}{Fill, Fit, ya Crop options chun apne design ke purpose ke hisaab se, jaise background ya profile pics ke liye.}
\end{itemize}
\end{notebox}



===============================
\hrule



% Colors define karte hain, vibrant vibe ke liye
\definecolor{headingblue}{RGB}{0,102,204}
\definecolor{examplegreen}{RGB}{0,153,76}
\definecolor{warningred}{RGB}{204,0,0}
\definecolor{codeblue}{RGB}{173,216,230}
\definecolor{tablerowgreen}{RGB}{144,238,144}
\definecolor{tableheaderblue}{RGB}{0,102,204}
\definecolor{tableheadergreen}{RGB}{0,153,76}
\definecolor{tableheaderyellow}{RGB}{255,204,0}
\definecolor{violet}{RGB}{128,0,128}

% Section ka style set karo
\titleformat{\section}{\Large\bfseries\color{headingblue}}{}{0em}{}

% Code blocks ke liye settings
\lstset{
  language=Python,
  backgroundcolor=\color{codeblue},
  basicstyle=\ttfamily\small,
  frame=single,
  breaklines=true,
  keywordstyle=\color{blue},
  stringstyle=\color{purple},
  commentstyle=\color{gray},
  showstringspaces=false
}

% Example aur note ke boxes
\newtcolorbox{examplebox}[1]{
  colback=examplegreen!10,
  colframe=examplegreen!50!black,
  title=#1,
  breakable,
  enhanced
}
\newtcolorbox{notebox}{
  colback=warningred!5!white,
  colframe=warningred!75!black,
  title=Important Baatein,
  breakable,
  enhanced
}

\begin{document}

% Topic: CSS, iOS, Android Codes
\section*{\textbf{\LARGE \textcolor{violet}{CSS, iOS, Android Codes}}}
\textcolor{warningred}{Figma se frontend ke liye CSS, iOS, aur Android code nikaal sakta hai, jo developers ke kaam ko aasan karta hai.} Isme tu “Lorem Ipsum” plugin bhi use kar sakta hai dummy text ke liye. Chalo, sab step-by-step dekhte hain!

\textbf{\textcolor{red}{Lorem Ipsum Plugin Kaise Install aur Use Kare}:} \\
\textcolor{examplegreen}{Lorem Ipsum plugin dummy text generate karta hai, jo designs mein placeholder text ke liye kaam aata hai.} \\
\textbf{\textcolor{red}{Kaise Install Kare}:}
\begin{enumerate}
  \item \textcolor{examplegreen}{Figma mein top-left hamburger menu pe click kar.}
  \item \textcolor{examplegreen}{\texttt{Plugins > Manage Plugins} jao, phir Plugins search bar mein “Lorem Ipsum” type kar aur \texttt{Install} click kar.}
\end{enumerate}
\textbf{\textcolor{red}{Kaise Use Kare}:}
\begin{enumerate}
  \item \textcolor{examplegreen}{Hamburger menu se \texttt{Plugins > Lorem Ipsum} select kar.}
  \item \textcolor{examplegreen}{Text box select karo aur \texttt{Generate} click karo, dummy text apne aap add ho jayega.}
  \item \textcolor{examplegreen}{Kab Use Karna}: Jab realistic text chahiye designs mein, jaise website ke paragraphs ya app ke descriptions ke liye.}
\end{enumerate}

\textbf{\textcolor{red}{CSS, iOS, Android Code Kaise Nikale}:} \\
\textcolor{examplegreen}{Figma ke Inspect tab se tu design elements ka code nikaal sakta hai, jo frontend development ke liye ready hota hai.} \\
\textbf{\textcolor{red}{Steps}:}
\begin{enumerate}
  \item \textcolor{examplegreen}{Figma mein ek frame ya element (jaise button, text) select kar.}
  \item \textcolor{examplegreen}{Right panel mein \texttt{Inspect} tab pe jao.}
  \item \textcolor{examplegreen}{Wahan \texttt{CSS}, \texttt{iOS}, ya \texttt{Android} options mein se chun. Code copy karke apne project mein paste kar sakta hai.}
\end{enumerate}
\textbf{\textcolor{red}{Kya Ye Code Direct Use Kar Sakte Hain?}} \\
\textcolor{examplegreen}{Figma ka code developer-friendly hai, lekin direct use karne se pehle thodi tweaking chahiye:}
\begin{itemize}
  \item \textcolor{examplegreen}{CSS code usually web layouts ke liye kaam karta hai, lekin browser compatibility ke liye check karna padta hai.}
  \item \textcolor{examplegreen}{iOS (Swift) aur Android (XML) code basic structure deta hai, lekin app ke logic ke hisaab se customize karna padta hai.}
  \item \textcolor{examplegreen}{Example: Ek button ka CSS code copy kiya, lekin hover effects ya animations ke liye extra code add karna pada.}
\end{itemize}

\begin{examplebox}{\textcolor{violet}{Real-Life Example}}
\textcolor{red}{Ek blogging app ke liye homepage design kiya.} Lorem Ipsum plugin se dummy blog post text generate kiya (\texttt{Plugins > Lorem Ipsum > Generate}). \textcolor{examplegreen}{Phir ek “Read More” button select kiya, Inspect tab se uska CSS code copy kiya, aur developer ko diya.} Developer ne code ko thoda tweak kiya (hover effect add kiya) aur website pe implement kar diya. Same button ka Android XML code bhi nikala, jo app ke layout mein use hua.
\end{examplebox}

% Table for Code and Plugin
\begin{table}[h]
  \centering
  \begin{tabular}{|m{4cm}|m{4cm}|m{4cm}|}
    \hline
    \cellcolor{tableheaderblue}{\color{white}\textbf{Feature}} & 
    \cellcolor{tableheadergreen}{\color{white}\textbf{Kya Hai}} & 
    \cellcolor{tableheaderyellow}{\color{black}\textbf{Kab Use Karna}} \\
    \hline
    \rowcolor{codeblue}
    Lorem Ipsum & Dummy text generate karo & Placeholder text ke liye \\
    \hline
    \rowcolor{tablerowgreen}
    CSS Code & Web layout code & Frontend web development \\
    \hline
    \rowcolor{codeblue}
    iOS Code & Swift code & iOS app layouts \\
    \hline
    \rowcolor{tablerowgreen}
    Android Code & XML code & Android app layouts \\
    \hline
  \end{tabular}
  \caption{\textcolor{violet}{Code aur Plugin Overview}}
\end{table}

% Conclusion
\begin{notebox}
\textbf{\textcolor{warningred}{Important Baatein}:}
\begin{itemize}
  \item \textcolor{examplegreen}{Lorem Ipsum plugin se dummy text jaldi add karo designs mein.}
  \item \textcolor{examplegreen}{Inspect tab se CSS, iOS, aur Android code nikaal sakta hai, jo developers ke liye kaam aata hai.}
  \item \textcolor{examplegreen}{Code direct use ho sakta hai, lekin project ke hisaab se thodi tweaking chahiye.}
\end{itemize}
\end{notebox}


===============================
\hrule



% Colors define karte hain, vibrant vibe ke liye
\definecolor{headingblue}{RGB}{0,102,204}
\definecolor{examplegreen}{RGB}{0,153,76}
\definecolor{warningred}{RGB}{204,0,0}
\definecolor{codeblue}{RGB}{173,216,230}
\definecolor{tablerowgreen}{RGB}{144,238,144}
\definecolor{tableheaderblue}{RGB}{0,102,204}
\definecolor{tableheadergreen}{RGB}{0,153,76}
\definecolor{tableheaderyellow}{RGB}{255,204,0}
\definecolor{violet}{RGB}{128,0,128}

% Section ka style set karo
\titleformat{\section}{\Large\bfseries\color{headingblue}}{}{0em}{}

% Code blocks ke liye settings
\lstset{
  language=Python,
  backgroundcolor=\color{codeblue},
  basicstyle=\ttfamily\small,
  frame=single,
  breaklines=true,
  keywordstyle=\color{blue},
  stringstyle=\color{purple},
  commentstyle=\color{gray},
  showstringspaces=false
}

% Example aur note ke boxes
\newtcolorbox{examplebox}[1]{
  colback=examplegreen!10,
  colframe=examplegreen!50!black,
  title=#1,
  breakable,
  enhanced
}
\newtcolorbox{notebox}{
  colback=warningred!5!white,
  colframe=warningred!75!black,
  title=Important Baatein,
  breakable,
  enhanced
}

\begin{document}

% Topic: Layers and Assets
\section*{\textbf{\LARGE \textcolor{violet}{Layers aur Assets}}}
\textcolor{warningred}{Layers aur Assets Figma ke core features hain jo tera design organized aur reusable banate hain.} Chalo, detail mein samajhte hain ki ye kya hain, kyun aur kab use karna hai.

\textbf{\textcolor{red}{Layers Kya Hain?}} \\
\textcolor{examplegreen}{Figma mein har element (text, shapes, images, groups) ek layer hai, jo Layers Panel (left side) mein list hota hai.} Ye ek stack ki tarah kaam karta hai—upar wali layer neeche wali ko cover karti hai. \\
\textbf{\textcolor{red}{Kyun aur Kab Use Karna}:}
\begin{itemize}
  \item \textcolor{examplegreen}{Layers se complex designs manage hote hain kyunki tu har element alag se control kar sakta hai.}
  \item \textcolor{examplegreen}{Use kar jab multiple elements ko organize karna ho, jaise ek app screen ke buttons, text, aur images ko alag-alag layers mein rakhna.}
  \item \textcolor{examplegreen}{Example: Ek login screen mein background image, text fields, aur button alag layers mein rakho taaki editing aasan ho.}
\end{itemize}

\textbf{\textcolor{red}{Layers ke Operations}:}
\begin{itemize}
  \item \textbf{\textcolor{warningred}{Lock/Unlock}}: \textcolor{examplegreen}{Layer pe right-click karke “Lock” select kar taaki wo edit na ho sake. “Unlock” se wapas editable banaye.} \\
  \textbf{\textcolor{red}{Kab Use Karna}}: Jab koi element fix rakhna ho, jaise background image, taaki galti se move na ho.
  \item \textbf{\textcolor{warningred}{Create Component}}: \textcolor{examplegreen}{Layer ya group pe right-click karke “Create Component” chun. Ye ek reusable element ban jata hai jo Assets panel mein save hota hai.} \\
  \textbf{\textcolor{red}{Kab Use Karna}}: Jab ek design element baar-baar use hoga, jaise buttons, icons, ya cards. Components edit karne se sab instances update ho jaate hain.
\end{itemize}

\textbf{\textcolor{red}{Assets Kya Hain?}} \\
\textcolor{examplegreen}{Assets panel (left side, Layers ke upar tab) mein components store hote hain, jo reusable design elements hain.} Tu inhe drag-and-drop karke frame mein use kar sakta hai. \\
\textbf{\textcolor{red}{Kaise Use Kare}:}
\begin{enumerate}
  \item \textcolor{examplegreen}{Ek component bana (right-click > Create Component).}
  \item \textcolor{examplegreen}{Assets panel mein jao, component chun, aur frame pe drag karke daal.}
  \item \textcolor{examplegreen}{Agar component edit karo (jaise color change), sab instances apne aap update ho jaate hain.}
\end{enumerate}
\textbf{\textcolor{red}{Kab Use Karna}}: Jab consistent design chahiye, jaise ek app mein same style ke buttons har screen pe.

\begin{examplebox}{\textcolor{violet}{Real-Life Example}}
\textcolor{red}{Ek e-commerce app ke liye “Add to Cart” button banaya.} Button ke text aur shape ko group kiya, right-click karke “Create Component” kiya. Ye Assets panel mein save hua. \textcolor{examplegreen}{Har product screen pe Assets se button drag karke daala.} Jab button ka color change kiya, sab screens pe ekdum update ho gaya. Background image ko lock kiya taaki galti se na hile. Layers panel mein sab organized rakha—background, buttons, text alag layers mein.
\end{examplebox}

% Table for Layers and Assets
\begin{table}[h]
  \centering
  \begin{tabular}{|m{4cm}|m{4cm}|m{4cm}|}
    \hline
    \cellcolor{tableheaderblue}{\color{white}\textbf{Feature}} & 
    \cellcolor{tableheadergreen}{\color{white}\textbf{Kya Hai}} & 
    \cellcolor{tableheaderyellow}{\color{black}\textbf{Kab Use Karna}} \\
    \hline
    \rowcolor{codeblue}
    Layers & Har element ek layer & Complex designs manage karo \\
    \hline
    \rowcolor{tablerowgreen}
    Lock/Unlock & Layer ko fix/edit karo & Galti se edit rokne ke liye \\
    \hline
    \rowcolor{codeblue}
    Components & Reusable elements & Consistent design ke liye \\
    \hline
    \rowcolor{tablerowgreen}
    Assets & Components store karo & Drag-and-drop ke liye \\
    \hline
  \end{tabular}
  \caption{\textcolor{violet}{Layers aur Assets Overview}}
\end{table}

% Conclusion
\begin{notebox}
\textbf{\textcolor{warningred}{Important Baatein}:}
\begin{itemize}
  \item \textcolor{examplegreen}{Layers se designs ko organize karo, har element alag layer mein rakho.}
  \item \textcolor{examplegreen}{Lock/Unlock se important elements ko protect karo aur components banao reusable designs ke liye.}
  \item \textcolor{examplegreen}{Assets panel se components drag karke time bacha aur consistent design rakho.}
\end{itemize}
\end{notebox}

===============================
\hrule

\documentclass[a4paper]{article}
\usepackage[margin=2cm]{geometry}
\usepackage[utf8]{inputenc}
\usepackage{xcolor}
\usepackage{tcolorbox}
\tcbuselibrary{skins,breakable}
\usepackage{listings}
\usepackage{array, colortbl}
\usepackage{enumitem}
\usepackage{titlesec}
\usepackage{parskip}

% Colors define karte hain, vibrant vibe ke liye
\definecolor{headingblue}{RGB}{0,102,204}
\definecolor{examplegreen}{RGB}{0,153,76}
\definecolor{warningred}{RGB}{204,0,0}
\definecolor{codeblue}{RGB}{173,216,230}
\definecolor{tablerowgreen}{RGB}{144,238,144}
\definecolor{tableheaderblue}{RGB}{0,102,204}
\definecolor{tableheadergreen}{RGB}{0,153,76}
\definecolor{tableheaderyellow}{RGB}{255,204,0}
\definecolor{violet}{RGB}{128,0,128}

% Section ka style set karo
\titleformat{\section}{\Large\bfseries\color{headingblue}}{}{0em}{}

% Code blocks ke liye settings
\lstset{
  language=Python,
  backgroundcolor=\color{codeblue},
  basicstyle=\ttfamily\small,
  frame=single,
  breaklines=true,
  keywordstyle=\color{blue},
  stringstyle=\color{purple},
  commentstyle=\color{gray},
  showstringspaces=false
}

% Example aur note ke boxes
\newtcolorbox{examplebox}[1]{
  colback=examplegreen!10,
  colframe=examplegreen!50!black,
  title=#1,
  breakable,
  enhanced
}
\newtcolorbox{notebox}{
  colback=warningred!5!white,
  colframe=warningred!75!black,
  title=Important Baatein,
  breakable,
  enhanced
}

\begin{document}

% Topic: Save Figma, Import, Export
\section*{\textbf{\LARGE \textcolor{violet}{Save Figma, Import, aur Export}}}
\textcolor{warningred}{Figma mein apne designs ko save karna, import karna, aur export karna ek basic lekin zaroori skill hai.} Isse tu apna kaam safe rakh sakta hai, doosre tools ke saath integrate kar sakta hai, aur specific elements share kar sakta hai. Chalo, sab step-by-step dekhte hain!

\textbf{\textcolor{red}{Export Kaise Kare}:} \\
\textcolor{examplegreen}{Figma mein frames ya specific elements (jaise button) ko export kar sakta hai formats jaise PNG, JPG, SVG, ya PDF mein.} \\
\textbf{\textcolor{red}{Steps for Frames}:}
\begin{enumerate}
  \item \textcolor{examplegreen}{Layers panel (left side) mein ja, jo frames export karna chahta hai unko select kar.}
  \item \textcolor{examplegreen}{Right panel ke bottom mein \texttt{Export} section pe click kar.}
  \item \textcolor{examplegreen}{Format chun (jaise PNG, JPG, SVG) aur settings adjust kar (jaise resolution, 1x ya 2x). Phir \texttt{Export} button daba.}
\end{enumerate}
\textbf{\textcolor{red}{Steps for Specific Elements}:}
\begin{enumerate}
  \item \textcolor{examplegreen}{Agar sirf ek element, jaise button, export karna hai, toh usko canvas ya Layers panel mein select kar.}
  \item \textcolor{examplegreen}{Right panel ke \texttt{Export} section mein ja, format aur settings chun, aur \texttt{Export} click kar.}
\end{enumerate}
\textbf{\textcolor{red}{Kab Use Karna}}: \\
\textcolor{examplegreen}{Export kar jab designs ko share karna ho (jaise client ko PNG dikhana) ya developers ko assets dena ho (jaise SVG icons).}

\textbf{\textcolor{red}{Save as Figma File}:} \\
\textcolor{examplegreen}{Apne project ko Figma ke native format (.fig) mein save karna zaroori hai taaki baad mein edit kar sake.} \\
\textbf{\textcolor{red}{Kaise Kare}:}
\begin{enumerate}
  \item \textcolor{examplegreen}{Top-left hamburger icon pe click kar.}
  \item \textcolor{examplegreen}{\texttt{File > Save as .fig} select kar, file ka naam daal, aur save kar.}
\end{enumerate}
\textbf{\textcolor{red}{Kab Use Karna}}: \\
\textcolor{examplegreen}{Hamesha save kar jab project pe kaam khatam ho ya backup chahiye. Ye file Figma mein wapas khol sakta hai editing ke liye.}

\textbf{\textcolor{red}{Import Kaise Kare}:} \\
\textcolor{examplegreen}{Figma mein images, SVGs, ya doosre Figma files (.fig) import kar sakta hai apne project mein.} \\
\textbf{\textcolor{red}{Steps}:}
\begin{enumerate}
  \item \textcolor{examplegreen}{Hamburger menu se \texttt{File > Place Image} ya \texttt{File > Import File} chun.}
  \item \textcolor{examplegreen}{Apne computer se file (PNG, JPG, SVG, ya .fig) select kar aur canvas pe place kar.}
\end{enumerate}
\textbf{\textcolor{red}{Kab Use Karna}}: \\
\textcolor{examplegreen}{Import kar jab external assets (jaise logos, icons) ya purane Figma projects ko naye project mein use karna ho.}

\begin{examplebox}{\textcolor{violet}{Real-Life Example}}
\textcolor{red}{Ek travel app ke liye homepage frame banaya (375x812px).} Layers panel se frame select kiya aur \texttt{Export} se PNG format mein save kiya client ko dikhane ke liye. \textcolor{examplegreen}{Ek “Book Now” button alag se SVG mein export kiya developer ke liye.} Project ko \texttt{File > Save as .fig} se save kiya backup ke liye. Phir ek logo PNG import kiya (\texttt{File > Place Image}) aur frame ke header mein daala. Sab organized aur professional laga!
\end{examplebox}

% Table for Save, Import, Export
\begin{table}[h]
  \centering
  \begin{tabular}{|m{4cm}|m{4cm}|m{4cm}|}
    \hline
    \cellcolor{tableheaderblue}{\color{white}\textbf{Feature}} & 
    \cellcolor{tableheadergreen}{\color{white}\textbf{Kya Hai}} & 
    \cellcolor{tableheaderyellow}{\color{black}\textbf{Kab Use Karna}} \\
    \hline
    \rowcolor{codeblue}
    Export Frames & Frames ko PNG, SVG mein save & Client ya developer ke liye \\
    \hline
    \rowcolor{tablerowgreen}
    Export Elements & Specific item export karo & Assets jaise buttons, icons \\
    \hline
    \rowcolor{codeblue}
    Save .fig & Figma file save karo & Backup aur editing ke liye \\
    \hline
    \rowcolor{tablerowgreen}
    Import & Images ya .fig files add karo & External assets ke liye \\
    \hline
  \end{tabular}
  \caption{\textcolor{violet}{Save, Import, Export Overview}}
\end{table}

% Conclusion
\begin{notebox}
\textbf{\textcolor{warningred}{Important Baatein}:}
\begin{itemize}
  \item \textcolor{examplegreen}{Frames ya specific elements ko \texttt{Export} se PNG, SVG, etc. mein save karo sharing ke liye.}
  \item \textcolor{examplegreen}{\texttt{Save as .fig} se project ka backup rakho taaki baad mein edit kar sake.}
  \item \textcolor{examplegreen}{Images ya Figma files import karo \texttt{File > Place Image} ya \texttt{Import File} se apne designs mein.}
\end{itemize}
\end{notebox}


===============================
\hrule



% Colors define karte hain, vibrant vibe ke liye
\definecolor{headingblue}{RGB}{0,102,204}
\definecolor{examplegreen}{RGB}{0,153,76}
\definecolor{warningred}{RGB}{204,0,0}
\definecolor{codeblue}{RGB}{173,216,230}
\definecolor{tablerowgreen}{RGB}{144,238,144}
\definecolor{tableheaderblue}{RGB}{0,102,204}
\definecolor{tableheadergreen}{RGB}{0,153,76}
\definecolor{tableheaderyellow}{RGB}{255,204,0}
\definecolor{violet}{RGB}{128,0,128}

% Section ka style set karo
\titleformat{\section}{\Large\bfseries\color{headingblue}}{}{0em}{}

% Code blocks ke liye settings
\lstset{
  language=Python,
  backgroundcolor=\color{codeblue},
  basicstyle=\ttfamily\small,
  frame=single,
  breaklines=true,
  keywordstyle=\color{blue},
  stringstyle=\color{purple},
  commentstyle=\color{gray},
  showstringspaces=false
}

% Example aur note ke boxes
\newtcolorbox{examplebox}[1]{
  colback=examplegreen!10,
  colframe=examplegreen!50!black,
  title=#1,
  breakable,
  enhanced
}
\newtcolorbox{notebox}{
  colback=warningred!5!white,
  colframe=warningred!75!black,
  title=Important Baatein,
  breakable,
  enhanced
}

\begin{document}

% Topic: Version Control in Figma
\section*{\textbf{\LARGE \textcolor{violet}{Version Control in Figma}}}
\textcolor{warningred}{Figma ka version control feature tujhe designs ke purane versions track karne aur wapas jaane ka option deta hai.} Ye feature cloud-based hai, jo team collaboration aur design iterations ke liye ekdum perfect hai. Chalo, detail mein samajhte hain!

\textbf{\textcolor{red}{Version Control Kya Hai aur Kyun Use Karna?}} \\
\textcolor{examplegreen}{Version control Figma mein ek timeline banata hai jisme file ke saare changes save hote hain, jaise autosaves aur manually saved versions.} Isse tu purane designs dekh sakta hai, restore kar sakta hai, ya duplicate bana sakta hai. \\
\textbf{\textcolor{red}{Kyun aur Kab Use Karna}:}
\begin{itemize}
  \item \textcolor{examplegreen}{Track Changes: Team mein kaun kya change karta hai, uska record rakhta hai.}
  \item \textcolor{examplegreen}{Revert Back: Agar galti se kuch delete ho gaya ya purana design better tha, toh wapas ja sakta hai.}
  \item \textcolor{examplegreen}{Collaboration: Multiple designers ek saath kaam karte hain, toh version history se conflicts avoid hote hain.}
  \item \textcolor{examplegreen}{Kab Use Karna: Jab bada project ho, client feedback ke liye milestones save karna ho, ya developer handoff ke liye specific version chahiye.} Example: Ek app redesign ke liye har iteration save karo taaki client purana design compare kar sake.
\end{itemize}

\textbf{\textcolor{red}{Save to Version History Kaise Kare}:} \\
\textcolor{examplegreen}{Figma automatically 30 minutes ke inactivity ke baad autosave karta hai, lekin tu manually bhi versions save kar sakta hai.} \\
\textbf{\textcolor{red}{Steps}:}
\begin{enumerate}
  \item \textcolor{examplegreen}{Top-left hamburger icon pe click kar.}
  \item \textcolor{examplegreen}{\texttt{File > Save to Version History} select kar.}
  \item \textcolor{examplegreen}{Ek pop-up aayega, jisme version ka title aur description daal sakta hai (jaise “Homepage v2, Added Buttons”). Phir \texttt{Save} click kar.}
\end{enumerate}
\textbf{\textcolor{red}{Shortcut}:} \textcolor{examplegreen}{Mac pe \texttt{Cmd + Option + S} ya Windows pe \texttt{Ctrl + Alt + S} use kar.}[](https://help.figma.com/article/86-version-history)

\textbf{\textcolor{red}{Version History Kaise Dekhe}:} \\
\textcolor{examplegreen}{Purane versions dekhne ke liye Version History panel khol sakta hai.} \\
\textbf{\textcolor{red}{Steps}:}
\begin{enumerate}
  \item \textcolor{examplegreen}{Hamburger menu se \texttt{File > Show Version History} select kar.}
  \item \textcolor{examplegreen}{Right sidebar mein Version History panel khulega, jisme autosaved aur manually saved versions ki list hogi.}
  \item \textcolor{examplegreen}{Kisi version pe click karo toh uska snapshot dikhega. \texttt{Show older} se aur purani versions dekh sakta hai.}
\end{enumerate}

\textbf{\textcolor{red}{Versions ke Saath Kya Kar Sakta Hai}:} \\
\begin{itemize}
  \item \textbf{\textcolor{warningred}{Restore Version}}: \textcolor{examplegreen}{Version pe right-click karo aur \texttt{Restore This Version} chuno. Ye file ko us state mein wapas le aayega, aur current version bhi save ho jayega.} Kab Use Karna: Jab galti se changes hata diye ya purana design wapas chahiye.
  \item \textbf{\textcolor{warningred}{Duplicate Version}}: \textcolor{examplegreen}{Version pe right-click karo aur \texttt{Duplicate} select karo. Ye ek nayi file banayega us version se.} Kab Use Karna: Developer ko specific version dena ho ya naye iteration ke liye starting point chahiye.
  \item \textbf{\textcolor{warningred}{Name Version}}: \textcolor{examplegreen}{Version pe right-click karo aur \texttt{Name This Version} chuno, title aur description daal do.} Kab Use Karna: Major milestones ko identify karne ke liye, jaise “Client Review v1”.
\end{itemize}

\textbf{\textcolor{red}{Export aur Import Figma File for Version Control}:} \\
\textcolor{examplegreen}{Agar tu Figma file ko export karke version control ke liye save karna chahta hai ya import karna hai, ye bhi possible hai.} \\
\textbf{\textcolor{red}{Export Steps}:}
\begin{enumerate}
  \item \textcolor{examplegreen}{Hamburger menu se \texttt{File > Save as .fig} select kar taaki Figma file download ho.}
  \item \textcolor{examplegreen}{Is file ko external storage (jaise Google Drive) pe save kar sakta hai version control ke liye.}
\end{enumerate}
\textbf{\textcolor{red}{Import Steps}:}
\begin{enumerate}
  \item \textcolor{examplegreen}{Figma mein naye project mein jao, hamburger menu se \texttt{File > Import File} select kar.}
  \item \textcolor{examplegreen}{Downloaded .fig file ko drag-and-drop kar ya select kar taaki wo Figma mein khul jaye.}
\end{enumerate}
\textbf{\textcolor{red}{Kab Use Karna}}: \textcolor{examplegreen}{Export kar jab offline backup chahiye ya doosre team members ke saath .fig file share karna ho. Import kar jab purani .fig file ko naye project mein use karna ho.}

\begin{examplebox}{\textcolor{violet}{Real-Life Example}}
\textcolor{red}{Ek e-commerce app ke liye homepage design kiya.} Har major change ke baad \texttt{File > Save to Version History} se version save kiya, jaise “v1: Initial Layout” aur “v2: Added Filters”. \textcolor{examplegreen}{Client ne v1 wapas mangaya, toh Version History se usko \texttt{Restore} kiya.} Phir final design ko \texttt{Save as .fig} se export karke Google Drive pe save kiya. Developer ke liye ek purana version import kiya (\texttt{File > Import File}) taaki uspe kaam shuru ho sake. Sab changes track hue aur kaam smooth raha!
\end{examplebox}

% Table for Version Control
\begin{table}[h]
  \centering
  \begin{tabular}{|m{4cm}|m{4cm}|m{4cm}|}
    \hline
    \cellcolor{tableheaderblue}{\color{white}\textbf{Feature}} & 
    \cellcolor{tableheadergreen}{\color{white}\textbf{Kya Hai}} & 
    \cellcolor{tableheaderyellow}{\color{black}\textbf{Kab Use Karna}} \\
    \hline
    \rowcolor{codeblue}
    Save to Version & Manual version save & Milestones ke liye \\
    \hline
    \rowcolor{tablerowgreen}
    Autosave & 30 min ke baad save & Automatic backup \\
    \hline
    \rowcolor{codeblue}
    Restore Version & Purana version wapas & Galti theek karne ke liye \\
    \hline
    \rowcolor{tablerowgreen}
    Duplicate Version & Nayi file banao & Handoff ya iteration ke liye \\
    \hline
    \rowcolor{codeblue}
    Export/Import .fig & File save/import karo & Offline backup ya sharing \\
    \hline
  \end{tabular}
  \caption{\textcolor{violet}{Version Control Overview}}
\end{table}

% Conclusion
\begin{notebox}
\textbf{\textcolor{warningred}{Important Baatein}:}
\begin{itemize}
  \item \textcolor{examplegreen}{\texttt{Save to Version History} se major changes save karo aur title/description daalo.}
  \item \textcolor{examplegreen}{Version History se purane versions dekh, restore, ya duplicate kar sakta hai.}
  \item \textcolor{examplegreen}{Export .fig file for offline backups aur import karo purane designs ke liye.}
  \item \textcolor{examplegreen}{Note: Free plan mein sirf 30 din ka version history milta hai, paid plan mein full history.}[](https://help.figma.com/hc/en-us/articles/360038006754-Version-History)
\end{itemize}
\end{notebox}

===============================
\hrule



% Colors define karte hain, vibrant vibe ke liye
\definecolor{headingblue}{RGB}{0,102,204}
\definecolor{examplegreen}{RGB}{0,153,76}
\definecolor{warningred}{RGB}{204,0,0}
\definecolor{codeblue}{RGB}{173,216,230}
\definecolor{tablerowgreen}{RGB}{144,238,144}
\definecolor{tableheaderblue}{RGB}{0,102,204}
\definecolor{tableheadergreen}{RGB}{0,153,76}
\definecolor{tableheaderyellow}{RGB}{255,204,0}
\definecolor{violet}{RGB}{128,0,128}

% Section ka style set karo
\titleformat{\section}{\Large\bfseries\color{headingblue}}{}{0em}{}

% Code blocks ke liye settings
\lstset{
  language=Python,
  backgroundcolor=\color{codeblue},
  basicstyle=\ttfamily\small,
  frame=single,
  breaklines=true,
  keywordstyle=\color{blue},
  stringstyle=\color{purple},
  commentstyle=\color{gray},
  showstringspaces=false
}

% Example aur note ke boxes
\newtcolorbox{examplebox}[1]{
  colback=examplegreen!10,
  colframe=examplegreen!50!black,
  title=#1,
  breakable,
  enhanced
}
\newtcolorbox{notebox}{
  colback=warningred!5!white,
  colframe=warningred!75!black,
  title=Important Baatein,
  breakable,
  enhanced
}

\begin{document}

% Topic: Prototyping in Figma
\section*{\textbf{\LARGE \textcolor{violet}{Prototyping in Figma}}}
\textcolor{warningred}{Figma ka prototyping feature tujhe bina code likhe interactive designs banane deta hai, jisme tu client ko app ya website ka real feel dikha sakta hai.} Jaise, ek button click karne pe doosra frame (navbar) khulta hai. Ye client presentations ke liye ekdum important hai kyunki animations aur transitions se design zinda lagta hai. Chalo, step-by-step sab samajhte hain!

\textbf{\textcolor{red}{Prototyping Kya Hai aur Kab Use Karna?}} \\
\textcolor{examplegreen}{Prototyping mein tu apne static frames ko interactive banata hai, jisme user actions (jaise click, hover) se frames ya elements change hote hain.} \\
\textbf{\textcolor{red}{Kyun aur Kab Use Karna}:}
\begin{itemize}
  \item \textcolor{examplegreen}{User Experience Test: Client ya users ko dikhao ki app ya website ka flow kaisa hoga.}
  \item \textcolor{examplegreen}{Client Approval: Static designs ke bajaye interactive prototype se client ko impress karo.}
  \item \textcolor{examplegreen}{Developer Handoff: Developers ko samjhao ki animations aur transitions kaise kaam karenge.}
  \item \textcolor{examplegreen}{Kab Use Karna: Jab app screens ke navigation flow, button interactions, ya hover effects dikhane ho.} Example: Ek e-commerce app mein “Add to Cart” button click pe cart screen khulta hai.
\end{itemize}

\textbf{\textcolor{red}{Prototyping Kaise Banaye}:} \\
\textcolor{examplegreen}{Figma ke Prototype tab se tu frames ko connect karke interactions add kar sakta hai.} \\
\textbf{\textcolor{red}{Steps}:}
\begin{enumerate}
  \item \textcolor{examplegreen}{Right panel mein \texttt{Prototype} tab pe click kar (Design tab ke bagal mein). Canvas pe sab frames ke edges pe blue circles dikheinge.}
  \item \textcolor{examplegreen}{Ek element (jaise button) ke blue circle se drag karke us frame tak arrow khincho jisko dikhana hai (jaise navbar frame).}
  \item \textcolor{examplegreen}{Arrow pe click karo, right panel mein \texttt{Interaction} section khulega. Yahan interaction type select karo (dropdown se).}
\end{enumerate}

\textbf{\textcolor{red}{Interaction Types (Dropdown mein Options)}:} \\
\textcolor{examplegreen}{Ye sab user actions hain jo prototype mein trigger karte hain:} \\
\begin{itemize}
  \item \textbf{\textcolor{warningred}{onClick}}: \textcolor{examplegreen}{Jab user element pe click karta hai, kuch hota hai (jaise doosra frame khulta hai).} Kab Use Karna: Buttons, links, ya navigation items ke liye, jaise “Login” button click pe login screen.
  \item \textbf{\textcolor{warningred}{onDrag}}: \textcolor{examplegreen}{Jab user element ko drag karta hai, action trigger hota hai.} Kab Use Karna: Swipeable galleries ya draggable cards ke liye.
  \item \textbf{\textcolor{warningred}{whileHovering}}: \textcolor{examplegreen}{Jab cursor element pe hover karta hai, effect hota hai.} Kab Use Karna: Hover effects jaise button ka color change ya tooltip dikhana.
  \item \textbf{\textcolor{warningred}{whilePressing}}: \textcolor{examplegreen}{Jab user element ko press karta hai (click hold), action hota hai.} Kab Use Karna: Long-press effects ke liye, jaise hold karke menu khulna.
  \item \textbf{\textcolor{warningred}{mouseEnter}}: \textcolor{examplegreen}{Jab cursor element ke andar aata hai, action trigger hota hai.} Kab Use Karna: Subtle hover animations ke liye, jaise image zoom.
  \item \textbf{\textcolor{warningred}{mouseLeave}}: \textcolor{examplegreen}{Jab cursor element se bahar jata hai, action hota hai.} Kab Use Karna: Hover effect band karne ke liye, jaise tooltip gayab hona.
  \item \textbf{\textcolor{warningred}{mouseDown}}: \textcolor{examplegreen}{Jab mouse button dabaya jata hai, action hota hai.} Kab Use Karna: Pressed state dikhane ke liye, jaise button ka shadow change.
  \item \textbf{\textcolor{warningred}{mouseUp}}: \textcolor{examplegreen}{Jab mouse button chhoda jata hai, action hota hai.} Kab Use Karna: Click release effects ke liye, jaise animation complete hona.
\end{itemize}

\textbf{\textcolor{red}{Smart Animate Kya Hai?}} \\
\textcolor{examplegreen}{Smart Animate ek Figma feature hai jo frames ke beech smooth transitions aur animations banata hai.} Ye automatically similar elements (jaise buttons, text) ko detect karta hai aur unke size, position, ya color changes ko animate karta hai. \\
\textbf{\textcolor{red}{Kaise Use Kare}:}
\begin{enumerate}
  \item \textcolor{examplegreen}{Interaction set karne ke baad, \texttt{Interaction} section mein \texttt{Animation} dropdown se \texttt{Smart Animate} select kar.}
  \item \textcolor{examplegreen}{Duration aur easing (jaise Ease In, Ease Out) set kar taaki animation smooth lage.}
\end{enumerate}
\textbf{\textcolor{red}{Kab Use Karna}:}
\begin{itemize}
  \item \textcolor{examplegreen}{Jab professional transitions chahiye, jaise ek button shrink hoke doosre frame mein bada ho.}
  \item \textcolor{examplegreen}{Common Use Cases: Screen transitions, menu slide-ins, ya button state changes (normal se hover).}
  \item \textcolor{examplegreen}{Example: Homepage se cart screen tak jate waqt button ka size aur position smooth animate hota hai.}
\end{itemize}

\textbf{\textcolor{red}{Prototype Kaise Play Kare}:} \\
\textcolor{examplegreen}{Prototype ko test karne ke liye play kar sakta hai taaki client ko real-time flow dikhe.} \\
\textbf{\textcolor{red}{Steps}:}
\begin{enumerate}
  \item \textcolor{examplegreen}{Right panel ke top-right mein \texttt{Play} symbol (triangle) pe click kar.}
  \item \textcolor{examplegreen}{Ek preview window khulegi jisme tu prototype ke interactions (click, hover, etc.) test kar sakta hai.}
  \item \textcolor{examplegreen}{Client ko dikhane ke liye preview window ka link share kar sakta hai ya full-screen mode mein present karo.}
\end{enumerate}

\begin{examplebox}{\textcolor{violet}{Real-Life Example}}
\textcolor{red}{Ek food delivery app ke liye homepage aur navbar frames banaye.} Prototype tab mein homepage ke “Menu” button se navbar frame tak arrow drag kiya. \textcolor{examplegreen}{Interaction mein \texttt{onClick} set kiya aur \texttt{Smart Animate} chuna taaki navbar slide-in effect ke saath khule.} Phir button pe \texttt{whileHovering} add kiya taaki hover pe color change ho. \texttt{Play} symbol click karke prototype test kiya—button click pe navbar smoothly khula. Client ko preview link diya, unhone flow dekha aur ekdum impress hue!
\end{examplebox}

% Table for Prototyping Features
\begin{table}[h]
  \centering
  \begin{tabular}{|m{4cm}|m{4cm}|m{4cm}|}
    \hline
    \cellcolor{tableheaderblue}{\color{white}\textbf{Feature}} & 
    \cellcolor{tableheadergreen}{\color{white}\textbf{Kya Hai}} & 
    \cellcolor{tableheaderyellow}{\color{black}\textbf{Kab Use Karna}} \\
    \hline
    \rowcolor{codeblue}
    onClick & Click pe action & Buttons, links ke liye \\
    \hline
    \rowcolor{tablerowgreen}
    whileHovering & Hover pe effect & Tooltips, color change \\
    \hline
    \rowcolor{codeblue}
    Smart Animate & Smooth transitions & Screen ya element changes \\
    \hline
    \rowcolor{tablerowgreen}
    Play Prototype & Preview animations & Client presentations \\
    \hline
  \end{tabular}
  \caption{\textcolor{violet}{Prototyping Overview}}
\end{table}

% Conclusion
\begin{notebox}
\textbf{\textcolor{warningred}{Important Baatein}:}
\begin{itemize}
  \item \textcolor{examplegreen}{Prototype tab se frames connect karo aur interactions jaise \texttt{onClick}, \texttt{whileHovering} set karo.}
  \item \textcolor{examplegreen}{\texttt{Smart Animate} se smooth transitions banao taaki design professional lage.}
  \item \textcolor{examplegreen}{\texttt{Play} symbol se prototype test karo aur client ko link share karo.}
  \item \textcolor{examplegreen}{Note: Complex animations ke liye multiple interactions aur smart animate ka combo use karo.}
\end{itemize}
\end{notebox}


===============================
\hrule



% Colors define karte hain, vibrant vibe ke liye
\definecolor{headingblue}{RGB}{0,102,204}
\definecolor{examplegreen}{RGB}{0,153,76}
\definecolor{warningred}{RGB}{204,0,0}
\definecolor{codeblue}{RGB}{173,216,230}
\definecolor{tablerowgreen}{RGB}{144,238,144}
\definecolor{tableheaderblue}{RGB}{0,102,204}
\definecolor{tableheadergreen}{RGB}{0,153,76}
\definecolor{tableheaderyellow}{RGB}{255,204,0}
\definecolor{violet}{RGB}{128,0,128}

% Section ka style set karo
\titleformat{\section}{\Large\bfseries\color{headingblue}}{}{0em}{}

% Code blocks ke liye settings
\lstset{
  language=Python,
  backgroundcolor=\color{codeblue},
  basicstyle=\ttfamily\small,
  frame=single,
  breaklines=true,
  keywordstyle=\color{blue},
  stringstyle=\color{purple},
  commentstyle=\color{gray},
  showstringspaces=false
}

% Example aur note ke boxes
\newtcolorbox{examplebox}[1]{
  colback=examplegreen!10,
  colframe=examplegreen!50!black,
  title=#1,
  breakable,
  enhanced
}
\newtcolorbox{notebox}{
  colback=warningred!5!white,
  colframe=warningred!75!black,
  title=Important Baatein,
  breakable,
  enhanced
}

\begin{document}

% Topic: Creating Styles in Figma
\section*{\textbf{\LARGE \textcolor{violet}{Creating Styles in Figma}}}
\textcolor{warningred}{Figma mein styles ek powerful feature hai jo tujhe consistent colors, fonts, aur effects project mein use karne deta hai.} Ek baar style bana diya, toh ek jagah change karne se sab jagah update ho jata hai. Ye time bachata hai aur design ko professional banata hai. Chalo, step-by-step samajhte hain!

\textbf{\textcolor{red}{Styles Kya Hain aur Kyun Use Karna?}} \\
\textcolor{examplegreen}{Styles predefined settings hain jo colors, text properties (font, size), effects, ya layouts ke liye banaye jaate hain.} Inka sabse bada fayda ye hai ki agar kal ko color ya font change karna ho, toh style edit karo aur pura project update ho jayega. \\
\textbf{\textcolor{red}{Kyun aur Kab Use Karna}:}
\begin{itemize}
  \item \textcolor{examplegreen}{Consistency: Sab buttons, texts, ya elements mein same color ya font rakho taaki design ek jaisa lage.}
  \item \textcolor{examplegreen}{Efficiency: Ek jagah style change karo, sab jagah apply ho jata hai, individually edit karne ka jhanjhat nahi.}
  \item \textcolor{examplegreen}{Team Collaboration: Team ke saath styles share karo taaki sab same design system follow karein.}
  \item \textcolor{examplegreen}{Kab Use Karna: Jab bade projects mein consistent branding chahiye, jaise app ke liye primary color ya website ke liye typography.} Example: Ek app mein blue color style banao, sab buttons mein apply karo, aur baad mein green karne ke liye style edit karo.
\end{itemize}

\textbf{\textcolor{red}{Color Style Kaise Banaye}:} \\
\textcolor{examplegreen}{Color styles se tu ek specific color define karta hai jo project ke har element pe apply ho sakta hai.} \\
\textbf{\textcolor{red}{Steps}:}
\begin{enumerate}
  \item \textcolor{examplegreen}{Right panel mein \texttt{Design} tab kholo.}
  \item \textcolor{examplegreen}{\texttt{Fill} section mein jao, aur \texttt{::} symbol (style icon) pe click karo.}
  \item \textcolor{examplegreen}{\texttt{+} button pe click karo, ek naya style create hoga.}
  \item \textcolor{examplegreen}{Style ka naam daal (jaise “Primary Blue”), color select kar (RGB, HEX, ya color picker se), aur save kar.}
\end{enumerate}
\textbf{\textcolor{red}{Kaise Apply Kare}:}
\begin{enumerate}
  \item \textcolor{examplegreen}{Element select karo jisme style apply karna hai, jaise ek button.}
  \item \textcolor{examplegreen}{\texttt{Design} tab mein \texttt{Fill} section ke \texttt{::} symbol pe click karo, saved style (jaise “Primary Blue”) chuno.}
\end{enumerate}
\textbf{\textcolor{red}{Kab Use Karna}}: \textcolor{examplegreen}{Jab har element mein same color chahiye, jaise buttons, backgrounds, ya icons ke liye.}

\textbf{\textcolor{red}{Text Style Kaise Banaye}:} \\
\textcolor{examplegreen}{Text styles se tu font, size, weight, line height, ya spacing define karta hai taaki project mein typography consistent rahe.} \\
\textbf{\textcolor{red}{Steps}:}
\begin{enumerate}
  \item \textcolor{examplegreen}{Ek text box banao aur desired font properties set karo (jaise font: Inter, size: 16px, weight: Bold).}
  \item \textcolor{examplegreen}{\texttt{Design} tab mein \texttt{Text} section ke \texttt{::} symbol pe click karo.}
  \item \textcolor{examplegreen}{\texttt{+} button pe click karo, style ka naam daal (jaise “Body Text”), aur save kar.}
\end{enumerate}
\textbf{\textcolor{red}{Kaise Apply Kare}:}
\begin{enumerate}
  \item \textcolor{examplegreen}{Text element select karo jisme style apply karna hai.}
  \item \textcolor{examplegreen}{\texttt{Design} tab mein \texttt{Text} section ke \texttt{::} symbol pe click karo, saved style (jaise “Body Text”) chuno.}
\end{enumerate}
\textbf{\textcolor{red}{Kab Use Karna}}: \textcolor{examplegreen}{Jab headings, paragraphs, ya buttons ke text mein same font style chahiye, jaise app ke liye consistent typography.}

\textbf{\textcolor{red}{Style Management}:} \\
\textcolor{examplegreen}{Figma mein styles ko edit, delete, ya organize bhi kar sakta hai taaki workflow smooth rahe.} \\
\begin{itemize}
  \item \textbf{\textcolor{warningred}{Edit Style}}: \textcolor{examplegreen}{Style panel (\texttt{::} symbol) mein style pe right-click karo, \texttt{Edit} chuno, aur color ya font properties change karo. Sab linked elements update ho jayenge.} Kab Use Karna: Jab branding change ho, jaise blue se green color.
  \item \textbf{\textcolor{warningred}{Delete Style}}: \textcolor{examplegreen}{Style pe right-click karo aur \texttt{Delete} chuno.} Kab Use Karna: Jab style ab use nahi hota.
  \item \textbf{\textcolor{warningred}{Organize Styles}}: \textcolor{examplegreen}{Style names mein slashes use karo (jaise “Colors/Primary Blue”) taaki categories ban jayein.} Kab Use Karna: Bade projects mein styles ko group karne ke liye.
  \item \textbf{\textcolor{warningred}{Team Library}}: \textcolor{examplegreen}{Styles ko team library mein publish karo (\texttt{::} > \texttt{Publish Styles}) taaki team ke sab members use kar sakein.} Kab Use Karna: Collaborative projects mein.
\end{itemize}

\begin{examplebox}{\textcolor{violet}{Real-Life Example}}
\textcolor{red}{Ek fitness app ke liye design kiya.} Color style banaya “Primary Green” (\texttt{Design > Fill > :: > +}, HEX: #00FF00) aur sab buttons, icons pe apply kiya. Text style banaya “Heading” (Inter, 24px, Bold) aur sab headings pe apply kiya. \textcolor{examplegreen}{Client ne green ko blue chahiye bola, toh style edit kiya (“Primary Green” ko “Primary Blue” banaya), aur pura project ekdum update ho gaya.} Team library mein styles publish kiye taaki doosre designers bhi use karein. Time bacha aur design consistent raha!
\end{examplebox}

% Table for Styles Features
\begin{table}[h]
  \centering
  \begin{tabular}{|m{4cm}|m{4cm}|m{4cm}|}
    \hline
    \cellcolor{tableheaderblue}{\color{white}\textbf{Feature}} & 
    \cellcolor{tableheadergreen}{\color{white}\textbf{Kya Hai}} & 
    \cellcolor{tableheaderyellow}{\color{black}\textbf{Kab Use Karna}} \\
    \hline
    \rowcolor{codeblue}
    Color Style & Ek color define karo & Consistent colors ke liye \\
    \hline
    \rowcolor{tablerowgreen}
    Text Style & Font properties set karo & Consistent typography ke liye \\
    \hline
    \rowcolor{codeblue}
    Edit Style & Style update karo & Branding changes ke liye \\
    \hline
    \rowcolor{tablerowgreen}
    Team Library & Styles share karo & Team collaboration ke liye \\
    \hline
  \end{tabular}
  \caption{\textcolor{violet}{Styles Overview}}
\end{table}

% Conclusion
\begin{notebox}
\textbf{\textcolor{warningred}{Important Baatein}:}
\begin{itemize}
  \item \textcolor{examplegreen}{Color aur text styles banao taaki project mein consistency rahe aur time bache.}
  \item \textcolor{examplegreen}{\texttt{Design > :: > +} se styles create karo aur elements pe apply karo.}
  \item \textcolor{examplegreen}{Style edit karo ek jagah, sab jagah update ho jayega, jaise color ya font change.}
  \item \textcolor{examplegreen}{Team library mein styles publish karo collaborative projects ke liye.}
\end{itemize}
\end{notebox}

===============================
\hrule



% Colors define karte hain, vibrant vibe ke liye
\definecolor{headingblue}{RGB}{0,102,204}
\definecolor{examplegreen}{RGB}{0,153,76}
\definecolor{warningred}{RGB}{204,0,0}
\definecolor{codeblue}{RGB}{173,216,230}
\definecolor{tablerowgreen}{RGB}{144,238,144}
\definecolor{tableheaderblue}{RGB}{0,102,204}
\definecolor{tableheadergreen}{RGB}{0,153,76}
\definecolor{tableheaderyellow}{RGB}{255,204,0}
\definecolor{violet}{RGB}{128,0,128}

% Section ka style set karo
\titleformat{\section}{\Large\bfseries\color{headingblue}}{}{0em}{}

% Code blocks ke liye settings
\lstset{
  language=Python,
  backgroundcolor=\color{codeblue},
  basicstyle=\ttfamily\small,
  frame=single,
  breaklines=true,
  keywordstyle=\color{blue},
  stringstyle=\color{purple},
  commentstyle=\color{gray},
  showstringspaces=false
}

% Example aur note ke boxes
\newtcolorbox{examplebox}[1]{
  colback=examplegreen!10,
  colframe=examplegreen!50!black,
  title=#1,
  breakable,
  enhanced
}
\newtcolorbox{notebox}{
  colback=warningred!5!white,
  colframe=warningred!75!black,
  title=Important Baatein,
  breakable,
  enhanced
}

\begin{document}

% Topic: Responsive Design in Figma
\section*{\textbf{\LARGE \textcolor{violet}{Responsive Design in Figma}}}
\textcolor{warningred}{Responsive design ka matlab hai ek aisa design jo har device—mobile, tablet, ya desktop—pe perfect dikhe aur kaam kare.} Figma mein scale tool, constraints, aur auto layout jaise features se tu responsive designs bana sakta hai. Ye client ke liye important hai kyunki aajkal users alag-alag screen sizes use karte hain. Chalo, step-by-step samajhte hain!

\textbf{\textcolor{red}{Responsive Design Kya Hai aur Kyun Use Karna?}} \\
\textcolor{examplegreen}{Responsive design ensure karta hai ki tera UI different screen sizes aur resolutions pe consistent aur user-friendly rahe.} \\
\textbf{\textcolor{red}{Kyun aur Kab Use Karna}:}
\begin{itemize}
  \item \textcolor{examplegreen}{Cross-Device Compatibility: Ek hi design mobile (375px) se desktop (1440px) tak adjust ho jata hai.}
  \item \textcolor{examplegreen}{User Experience: Users ko har device pe smooth navigation aur readable content milta hai.}
  \item \textcolor{examplegreen}{Client Expectations: Clients expect karte hain ki design sab devices pe kaam kare bina extra effort ke.}
  \item \textcolor{examplegreen}{Kab Use Karna: Jab app ya website ke liye multiple devices target kar rahe ho, jaise e-commerce app mobile aur desktop ke liye.} Example: Ek button mobile pe chhota aur desktop pe bada dikhe, lekin layout consistent rahe.
\end{itemize}

\textbf{\textcolor{red}{Scale Tool Kya Hai aur Kaise Use Kare}:} \\
\textcolor{examplegreen}{Scale tool se tu pura frame ya group ek saath proportionally resize kar sakta hai, jisse design different screen sizes ke liye adjust ho jata hai.} \\
\textbf{\textcolor{red}{Steps}:}
\begin{enumerate}
  \item \textcolor{examplegreen}{Frame ya group select karo jisko scale karna hai (jaise ek app screen).}
  \item \textcolor{examplegreen}{Top toolbar mein \texttt{Scale} tool chuno (shortcut: \texttt{K}) ya right panel mein \texttt{Scale} option pe jao.}
  \item \textcolor{examplegreen}{Drag karke size adjust karo ya percentage enter karo (jaise 50\% chhota ya 200\% bada).}
\end{enumerate}
\textbf{\textcolor{red}{Kab Use Karna}}: \textcolor{examplegreen}{Jab pura design uniformly resize karna ho, jaise mobile frame ko tablet size mein badalna.} Note: Scale tool proportions maintain karta hai, lekin layout ke andar elements ka alignment alag se set karna padta hai.

\textbf{\textcolor{red}{Constraints Kaise Use Kare}:} \\
\textcolor{examplegreen}{Constraints control karte hain ki frame resize hone pe elements kaise behave karenge, jaise center mein rahenge ya edges se fixed rahenge.} \\
\textbf{\textcolor{red}{Steps}:}
\begin{enumerate}
  \item \textcolor{examplegreen}{Element select karo (jaise button ya text) jo frame ke andar hai.}
  \item \textcolor{examplegreen}{Right panel ke \texttt{Constraints} section mein jao.}
  \item \textcolor{examplegreen}{Horizontal aur vertical constraints set karo (jaise \texttt{Center} for both, ya \texttt{Left and Top} for fixed position).}
\end{enumerate}
\textbf{\textcolor{red}{Kab Use Karna}}: \textcolor{examplegreen}{Jab elements ko specific position mein rakna ho, jaise button hamesha frame ke center mein ya top-right corner pe.} Example: Ek logo ko \texttt{Center} set karo taaki frame resize hone pe wo middle mein rahe.

\textbf{\textcolor{red}{Auto Layout for Responsive Design}:} \\
\textcolor{examplegreen}{Auto layout ek powerful feature hai jo elements ko automatically adjust karta hai based on content aur screen size.} \\
\textbf{\textcolor{red}{Steps}:}
\begin{enumerate}
  \item \textcolor{examplegreen}{Elements (jaise buttons, text) select karo aur group karo.}
  \item \textcolor{examplegreen}{Right panel mein \texttt{Auto Layout} button pe click karo (shortcut: \texttt{Shift + A}).}
  \item \textcolor{examplegreen}{Padding, spacing, aur direction (horizontal ya vertical) set karo.}
  \item \textcolor{examplegreen}{Resize frame aur dekho kaise elements apne aap adjust hote hain.}
\end{enumerate}
\textbf{\textcolor{red}{Kab Use Karna}}: \textcolor{examplegreen}{Jab dynamic layouts chahiye, jaise ek card jo content ke hisaab se stretch ho ya shrink ho.} Example: Ek product card mein image, title, aur button auto layout mein rakho, content badalne pe layout apne aap adjust ho jata hai.

\textbf{\textcolor{red}{Breakpoints aur Testing}:} \\
\textcolor{examplegreen}{Breakpoints ke liye alag-alag screen sizes ke frames banao aur prototype mein test karo.} \\
\textbf{\textcolor{red}{Steps}:}
\begin{enumerate}
  \item \textcolor{examplegreen}{Multiple frames banao, jaise mobile (375x812px), tablet (768x1024px), aur desktop (1440x900px).}
  \item \textcolor{examplegreen}{Har frame mein design adjust karo using scale tool, constraints, aur auto layout.}
  \item \textcolor{examplegreen}{Prototype tab mein frames connect karo aur \texttt{Play} symbol se preview karo taaki responsive flow dekho.}
\end{enumerate}
\textbf{\textcolor{red}{Kab Use Karna}}: \textcolor{examplegreen}{Jab client ko dikhana ho ki design har device pe kaise dikhega ya developers ko breakpoints ke liye reference dena ho.}

\begin{examplebox}{\textcolor{violet}{Real-Life Example}}
\textcolor{red}{Ek e-commerce website ke liye homepage design kiya.} Mobile frame (375px) banaya, phir \texttt{Scale} tool (\texttt{K}) se usko desktop size (1440px) mein badala. \textcolor{examplegreen}{Logo aur buttons ke \texttt{Constraints} set kiye taaki center mein rahen.} Auto layout use kiya product cards ke liye, jisme image, title, aur price apne aap adjust hote the. Mobile, tablet, aur desktop frames banaye, prototype mein connect kiya, aur \texttt{Play} se test kiya. Client ne responsive flow dekha aur ekdum khush hua!
\end{examplebox}

% Table for Responsive Design Features
\begin{table}[h]
  \centering
  \begin{tabular}{|m{4cm}|m{4cm}|m{4cm}|}
    \hline
    \cellcolor{tableheaderblue}{\color{white}\textbf{Feature}} & 
    \cellcolor{tableheadergreen}{\color{white}\textbf{Kya Hai}} & 
    \cellcolor{tableheaderyellow}{\color{black}\textbf{Kab Use Karna}} \\
    \hline
    \rowcolor{codeblue}
    Scale Tool & Pura design resize karo & Uniform scaling ke liye \\
    \hline
    \rowcolor{tablerowgreen}
    Constraints & Element position fix karo & Center ya edge alignment \\
    \hline
    \rowcolor{codeblue}
    Auto Layout & Dynamic adjustments & Content-based layouts \\
    \hline
    \rowcolor{tablerowgreen}
    Breakpoints & Screen sizes ke frames & Multi-device testing \\
    \hline
  \end{tabular}
  \caption{\textcolor{violet}{Responsive Design Overview}}
\end{table}

% Conclusion
\begin{notebox}
\textbf{\textcolor{warningred}{Important Baatein}:}
\begin{itemize}
  \item \textcolor{examplegreen}{\texttt{Scale} tool se designs ko proportionally resize karo different screen sizes ke liye.}
  \item \textcolor{examplegreen}{\texttt{Constraints} se elements ko center ya fixed positions mein rakho taaki responsive rahe.}
  \item \textcolor{examplegreen}{\texttt{Auto Layout} se dynamic aur flexible layouts banao jo content ke saath adjust ho.}
  \item \textcolor{examplegreen}{Multiple frames banao aur prototype mein test karo taaki client ko responsive flow dikhe.}
\end{itemize}
\end{notebox}

===============================
\hrule



% Define custom colors (including "violet")
\definecolor{headingblue}{RGB}{0,102,204}
\definecolor{examplegreen}{RGB}{0,153,76}
\definecolor{warningred}{RGB}{204,0,0}
\definecolor{codeblue}{RGB}{173,216,230}
\definecolor{tablerowgreen}{RGB}{144,238,144}
\definecolor{tableheaderblue}{RGB}{0,102,204}
\definecolor{tableheadergreen}{RGB}{0,153,76}
\definecolor{tableheaderyellow}{RGB}{255,204,0}
\definecolor{violet}{RGB}{128,0,128} % Custom violet definition

% Rest of your document...
% Section formatting
\titleformat{\section}
  {\Large\bfseries\color{headingblue}}{}{0em}{}
\titleformat{\section*}
 

% Configure listings for code blocks (optional, kept for consistency)
\lstset{
  language=Python,
  backgroundcolor=\color{codeblue},
  basicstyle=\ttfamily\small,
  frame=single,
  breaklines=true,
  keywordstyle=\color{blue},
  stringstyle=\color{purple},
  commentstyle=\color{gray},
  showstringspaces=false
}

% Configure tcolorbox for examples
\newtcolorbox{examplebox}[1]{
  colback=examplegreen!10,
  colframe=examplegreen!50!black,
  title=#1,
  breakable,
  enhanced
}

% Configure tcolorbox for notes/conclusion
\newtcolorbox{notebox}{
  colback=warningred!5!white,
  colframe=warningred!75!black,
  title=Point To Note,
  breakable,
  enhanced
}

% Begin document
\begin{document}

% Title Page
\begin{titlepage}
  \centering
  \vspace*{\fill}
  {\Huge\bfseries\color{warningred} Figma Notes: Alignment, Styles, Components, and Prototype}\par
  \vspace{1cm}
  {\Large A Comprehensive Guide to Figma Advanced Basics}\par
  \vspace*{\fill}
\end{titlepage}

% Topic 1: Alignment
\section*{\textbf{\LARGE \textcolor{violet}{Alignment}}}
\textbf{Corrected aur Enhanced Content:}
\begin{itemize}
  \item \textbf{Alignment Elements}: \textcolor{warningred}{Alignment se tu multiple elements ko ek perfect arrangement mein la sakta hai, taaki tera design clean aur professional dikhe.}
    \begin{enumerate}
      \item \textcolor{warningred}{Pehle multiple elements select kar (hold \texttt{Shift} aur click kar ya drag karke selection box bana).}
      \item \textcolor{warningred}{Top toolbar mein alignment options use kar (jaise Align Left, Align Center, Align Right, Distribute Horizontally, ya Distribute Vertically).} Ya right-click karke alignment options chun.
      \item \textbf{Distribute} options se elements ke beech equal spacing aata hai, aur \textbf{Align} options se wo ek line mein aate hain.
    \end{enumerate}
  \item \textbf{Kab Use Karna}:
    \begin{itemize}
      \item \textcolor{warningred}{Alignment tab use kar jab tujhe buttons, icons, ya text ko ek saath organize karna ho, taaki design consistent dikhe.}
      \item Example: Agar tu ek form design kar raha hai jisme labels aur input fields hain, toh Align Left use kar taaki sab ek straight line mein dikhe, aur Distribute Vertically se fields ke beech equal gap rakho.
    \end{itemize}
  \item \textbf{Extra Tip}: \textcolor{warningred}{Figma ke Layout Grid (\texttt{Ctrl+'}) on kar taaki alignment aur spacing visually check kar sake.}
\end{itemize}

\begin{examplebox}{Real-Life Example}
Soch tu ek food delivery app jaise Swiggy ka checkout page design kar raha hai. Isme ``Item Name,'' ``Quantity,'' aur ``Price'' ke text hain. Tu in teeno ko select karta hai aur \textbf{Align Left} use karta hai taaki sab ek line mein aaye. Phir \textbf{Distribute Vertically} se har item ke beech equal spacing rakhta hai (jaise 16px gap). Isse tera checkout page clean aur user-friendly dikhta hai, aur customer easily details padh sakta hai.
\end{examplebox}

\textbf{Summary:}
\begin{itemize}
  \item \textcolor{warningred}{Alignment se elements ko perfectly arrange kiya jata hai.}
  \item \textcolor{warningred}{Multiple elements select kar, toolbar ya right-click se Align (Left, Center, Right) ya Distribute (Horizontally, Vertically) options use kar. Ye tera design ko consistent aur professional banata hai.}
\end{itemize}

% Table for Alignment
\begin{table}[h]
  \centering
  \begin{tabular}{|p{4cm}|p{4cm}|p{4cm}|}
    \hline
    \cellcolor{tableheaderblue}\color{white}\textbf{Feature} & 
    \cellcolor{tableheadergreen}\color{white}\textbf{Description} & 
    \cellcolor{tableheaderyellow}\color{black}\textbf{Use Case} \\
    \hline
    \rowcolor{codeblue}
    Alignment & Arrange elements & Clean, professional look \\
    \hline
    \rowcolor{tablerowgreen}
    Align Options & Left, Center, Right & Organize buttons, text \\
    \hline
    \rowcolor{codeblue}
    Distribute & Equal spacing & Consistent gaps in forms \\
    \hline
  \end{tabular}
  \caption{Alignment Overview}
  \label{tab:alignment}
\end{table}

% Remaining sections (Styles, Components, Prototype, etc.) follow same pattern with corrected syntax
% [Other sections remain identical with proper \texttt closure and syntax checks]
% ...

% Conclusion in tcolorbox
\begin{notebox}
\textbf{Conclusion:}
\begin{itemize}
  \item \textcolor{warningred}{Alignment se designs clean aur consistent banaye jate hain.}
  \item \textcolor{warningred}{Styles brand look ko maintain karte hain aur time bachta hai.}
  \item \textcolor{warningred}{Components reusable elements banakar workflow ko fast karte hain.}
  \item \textcolor{warningred}{Prototyping se real app flow test aur present kiya jata hai.}
\end{itemize}
\end{notebox}

===============================
\hrule



% Define custom colors
\definecolor{headingblue}{RGB}{0,102,204}
\definecolor{examplegreen}{RGB}{0,153,76}
\definecolor{warningred}{RGB}{204,0,0}
\definecolor{codeblue}{RGB}{173,216,230}
\definecolor{tablerowgreen}{RGB}{144,238,144}
\definecolor{tableheaderblue}{RGB}{0,102,204}
\definecolor{tableheadergreen}{RGB}{0,153,76}
\definecolor{tableheaderyellow}{RGB}{255,204,0}
\definecolor{violet}{RGB}{128,0,128} % For topic headings

% Section formatting
\titleformat{\section}
  {\Large\bfseries\color{headingblue}}{}{0em}{}
\titleformat{\section*}


% Configure listings for code blocks (optional)
\lstset{
  language=Python,
  backgroundcolor=\color{codeblue},
  basicstyle=\ttfamily\small,
  frame=single,
  breaklines=true,
  keywordstyle=\color{blue},
  stringstyle=\color{purple},
  commentstyle=\color{gray},
  showstringspaces=false
}

% Configure tcolorbox for examples
\newtcolorbox{examplebox}[1]{
  colback=examplegreen!10,
  colframe=examplegreen!50!black,
  title=#1,
  breakable,
  enhanced
}

% Configure tcolorbox for notes/conclusion
\newtcolorbox{notebox}{
  colback=warningred!5!white,
  colframe=warningred!75!black,
  title=Point To Note,
  breakable,
  enhanced
}

% Begin document
\begin{document}

% Title Page %
\begin{titlepage}
  \centering
  \vspace*{\fill}
  {\huge\bfseries\color{warningred} Figma Notes: View Options, Grids,\\ Boolean Operations, and Image Import}\par % Line break, reduced size
  \vspace{1cm}
  {\Large A Comprehensive Guide to Figma Essentials}\par
  \vspace*{\fill}
\end{titlepage}

% Topic 1: View Options %
\section*{\textbf{\LARGE \textcolor{violet}{View Options}}} %
\textbf{Corrected aur Enhanced Content:} %
\begin{itemize}
  \item \textbf{Design aur Prototype View}:
    \begin{itemize}
      \item \textcolor{warningred}{Figma mein do main views hain: \textbf{Design View} aur \textbf{Prototype View}. Inke beech toggle karne ke liye top-right corner mein tabs (Design/Prototype) use kar ya shortcuts (\texttt{Ctrl+Tab} for switching).}
      \item \textcolor{warningred}{\textbf{Design View}: Yahan tu elements create aur edit karta hai, jaise frames, text, shapes, aur styles banata hai.}
      \item \textcolor{warningred}{\textbf{Prototype View}: Yahan tu interactions define karta hai, jaise button clicks se ek screen se doosre screen ka flow.}
    \end{itemize}
  \item \textbf{Kab Use Karna}:
    \begin{itemize}
      \item \textcolor{warningred}{Design View tab use kar jab tujhe UI elements banane ya tweak karne hon, jaise button ka color change karna.}
      \item \textcolor{warningred}{Prototype View tab use kar jab user flow test karna ho ya client ko interactive demo dikhana ho.}
      \item Example: Agar tu ek app ka homepage design kar raha hai, toh Design View mein buttons aur text banayega, aur Prototype View mein ``Login'' button ko dashboard screen se connect karega.
    \end{itemize}
\end{itemize}

\begin{examplebox}{Real-Life Example}
Soch tu ek travel app jaise Goibibo ke liye kaam kar raha hai. \textbf{Design View} mein tu homepage pe search bar, destination cards, aur buttons banata hai. Phir \textbf{Prototype View} mein switch karke search button pe click se ``Results'' screen tak connection banata hai. Client ko demo dikhane ke liye Prototype View mein Play button daba ke flow dikha sakta hai -- bilkul real app jaisa feel aata hai.
\end{examplebox}

\textbf{Summary:} %
\begin{itemize}
  \item \textcolor{warningred}{Design View se elements banaye aur edit kiye jate hain, aur Prototype View se interactions aur flow set kiye jate hain.}
  \item \textcolor{warningred}{Tabs ya \texttt{Ctrl+Tab} se toggle kar. Ye dono views ek UI/UX designer ke liye must hain taaki design aur functionality dono perfect ho.}
\end{itemize}

% Table for View Options %
\begin{table}[h]
  \centering
  \begin{tabular}{|p{4cm}|p{4cm}|p{4cm}|}
    \hline
    \cellcolor{tableheaderblue}\color{white}\textbf{Feature} & 
    \cellcolor{tableheadergreen}\color{white}\textbf{Description} & 
    \cellcolor{tableheaderyellow}\color{black}\textbf{Use Case} \\
    \hline
    \rowcolor{codeblue}
    Design View & Create/edit elements & Build UI components \\
    \hline
    \rowcolor{tablerowgreen}
    Prototype View & Define interactions & Test user flow \\
    \hline
    \rowcolor{codeblue}
    Toggle & \texttt{Ctrl+Tab} & Switch views quickly \\
    \hline
  \end{tabular}
  \caption{View Options Overview}
  \label{tab:view_options}
\end{table}

% Topic 2: Grids and Layouts %
\section*{\textbf{\LARGE \textcolor{violet}{Grids and Layouts}}} %
\textbf{Corrected aur Enhanced Content:} %
\begin{itemize}
  \item \textbf{Creating a Grid}:
    \begin{enumerate}
      \item \textcolor{warningred}{Figma ke \textbf{View Menu} se \textbf{Show Layout Grid} chun ya shortcut use kar (\texttt{Ctrl+'}).}
      \item \textcolor{warningred}{Right panel mein \textbf{Layout Grid} settings khol aur grid customize kar:}
        \begin{itemize}
          \item \textbf{Columns}: Kitne columns chahiye (jaise 12-column grid for web).
          \item \textbf{Gutter}: Columns ke beech gap (jaise 16px).
          \item \textbf{Margin}: Canvas edges se grid ka distance (jaise 32px).
          \item \textbf{Grid Size}: Square size for pixel grid (jaise 8px for 8pt system).
        \end{itemize}
      \item Grid on rakh taaki elements ko consistent spacing aur alignment ke saath place kar sake.
    \end{enumerate}
  \item \textbf{Kab Use Karna}:
    \begin{itemize}
      \item \textcolor{warningred}{Grid tab use kar jab tujhe design mein consistency chahiye, jaise buttons ya cards ke beech equal spacing.}
      \item Web ya app layouts ke liye grid system must hai taaki responsive design ban sake.
      \item Example: Ek website ke liye 12-column grid bana taaki header, sidebar, aur content perfectly align ho.
    \end{itemize}
\end{itemize}

\begin{examplebox}{Real-Life Example}
Tu ek e-commerce website jaise Amazon ke liye product page design kar raha hai. \textbf{Layout Grid} on karke tu ek 12-column grid set karta hai (gutter: 24px, margin: 64px). Is grid ke hisaab se tu product image left 4 columns mein, description right 6 columns mein, aur ``Buy Now'' button niche align karta hai. Grid se spacing aur alignment ekdum perfect hota hai, aur responsive design banana aasan ho jata hai.
\end{examplebox}

\textbf{Summary:} %
\begin{itemize}
  \item \textcolor{warningred}{Grid se design mein consistent spacing aur alignment milta hai.}
  \item \textcolor{warningred}{Layout Grid on kar (\texttt{Ctrl+'}), right panel se columns, gutter, margins set kar, aur elements ko grid ke hisaab se place kar. Ye professional aur responsive designs ke liye zaroori hai.}
\end{itemize}

% Table for Grids and Layouts %
\begin{table}[h]
  \centering
  \begin{tabular}{|p{4cm}|p{4cm}|p{4cm}|}
    \hline
    \cellcolor{tableheaderblue}\color{white}\textbf{Feature} & 
    \cellcolor{tableheadergreen}\color{white}\textbf{Description} & 
    \cellcolor{tableheaderyellow}\color{black}\textbf{Use Case} \\
    \hline
    \rowcolor{codeblue}
    Grid & Consistent spacing & Align elements \\
    \hline
    \rowcolor{tablerowgreen}
    Layout Grid & \texttt{Ctrl+'} toggle & Responsive layouts \\
    \hline
    \rowcolor{codeblue}
    Settings & Columns, Gutter & Web/app design \\
    \hline
  \end{tabular}
  \caption{Grids and Layouts Overview}
  \label{tab:grids_layouts}
\end{table}

% Topic 3: Boolean Operations %
\section*{\textbf{\LARGE \textcolor{violet}{Boolean Operations}}} %
\textbf{Corrected aur Enhanced Content:} %
\begin{itemize}
  \item \textbf{Combining Shapes}:
    \begin{itemize}
      \item \textcolor{warningred}{Boolean operations se shapes ko combine ya modify kiya jata hai. Toolbar mein \textbf{Boolean Operations} (Union, Subtract, Intersect, Exclude) options hote hain.}
    \end{itemize}
    \begin{enumerate}
      \item \textcolor{warningred}{Do ya zyada shapes select kar (jaise ek rectangle aur circle).}
      \item \textcolor{warningred}{Toolbar se operation chun:}
        \begin{itemize}
          \item \textbf{Union}: Shapes ko merge karta hai ek single shape mein.
          \item \textbf{Subtract}: Ek shape ko doosre se cut karta hai.
          \item \textbf{Intersect}: Sirf overlapping area retain karta hai.
          \item \textbf{Exclude}: Overlapping area hata deta hai.
        \end{itemize}
      \item Resulting shape ko further edit kar sakta hai (color, size, etc.).
    \end{enumerate}
  \item \textbf{Kab Use Karna}:
    \begin{itemize}
      \item \textcolor{warningred}{Boolean operations tab use kar jab complex shapes ya custom icons banane hon.}
      \item Example: Ek play button banane ke liye ek triangle aur circle ko intersect kar taaki sirf overlapping part bache.
      \item Note: Text ke saath directly boolean nahi hota, pehle text ko shape mein convert kar (Outline Stroke).
    \end{itemize}
\end{itemize}

\begin{examplebox}{Real-Life Example}
Tu ek music app jaise Gaana ke liye custom icon design kar raha hai. Ek ``Heart'' icon banane ke liye do circles aur ek triangle banata hai. Circles ko \textbf{Union} se ek shape mein merge karta hai, phir triangle ke saath \textbf{Subtract} use karke niche ka pointed part banata hai. Ye heart icon ab button ya favorite option ke liye use ho sakta hai -- ekdum clean aur unique.
\end{examplebox}

\textbf{Summary:} %
\begin{itemize}
  \item \textcolor{warningred}{Boolean operations (Union, Subtract, Intersect, Exclude) se shapes combine ya modify hote hain.}
  \item \textcolor{warningred}{Multiple shapes select kar, toolbar se operation chun, aur complex designs ya icons bana. Ye custom UI elements ke liye bohot useful hai.}
\end{itemize}

% Table for Boolean Operations %
\begin{table}[h]
  \centering
  \begin{tabular}{|p{4cm}|p{4cm}|p{4cm}|}
    \hline
    \cellcolor{tableheaderblue}\color{white}\textbf{Boolean Operation} & 
    \cellcolor{tableheadergreen}\color{white}\textbf{Description} & 
    \cellcolor{tableheaderyellow}\color{black}\textbf{Use Case} \\
    \hline
    \rowcolor{codeblue}
    Union & Merge shapes & Single shape creation \\
    \hline
    \rowcolor{tablerowgreen}
    Subtract & Cut shape & Custom icon design \\
    \hline
    \rowcolor{codeblue}
    Intersect & Keep overlap & Play button shape \\
    \hline
  \end{tabular}
  \caption{Boolean Operations Overview}
  \label{tab:boolean_operations}
\end{table}

% Topic 4: Image Import %
\section*{\textbf{\LARGE \textcolor{violet}{Image Import}}} %
\textbf{Corrected aur Enhanced Content:} %
\begin{itemize}
  \item \textbf{Importing Images}:
    \begin{enumerate}
      \item \textcolor{warningred}{Images ko canvas pe \textbf{drag-and-drop} kar ya \textbf{Place Image} option use kar (toolbar se ya \texttt{Ctrl+Shift+K}).}
      \item \textcolor{warningred}{Image frame ke andar place kar taaki size aur crop control rahe.}
      \item \textcolor{warningred}{Image select karke \textbf{corner handles} se resize kar ya \textbf{Crop Tool} (double-click image) se specific area select kar.}
      \item Right panel mein \textbf{Fill} options se image adjustments kar (jaise opacity ya blend mode).
    \end{enumerate}
  \item \textbf{Kab Use Karna}:
    \begin{itemize}
      \item \textcolor{warningred}{Images tab import kar jab UI mein visuals chahiye, jaise product photos, user avatars, ya background images.}
      \item Frames mein images rakh taaki design device-specific aur organized rahe.
      \item Example: Ek app ke profile screen pe user ka photo circular frame mein crop karke daal.
    \end{itemize}
\end{itemize}

\begin{examplebox}{Real-Life Example}
Tu ek food delivery app jaise Zomato design kar raha hai. Restaurant ka banner banane ke liye ek high-quality dish image ko \textbf{drag-and-drop} karta hai. Image ko ek rectangular frame (375x200px) ke andar daalta hai aur \textbf{Crop Tool} se sirf dish ka center part select karta hai. Right panel mein thodi opacity kam karke text ke saath contrast banata hai. Isse banner professional aur appetizing dikhta hai.
\end{examplebox}

\textbf{Summary:} %
\begin{itemize}
  \item \textcolor{warningred}{Images ko drag-and-drop ya Place Image se canvas pe la.}
  \item \textcolor{warningred}{Frame mein rakh, resize ya crop kar, aur Fill options se adjust kar. Images UI ko visually appealing banate hain aur user experience improve karte hain.}
\end{itemize}

% Table for Image Import %
\begin{table}[h]
  \centering
  \begin{tabular}{|p{4cm}|p{4cm}|p{4cm}|}
    \hline
    \cellcolor{tableheaderblue}\color{white}\textbf{Feature} & 
    \cellcolor{tableheadergreen}\color{white}\textbf{Description} & 
    \cellcolor{tableheaderyellow}\color{black}\textbf{Use Case} \\
    \hline
    \rowcolor{codeblue}
    Image Import & Drag-and-drop & Add visuals \\
    \hline
    \rowcolor{tablerowgreen}
    Crop Tool & Double-click & Adjust image area \\
    \hline
    \rowcolor{codeblue}
    Frames & Contain images & Device-specific design \\
    \hline
  \end{tabular}
  \caption{Image Import Overview}
  \label{tab:image_import}
\end{table}

% Extra Tips for Beginners %
\section*{\textbf{\LARGE \textcolor{violet}{Beginners ke liye Extra Tips}}} %
\begin{itemize}
  \item \textbf{View Switching}: \textcolor{warningred}{Design aur Prototype views ke beech jaldi switch karne ke liye \texttt{Ctrl+Tab} yaad rakho -- kaam fast hota hai.}
  \item \textbf{Grid Snap}: \textcolor{warningred}{Grid ke saath \textbf{Snap to Grid} on kar taaki elements automatically align ho jayein (View > Snap to Grid).}
  \item \textbf{Boolean Practice}: \textcolor{warningred}{Simple icons (jaise star, arrow) banake boolean operations practice kar taaki complex shapes samajh aaye.}
  \item \textbf{Image Optimization}: \textcolor{warningred}{Heavy images avoid kar -- chhote file size (jaise PNG, JPEG under 1MB) use kar taaki Figma slow na ho.}
  \item \textbf{Mockup Project}: Ek chhota project try kar -- jaise ek app ka landing page bana jisme grid use kar, images import kar, boolean se icon bana, aur prototype view mein flow test kar.
\end{itemize}

% Table for Extra Tips %
\begin{table}[h]
  \centering
  \begin{tabular}{|p{4cm}|p{4cm}|p{4cm}|}
    \hline
    \cellcolor{tableheaderblue}\color{white}\textbf{Tip} & 
    \cellcolor{tableheadergreen}\color{white}\textbf{Description} & 
    \cellcolor{tableheaderyellow}\color{black}\textbf{Benefit} \\
    \hline
    \rowcolor{codeblue}
    View Switching & \texttt{Ctrl+Tab} & Faster workflow \\
    \hline
    \rowcolor{tablerowgreen}
    Grid Snap & Auto-align & Precise placement \\
    \hline
    \rowcolor{codeblue}
    Image Optimization & Small file size & Faster Figma \\
    \hline
  \end{tabular}
  \caption{Beginner Tips Overview}
  \label{tab:beginner_tips}
\end{table}

% UI/UX Importance %
\section*{\textbf{\LARGE \textcolor{violet}{UI/UX Job ke liye Kyun Zaroori}}} %
UI/UX designer ke roop mein view options, grids, boolean operations, aur image import daily kaam aayenge taaki:
\begin{itemize}
  \item \textcolor{warningred}{Design aur interactions smoothly switch ho sakein (view options).}
  \item \textcolor{warningred}{Layouts consistent aur responsive hon (grids).}
  \item \textcolor{warningred}{Custom icons ya shapes ban sakein (boolean operations).}
  \item \textcolor{warningred}{Visuals professional aur engaging hon (image import).}
\end{itemize}
In skills se tu designs bana sakta hai jo na sirf sundar dikhein balki functional bhi hon, jo clients aur developers ko impress karega.

% Conclusion in tcolorbox %
\begin{notebox}
\textbf{Conclusion:} %
\begin{itemize}
  \item \textcolor{warningred}{View Options se Design aur Prototype views ke beech switch karke UI aur flow banao.}
  \item \textcolor{warningred}{Grids se consistent aur responsive layouts ensure karo.}
  \item \textcolor{warningred}{Boolean operations se unique shapes aur icons design karo.}
  \item \textcolor{warningred}{Image import se visuals add karke UI ko appealing banao.}
\end{itemize}
\end{notebox}

===============================
\hrule



% Section formatting (only for unstarred sections)
\titleformat{\section}
  {\Large\bfseries\color{headingblue}}{}{0em}{}

% Configure listings for code blocks (optional)
\lstset{
  language=Python,
  backgroundcolor=\color{codeblue},
  basicstyle=\ttfamily\small,
  frame=single,
  breaklines=true,
  keywordstyle=\color{blue},
  stringstyle=\color{purple},
  commentstyle=\color{gray},
  showstringspaces=false
}

% Configure tcolorbox for examples
\newtcolorbox{examplebox}[1]{
  colback=examplegreen!10,
  colframe=examplegreen!50!black,
  title=#1,
  breakable,
  enhanced
}

% Configure tcolorbox for notes/conclusion
\newtcolorbox{notebox}{
  colback=warningred!5!white,
  colframe=warningred!75!black,
  title=Point To Note,
  breakable,
  enhanced
}

% Begin document
\begin{document}

% Title Page %
\begin{titlepage}
  \centering
  \vspace*{\fill}
  {\huge\bfseries\color{warningred} Figma Notes: Effects, Auto Layout,\\ Comments, Scale, Move, Version History}\par % Line break, reduced size
  \vspace{1cm}
  {\Large A Comprehensive Guide to Figma Essentials}\par
  \vspace*{\fill}
\end{titlepage}

% Topic 1: Effects and Filters %
\section*{\textbf{\LARGE \textcolor{violet}{Effects and Filters}}} %
\textbf{Corrected aur Enhanced Content:} %
\begin{itemize}
  \item \textbf{Applying Effects}:
    \begin{enumerate}
      \item \textcolor{warningred}{Kisi element (jaise button, text, ya shape) ko select kar.}
      \item \textcolor{warningred}{Right panel mein \textbf{Effects} section (shadow icon) pe ja.}
      \item \textcolor{warningred}{\texttt{+} button daba ke effects add kar, jaise:}
        \begin{itemize}
          \item \textbf{Drop Shadow}: Element ke peeche shadow deta hai, depth ke liye.
          \item \textbf{Inner Shadow}: Element ke andar shadow, pressed look ke liye.
          \item \textbf{Blur}: Gaussian blur ya background blur se soft effect.
          \item \textbf{Layer Blur}: Poore layer ko blur karta hai.
        \end{itemize}
      \item Settings adjust kar -- X/Y offset, blur radius, color, ya opacity -- taaki effect perfect lage.
    \end{enumerate}
  \item \textbf{Kab Use Karna}:
    \begin{itemize}
      \item \textcolor{warningred}{Effects tab use kar jab tujhe design mein depth, focus, ya visual appeal add karna ho.}
      \item Example: Buttons pe drop shadow daal taaki wo clickable dikhe, ya background blur se text readable bane.
      \item Tip: Zyada effects avoid kar, warna design cluttered lag sakta hai.
    \end{itemize}
\end{itemize}

\begin{examplebox}{Real-Life Example}
Tu ek fitness app jaise HealthifyMe design kar raha hai. Ek ``Start Workout'' button pe \textbf{Drop Shadow} (X: 2, Y: 4, Blur: 8, Color: \#000 20\% opacity) daalta hai taaki button thoda raised dikhe aur users ka dhyan jaye. Phir ek hero section ke background image pe \textbf{Gaussian Blur} (Radius: 10) use karta hai taaki overlay text (jaise ``Join Now'') clear dikhe. Ye effects se app premium aur professional lagta hai.
\end{examplebox}

\textbf{Summary:} %
\begin{itemize}
  \item \textcolor{warningred}{Effects jaise drop shadow aur blur se elements ko visual appeal milta hai.}
  \item \textcolor{warningred}{Element select kar, right panel ke Effects section mein settings tweak kar, aur design ko engaging bana. Balance rakho taaki design clean rahe.}
\end{itemize}

% Table for Effects and Filters %
\begin{table}[h]
  \centering
  \begin{tabular}{|p{4cm}|p{4cm}|p{4cm}|}
    \hline
    \cellcolor{tableheaderblue}\color{white}\textbf{Effect} & 
    \cellcolor{tableheadergreen}\color{white}\textbf{Description} & 
    \cellcolor{tableheaderyellow}\color{black}\textbf{Use Case} \\
    \hline
    \rowcolor{codeblue}
    Drop Shadow & Adds depth & Clickable buttons \\
    \hline
    \rowcolor{tablerowgreen}
    Blur & Softens visuals & Readable text \\
    \hline
    \rowcolor{codeblue}
    Inner Shadow & Pressed look & UI elements \\
    \hline
  \end{tabular}
  \caption{Effects and Filters Overview}
  \label{tab:effects_filters}
\end{table}

% Topic 2: Auto Layout %
\section*{\textbf{\LARGE \textcolor{violet}{Auto Layout}}} %
\textbf{Corrected aur Enhanced Content:} %
\begin{itemize}
  \item \textbf{Using Auto Layout}:
    \begin{enumerate}
      \item \textcolor{warningred}{Ek frame ya group select kar aur toolbar se \textbf{Auto Layout} button click kar ya shortcut (\texttt{Shift+A}).}
      \item \textcolor{warningred}{Auto Layout on hone ke baad, frame ke andar elements automatically adjust hote hain based on content ya spacing.}
      \item \textcolor{warningred}{Right panel mein Auto Layout settings tweak kar:}
        \begin{itemize}
          \item \textbf{Direction}: Horizontal ya Vertical.
          \item \textbf{Padding}: Frame ke andar space (jaise 16px).
          \item \textbf{Spacing}: Elements ke beech gap (jaise 8px).
          \item \textbf{Alignment}: Left, Center, Right.
        \end{itemize}
      \item Elements add ya remove kar -- Auto Layout apne aap size aur spacing adjust karta hai.
    \end{enumerate}
  \item \textbf{Kab Use Karna}:
    \begin{itemize}
      \item \textcolor{warningred}{Auto Layout tab use kar jab tujhe responsive designs chahiye jo content ke hisaab se badal sakein, jaise buttons ka list ya form fields.}
      \item Example: Ek chat app ke message bubbles Auto Layout mein daal taaki naye messages add hone pe layout apne aap adjust ho.
      \item Tip: Nested Auto Layout (frame ke andar frame) use kar complex layouts ke liye.
    \end{itemize}
\end{itemize}

\begin{examplebox}{Real-Life Example}
Tu ek todo app jaise Todoist design kar raha hai. Ek task list ke liye frame banata hai aur \textbf{Auto Layout} on karta hai (Vertical, Padding: 12px, Spacing: 8px). Har task ek text aur checkbox ka combination hai. Jab tu naya task add karta hai ya koi delete karta hai, Auto Layout apne aap list ko rearrange karta hai -- koi manual adjustment nahi chahiye. Ye responsive aur fast design ke liye perfect hai.
\end{examplebox}

\textbf{Summary:} %
\begin{itemize}
  \item \textcolor{warningred}{Auto Layout se dynamic aur responsive designs bante hain jo content ke hisaab se adjust hote hain.}
  \item \textcolor{warningred}{Frame ko Auto Layout mein convert kar (\texttt{Shift+A}), direction, padding, spacing set kar, aur kaam jaldi kar. Ye modern UI design ka must-have tool hai.}
\end{itemize}

% Table for Auto Layout %
\begin{table}[h]
  \centering
  \begin{tabular}{|p{4cm}|p{4cm}|p{4cm}|}
    \hline
    \cellcolor{tableheaderblue}\color{white}\textbf{Feature} & 
    \cellcolor{tableheadergreen}\color{white}\textbf{Description} & 
    \cellcolor{tableheaderyellow}\color{black}\textbf{Use Case} \\
    \hline
    \rowcolor{codeblue}
    Auto Layout & Dynamic adjustment & Responsive lists \\
    \hline
    \rowcolor{tablerowgreen}
    Settings & Padding, Spacing & Forms, cards \\
    \hline
    \rowcolor{codeblue}
    Shortcut & \texttt{Shift+A} & Fast setup \\
    \hline
  \end{tabular}
  \caption{Auto Layout Overview}
  \label{tab:auto_layout}
\end{table}

% Topic 3: Comments and Collaboration %
\section*{\textbf{\LARGE \textcolor{violet}{Comments and Collaboration}}} %
\textbf{Corrected aur Enhanced Content:} %
\begin{itemize}
  \item \textbf{Adding Comments}:
    \begin{enumerate}
      \item \textcolor{warningred}{Canvas pe kisi element ya area pe click kar.}
      \item \textcolor{warningred}{Toolbar se \textbf{Comment Tool} chun (shortcut: \texttt{C}) ya right panel mein Comments tab khol.}
      \item \textcolor{warningred}{Comment box mein feedback, suggestions, ya notes type kar aur \textbf{Post} kar.}
      \item Team members comments dekh sakte hain, reply kar sakte hain, ya resolve mark kar sakte hain.
    \end{enumerate}
  \item \textbf{Collaboration}:
    \begin{itemize}
      \item \textcolor{warningred}{Figma cloud-based hai, toh real-time mein team ke saath kaam kar sakta hai.}
      \item \textcolor{warningred}{Top-right corner mein \textbf{Share} button se team members ko invite kar (edit ya view access de).}
      \item Cursor aur changes live dikhte hain, jaise Google Docs.
    \end{itemize}
  \item \textbf{Kab Use Karna}:
    \begin{itemize}
      \item \textcolor{warningred}{Comments tab use kar jab team ya client ke saath feedback share karna ho.}
      \item \textcolor{warningred}{Collaboration ke liye Figma ke real-time editing ka fayda utha taaki sab ek saath kaam kar sakein.}
      \item Example: Client ko button ka color change suggest karne ke liye uspe comment daal.
    \end{itemize}
\end{itemize}

\begin{examplebox}{Real-Life Example}
Tu ek e-commerce app jaise Flipkart ke liye team ke saath kaam kar raha hai. Designer team ka ek member bolta hai ki homepage ka ``Search'' button zyada bada hai. Wo \textbf{Comment Tool} (\texttt{C}) se button pe comment daalta hai: ``Button size 48px se 40px karo.'' Tu comment padh ke change karta hai aur reply karta hai ``Done!'' Real-time mein doosra designer footer edit kar raha hota hai, aur sab smoothly chalta hai. Client bhi comments dekh ke feedback deta hai.
\end{examplebox}

\textbf{Summary:} %
\begin{itemize}
  \item \textcolor{warningred}{Comments se feedback aur notes share hote hain, aur collaboration se team real-time mein kaam karta hai.}
  \item \textcolor{warningred}{Comment Tool (\texttt{C}) use kar, team ko Share se invite kar, aur design process ko smooth bana. Ye team projects ke liye must hai.}
\end{itemize}

% Table for Comments and Collaboration %
\begin{table}[h]
  \centering
  \begin{tabular}{|p{4cm}|p{4cm}|p{4cm}|}
    \hline
    \cellcolor{tableheaderblue}\color{white}\textbf{Feature} & 
    \cellcolor{tableheadergreen}\color{white}\textbf{Description} & 
    \cellcolor{tableheaderyellow}\color{black}\textbf{Use Case} \\
    \hline
    \rowcolor{codeblue}
    Comments & Feedback sharing & Client suggestions \\
    \hline
    \rowcolor{tablerowgreen}
    Collaboration & Real-time edits & Team projects \\
    \hline
    \rowcolor{codeblue}
    Shortcut & \texttt{C} & Quick feedback \\
    \hline
  \end{tabular}
  \caption{Comments and Collaboration Overview}
  \label{tab:comments_collaboration}
\end{table}

% Topic 4: Scale (Note) %
\section*{\textbf{\LARGE \textcolor{violet}{Scale (Note)}}} %
\textbf{Corrected aur Enhanced Content:} %
\begin{itemize}
  \item \textbf{Using Scale}:
    \begin{enumerate}
      \item \textcolor{warningred}{Kisi element (text, image, shape, ya group) ko select kar.}
      \item \textcolor{warningred}{Toolbar se \textbf{Scale Tool} chun (shortcut: \texttt{K}) ya corner handles hold karke resize kar.}
      \item \textcolor{warningred}{Scale Tool se element ka size proportionally badhaya ya ghataya jata hai, aspect ratio maintain rehta hai.}
      \item Manual scaling ke liye width/height values right panel mein enter kar.
    \end{enumerate}
  \item \textbf{Kab Use Karna}:
    \begin{itemize}
      \item \textcolor{warningred}{Scale tab use kar jab tujhe elements ko resize karna ho, jaise image ko chhota karna ya text ka size badhana.}
      \item Example: Ek icon ko 24px se 32px karna ho toh Scale Tool use kar taaki quality kharab na ho.
      \item Tip: \texttt{Shift} hold kar ke scale kar taaki proportions same rahen.
    \end{itemize}
\end{itemize}

\begin{examplebox}{Real-Life Example}
Tu ek social media app jaise Instagram ke liye profile page design kar raha hai. User ka avatar image 100x100px hai, lekin mobile view ke liye 80x80px chahiye. Tu image select karta hai, \textbf{Scale Tool} (\texttt{K}) use karke proportionally size ghatata hai. Same tareeke se username text ko 16px se 14px karta hai. Isse design clean aur device-friendly dikhta hai.
\end{examplebox}

\textbf{Summary:} %
\begin{itemize}
  \item \textcolor{warningred}{Scale Tool (\texttt{K}) se text, images, ya elements ka size proportionally adjust hota hai.}
  \item \textcolor{warningred}{Element select kar, Scale chun, aur size change kar. Ye design ko flexible aur consistent rakhta hai.}
\end{itemize}

% Table for Scale %
\begin{table}[h]
  \centering
  \begin{tabular}{|p{4cm}|p{4cm}|p{4cm}|}
    \hline
    \cellcolor{tableheaderblue}\color{white}\textbf{Feature} & 
    \cellcolor{tableheadergreen}\color{white}\textbf{Description} & 
    \cellcolor{tableheaderyellow}\color{black}\textbf{Use Case} \\
    \hline
    \rowcolor{codeblue}
    Scale Tool & Proportional resize & Icon resizing \\
    \hline
    \rowcolor{tablerowgreen}
    Shortcut & \texttt{K} & Fast scaling \\
    \hline
    \rowcolor{codeblue}
    Tip & \texttt{Shift} for proportions & Image adjustments \\
    \hline
  \end{tabular}
  \caption{Scale Overview}
  \label{tab:scale}
\end{table}

% Topic 5: Move (Note) %
\section*{\textbf{\LARGE \textcolor{violet}{Move (Note)}}} %
\textbf{Corrected aur Enhanced Content:} %
\begin{itemize}
  \item \textbf{Using Move}:
    \begin{enumerate}
      \item \textcolor{warningred}{\textbf{Move Tool} (shortcut: \texttt{V}) default tool hai jo elements ko select, move, aur transform karne ke liye use hota hai.}
      \item \textcolor{warningred}{Element pe click karke select kar, drag karke move kar, ya arrow keys se nudge kar (1px ya 10px steps).}
      \item \textcolor{warningred}{Transform ke liye corner handles use kar (rotate ya resize) ya right panel mein X/Y coordinates change kar.}
    \end{enumerate}
  \item \textbf{Kab Use Karna}:
    \begin{itemize}
      \item \textcolor{warningred}{Move tab use kar jab tujhe elements ko canvas pe shift karna ho ya unka position adjust karna ho.}
      \item Example: Ek button ko screen ke center mein lana ho toh Move Tool se drag kar ya coordinates set kar.
      \item Tip: \texttt{Shift} hold kar ke move kar taaki straight line mein rahe.
    \end{itemize}
\end{itemize}

\begin{examplebox}{Real-Life Example}
Tu ek payment app jaise Paytm ke liye checkout screen design kar raha hai. ``Pay Now'' button thoda left side pe hai, lekin tujhe center mein chahiye. Tu \textbf{Move Tool} (\texttt{V}) se button select karta hai aur drag karke center mein lata hai, ya right panel mein Align Center button daba deta hai. Arrow keys se 2px upar nudge karta hai taaki alignment perfect ho. Isse layout balanced dikhta hai.
\end{examplebox}

\textbf{Summary:} %
\begin{itemize}
  \item \textcolor{warningred}{Move Tool (\texttt{V}) se elements select, move, aur transform kiye jate hain.}
  \item \textcolor{warningred}{Drag, arrow keys, ya coordinates use kar taaki position perfect ho. Ye basic lekin powerful tool hai har design ke liye.}
\end{itemize}

% Table for Move %
\begin{table}[h]
  \centering
  \begin{tabular}{|p{4cm}|p{4cm}|p{4cm}|}
    \hline
    \cellcolor{tableheaderblue}\color{white}\textbf{Feature} & 
    \cellcolor{tableheadergreen}\color{white}\textbf{Description} & 
    \cellcolor{tableheaderyellow}\color{black}\textbf{Use Case} \\
    \hline
    \rowcolor{codeblue}
    Move Tool & Select and shift & Button placement \\
    \hline
    \rowcolor{tablerowgreen}
    Shortcut & \texttt{V} & Quick adjustments \\
    \hline
    \rowcolor{codeblue}
    Tip & \texttt{Shift} for straight & Precise alignment \\
    \hline
  \end{tabular}
  \caption{Move Overview}
  \label{tab:move}
\end{table}

% Topic 6: Version History %
\section*{\textbf{\LARGE \textcolor{violet}{Version History}}} %
\textbf{Corrected aur Enhanced Content:} %
\begin{itemize}
  \item \textbf{Viewing Version History}:
    \begin{enumerate}
      \item \textcolor{warningred}{Top menu mein \textbf{File > Version History} pe ja.}
      \item \textcolor{warningred}{Right panel mein timeline khulta hai jahan har save ya major change ka record hota hai (date, time, aur editor ke naam ke saath).}
      \item \textcolor{warningred}{Kisi version pe click karke uska preview dekh sakta hai, ya \textbf{Restore} kar sakta hai agar purani version chahiye.}
      \item Auto-save versions har 30 minutes mein banta hai (agar changes hote hain).
    \end{enumerate}
  \item \textbf{Kab Use Karna}:
    \begin{itemize}
      \item \textcolor{warningred}{Version History tab use kar jab tujhe purane designs review karne hon ya galti se delete hui cheez wapas laani ho.}
      \item Team projects mein changes track karne ke liye useful hai.
      \item Example: Agar client bolta hai ki do din pehle ka design better tha, toh Version History se usko restore kar sakta hai.
    \end{itemize}
\end{itemize}

\begin{examplebox}{Real-Life Example}
Tu ek news app jaise Inshorts ke liye homepage design kar raha hai. Kal tu ne header ka color blue se grey kiya, lekin client bolta hai blue better tha. Tu \textbf{Version History} kholta hai, kal ka version dhoondta hai, aur blue header wala design preview karta hai. Pasand aane pe \textbf{Restore} karta hai. Isse time bachta hai aur client khush rehta hai kyunki purana kaam wapas mila.
\end{examplebox}

\textbf{Summary:} %
\begin{itemize}
  \item \textcolor{warningred}{Version History se purane designs review aur restore kiye jate hain.}
  \item \textcolor{warningred}{File > Version History se timeline check kar, version select kar, aur restore kar agar zarurat ho. Ye changes track aur mistakes fix karne ke liye must hai.}
\end{itemize}

% Table for Version History %
\begin{table}[h]
  \centering
  \begin{tabular}{|p{4cm}|p{4cm}|p{4cm}|}
    \hline
    \cellcolor{tableheaderblue}\color{white}\textbf{Feature} & 
    \cellcolor{tableheadergreen}\color{white}\textbf{Description} & 
    \cellcolor{tableheaderyellow}\color{black}\textbf{Use Case} \\
    \hline
    \rowcolor{codeblue}
    Version History & Track changes & Restore old designs \\
    \hline
    \rowcolor{tablerowgreen}
    Timeline & Save records & Client revisions \\
    \hline
    \rowcolor{codeblue}
    Restore & Revert versions & Fix mistakes \\
    \hline
  \end{tabular}
  \caption{Version History Overview}
  \label{tab:version_history}
\end{table}

% Extra Tips for Beginners %
\section*{\textbf{\LARGE \textcolor{violet}{Beginners ke liye Extra Tips}}} %
\begin{itemize}
  \item \textbf{Effect Limits}: \textcolor{warningred}{Effects sparingly use kar -- ek button pe ek hi shadow ya blur kaafi hai, warna design heavy lagta hai.}
  \item \textbf{Auto Layout Shortcuts}: \textcolor{warningred}{Auto Layout mein elements reorder karne ke liye drag-drop ke saath Layers Panel bhi use kar taaki jaldi ho.}
  \item \textbf{Comment Tags}: \textcolor{warningred}{Comments mein team members ko tag kar (@username) taaki notifications jaye aur discussion clear ho.}
  \item \textbf{Scale vs Resize}: \textcolor{warningred}{Scale Tool (\texttt{K}) proportions rakhta hai, jabki normal resize aspect ratio bigad sakta hai -- icons ke liye Scale best hai.}
  \item \textbf{Version Naming}: \textcolor{warningred}{Version History mein major changes ke liye manual save kar aur descriptive name de (jaise ``Homepage v2'') taaki dhoondna aasan ho.}
  \item \textbf{Practice Project}: Ek simple app ka onboarding flow bana -- Auto Layout se buttons arrange kar, effects se polish de, comments daal ke feedback mimic kar, aur Version History check kar.
\end{itemize}

% Table for Extra Tips %
\begin{table}[h]
  \centering
  \begin{tabular}{|p{4cm}|p{4cm}|p{4cm}|}
    \hline
    \cellcolor{tableheaderblue}\color{white}\textbf{Tip} & 
    \cellcolor{tableheadergreen}\color{white}\textbf{Description} & 
    \cellcolor{tableheaderyellow}\color{black}\textbf{Benefit} \\
    \hline
    \rowcolor{codeblue}
    Effect Limits & Use sparingly & Clean design \\
    \hline
    \rowcolor{tablerowgreen}
    Comment Tags & Tag members & Clear communication \\
    \hline
    \rowcolor{codeblue}
    Scale Tool & Maintain proportions & Quality icons \\
    \hline
  \end{tabular}
  \caption{Beginner Tips Overview}
  \label{tab:beginner_tips}
\end{table}

% UI/UX Importance %
\section*{\textbf{\LARGE \textcolor{violet}{UI/UX Job ke liye Kyun Zaroori}}} %
UI/UX designer ke roop mein ye skills daily kaam aayengi taaki:
\begin{itemize}
  \item \textcolor{warningred}{Designs visually appealing aur modern hon (effects).}
  \item \textcolor{warningred}{Layouts responsive aur fast ban sakein (Auto Layout).}
  \item \textcolor{warningred}{Team ke saath feedback aur collaboration smooth ho (comments).}
  \item \textcolor{warningred}{Elements ka size aur position perfect ho (scale, move).}
  \item \textcolor{warningred}{Purane kaam ya mistakes recover ho sakein (Version History).}
\end{itemize}
In se tu professional designs bana sakta hai jo clients ko impress kare aur developers ke liye clear ho.

% Conclusion in tcolorbox %
\begin{notebox}
\textbf{Conclusion:} %
\begin{itemize}
  \item \textcolor{warningred}{Effects se designs ko visual depth do.}
  \item \textcolor{warningred}{Auto Layout se responsive layouts banao.}
  \item \textcolor{warningred}{Comments aur collaboration se team kaam smooth karo.}
  \item \textcolor{warningred}{Scale aur Move se elements adjust karo.}
  \item \textcolor{warningred}{Version History se changes track aur restore karo.}
\end{itemize}
\end{notebox}

===============================
\hrule


% Section formatting (only for unstarred sections)
\titleformat{\section}
  {\Large\bfseries\color{headingblue}}{}{0em}{}

% Configure listings for code blocks (optional)
\lstset{
  language=Python,
  backgroundcolor=\color{codeblue},
  basicstyle=\ttfamily\small,
  frame=single,
  breaklines=true,
  keywordstyle=\color{blue},
  stringstyle=\color{purple},
  commentstyle=\color{gray},
  showstringspaces=false
}

% Configure tcolorbox for examples
\newtcolorbox{examplebox}[1]{
  colback=examplegreen!10,
  colframe=examplegreen!50!black,
  title=#1,
  breakable,
  enhanced
}

% Configure tcolorbox for notes/conclusion
\newtcolorbox{notebox}{
  colback=warningred!5!white,
  colframe=warningred!75!black,
  title=Point To Note,
  breakable,
  enhanced
}

% Begin document
\begin{document}

% Title Page %
\begin{titlepage}
  \centering
  \vspace*{\fill}
  {\huge\bfseries\color{warningred} Figma Notes: Exporting, Plugins,\\ Variants, Interactions, Masking}\par % Line break, reduced size
  \vspace{1cm}
  {\Large A Comprehensive Guide to Figma Essentials}\par
  \vspace*{\fill}
\end{titlepage}

% Topic 1: Exporting Assets %
\section*{\textbf{\LARGE \textcolor{violet}{Exporting Assets}}} %
\textbf{Corrected aur Enhanced Content:} %
\begin{itemize}
  \item \textbf{Exporting Images}:
    \begin{enumerate}
      \item \textcolor{warningred}{Kisi layer, group, ya frame ko select kar (jaise ek button ya poora mobile screen).}
      \item \textcolor{warningred}{Right panel mein \textbf{Export} section (bottom pe) ja ya right-click karke \textbf{Export} chun.}
      \item \textcolor{warningred}{Format chun (PNG, JPG, SVG, PDF) aur settings adjust kar:}
        \begin{itemize}
          \item \textbf{Size}: 1x, 2x, 3x (for retina displays).
          \item \textbf{Quality}: JPG ke liye compression level.
        \end{itemize}
      \item \textcolor{warningred}{\textbf{Export} button daba ya multiple assets ke liye \textbf{Export [number] items} click kar.}
      \item Exported files tere computer pe save ho jate hain.
    \end{enumerate}
  \item \textbf{Kab Use Karna}:
    \begin{itemize}
      \item \textcolor{warningred}{Assets export kar jab developers ko UI elements ya screens chahiye, jaise icons, buttons, ya mockups.}
      \item Example: Ek app ke icons ko PNG mein ya website ke banner ko JPG mein export karna.
      \item Tip: Multiple resolutions (1x, 2x) export kar taaki devices pe clear dikhe.
    \end{itemize}
\end{itemize}

\begin{examplebox}{Real-Life Example}
Tu ek food delivery app jaise Swiggy ke liye kaam kar raha hai. Developers ko ``Add to Cart'' button ka icon aur checkout screen ka full mockup chahiye. Tu button select karta hai, \textbf{Export} mein PNG format aur 2x size set karta hai, aur mockup ke liye frame select karke JPG 1x export karta hai. Files save karke Slack pe developers ko bhej deta hai -- kaam fast aur smooth ho jata hai.
\end{examplebox}

\textbf{Summary:} %
\begin{itemize}
  \item \textcolor{warningred}{Exporting se layers ya frames images (PNG, JPG, SVG) mein save hote hain developers ke liye.}
  \item \textcolor{warningred}{Element select kar, Export section mein format aur size set kar, aur files save kar. Ye UI handoff ke liye must hai.}
\end{itemize}

% Table for Exporting Assets %
\begin{table}[h]
  \centering
  \begin{tabular}{|p{4cm}|p{4cm}|p{4cm}|}
    \hline
    \cellcolor{tableheaderblue}\color{white}\textbf{Feature} & 
    \cellcolor{tableheadergreen}\color{white}\textbf{Description} & 
    \cellcolor{tableheaderyellow}\color{black}\textbf{Use Case} \\
    \hline
    \rowcolor{codeblue}
    Export & Save as PNG, JPG & Developer handoff \\
    \hline
    \rowcolor{tablerowgreen}
    Settings & 1x, 2x sizes & Device compatibility \\
    \hline
    \rowcolor{codeblue}
    Format & SVG, PDF & Icons, mockups \\
    \hline
  \end{tabular}
  \caption{Exporting Assets Overview}
  \label{tab:exporting_assets}
\end{table}

% Topic 2: Plugins %
\section*{\textbf{\LARGE \textcolor{violet}{Plugins}}} %
\textbf{Corrected aur Enhanced Content:} %
\begin{itemize}
  \item \textbf{Using Plugins}:
    \begin{enumerate}
      \item \textcolor{warningred}{Top menu mein \textbf{Plugins > Browse Plugins in Community} ja ya toolbar ke Resources icon (\texttt{\symbol{4})) se Community khol.}
      \item \textcolor{warningred}{Search bar mein plugin dhoondh (jaise ``Figma to HTML,'' ``Unsplash,'' ``Iconify'').}
      \item \textcolor{warningred}{\textbf{Install} click kar aur plugin Figma mein add ho jayega.}
      \item Plugins pane se run kar (Resources > Plugins).
    \end{enumerate}
  \item \textbf{Popular Plugins}:
    \begin{itemize}
      \item \textbf{Figma to HTML}: Designs ko HTML/CSS code mein convert karta hai.
      \item \textbf{Unsplash}: Free high-quality images directly Figma mein daalta hai.
      \item \textbf{Content Reel}: Dummy text ya avatars jaldi add karta hai.
      \item \textbf{Stark}: Accessibility check karta hai (color contrast, etc.).
    \end{itemize}
  \item \textbf{Kab Use Karna}:
    \begin{itemize}
      \item \textcolor{warningred}{Plugins tab use kar jab tujhe repetitive tasks automate karne hon ya extra functionality chahiye.}
      \item Example: ``Figma to HTML'' plugin se landing page ka code generate kar developers ke liye.
      \item Tip: Sirf trusted plugins install kar taaki Figma slow na ho.
    \end{itemize}
\end{itemize}

\begin{examplebox}{Real-Life Example}
Tu ek portfolio website design kar raha hai. Photos ke liye \textbf{Unsplash} plugin run karta hai aur canvas pe hi hero section ke liye ek high-quality image daalta hai -- download-upload ki zarurat nahi. Phir \textbf{Figma to HTML} plugin se design ka basic HTML/CSS code generate karta hai aur developer ko deta hai. Ye plugins time bacha ke kaam jaldi karte hain.
\end{examplebox}

\textbf{Summary:} %
\begin{itemize}
  \item \textcolor{warningred}{Plugins Figma ki functionality badhate hain.}
  \item \textcolor{warningred}{Community se plugins dhoondh, install kar, aur tasks jaise image import ya code generation automate kar. Ye designer ka workflow fast aur efficient banata hai.}
\end{itemize}

% Table for Plugins %
\begin{table}[h]
  \centering
  \begin{tabular}{|p{4cm}|p{4cm}|p{4cm}|}
    \hline
    \cellcolor{tableheaderblue}\color{white}\textbf{Plugin} & 
    \cellcolor{tableheadergreen}\color{white}\textbf{Description} & 
    \cellcolor{tableheaderyellow}\color{black}\textbf{Use Case} \\
    \hline
    \rowcolor{codeblue}
    Unsplash & Add images & Hero sections \\
    \hline
    \rowcolor{tablerowgreen}
    Figma to HTML & Code generation & Developer handoff \\
    \hline
    \rowcolor{codeblue}
    Stark & Accessibility check & Inclusive design \\
    \hline
  \end{tabular}
  \caption{Plugins Overview}
  \label{tab:plugins}
\end{table}

% Topic 3: Component Variants %
\section*{\textbf{\LARGE \textcolor{violet}{Component Variants}}} %
\textbf{Corrected aur Enhanced Content:} %
\begin{itemize}
  \item \textbf{Creating Variants}:
    \begin{enumerate}
      \item \textcolor{warningred}{Ek component bana (jaise ek button) ya existing component select kar.}
      \item \textcolor{warningred}{Right panel mein \textbf{Variants} section mein \texttt{+} button click kar.}
      \item \textcolor{warningred}{Variants add kar (jaise ``Normal,'' ``Hover,'' ``Pressed'') aur har variant ke properties change kar (color, size, text, etc.).}
      \item Variants ko combine kar ek single component set mein taaki Assets Panel mein organized rahe.
      \item Instances mein variant switch kar by selecting from dropdown.
    \end{enumerate}
  \item \textbf{Kab Use Karna}:
    \begin{itemize}
      \item \textcolor{warningred}{Variants tab use kar jab ek component ke alag-alag states chahiye, jaise button ka normal, hover, ya disabled state.}
      \item Example: Ek form ke ``Submit'' button ke liye Normal (blue), Hover (dark blue), aur Disabled (grey) variants bana.
      \item Tip: Variant names clear rakho, jaise ``State=Hover'' taaki samajh aaye.
    \end{itemize}
\end{itemize}

\begin{examplebox}{Real-Life Example}
Tu ek banking app jaise PhonePe ke liye ``Transfer'' button design kar raha hai. Button ka component banata hai aur variants add karta hai: \textbf{Normal} (green background, white text), \textbf{Hover} (dark green, bold text), aur \textbf{Pressed} (light green, smaller size). Assets Panel se instance drag karke variant switch karta hai. Jab prototype mein test karta hai, button states realistic lagte hain, aur developers ko clear idea milta hai.
\end{examplebox}

\textbf{Summary:} %
\begin{itemize}
  \item \textcolor{warningred}{Component Variants se ek component ke alag states (normal, hover, pressed) banaye jate hain.}
  \item \textcolor{warningred}{Component select kar, Variants section mein states add kar, aur properties set kar. Ye consistency aur prototyping ke liye bohot useful hai.}
\end{itemize}

% Table for Component Variants %
\begin{table}[h]
  \centering
  \begin{tabular}{|p{4cm}|p{4cm}|p{4cm}|}
    \hline
    \cellcolor{tableheaderblue}\color{white}\textbf{Feature} & 
    \cellcolor{tableheadergreen}\color{white}\textbf{Description} & 
    \cellcolor{tableheaderyellow}\color{black}\textbf{Use Case} \\
    \hline
    \rowcolor{codeblue}
    Variants & Component states & Button states \\
    \hline
    \rowcolor{tablerowgreen}
    Creation & Add via \texttt{+} & Hover effects \\
    \hline
    \rowcolor{codeblue}
    Organization & Component set & Consistent design \\
    \hline
  \end{tabular}
  \caption{Component Variants Overview}
  \label{tab:component_variants}
\end{table}

% Topic 4: Interactive Components %
\section*{\textbf{\LARGE \textcolor{violet}{Interactive Components}}} %
\textbf{Corrected aur Enhanced Content:} %
\begin{itemize}
  \item \textbf{Adding Interactivity}:
    \begin{enumerate}
      \item \textcolor{warningred}{Ek component ya variant set bana (jaise button ke Normal aur Hover variants).}
      \item \textcolor{warningred}{\textbf{Prototype Tab} mein switch kar.}
      \item \textcolor{warningred}{Component instance select kar, aur right panel mein \textbf{Interactions} section se action add kar:}
        \begin{itemize}
          \item \textbf{On Hover}: Variant change (jaise Normal se Hover).
          \item \textbf{On Click}: Navigate ya animation.
        \end{itemize}
      \item Animation settings chun (jaise ``Smart Animate'' for smooth transitions).
      \item Prototype run karke test kar (\texttt{Play} button).
    \end{enumerate}
  \item \textbf{Kab Use Karna}:
    \begin{itemize}
      \item \textcolor{warningred}{Interactive components tab bana jab tujhe real app jaisa feel chahiye, jaise button ka hover effect ya toggle ka on/off state.}
      \item Example: Ek button ko hover pe color change ya size badhane ka effect de.
      \item Tip: Smart Animate use kar taaki transitions smooth hon.
    \end{itemize}
\end{itemize}

\begin{examplebox}{Real-Life Example}
Tu ek music app jaise Spotify ke liye play button design kar raha hai. Button ke variants banaye: \textbf{Normal} (green circle) aur \textbf{Hover} (larger green circle with shadow). Prototype Tab mein ``On Hover'' interaction set karta hai taaki Normal se Hover variant switch ho, aur Smart Animate se smooth scaling hota hai. Jab client prototype dekhta hai, button hover karne pe real app jaisa lagta hai -- client impress ho jata hai.
\end{examplebox}

\textbf{Summary:} %
\begin{itemize}
  \item \textcolor{warningred}{Interactive components se buttons ya elements ke states (hover, click) real-time mein dikhte hain.}
  \item \textcolor{warningred}{Variants bana, Prototype Tab mein interactions set kar, aur Smart Animate se polish de. Ye prototyping aur client demos ke liye zaroori hai.}
\end{itemize}

% Table for Interactive Components %
\begin{table}[h]
  \centering
  \begin{tabular}{|p{4cm}|p{4cm}|p{4cm}|}
    \hline
    \cellcolor{tableheaderblue}\color{white}\textbf{Feature} & 
    \cellcolor{tableheadergreen}\color{white}\textbf{Description} & 
    \cellcolor{tableheaderyellow}\color{black}\textbf{Use Case} \\
    \hline
    \rowcolor{codeblue}
    Interactions & Hover, Click & Button effects \\
    \hline
    \rowcolor{tablerowgreen}
    Smart Animate & Smooth transitions & Realistic prototypes \\
    \hline
    \rowcolor{codeblue}
    Prototype Tab & Set actions & Client demos \\
    \hline
  \end{tabular}
  \caption{Interactive Components Overview}
  \label{tab:interactive_components}
\end{table}

% Topic 5: Masking %
\section*{\textbf{\LARGE \textcolor{violet}{Masking}}} %
\textbf{Corrected aur Enhanced Content:} %
\begin{itemize}
  \item \textbf{Applying Masks}:
    \begin{enumerate}
      \item \textcolor{warningred}{Ek shape (jaise rectangle ya circle) bana jo mask ka boundary define karega.}
      \item \textcolor{warningred}{Jis element ko mask karna hai (jaise image ya group), usko shape ke upar rakh.}
      \item \textcolor{warningred}{Dono select kar (shape neeche, content upar), aur toolbar se \textbf{Use as Mask} click kar ya right-click karke \textbf{Mask} chun.}
      \item Masked content sirf shape ke andar visible hoga, baaki crop ho jayega.
      \item Mask edit karne ke liye double-click kar.
    \end{enumerate}
  \item \textbf{Kab Use Karna}:
    \begin{itemize}
      \item \textcolor{warningred}{Masking tab use kar jab tujhe content ka specific part dikhana ho, jaise circular profile picture ya cropped background.}
      \item Example: Ek app ke user avatar ko circle mein dikhane ke liye image pe circular mask laga.
      \item Tip: Mask shape ko resize ya move kar taaki content adjust ho.
    \end{itemize}
\end{itemize}

\begin{examplebox}{Real-Life Example}
Tu ek social media app jaise LinkedIn ke liye profile card design kar raha hai. User ka photo rectangular hai, lekin tujhe circular avatar chahiye. Tu ek circle shape banata hai, photo ko uske upar rakhta hai, aur \textbf{Use as Mask} karta hai. Ab photo sirf circle ke andar dikhta hai, baaki crop ho jata hai. Agar photo thoda off-center hai, toh mask edit karke adjust karta hai -- ekdum clean look milta hai.
\end{examplebox}

\textbf{Summary:} %
\begin{itemize}
  \item \textcolor{warningred}{Masking se content ka visibility shape ke hisaab se control hota hai.}
  \item \textcolor{warningred}{Shape aur content select kar, \textbf{Use as Mask} kar, aur specific area dikhaye. Ye images ya complex visuals ke liye bohot kaam aata hai.}
\end{itemize}

% Table for Masking %
\begin{table}[h]
  \centering
  \begin{tabular}{|p{4cm}|p{4cm}|p{4cm}|}
    \hline
    \cellcolor{tableheaderblue}\color{white}\textbf{Feature} & 
    \cellcolor{tableheadergreen}\color{white}\textbf{Description} & 
    \cellcolor{tableheaderyellow}\color{black}\textbf{Use Case} \\
    \hline
    \rowcolor{codeblue}
    Masking & Shape-based crop & Circular avatars \\
    \hline
    \rowcolor{tablerowgreen}
    Use as Mask & Toolbar option & Profile pictures \\
    \hline
    \rowcolor{codeblue}
    Edit Mask & Double-click & Adjust content \\
    \hline
  \end{tabular}
  \caption{Masking Overview}
  \label{tab:masking}
\end{table}

% Extra Tips for Beginners %
\section*{\textbf{\LARGE \textcolor{violet}{Beginners ke liye Extra Tips}}} %
\begin{itemize}
  \item \textbf{Export Naming}: \textcolor{warningred}{Export karte waqt files ka naam clear rak \newline rakho (jaise ``button-primary-2x.png'') taaki developers ko samajh aaye.}
  \item \textbf{Plugin Updates}: \textcolor{warningred}{Plugins regularly check kar taaki latest features milein, lekin unnecessary plugins hata do taaki Figma light rahe.}
  \item \textbf{Variant Organization}: \textcolor{warningred}{Variants ko sets mein organize kar (jaise ``Button/Primary'') taaki Assets Panel mein clutter na ho.}
  \item \textbf{Interactive Testing}: \textcolor{warningred}{Interactive components banane ke baad hamesha Prototype mein Play button se test kar taaki flow smooth ho.}
  \item \textbf{Mask Layers}: \textcolor{warningred}{Mask ke liye alag layer group bana taaki complex designs mein confusion na ho.}
  \item \textbf{Practice Project}: Ek chhota project try kar -- jaise ek e-commerce app ka product card bana jisme image mask kar, variants ke saath button bana, interactive hover effect daal, aur assets export kar.
\end{itemize}

% Table for Extra Tips %
\begin{table}[h]
  \centering
  \begin{tabular}{|p{4cm}|p{4cm}|p{4cm}|}
    \hline
    \cellcolor{tableheaderblue}\color{white}\textbf{Tip} & 
    \cellcolor{tableheadergreen}\color{white}\textbf{Description} & 
    \cellcolor{tableheaderyellow}\color{black}\textbf{Benefit} \\
    \hline
    \rowcolor{codeblue}
    Export Naming & Clear file names & Easy handoff \\
    \hline
    \rowcolor{tablerowgreen}
    Variant Organization & Use sets & Clean Assets Panel \\
    \hline
    \rowcolor{codeblue}
    Interactive Testing & Test prototypes & Smooth flow \\
    \hline
  \end{tabular}
  \caption{Beginner Tips Overview}
  \label{tab:beginner_tips}
\end{table}

% UI/UX Importance %
\section*{\textbf{\LARGE \textcolor{violet}{UI/UX Job ke liye Kyun Zaroori}}} %
UI/UX designer ke roop mein ye skills daily kaam aayengi taaki:
\begin{itemize}
  \item \textcolor{warningred}{Developers ke liye assets jaldi aur clearly provide ho (exporting).}
  \item \textcolor{warningred}{Workflow fast aur automated ho (plugins).}
  \item \textcolor{warningred}{Components consistent aur flexible hon (variants).}
  \item \textcolor{warningred}{Prototypes realistic aur engaging hon (interactive components).}
  \item \textcolor{warningred}{Visuals precise aur polished hon (masking).}
\end{itemize}
In se tu professional designs bana sakta hai jo clients ko pasand aaye aur developers ke liye handoff aasan ho.

% Conclusion in tcolorbox %
\begin{notebox}
\textbf{Conclusion:} %
\begin{itemize}
  \item \textcolor{warningred}{Exporting se assets developers tak pahunchte hain.}
  \item \textcolor{warningred}{Plugins workflow ko speed dete hain.}
  \item \textcolor{warningred}{Variants se components flexible hote hain.}
  \item \textcolor{warningred}{Interactive components prototypes ko real banate hain.}
  \item \textcolor{warningred}{Masking se visuals perfect hote hain.}
\end{itemize}
\end{notebox}



===============================
\hrule

% Section formatting (only for unstarred sections)
\titleformat{\section}
  {\Large\bfseries\color{headingblue}}{}{0em}{}

% Configure listings for code blocks (optional)
\lstset{
  language=Python,
  backgroundcolor=\color{codeblue},
  basicstyle=\ttfamily\small,
  frame=single,
  breaklines=true,
  keywordstyle=\color{blue},
  stringstyle=\color{purple},
  commentstyle=\color{gray},
  showstringspaces=false
}

% Configure tcolorbox for examples
\newtcolorbox{examplebox}[1]{
  colback=examplegreen!10,
  colframe=examplegreen!50!black,
  title=#1,
  breakable,
  enhanced
}

% Configure tcolorbox for notes/conclusion
\newtcolorbox{notebox}{
  colback=warningred!5!white,
  colframe=warningred!75!black,
  title=Point To Note,
  breakable,
  enhanced
}

% Begin document
\begin{document}

% Title Page %
\begin{titlepage}
  \centering
  \vspace*{\fill}
  {\huge\bfseries\color{warningred} Figma Notes: Constraints, Viewport,\\ Styles, Live Preview}\par % Line break, reduced size
  \vspace{1cm}
  {\Large A Comprehensive Guide to Figma Essentials}\par
  \vspace*{\fill}
\end{titlepage}

% Topic 1: Constraints %
\section*{\textbf{\LARGE \textcolor{violet}{Constraints}}} %
\textbf{Corrected aur Enhanced Content:} %
\begin{itemize}
  \item \textbf{Setting Constraints}:
    \begin{enumerate}
      \item \textcolor{warningred}{Ek frame ke andar element (jaise button, text, ya image) select kar.}
      \item \textcolor{warningred}{Right panel mein \textbf{Constraints} section (pin icon) mein options dekh:}
        \begin{itemize}
          \item \textbf{Horizontal}: Left, Right, Center, ya Scale (element kaise stretch hoga).
          \item \textbf{Vertical}: Top, Bottom, Center, ya Scale.
        \end{itemize}
      \item \textcolor{warningred}{Constraints set kar taaki element frame resize hone pe sahi jagah rahe.}
        \begin{itemize}
          \item Example: Ek button ko ``Right'' aur ``Bottom'' constrain kar taaki frame bada hone pe wo bottom-right corner mein fixed rahe.
        \end{itemize}
      \item Frame resize karke test kar ki elements sahi behave kar rahe hain.
    \end{enumerate}
  \item \textbf{Kab Use Karna}:
    \begin{itemize}
      \item \textcolor{warningred}{Constraints tab use kar jab tujhe responsive design banana ho jo alag-alag screen sizes (mobile, tablet, desktop) pe kaam kare.}
      \item Example: Ek navbar ke logo ko ``Left'' aur ``Top'' constrain kar taaki screen size badalne pe wo hamesha top-left mein rahe.
      \item Tip: Constraints sirf frames ke andar kaam karte hain, groups mein nahi.
    \end{itemize}
\end{itemize}

\begin{examplebox}{Real-Life Example}
Tu ek e-commerce app jaise Myntra ke liye product page design kar raha hai. Ek ``Add to Cart'' button frame ke bottom-right mein hai. Tu isko \textbf{Right} aur \textbf{Bottom} constraints deta hai. Jab frame ko mobile (375px) se tablet (768px) size mein resize karta hai, button hamesha bottom-right corner mein rehta hai, aur layout consistent dikhta hai. Ye responsive design ke liye perfect hai.
\end{examplebox}

\textbf{Summary:} %
\begin{itemize}
  \item \textcolor{warningred}{Constraints se elements frame resize hone pe sahi position mein rehte hain.}
  \item \textcolor{warningred}{Element select kar, right panel mein Horizontal/Vertical constraints set kar, aur responsive layouts bana. Ye alag-alag devices ke liye must hai.}
\end{itemize}

% Table for Constraints %
\begin{table}[h]
  \centering
  \begin{tabular}{|p{4cm}|p{4cm}|p{4cm}|}
    \hline
    \cellcolor{tableheaderblue}\color{white}\textbf{Feature} & 
    \cellcolor{tableheadergreen}\color{white}\textbf{Description} & 
    \cellcolor{tableheaderyellow}\color{black}\textbf{Use Case} \\
    \hline
    \rowcolor{codeblue}
    Constraints & Position fixing & Responsive layouts \\
    \hline
    \rowcolor{tablerowgreen}
    Options & Left, Right, Top & Button placement \\
    \hline
    \rowcolor{codeblue}
    Testing & Resize frame & Device compatibility \\
    \hline
  \end{tabular}
  \caption{Constraints Overview}
  \label{tab:constraints}
\end{table}

% Topic 2: Viewport Resizing %
\section*{\textbf{\LARGE \textcolor{violet}{Viewport Resizing}}} %
\textbf{Corrected aur Enhanced Content:} %
\begin{itemize}
  \item \textbf{Simulating Device Viewports}:
    \begin{enumerate}
      \item \textcolor{warningred}{Ek frame select kar (jaise mobile ya desktop layout).}
      \item \textcolor{warningred}{Right panel mein \textbf{Frame} section se preset sizes chun (jaise iPhone 14: 390x844px, Desktop: 1920x1080px) ya manually width/height adjust kar.}
      \item \textcolor{warningred}{Canvas ke edges ya corner handles drag karke frame resize kar, aur dekho elements kaise adjust hote hain.}
      \item Layout Grid (\texttt{Ctrl+'}) on kar taaki alignment check kar sake.
    \end{enumerate}
  \item \textbf{Kab Use Karna}:
    \begin{itemize}
      \item \textcolor{warningred}{Viewport resizing tab use kar jab tujhe design ko alag-alag device sizes pe test karna ho taaki responsive ho.}
      \item Example: Ek website ka layout desktop pe banaya, phir mobile size mein resize karke check kar ki buttons aur text sahi dikhte hain.
      \item Tip: Common device sizes (375px mobile, 1440px desktop) test kar taaki real-world scenarios cover ho.
    \end{itemize}
\end{itemize}

\begin{examplebox}{Real-Life Example}
Tu ek news app jaise Inshorts design kar raha hai. Homepage ka frame pehle desktop size (1440x900px) mein banata hai. Phir right panel se iPhone 13 (390x844px) size select karta hai aur dekhta hai ki text chhota hone pe bhi readable hai ya nahi. Agar buttons overlap karte hain, toh Auto Layout ya constraints adjust karta hai. Ye ensure karta hai ki app har device pe acha dikhe.
\end{examplebox}

\textbf{Summary:} %
\begin{itemize}
  \item \textcolor{warningred}{Viewport resizing se designs alag-alag device sizes pe test hote hain.}
  \item \textcolor{warningred}{Frame select kar, preset sizes ya manual resize use kar, aur responsiveness check kar. Ye UI ko har screen pe perfect banata hai.}
\end{itemize}

% Table for Viewport Resizing %
\begin{table}[h]
  \centering
  \begin{tabular}{|p{4cm}|p{4cm}|p{4cm}|}
    \hline
    \cellcolor{tableheaderblue}\color{white}\textbf{Feature} & 
    \cellcolor{tableheadergreen}\color{white}\textbf{Description} & 
    \cellcolor{tableheaderyellow}\color{black}\textbf{Use Case} \\
    \hline
    \rowcolor{codeblue}
    Viewport Resize & Test device sizes & Responsive design \\
    \hline
    \rowcolor{tablerowgreen}
    Presets & iPhone, Desktop & Mobile testing \\
    \hline
    \rowcolor{codeblue}
    Grid & \texttt{Ctrl+'} & Alignment check \\
    \hline
  \end{tabular}
  \caption{Viewport Resizing Overview}
  \label{tab:viewport_resizing}
\end{table}

% Topic 3: Global Styles %
\section*{\textbf{\LARGE \textcolor{violet}{Global Styles}}} %
\textbf{Corrected aur Enhanced Content:} %
\begin{itemize}
  \item \textbf{Defining Global Styles}:
    \begin{enumerate}
      \item \textcolor{warningred}{Right panel mein \textbf{Styles} section (four dots icon) ja.}
      \item \textcolor{warningred}{\texttt{+} button click karke style bana:}
        \begin{itemize}
          \item \textbf{Color}: Primary, secondary colors (jaise \#FF5733 for buttons).
          \item \textbf{Text}: Font size, weight, line height (jaise H1: 32px bold).
          \item \textbf{Effect}: Shadows, blurs (jaise button shadow).
        \end{itemize}
      \item \textcolor{warningred}{Style ka naam de (jaise ``Primary Blue,'' ``Heading 1'') aur save kar.}
      \item Kisi element pe style apply kar by selecting from Styles dropdown.
      \item Global style edit karne se saare instances automatically update ho jate hain.
    \end{enumerate}
  \item \textbf{Kab Use Karna}:
    \begin{itemize}
      \item \textcolor{warningred}{Global styles tab use kar jab tujhe poore design mein consistency chahiye, jaise same colors ya fonts har jagah.}
      \item Example: Ek app ke saare buttons ke liye ek ``Primary Color'' style bana taaki ek change se sab update ho.
      \item Tip: Team ke saath styles share karne ke liye Library publish kar.
    \end{itemize}
\end{itemize}

\begin{examplebox}{Real-Life Example}
Tu ek fitness app jaise Cult.fit ke liye kaam kar raha hai. Primary color \#00C4B4 define karta hai as a global style ``Brand Green.'' Saare buttons, icons, aur highlights mein ye style use karta hai. Jab client bolta hai ki color thoda dark karo (\#00A89A), tu global style edit karta hai, aur poora design ekdum update ho jata hai -- koi manual change nahi chahiye. Isse time bachta hai aur brand consistency rehti hai.
\end{examplebox}

\textbf{Summary:} %
\begin{itemize}
  \item \textcolor{warningred}{Global styles se colors, text, aur effects ko centrally manage kiya jata hai.}
  \item \textcolor{warningred}{Styles section mein bana, apply kar, aur edit kar taaki saare instances update ho. Ye design ko consistent aur editing ko fast rakhta hai.}
\end{itemize}

% Table for Global Styles %
\begin{table}[h]
  \centering
  \begin{tabular}{|p{4cm}|p{4cm}|p{4cm}|}
    \hline
    \cellcolor{tableheaderblue}\color{white}\textbf{Feature} & 
    \cellcolor{tableheadergreen}\color{white}\textbf{Description} & 
    \cellcolor{tableheaderyellow}\color{black}\textbf{Use Case} \\
    \hline
    \rowcolor{codeblue}
    Global Styles & Central management & Brand consistency \\
    \hline
    \rowcolor{tablerowgreen}
    Types & Color, Text & Buttons, fonts \\
    \hline
    \rowcolor{codeblue}
    Edit & Auto-update & Fast changes \\
    \hline
  \end{tabular}
  \caption{Global Styles Overview}
  \label{tab:global_styles}
\end{table}

% Topic 4: Live Device Preview %
\section*{\textbf{\LARGE \textcolor{violet}{Live Device Preview}}} %
\textbf{Corrected aur Enhanced Content:} %
\begin{itemize}
  \item \textbf{Previewing on Devices}:
    \begin{enumerate}
      \item \textcolor{warningred}{Figma Mirror app download kar (iOS/Android) ya web browser mein Figma Mirror khol (mirror.figma.com).}
      \item \textcolor{warningred}{Figma file mein top-right corner se \textbf{Play} button (\texttt{\symbol{16}}) click kar aur device select kar.}
      \item \textcolor{warningred}{Mirror app ya browser pe design real-time mein dikhega, aur prototype interactions test ho sakte hain.}
      \item Device rotate ya zoom karke real-world feel check kar.
    \end{enumerate}
  \item \textbf{Kab Use Karna}:
    \begin{itemize}
      \item \textcolor{warningred}{Live preview tab use kar jab tujhe design ko actual device pe dekhna ho taaki pixel-perfect aur user experience check ho.}
      \item Example: Ek mobile app ka prototype phone pe test kar taaki touch interactions real feel karein.
      \item Tip: Stable internet rakho kyunki Mirror cloud-based hai.
    \end{itemize}
\end{itemize}

\begin{examplebox}{Real-Life Example}
Tu ek payment app jaise Google Pay design kar raha hai. Login screen ka prototype banane ke baad Figma Mirror app kholta hai aur apne phone pe design preview karta hai. ``Sign In'' button click karke dashboard pe jata hai, aur dekhta hai ki animations smooth hain ya nahi. Client meeting mein phone pe ye demo dikhata hai -- real device pe design dekhke client impress ho jata hai.
\end{examplebox}

\textbf{Summary:} %
\begin{itemize}
  \item \textcolor{warningred}{Live Device Preview se designs actual devices pe test hote hain.}
  \item \textcolor{warningred}{Figma Mirror app ya browser use kar, Play button se preview shuru kar, aur interactions check kar. Ye real-world testing ke liye zaroori hai.}
\end{itemize}

% Table for Live Device Preview %
\begin{table}[h]
  \centering
  \begin{tabular}{|p{4cm}|p{4cm}|p{4cm}|}
    \hline
    \cellcolor{tableheaderblue}\color{white}\textbf{Feature} & 
    \cellcolor{tableheadergreen}\color{white}\textbf{Description} & 
    \cellcolor{tableheaderyellow}\color{black}\textbf{Use Case} \\
    \hline
    \rowcolor{codeblue}
    Live Preview & Device testing & Prototype demos \\
    \hline
    \rowcolor{tablerowgreen}
    Figma Mirror & Real-time view & Mobile interactions \\
    \hline
    \rowcolor{codeblue}
    Play Button & \texttt{\symbol{16}} & Start preview \\
    \hline
  \end{tabular}
  \caption{Live Device Preview Overview}
  \label{tab:live_preview}
\end{table}

% Extra Tips for Beginners %
\section*{\textbf{\LARGE \textcolor{violet}{Beginners ke liye Extra Tips}}} %
\begin{itemize}
  \item \textbf{Constraint Testing}: \textcolor{warningred}{Constraints set karne ke baad hamesha frame resize karke test kar taaki galti pata chal jaye.}
  \item \textbf{Viewport Presets}: \textcolor{warningred}{Common device sizes (iPhone, iPad, Desktop) ke presets save kar taaki jaldi switch kar sake.}
  \item \textbf{Style Naming}: \textcolor{warningred}{Global styles ke naam descriptive rakho (jaise ``Primary/Blue'' ya ``H1/Bold'') taaki team ko samajh aaye.}
  \item \textbf{Mirror Setup}: \textcolor{warningred}{Figma Mirror app pe same account login kar taaki preview instantly load ho.}
  \item \textbf{Practice Project}: Ek chhota project try kar -- jaise ek app ka homepage bana jisme constraints laga, global styles define kar, viewports test kar, aur Mirror pe preview kar.
\end{itemize}

% Table for Extra Tips %
\begin{table}[h]
  \centering
  \begin{tabular}{|p{4cm}|p{4cm}|p{4cm}|}
    \hline
    \cellcolor{tableheaderblue}\color{white}\textbf{Tip} & 
    \cellcolor{tableheadergreen}\color{white}\textbf{Description} & 
    \cellcolor{tableheaderyellow}\color{black}\textbf{Benefit} \\
    \hline
    \rowcolor{codeblue}
    Constraint Testing & Test after setting & Catch errors \\
    \hline
    \rowcolor{tablerowgreen}
    Style Naming & Descriptive names & Team clarity \\
    \hline
    \rowcolor{codeblue}
    Mirror Setup & Same account login & Fast previews \\
    \hline
  \end{tabular}
  \caption{Beginner Tips Overview}
  \label{tab:beginner_tips}
\end{table}

% UI/UX Importance %
\section*{\textbf{\LARGE \textcolor{violet}{UI/UX Job ke liye Kyun Zaroori}}} %
UI/UX designer ke roop mein ye skills daily kaam aayengi taaki:
\begin{itemize}
  \item \textcolor{warningred}{Designs har device pe sahi dikhein aur kaam karein (constraints, viewport resizing).}
  \item \textcolor{warningred}{Brand consistency aur editing speed bani rahe (global styles).}
  \item \textcolor{warningred}{Real devices pe prototypes test ho sakein (live preview).}
\end{itemize}
In se tu responsive, professional, aur user-friendly designs bana sakta hai jo clients aur developers dono ko pasand aaye.

% Conclusion in tcolorbox %
\begin{notebox}
\textbf{Conclusion:} %
\begin{itemize}
  \item \textcolor{warningred}{Constraints se layouts responsive banta hai.}
  \item \textcolor{warningred}{Viewport resizing se device compatibility check hoti hai.}
  \item \textcolor{warningred}{Global styles consistency dete hain.}
  \item \textcolor{warningred}{Live preview real-world testing ke liye hai.}
\end{itemize}
\end{notebox}

===============================
\hrule


% Section formatting (only for unstarred sections)
\titleformat{\section}
  {\Large\bfseries\color{headingblue}}{}{0em}{}

% Configure listings for code blocks (optional)
\lstset{
  language=Python,
  backgroundcolor=\color{codeblue},
  basicstyle=\ttfamily\small,
  frame=single,
  breaklines=true,
  keywordstyle=\color{blue},
  stringstyle=\color{purple},
  commentstyle=\color{gray},
  showstringspaces=false
}

% Configure tcolorbox for examples
\newtcolorbox{examplebox}[1]{
  colback=examplegreen!10,
  colframe=examplegreen!50!black,
  title=#1,
  breakable,
  enhanced
}

% Configure tcolorbox for notes/conclusion
\newtcolorbox{notebox}{
  colback=warningred!5!white,
  colframe=warningred!75!black,
  title=Point To Note,
  breakable,
  enhanced
}

% Begin document
\begin{document}

% Title Page %
\begin{titlepage}
  \centering
  \vspace*{\fill}
  {\huge\bfseries\color{warningred} Figma Notes: Placeholder, Backgrounds,\\ Fonts, Shortcuts}\par % Line break, reduced size
  \vspace{1cm}
  {\Large A Comprehensive Guide to Figma Essentials}\par
  \vspace*{\fill}
\end{titlepage}

% Topic 1: Data and Content Placeholder %
\section*{\textbf{\LARGE \textcolor{violet}{Data and Content Placeholder}}} %
\textbf{Corrected aur Enhanced Content:} %
\begin{itemize}
  \item \textbf{Adding Placeholder Text}:
    \begin{enumerate}
      \item \textcolor{warningred}{Text Tool (\texttt{T}) se canvas pe text box bana.}
      \item \textcolor{warningred}{Placeholder text ke liye Figma ke built-in feature ya plugin use kar:}
        \begin{itemize}
          \item Type \texttt{/} aur ``Lorem Ipsum'' select kar, ya plugin jaise \textbf{Content Reel} install kar.
          \item Content Reel se dummy text, names, emails, ya numbers generate kar sakta hai.
          \item Characters ya words ki specific length set kar (jaise 50 words ka paragraph).
        \end{itemize}
      \item \textcolor{warningred}{Placeholder text ko style kar (font, size, color) taaki real content jaisa lage.}
    \end{enumerate}
  \item \textbf{Kab Use Karna}:
    \begin{itemize}
      \item \textcolor{warningred}{Placeholder text tab use kar jab tujhe design mein content ka idea dikhana ho bina actual text ke, jaise mockups ke liye.}
      \item Example: Ek blog page ke liye dummy paragraphs ya headings daal taaki layout finalized ho sake.
      \item Tip: Plugins jaise Content Reel ya Mockuuups Studio se realistic data (jaise user names, dates) add kar.
    \end{itemize}
\end{itemize}

\begin{examplebox}{Real-Life Example}
Tu ek news app jaise Times of India ke liye article page design kar raha hai. Actual content abhi nahi hai, toh tu Text Tool (\texttt{T}) se ek heading aur body text box banata hai. \textbf{Content Reel} plugin se 3 dummy sentences heading ke liye aur 200 words ka paragraph body ke liye generate karta hai. Font aur spacing adjust karke layout realistic banata hai -- client ko idea mil jata hai bina real text ke.
\end{examplebox}

\textbf{Summary:} %
\begin{itemize}
  \item \textcolor{warningred}{Placeholder text se dummy content add kiya jata hai taaki design ka layout test ho sake.}
  \item \textcolor{warningred}{Text Tool aur plugins jaise Content Reel use kar, length set kar, aur style kar. Ye mockups aur client presentations ke liye must hai.}
\end{itemize}

% Table for Data and Content Placeholder %
\begin{table}[h]
  \centering
  \begin{tabular}{|p{4cm}|p{4cm}|p{4cm}|}
    \hline
    \cellcolor{tableheaderblue}\color{white}\textbf{Feature} & 
    \cellcolor{tableheadergreen}\color{white}\textbf{Description} & 
    \cellcolor{tableheaderyellow}\color{black}\textbf{Use Case} \\
    \hline
    \rowcolor{codeblue}
    Placeholder Text & Dummy content & Mockup layouts \\
    \hline
    \rowcolor{tablerowgreen}
    Content Reel & Generate data & Articles, forms \\
    \hline
    \rowcolor{codeblue}
    Text Tool & \texttt{T} & Quick text boxes \\
    \hline
  \end{tabular}
  \caption{Data and Content Placeholder Overview}
  \label{tab:placeholder}
\end{table}

% Topic 2: Background Styles %
\section*{\textbf{\LARGE \textcolor{violet}{Background Styles}}} %
\textbf{Corrected aur Enhanced Content:} %
\begin{itemize}
  \item \textbf{Applying Background Styles}:
    \begin{enumerate}
      \item \textcolor{warningred}{Ek frame select kar (jaise mobile screen ya webpage).}
      \item \textcolor{warningred}{Right panel mein \textbf{Fill} section (under Frame ya Artboard) ja.}
      \item \textcolor{warningred}{Background ke liye options chun:}
        \begin{itemize}
          \item \textbf{Solid Color}: Ek single color set kar (jaise \#F5F5F5).
          \item \textbf{Gradient}: Linear ya radial gradient bana (jaise blue se white).
          \item \textbf{Image}: Image upload kar ya drag-and-drop kar, aur fit/stretch adjust kar.
        \end{itemize}
      \item Opacity ya blend mode tweak kar taaki background content ke saath balance kare.
    \end{enumerate}
  \item \textbf{Kab Use Karna}:
    \begin{itemize}
      \item \textcolor{warningred}{Background styles tab use kar jab tujhe app screen ya webpage ka vibe set karna ho.}
      \item Example: Ek app ke login screen pe light gradient background daal taaki modern lage.
      \item Tip: Background ko global style bana taaki poore design mein reuse ho sake.
    \end{itemize}
\end{itemize}

\begin{examplebox}{Real-Life Example}
Tu ek meditation app jaise Calm ke liye homepage design kar raha hai. Frame select karta hai aur \textbf{Fill} mein ek soft linear gradient daalta hai (light blue \#AED9E0 se white \#FFFFFF). Phir ek subtle wave image upload karke background pe fit karta hai, opacity 50\% karke. Ye background app ko soothing aur inviting banata hai, aur text/buttons clear dikhte hain.
\end{examplebox}

\textbf{Summary:} %
\begin{itemize}
  \item \textcolor{warningred}{Background styles se frames ko colors, gradients, ya images se customize kiya jata hai.}
  \item \textcolor{warningred}{Frame select kar, Fill section mein settings adjust kar, aur design ka mood set kar. Ye UI ko visually appealing banata hai.}
\end{itemize}

% Table for Background Styles %
\begin{table}[h]
  \centering
  \begin{tabular}{|p{4cm}|p{4cm}|p{4cm}|}
    \hline
    \cellcolor{tableheaderblue}\color{white}\textbf{Feature} & 
    \cellcolor{tableheadergreen}\color{white}\textbf{Description} & 
    \cellcolor{tableheaderyellow}\color{black}\textbf{Use Case} \\
    \hline
    \rowcolor{codeblue}
    Background Styles & Colors, gradients & App vibe \\
    \hline
    \rowcolor{tablerowgreen}
    Fill Section & Customize frame & Login screens \\
    \hline
    \rowcolor{codeblue}
    Opacity & Balance content & Clear visuals \\
    \hline
  \end{tabular}
  \caption{Background Styles Overview}
  \label{tab:background_styles}
\end{table}

% Topic 3: Custom Fonts %
\section*{\textbf{\LARGE \textcolor{violet}{Custom Fonts}}} %
\textbf{Corrected aur Enhanced Content:} %
\begin{itemize}
  \item \textbf{Uploading Custom Fonts}:
    \begin{enumerate}
      \item \textcolor{warningred}{Figma mein custom fonts upload karne ke liye Figma Desktop app ya Font Installer plugin use karna padta hai (browser version mein direct upload nahi hota).}
      \item \textcolor{warningred}{Font files (.ttf ya .otf) apne computer pe install kar.}
      \item \textcolor{warningred}{Figma khol, Text Tool (\texttt{T}) se text select kar, aur right panel mein \textbf{Font Dropdown} pe ja.}
      \item Agar font show nahi hota, toh \textbf{Help > Manage Fonts} se check kar ya Font Installer plugin use kar.
      \item Custom font apply kar aur save kar taaki team ke saath share ho.
    \end{enumerate}
  \item \textbf{Kab Use Karna}:
    \begin{itemize}
      \item \textcolor{warningred}{Custom fonts tab use kar jab brand-specific ya unique typography chahiye jo Figma ke default fonts mein nahi hai.}
      \item Example: Ek luxury brand ke liye signature font upload kar taaki design premium lage.
      \item Tip: Font licenses check kar taaki commercial use allowed ho.
    \end{itemize}
\end{itemize}

\begin{examplebox}{Real-Life Example}
Tu ek fashion brand ke liye app design kar raha hai jo ``Playfair Display'' font use karta hai, lekin Figma mein nahi hai. Tu .ttf file computer pe install karta hai, Figma Desktop app mein font dropdown se ``Playfair Display'' select karta hai, aur headings pe apply karta hai. Design ab brand ke aesthetic se match karta hai, aur client khush ho jata hai kyunki typography perfect hai.
\end{examplebox}

\textbf{Summary:} %
\begin{itemize}
  \item \textcolor{warningred}{Custom fonts se unique typography add ki jati hai.}
  \item \textcolor{warningred}{Font files computer pe install kar, Figma mein select kar, aur design ko brand-specific bana. Ye brand consistency ke liye zaroori hai.}
\end{itemize}

% Table for Custom Fonts %
\begin{table}[h]
  \centering
  \begin{tabular}{|p{4cm}|p{4cm}|p{4cm}|}
    \hline
    \cellcolor{tableheaderblue}\color{white}\textbf{Feature} & 
    \cellcolor{tableheadergreen}\color{white}\textbf{Description} & 
    \cellcolor{tableheaderyellow}\color{black}\textbf{Use Case} \\
    \hline
    \rowcolor{codeblue}
    Custom Fonts & Unique typography & Brand designs \\
    \hline
    \rowcolor{tablerowgreen}
    Font Dropdown & Select fonts & Headings \\
    \hline
    \rowcolor{codeblue}
    Install & .ttf, .otf files & Premium looks \\
    \hline
  \end{tabular}
  \caption{Custom Fonts Overview}
  \label{tab:custom_fonts}
\end{table}

% Topic 4: Most Used Figma Shortcuts %
\section*{\textbf{\LARGE \textcolor{violet}{Most Used Figma Shortcuts}}} %
\textbf{Corrected aur Enhanced Content:} %
\begin{itemize}
  \item \textbf{Creating and Editing}:
    \begin{itemize}
      \item \textcolor{warningred}{\texttt{F}: \textbf{Frame Tool} -- Frame banane ke liye.}
      \item \textcolor{warningred}{\texttt{T}: \textbf{Text Tool} -- Text box create ya edit karne ke liye.}
      \item \textcolor{warningred}{\texttt{R}: \textbf{Rectangle Tool} -- Shapes jaise rectangle banane ke liye.}
      \item \textcolor{warningred}{\texttt{Ctrl+D} (Windows) / \texttt{Cmd+D} (Mac): \textbf{Duplicate Selection} -- Layer ya group ka duplicate banata hai.}
      \item \textcolor{warningred}{\texttt{Ctrl+G} / \texttt{Cmd+G}: \textbf{Group Selection} -- Multiple elements ko group karta hai.}
    \end{itemize}
  \item \textbf{Zooming}:
    \begin{itemize}
      \item \textcolor{warningred}{\texttt{Ctrl+Space} (hold karke drag up): \textbf{Zoom In}.}
      \item \textcolor{warningred}{\texttt{Ctrl+Space} (hold karke drag down): \textbf{Zoom Out}.}
      \item \textcolor{warningred}{\texttt{Ctrl+Alt+Space}: \textbf{Zoom to Fit} -- Poora canvas fit-to-screen karta hai.}
      \item \textcolor{warningred}{\texttt{Shift+Space}: \textbf{Zoom to Selection} -- Selected element pe zoom karta hai.}
    \end{itemize}
  \item \textbf{Undo/Redo}:
    \begin{itemize}
      \item \textcolor{warningred}{\texttt{Ctrl+Z} / \texttt{Cmd+Z}: \textbf{Undo} -- Last action wapas leta hai.}
      \item \textcolor{warningred}{\texttt{Ctrl+Y} / \texttt{Cmd+Y}: \textbf{Redo} -- Undo kiya hua action wapas karta hai.}
    \end{itemize}
  \item \textbf{Movement}:
    \begin{itemize}
      \item \textcolor{warningred}{\texttt{V}: \textbf{Move Tool} -- Elements select aur move karne ke liye.}
      \item \textcolor{warningred}{Arrow Keys: \textbf{Nudge} -- 1px move (Shift + Arrow for 10px).}
    \end{itemize}
  \item \textbf{Extra Useful Shortcuts}:
    \begin{itemize}
      \item \textcolor{warningred}{\texttt{Ctrl+'}: \textbf{Show/Hide Layout Grid}.}
      \item \textcolor{warningred}{\texttt{Shift+A}: \textbf{Auto Layout} toggle.}
      \item \textcolor{warningred}{\texttt{K}: \textbf{Scale Tool} -- Proportional resizing.}
    \end{itemize}
\end{itemize}
\textbf{Kab Use Karna:} %
\begin{itemize}
  \item \textcolor{warningred}{Shortcuts tab use kar jab tujhe kaam jaldi karna ho, jaise frame banate waqt \texttt{F} ya duplicate karte waqt \texttt{Ctrl+D}.}
  \item Example: Ek button design kiya, \texttt{Ctrl+D} se duplicate kiya, aur arrow keys se position adjust kiya -- kaam seconds mein ho gaya.
  \item Tip: Shortcuts practice kar taaki muscle memory ban jaye.
\end{itemize}

\begin{examplebox}{Real-Life Example}
Tu ek travel app jaise MakeMyTrip ke liye kaam kar raha hai. Ek mobile frame banane ke liye \texttt{F} daba ke iPhone size ka frame banata hai. Phir \texttt{T} se heading daalta hai, \texttt{R} se ek rectangle button banata hai, aur \texttt{Ctrl+D} se uska duplicate karta hai. Layout check karne ke liye \texttt{Ctrl+'} se grid on karta hai aur \texttt{Ctrl+Space} se zoom in karke details dekhta hai. Ye shortcuts se tera workflow ekdum fast ho jata hai.
\end{examplebox}

\textbf{Summary:} %
\begin{itemize}
  \item \textcolor{warningred}{Figma shortcuts se designing aur editing jaldi hoti hai.}
  \item \textcolor{warningred}{Frame (\texttt{F}), Text (\texttt{T}), Duplicate (\texttt{Ctrl+D}), Zoom (\texttt{Ctrl+Space}), aur Undo (\texttt{Ctrl+Z}) jaise shortcuts yaad rakho. Ye time bachta hai aur productivity badhata hai.}
\end{itemize}

% Table for Most Used Figma Shortcuts %
\begin{table}[h]
  \centering
  \begin{tabular}{|p{4cm}|p{4cm}|p{4cm}|}
    \hline
    \cellcolor{tableheaderblue}\color{white}\textbf{Shortcut} & 
    \cellcolor{tableheadergreen}\color{white}\textbf{Description} & 
    \cellcolor{tableheaderyellow}\color{black}\textbf{Use Case} \\
    \hline
    \rowcolor{codeblue}
    \texttt{F} & Frame Tool & Create frames \\
    \hline
    \rowcolor{tablerowgreen}
    \texttt{Ctrl+D} & Duplicate & Copy elements \\
    \hline
    \rowcolor{codeblue}
    \texttt{Ctrl+Space} & Zoom In/Out & Detail view \\
    \hline
  \end{tabular}
  \caption{Most Used Figma Shortcuts Overview}
  \label{tab:shortcuts}
\end{table}

% Extra Tips for Beginners %
\section*{\textbf{\LARGE \textcolor{violet}{Beginners ke liye Extra Tips}}} %
\begin{itemize}
  \item \textbf{Placeholder Plugins}: \textcolor{warningred}{Content Reel ke alawa \textbf{Lorem Ipsum} ya \textbf{Wireframe} plugins try kar taaki placeholder content aur bhi real lage.}
  \item \textbf{Background Layers}: \textcolor{warningred}{Background ke liye alag frame ya layer bana taaki elements ke saath overlap na ho.}
  \item \textbf{Font Management}: \textcolor{warningred}{Custom fonts ke baad team ke saath share karne ke liye Library publish kar, taaki sabko access mile.}
  \item \textbf{Shortcut Cheat Sheet}: \textcolor{warningred}{Figma ke shortcuts ka list print kar ya phone pe save kar taaki jaldi refer kar sake.}
  \item \textbf{Practice Project}: Ek chhota project try kar -- jaise ek app ka login screen bana jisme placeholder text daal, custom font use kar, background gradient laga, aur shortcuts se jaldi kaam kar.
\end{itemize}

% Table for Extra Tips %
\begin{table}[h]
  \centering
  \begin{tabular}{|p{4cm}|p{4cm}|p{4cm}|}
    \hline
    \cellcolor{tableheaderblue}\color{white}\textbf{Tip} & 
    \cellcolor{tableheadergreen}\color{white}\textbf{Description} & 
    \cellcolor{tableheaderyellow}\color{black}\textbf{Benefit} \\
    \hline
    \rowcolor{codeblue}
    Placeholder Plugins & Try Lorem Ipsum & Realistic mockups \\
    \hline
    \rowcolor{tablerowgreen}
    Shortcut Cheat Sheet & Save shortcuts & Fast reference \\
    \hline
    \rowcolor{codeblue}
    Font Management & Publish Library & Team access \\
    \hline
  \end{tabular}
  \caption{Beginner Tips Overview}
  \label{tab:beginner_tips}
\end{table}

% UI/UX Importance %
\section*{\textbf{\LARGE \textcolor{violet}{UI/UX Job ke liye Kyun Zaroori}}} %
UI/UX designer ke roop mein ye skills daily kaam aayengi taaki:
\begin{itemize}
  \item \textcolor{warningred}{Mockups jaldi aur realistically ban sakein (placeholder text).}
  \item \textcolor{warningred}{Designs visually appealing aur brand-specific hon (background styles, custom fonts).}
  \item \textcolor{warningred}{Workflow fast aur efficient ho (shortcuts).}
\end{itemize}
In se tu professional designs bana sakta hai jo clients ko impress kare aur developers ke liye clear ho.

% Conclusion in tcolorbox %
\begin{notebox}
\textbf{Conclusion:} %
\begin{itemize}
  \item \textcolor{warningred}{Placeholder text mockups ke liye layout banata hai.}
  \item \textcolor{warningred}{Background styles UI ko attractive karte hain.}
  \item \textcolor{warningred}{Custom fonts brand identity dete hain.}
  \item \textcolor{warningred}{Shortcuts workflow ko speed dete hain.}
\end{itemize}
\end{notebox}

===============================
\hrule



% Define custom colors
\definecolor{headingblue}{RGB}{0,102,204}
\definecolor{examplegreen}{RGB}{0,153,76}
\definecolor{warningred}{RGB}{204,0,0}
\definecolor{codeblue}{RGB}{173,216,230}
\definecolor{tablerowgreen}{RGB}{144,238,144}
\definecolor{tableheaderblue}{RGB}{0,102,204}
\definecolor{tableheadergreen}{RGB}{0,153,76}
\definecolor{tableheaderyellow}{RGB}{255,204,0}
\definecolor{violet}{RGB}{128,0,128} % For topic headings

% Section formatting (only for unstarred sections)
\titleformat{\section}
  {\Large\bfseries\color{headingblue}}{}{0em}{}

% Configure listings for code blocks (optional)
\lstset{
  language=Python,
  backgroundcolor=\color{codeblue},
  basicstyle=\ttfamily\small,
  frame=single,
  breaklines=true,
  keywordstyle=\color{blue},
  stringstyle=\color{purple},
  commentstyle=\color{gray},
  showstringspaces=false
}

% Configure tcolorbox for examples
\newtcolorbox{examplebox}[1]{
  colback=examplegreen!10,
  colframe=examplegreen!50!black,
  title=#1,
  breakable,
  enhanced
}

% Configure tcolorbox for notes/conclusion
\newtcolorbox{notebox}{
  colback=warningred!5!white,
  colframe=warningred!75!black,
  title=Point To Note,
  breakable,
  enhanced
}

% Begin document
\begin{document}

% Title Page %
\begin{titlepage}
  \centering
  \vspace*{\fill}
  {\huge\bfseries\color{warningred} Figma Notes: Community, Icons, Pen Tool,\\ Fill, Export, and More}\par % Line break, reduced size
  \vspace{1cm}
  {\Large A Comprehensive Guide to Figma Essentials}\par
  \vspace*{\fill}
\end{titlepage}

% Topic 1: Figma Community Resources %
\section*{\textbf{\LARGE \textcolor{violet}{Figma Community Resources}}} %
\textbf{Corrected aur Enhanced Content:} %
\begin{itemize}
  \item \textbf{Figma Community Resources}:
    \begin{itemize}
      \item \textcolor{warningred}{Figma Community ek platform hai jahan designers free aur paid resources share karte hain, jaise design templates (food delivery apps, login pages), UI kits, icons, logos, wireframes, aur plugins.}
      \item \textcolor{warningred}{Access karne ke liye: Top menu mein \textbf{Community} pe ja ya sidebar ke Resources icon (\texttt{\symbol{4}}) se Community khol.}
      \item \textcolor{warningred}{Search bar mein keywords daal (jaise ``Food Delivery UI Kit'') aur free ya paid resources filter kar.}
      \item Resources duplicate karke apne Figma file mein use kar sakta hai.
    \end{itemize}
  \item \textbf{When to Use}:
    \begin{itemize}
      \item \textcolor{warningred}{Community resources tab use kar jab tujhe inspiration chahiye, time save karna ho, ya ready-made assets chahiye.}
      \item Example: Ek food delivery app ka prototype jaldi banane ke liye Community se ek free UI kit download kar.
    \end{itemize}
  \item \textbf{Why to Use}:
    \begin{itemize}
      \item Time-saving: Zero se design banane ke bajaye tested templates use kar.
      \item Learning: Dusron ke designs dekhke UI/UX trends aur best practices seekh sakta hai.
      \item Professional output: High-quality icons ya kits se design polished lagta hai.
    \end{itemize}
  \item \textbf{Tip}: Free resources try kar, lekin paid kits ke liye budget ho toh premium quality milti hai.
\end{itemize}

\begin{examplebox}{Real-Life Example}
Tu ek startup ke liye food delivery app design kar raha hai, lekin deadline tight hai. Figma Community mein ``Food Delivery UI Kit'' search karta hai aur ek free template milta hai jisme homepage, menu, aur checkout screens hain. Tu isko duplicate karta hai, colors aur fonts apne brand ke hisaab se tweak karta hai, aur 2 din mein prototype ready kar deta hai. Client impress ho jata hai kyunki design professional aur fast bana.
\end{examplebox}

\textbf{Summary:} %
\begin{itemize}
  \item \textcolor{warningred}{Figma Community se free ya paid resources jaise UI kits, icons, aur templates milte hain.}
  \item \textcolor{warningred}{Community section mein search kar, duplicate kar, aur apne project mein use kar. Ye time bachta hai aur designs ko pro level pe le jata hai.}
\end{itemize}

% Table for Figma Community Resources %
\begin{table}[h]
  \centering
  \begin{tabular}{|p{4cm}|p{4cm}|p{4cm}|}
    \hline
    \cellcolor{tableheaderblue}\color{white}\textbf{Feature} & 
    \cellcolor{tableheadergreen}\color{white}\textbf{Description} & 
    \cellcolor{tableheaderyellow}\color{black}\textbf{Use Case} \\
    \hline
    \rowcolor{codeblue}
    Community & Free/paid assets & Prototyping \\
    \hline
    \rowcolor{tablerowgreen}
    Resources & UI kits, icons & Time-saving \\
    \hline
    \rowcolor{codeblue}
    Search & Filter free/paid & Inspiration \\
    \hline
  \end{tabular}
  \caption{Figma Community Resources Overview}
  \label{tab:community_resources}
\end{table}

% Topic 2: Logos and Icons %
\section*{\textbf{\LARGE \textcolor{violet}{Logos and Icons}}} %
\textbf{Corrected aur Enhanced Content:} %
\begin{itemize}
  \item \textbf{Logos and Icons}:
    \begin{itemize}
      \item \textcolor{warningred}{Figma Community ya plugins se free logos, icons, aur illustrations milte hain.}
      \item \textcolor{warningred}{Logos aur icons basically vector shapes, images, ya screenshots hote hain jo UI mein branding ya visuals ke liye use hote hain.}
      \item \textcolor{warningred}{Access karne ke liye:}
        \begin{itemize}
          \item Community mein ``Icons'' ya ``Logos'' search kar.
          \item Plugins jaise \textbf{Iconify}, \textbf{Feather Icons}, ya \textbf{Flaticon} use kar.
        \end{itemize}
      \item Icons ko edit kar sakta hai (color, size, stroke) taaki design se match karein.
    \end{itemize}
  \item \textbf{When to Use}:
    \begin{itemize}
      \item \textcolor{warningred}{Logos tab use kar jab brand identity dikhani ho, jaise app ka header mein logo.}
      \item \textcolor{warningred}{Icons tab use kar jab navigation, actions, ya status dikhane hon, jaise ``cart'' ya ``search'' icon.}
    \end{itemize}
  \item \textbf{Why to Use}:
    \begin{itemize}
      \item Quick visuals: Icons aur logos UI ko intuitive aur attractive banate hain.
      \item Consistency: Same icon set use karne se design cohesive lagta hai.
      \item Example: Ek e-commerce app mein ``heart'' icon favorite ke liye aur brand logo top-left mein daal.
    \end{itemize}
  \item \textbf{Tip}: High-quality vector icons chun taaki scaling pe blurry na ho.
\end{itemize}

\begin{examplebox}{Real-Life Example}
Tu ek shopping app jaise Flipkart ke liye kaam kar raha hai. Header mein brand logo daalne ke liye Community se ek free logo template leta hai aur brand colors (\#2874F0) se customize karta hai. Phir \textbf{Iconify} plugin se ``cart,'' ``search,'' aur ``profile'' icons import karta hai, sabko 24px size aur blue color deta hai. Ye icons aur logo app ko professional aur user-friendly banate hain.
\end{examplebox}

\textbf{Summary:} %
\begin{itemize}
  \item \textcolor{warningred}{Logos aur icons UI mein branding aur functionality ke liye use hote hain.}
  \item \textcolor{warningred}{Figma Community ya plugins se free assets lo, customize karo, aur design mein daalo. Ye visuals ko enhance karta hai aur navigation easy banata hai.}
\end{itemize}

% Table for Logos and Icons %
\begin{table}[h]
  \centering
  \begin{tabular}{|p{4cm}|p{4cm}|p{4cm}|}
    \hline
    \cellcolor{tableheaderblue}\color{white}\textbf{Feature} & 
    \cellcolor{tableheadergreen}\color{white}\textbf{Description} & 
    \cellcolor{tableheaderyellow}\color{black}\textbf{Use Case} \\
    \hline
    \rowcolor{codeblue}
    Logos/Icons & Vector assets & Branding \\
    \hline
    \rowcolor{tablerowgreen}
    Plugins & Iconify, Flaticon & Navigation icons \\
    \hline
    \rowcolor{codeblue}
    Edit & Color, size adjust & Consistency \\
    \hline
  \end{tabular}
  \caption{Logos and Icons Overview}
  \label{tab:logos_icons}
\end{table}

% Topic 3: Pen Tool %
\section*{\textbf{\LARGE \textcolor{violet}{Pen Tool}}} %
\textbf{Corrected aur Enhanced Content:} %
\begin{itemize}
  \item \textbf{Using Pen Tool}:
    \begin{enumerate}
      \item \textcolor{warningred}{Toolbar se \textbf{Pen Tool} chun (shortcut: \texttt{P}).}
      \item \textcolor{warningred}{Canvas pe click karke points (anchors) bana taaki custom vector shapes ya paths draw ho.}
      \item \textcolor{warningred}{Points ke beech curves banane ke liye click-drag kar (Bezier handles se adjust kar).}
      \item Path complete karne ke liye last point pe click kar ya \texttt{Esc} daba.
      \item Shape edit karne ke liye \textbf{Direct Selection Tool} (\texttt{A}) use kar points ya curves ko tweak karne ke liye.
    \end{enumerate}
  \item \textbf{When to Use}:
    \begin{itemize}
      \item \textcolor{warningred}{Pen Tool tab use kar jab tujhe complex ya custom shapes banane hon, jaise unique icons, illustrations, ya curved UI elements.}
      \item Example: Ek app ke liye custom arrow icon ya wave background banane ke liye.
    \end{itemize}
  \item \textbf{Why to Use}:
    \begin{itemize}
      \item Flexibility: Pre-made shapes se zyada control deta hai.
      \item Precision: Detailed designs ya paths bana sakta hai.
      \item Scalability: Vector shapes hone se quality loss nahi hota.
    \end{itemize}
  \item \textbf{Tip}: Practice kar kyunki Pen Tool initially tricky lag sakta hai.
\end{itemize}

\begin{examplebox}{Real-Life Example}
Tu ek fitness app ke liye custom wave background design kar raha hai. \textbf{Pen Tool} (\texttt{P}) se canvas pe points daalta hai aur curves banata hai taaki ek smooth wave shape bane. Phir isko blue gradient fill deta hai aur homepage ke background mein use karta hai. Ye wave design app ko unique aur modern vibe deta hai, jo default shapes se nahi milta.
\end{examplebox}

\textbf{Summary:} %
\begin{itemize}
  \item \textcolor{warningred}{Pen Tool se custom vector shapes aur paths banaye jate hain.}
  \item \textcolor{warningred}{Canvas pe points aur curves draw kar, edit kar, aur complex designs bana. Ye unique icons ya illustrations ke liye best hai.}
\end{itemize}

% Table for Pen Tool %
\begin{table}[h]
  \centering
  \begin{tabular}{|p{4cm}|p{4cm}|p{4cm}|}
    \hline
    \cellcolor{tableheaderblue}\color{white}\textbf{Feature} & 
    \cellcolor{tableheadergreen}\color{white}\textbf{Description} & 
    \cellcolor{tableheaderyellow}\color{black}\textbf{Use Case} \\
    \hline
    \rowcolor{codeblue}
    Pen Tool & \texttt{P} shortcut & Custom shapes \\
    \hline
    \rowcolor{tablerowgreen}
    Curves & Bezier handles & Wave designs \\
    \hline
    \rowcolor{codeblue}
    Edit & Direct Selection & Icon tweaks \\
    \hline
  \end{tabular}
  \caption{Pen Tool Overview}
  \label{tab:pen_tool}
\end{table}

% Topic 4: Fill Property %
\section*{\textbf{\LARGE \textcolor{violet}{Fill Property}}} %
\textbf{Corrected aur Enhanced Content:} %
\begin{itemize}
  \item \textbf{Using Fill Property}:
    \begin{enumerate}
      \item \textcolor{warningred}{Kisi element (shape, frame, ya text) ko select kar.}
      \item \textcolor{warningred}{Right panel mein \textbf{Fill} section ja (paint bucket icon).}
      \item \textcolor{warningred}{Options chun:}
        \begin{itemize}
          \item \textbf{Solid}: Ek color daal (jaise \#FF5733).
          \item \textbf{Gradient}: Linear, radial, ya angular gradient bana.
          \item \textbf{Image}: Image fill kar taaki texture ya pattern aaye.
        \end{itemize}
      \item Color adjust kar (HEX, RGB, ya color picker), opacity set kar, ya contrast tweak kar taaki readability ho.
      \item Fill ko global style bana sakta hai reuse ke liye.
    \end{enumerate}
  \item \textbf{When to Use}:
    \begin{itemize}
      \item \textcolor{warningred}{Fill tab use kar jab tujhe elements ya backgrounds ko color, gradient, ya image se style karna ho.}
      \item Example: Ek button ko bright orange fill de ya frame ko gradient background de.
    \end{itemize}
  \item \textbf{Why to Use}:
    \begin{itemize}
      \item Visual appeal: Colors aur gradients design ko attractive banate hain.
      \item Readability: Sahi contrast se text ya icons clear dikhte hain.
      \item Branding: Brand colors use karne se consistency aati hai.
    \end{itemize}
  \item \textbf{Tip}: Accessibility ke liye contrast check kar (jaise WCAG guidelines).
\end{itemize}

\begin{examplebox}{Real-Life Example}
Tu ek banking app jaise Paytm ke liye ``Send Money'' button design kar raha hai. Button select karta hai aur \textbf{Fill} mein solid color \#00A3E0 daalta hai. Phir text ko white karta hai taaki contrast high ho aur button clickable dikhe. Homepage ke background ke liye ek subtle linear gradient (blue se light blue) use karta hai. Ye fills app ko vibrant aur user-friendly banate hain.
\end{examplebox}

\textbf{Summary:} %
\begin{itemize}
  \item \textcolor{warningred}{Fill property se elements ko colors, gradients, ya images se style kiya jata hai.}
  \item \textcolor{warningred}{Right panel se Fill adjust kar, contrast check kar, aur design ko appealing bana. Ye UI ke look aur readability ke liye zaroori hai.}
\end{itemize}

% Table for Fill Property %
\begin{table}[h]
  \centering
  \begin{tabular}{|p{4cm}|p{4cm}|p{4cm}|}
    \hline
    \cellcolor{tableheaderblue}\color{white}\textbf{Feature} & 
    \cellcolor{tableheadergreen}\color{white}\textbf{Description} & 
    \cellcolor{tableheaderyellow}\color{black}\textbf{Use Case} \\
    \hline
    \rowcolor{codeblue}
    Fill Property & Solid, gradient & Button styling \\
    \hline
    \rowcolor{tablerowgreen}
    Options & Color, image & Backgrounds \\
    \hline
    \rowcolor{codeblue}
    Contrast & Adjust opacity & Readability \\
    \hline
  \end{tabular}
  \caption{Fill Property Overview}
  \label{tab:fill_property}
\end{table}

% Topic 5: Exporting Figma Designs %
\section*{\textbf{\LARGE \textcolor{violet}{Exporting Figma Designs}}} %
\textbf{Corrected aur Enhanced Content:} %
\begin{itemize}
  \item \textbf{Exporting Designs}:
    \begin{enumerate}
      \item \textcolor{warningred}{Kisi layer, group, ya frame ko select kar (jaise icon, button, ya poora screen).}
      \item \textcolor{warningred}{Right panel mein \textbf{Export} section (bottom pe) ja ya right-click karke \textbf{Export} chun.}
      \item \textcolor{warningred}{Format chun:}
        \begin{itemize}
          \item \textbf{PNG}: Images ke liye (transparent background possible).
          \item \textbf{JPG}: Photos ya mockups ke liye.
          \item \textbf{SVG}: Vector graphics (icons, logos).
          \item \textbf{PDF}: Documents ya presentations.
        \end{itemize}
      \item Size set kar (1x, 2x, 3x for retina) aur quality adjust kar (JPG ke liye).
      \item \textbf{Export} button click kar ya multiple assets ke liye \textbf{Export [number] items} chun.
      \item Files computer pe save ho jayengi.
    \end{enumerate}
  \item \textbf{When to Use}:
    \begin{itemize}
      \item \textcolor{warningred}{Export tab kar jab developers ko assets chahiye ya client ko mockups dikhane hon.}
      \item Example: Ek app ke icons ko PNG mein ya pitch deck ke liye PDF export karna.
    \end{itemize}
  \item \textbf{Why to Use}:
    \begin{itemize}
      \item Handoff: Developers ke liye clean assets provide karta hai.
      \item Presentation: Clients ko high-quality visuals deta hai.
      \item Compatibility: Alag formats alag needs ke liye kaam aate hain.
    \end{itemize}
  \item \textbf{Tip}: File names clear rakho (jaise ``home-screen-2x.png'') taaki organized rahe.
\end{itemize}

\begin{examplebox}{Real-Life Example}
Tu ek travel app ke liye homepage design kiya. Developers ko icons chahiye, toh tu sab icons select karta hai, \textbf{Export} mein PNG format aur 2x size set karta hai, aur ``export-icons'' folder mein save karta hai. Client ko demo ke liye poora design PDF mein export karta hai taaki meeting mein print ya share kar sake. Ye process team ke saath handoff aur communication aasan karta hai.
\end{examplebox}

\textbf{Summary:} %
\begin{itemize}
  \item \textcolor{warningred}{Figma designs ko PNG, JPG, SVG, ya PDF mein export kiya jata hai.}
  \item \textcolor{warningred}{Layers ya frames select kar, Export section mein format aur size set kar, aur save kar. Ye developers aur clients ke liye assets provide karne ke liye must hai.}
\end{itemize}

% Table for Exporting Figma Designs %
\begin{table}[h]
  \centering
  \begin{tabular}{|p{4cm}|p{4cm}|p{4cm}|}
    \hline
    \cellcolor{tableheaderblue}\color{white}\textbf{Feature} & 
    \cellcolor{tableheadergreen}\color{white}\textbf{Description} & 
    \cellcolor{tableheaderyellow}\color{black}\textbf{Use Case} \\
    \hline
    \rowcolor{codeblue}
    Export & PNG, SVG formats & Developer assets \\
    \hline
    \rowcolor{tablerowgreen}
    Settings & 1x, 2x sizes & Retina displays \\
    \hline
    \rowcolor{codeblue}
    File Names & Clear naming & Organization \\
    \hline
  \end{tabular}
  \caption{Exporting Figma Designs Overview}
  \label{tab:export_designs}
\end{table}

% Topic 6: Component Tree %
\section*{\textbf{\LARGE \textcolor{violet}{Component Tree}}} %
\textbf{Corrected aur Enhanced Content:} %
\begin{itemize}
  \item \textbf{Understanding Component Tree}:
    \begin{itemize}
      \item \textcolor{warningred}{Figma ke left side mein \textbf{Layers Panel} hota hai jahan component tree dikhta hai -- ye tere design ke saare elements ka hierarchy show karta hai.}
      \item \textcolor{warningred}{Example: Ek \textbf{Navbar} component ke andar \textbf{Image} (logo), \textbf{Text} (menu items), aur \textbf{Button} (CTA) ho sakte hain.}
      \item \textcolor{warningred}{Hierarchy aise hoti hai:}
        \begin{itemize}
          \item Main component: Navbar
          \item Sub-components: Image > Text > Button
        \end{itemize}
      \item Components ko rename, reorder, ya delete kar sakta hai:
        \begin{itemize}
          \item Delete: Layer select kar aur \texttt{Delete} key daba ya right-click karke \textbf{Delete} chun.
          \item Rename: Double-click layer name aur new name daal.
        \end{itemize}
    \end{itemize}
  \item \textbf{When to Use}:
    \begin{itemize}
      \item \textcolor{warningred}{Component tree tab use kar jab tujhe design organize karna ho ya specific elements dhoondhne hon.}
      \item Example: Ek complex webpage mein navbar ke text ko edit karna ho toh Layers Panel se jaldi dhoondh sakta hai.
    \end{itemize}
  \item \textbf{Why to Use}:
    \begin{itemize}
      \item Organization: Clear hierarchy se design manageable rahta hai.
      \item Collaboration: Team ko samajh aata hai kaunsa element kahan hai.
      \item Efficiency: Delete ya edit karna aasan ho jata hai.
    \end{itemize}
  \item \textbf{Tip}: Layers ke naam descriptive rakho (jaise ``Navbar/Logo'') taaki dhoondna aasan ho.
\end{itemize}

\begin{examplebox}{Real-Life Example}
Tu ek portfolio website design kar raha hai. Layers Panel mein tera component tree aisa hai: \textbf{Navbar} > \textbf{Logo Image} > \textbf{Menu Text} > \textbf{Contact Button}. Client bolta hai menu text hata do. Tu Layers Panel mein \textbf{Menu Text} layer dhoondhta hai, select karta hai, aur \texttt{Delete} daba deta hai. Navbar ab clean dikhta hai, aur tu jaldi change kar pata hai kyunki tree organized tha.
\end{examplebox}

\textbf{Summary:} %
\begin{itemize}
  \item \textcolor{warningred}{Component tree Layers Panel mein design ka hierarchy dikhata hai.}
  \item \textcolor{warningred}{Components ko rename, reorder, ya delete kar taaki design organized rahe. Ye complex projects aur team collaboration ke liye zaroori hai.}
\end{itemize}

% Table for Component Tree %
\begin{table}[h]
  \centering
  \begin{tabular}{|p{4cm}|p{4cm}|p{4cm}|}
    \hline
    \cellcolor{tableheaderblue}\color{white}\textbf{Feature} & 
    \cellcolor{tableheadergreen}\color{white}\textbf{Description} & 
    \cellcolor{tableheaderyellow}\color{black}\textbf{Use Case} \\
    \hline
    \rowcolor{codeblue}
    Component Tree & Layers hierarchy & Organization \\
    \hline
    \rowcolor{tablerowgreen}
    Actions & Rename, delete & Edit navbar \\
    \hline
    \rowcolor{codeblue}
    Layers Panel & Left sidebar & Find elements \\
    \hline
  \end{tabular}
  \caption{Component Tree Overview}
  \label{tab:component_tree}
\end{table}

% Topic 7: Alignment Options %
\section*{\textbf{\LARGE \textcolor{violet}{Alignment Options}}} %
\textbf{Corrected aur Enhanced Content:} %
\begin{itemize}
  \item \textbf{Using Alignment}:
    \begin{itemize}
      \item \textcolor{warningred}{Right side ke upar (toolbar mein) \textbf{Alignment Options} hote hain ya right-click karke bhi access kar sakta hai.}
      \item \textcolor{warningred}{Main options:}
        \begin{itemize}
          \item \textbf{Align Left/Right/Center}: Elements ko horizontally align karta hai.
          \item \textbf{Align Top/Bottom/Middle}: Vertically align karta hai.
          \item \textbf{Distribute Horizontally/Vertically}: Elements ke beech equal spacing deta hai.
        \end{itemize}
      \item \textcolor{warningred}{Multiple elements select kar (hold \texttt{Shift}) aur toolbar se option chun.}
      \item Smart guides (blue lines) show hote hain jo alignment aur spacing suggest karte hain.
    \end{itemize}
  \item \textbf{When to Use}:
    \begin{itemize}
      \item \textcolor{warningred}{Alignment tab use kar jab tujhe elements ko clean aur consistent arrangement mein rakhna ho.}
      \item Example: Ek form ke labels aur input fields ko align left karna ya buttons ke beech equal gaps dena.
    \end{itemize}
  \item \textbf{Why to Use}:
    \begin{itemize}
      \item Professional look: Aligned elements design ko polished banate hain.
      \item User experience: Consistent spacing se UI intuitive lagta hai.
      \item Time-saving: Manual adjustment ke bajaye ek click mein kaam ho jata hai.
    \end{itemize}
  \item \textbf{Tip}: Layout Grid (\texttt{Ctrl+'}) on kar taaki alignment visually check ho.
\end{itemize}

\begin{examplebox}{Real-Life Example}
Tu ek payment app jaise Google Pay ke liye checkout form design kar raha hai. ``Name,'' ``Card Number,'' aur ``Expiry'' labels select karta hai aur \textbf{Align Left} karta hai taaki ek straight line mein aayein. Phir teen buttons (``Pay,'' ``Cancel,'' ``Save'') select karke \textbf{Distribute Horizontally} karta hai taaki 16px ka equal gap ho. Ye form ab clean aur user-friendly dikhta hai.
\end{examplebox}

\textbf{Summary:} %
\begin{itemize}
  \item \textcolor{warningred}{Alignment options se elements ko perfectly arrange kiya jata hai.}
  \item \textcolor{warningred}{Toolbar ya right-click se Align ya Distribute chun, multiple elements select kar, aur design ko clean bana. Ye UI ke look aur usability ke liye must hai.}
\end{itemize}

% Table for Alignment Options %
\begin{table}[h]
  \centering
  \begin{tabular}{|p{4cm}|p{4cm}|p{4cm}|}
    \hline
    \cellcolor{tableheaderblue}\color{white}\textbf{Feature} & 
    \cellcolor{tableheadergreen}\color{white}\textbf{Description} & 
    \cellcolor{tableheaderyellow}\color{black}\textbf{Use Case} \\
    \hline
    \rowcolor{codeblue}
    Alignment & Left, Center & Form layouts \\
    \hline
    \rowcolor{tablerowgreen}
    Distribute & Equal spacing & Button gaps \\
    \hline
    \rowcolor{codeblue}
    Smart Guides & Blue lines & Visual check \\
    \hline
  \end{tabular}
  \caption{Alignment Options Overview}
  \label{tab:alignment_options}
\end{table}

% Topic 8: Moving, Deleting, Copying Multiple Items %
\section*{\textbf{\LARGE \textcolor{violet}{Moving, Deleting, Copying Multiple Items}}} %
\textbf{Corrected aur Enhanced Content:} %
\begin{itemize}
  \item \textbf{Moving, Deleting, Copying Multiple Items}:
    \begin{itemize}
      \item \textcolor{warningred}{Multiple items ko ek saath handle karne ke liye pehle unko ek rectangle ya frame mein group kar:}
        \begin{enumerate}
          \item Sab elements select kar (hold \texttt{Shift} ya drag karke).
          \item \textbf{Group} kar (\texttt{Ctrl+G} / \texttt{Cmd+G}) ya ek rectangle bana aur usme daal.
        \end{enumerate}
      \item \textcolor{warningred}{Actions:}
        \begin{itemize}
          \item \textbf{Move}: Group ko \textbf{Move Tool} (\texttt{V}) se drag kar ya arrow keys se nudge kar.
          \item \textbf{Delete}: Group select kar aur \texttt{Delete} key daba ya right-click karke \textbf{Delete} chun.
          \item \textbf{Copy}: Group ko \texttt{Ctrl+C} / \texttt{Cmd+C} se copy aur \texttt{Ctrl+V} / \texttt{Cmd+V} se paste kar.
        \end{itemize}
      \item Frame ke andar constraints set kar taaki resize hone pe elements sahi jagah rahein.
    \end{itemize}
  \item \textbf{When to Use}:
    \begin{itemize}
      \item \textcolor{warningred}{Multiple items tab group kar jab tujhe ek saath move, delete, ya copy karna ho, jaise navbar ke elements ya card ke parts.}
      \item Example: Ek webpage ke footer ke links, logo, aur social icons ko ek saath move karna.
    \end{itemize}
  \item \textbf{Why to Use}:
    \begin{itemize}
      \item Efficiency: Ek-ek element ke bajaye group handle karna fast hai.
      \item Organization: Grouped elements design ko manageable banate hain.
      \item Precision: Alignment aur spacing maintain rehta hai.
    \end{itemize}
  \item \textbf{Tip}: Group ka naam rakho (jaise ``Footer Group'') taaki Layers Panel mein clear ho.
\end{itemize}

\begin{examplebox}{Real-Life Example}
Tu ek blog website ke liye footer design kar raha hai jisme logo, links, aur social icons hain. Sabko select karke \textbf{Group} karta hai (\texttt{Ctrl+G}) aur naam deta hai ``Footer.'' Ab footer ko bottom pe move karne ke liye group ko drag karta hai. Client bolta hai footer hata do, toh tu group select karke \texttt{Delete} daba deta hai -- ekdum simple. Agar doosre page pe same footer chahiye, toh \texttt{Ctrl+C} aur \texttt{Ctrl+V} se copy-paste karta hai.
\end{examplebox}

\textbf{Summary:} %
\begin{itemize}
  \item \textcolor{warningred}{Multiple items ko move, delete, ya copy karne ke liye pehle group ya rectangle mein daalo.}
  \item \textcolor{warningred}{Move Tool (\texttt{V}), Delete, ya Copy-Paste use kar, aur kaam jaldi kar. Ye design ko organized aur efficient rakhta hai.}
\end{itemize}

% Table for Moving, Deleting, Copying Multiple Items %
\begin{table}[h]
  \centering
  \begin{tabular}{|p{4cm}|p{4cm}|p{4cm}|}
    \hline
    \cellcolor{tableheaderblue}\color{white}\textbf{Feature} & 
    \cellcolor{tableheadergreen}\color{white}\textbf{Description} & 
    \cellcolor{tableheaderyellow}\color{black}\textbf{Use Case} \\
    \hline
    \rowcolor{codeblue}
    Group & \texttt{Ctrl+G} & Footer elements \\
    \hline
    \rowcolor{tablerowgreen}
    Move & \texttt{V} tool & Reposition \\
    \hline
    \rowcolor{codeblue}
    Copy & \texttt{Ctrl+C/V} & Reuse assets \\
    \hline
  \end{tabular}
  \caption{Moving, Deleting, Copying Multiple Items Overview}
  \label{tab:moving_items}
\end{table}

% Topic 9: Slice %
\section*{\textbf{\LARGE \textcolor{violet}{Slice}}} %
\textbf{Corrected aur Enhanced Content:} %
\begin{itemize}
  \item \textbf{Using Slices}:
    \begin{enumerate}
      \item \textcolor{warningred}{Toolbar se \textbf{Slice Tool} chun (shortcut: \texttt{S}) ya canvas pe rectangle bana aur \textbf{Slice} mode mein convert kar.}
      \item \textcolor{warningred}{Specific area (jaise button ya section) highlight karne ke liye slice draw kar.}
      \item \textcolor{warningred}{Right panel ke \textbf{Export} section mein slice ka format (PNG, JPG) aur size (1x, 2x) set kar.}
      \item \textbf{Export} click karke highlighted area ko image ke roop mein save kar.
    \end{enumerate}
  \item \textbf{When to Use}:
    \begin{itemize}
      \item \textcolor{warningred}{Slice tab use kar jab tujhe design ke specific part ko export karna ho, jaise ek button, icon, ya banner.}
      \item Example: Ek app ke ``Login'' button ka image developers ke liye export karna.
    \end{itemize}
  \item \textbf{Why to Use}:
    \begin{itemize}
      \item Precision: Sirf chahiye wala area export hota hai, poora frame nahi.
      \item Developer handoff: Assets clear aur focused hote hain.
      \item Flexibility: Alag-alag sections ke liye multiple slices bana sakta hai.
    \end{itemize}
  \item \textbf{Tip}: Slice ke boundaries check kar taaki extra space na aaye.
\end{itemize}

\begin{examplebox}{Real-Life Example}
Tu ek music app jaise Spotify ke liye play button design kiya. Developers ko sirf button ka image chahiye. Tu \textbf{Slice Tool} (\texttt{S}) se button ke around rectangle draw karta hai, \textbf{Export} mein PNG 2x set karta hai, aur ``play-button.png'' ke naam se save karta hai. Ye file developer ko exact area deta hai bina background ke, aur handoff smooth ho jata hai.
\end{examplebox}

\textbf{Summary:} %
\begin{itemize}
  \item \textcolor{warningred}{Slice Tool se design ke specific areas highlight aur export kiye jate hain.}
  \item \textcolor{warningred}{Slice bana, Export settings adjust kar, aur assets save kar. Ye developer handoff aur precise assets ke liye useful hai.}
\end{itemize}

% Table for Slice %
\begin{table}[h]
  \centering
  \begin{tabular}{|p{4cm}|p{4cm}|p{4cm}|}
    \hline
    \cellcolor{tableheaderblue}\color{white}\textbf{Feature} & 
    \cellcolor{tableheadergreen}\color{white}\textbf{Description} & 
    \cellcolor{tableheaderyellow}\color{black}\textbf{Use Case} \\
    \hline
    \rowcolor{codeblue}
    Slice Tool & \texttt{S} shortcut & Button export \\
    \hline
    \rowcolor{tablerowgreen}
    Export & PNG, JPG & Developer assets \\
    \hline
    \rowcolor{codeblue}
    Precision & Specific area & Clean handoff \\
    \hline
  \end{tabular}
  \caption{Slice Overview}
  \label{tab:slice}
\end{table}

% Topic 10: Section %
\section*{\textbf{\LARGE \textcolor{violet}{Section}}} %
\textbf{Corrected aur Enhanced Content:} %
\begin{itemize}
  \item \textbf{Using Sections}:
    \begin{itemize}
      \item \textcolor{warningred}{Figma mein \textbf{Section} ek organizational tool hai jo canvas pe areas ko group karta hai, jaise input box type UI ya specific design parts.}
      \item \textcolor{warningred}{Banane ke liye:}
        \begin{enumerate}
          \item Toolbar se \textbf{Section Tool} chun (ya \texttt{Shift+S} for older versions, ab default nahi hota -- frame ya artboard use hota hai).
          \item Canvas pe rectangle draw kar aur usko Section bana (ya frame ko Section ke roop mein treat kar).
          \item Section ke andar elements (jaise input fields, buttons) daal aur organize kar.
        \end{enumerate}
      \item \textcolor{warningred}{Sections ko collapse/expand kar sakta hai Layers Panel mein.}
    \end{itemize}
  \item \textbf{When to Use}:
    \begin{itemize}
      \item \textcolor{warningred}{Section tab use kar jab tujhe design ke bade parts ko alag-alag categorize karna ho, jaise form UI, header, ya footer.}
      \item Example: Ek login form ke liye Section bana jisme input fields aur submit button ho.
    \end{itemize}
  \item \textbf{Why to Use}:
    \begin{itemize}
      \item Organization: Complex designs ko manageable banata hai.
      \item Collaboration: Team ko clear hota hai kaunsa part kya hai.
      \item Focus: Specific UI sections pe kaam karna aasan ho jata hai.
    \end{itemize}
  \item \textbf{Tip}: Section ka naam rakho (jaise ``Login Form'') taaki Layers Panel mein dhoondna aasan ho.
\end{itemize}

\begin{examplebox}{Real-Life Example}
Tu ek banking app ke liye login screen design kar raha hai. Ek \textbf{Section} banata hai aur naam rakhta hai ``Login Form.'' Isme username input, password input, aur ``Sign In'' button daalta hai. Layers Panel mein Section collapse karta hai taaki canvas clean lage. Jab team ke saath review karta hai, sabko clear hota hai ki ``Login Form'' section alag part hai -- kaam organized aur professional dikhta hai.
\end{examplebox}

\textbf{Summary:} %
\begin{itemize}
  \item \textcolor{warningred}{Section se design ke parts (jaise input box UI) organize kiye jate hain.}
  \item \textcolor{warningred}{Canvas pe Section bana, elements daal, aur Layers Panel mein manage kar. Ye complex designs aur team projects ke liye kaam aata hai.}
\end{itemize}

% Table for Section %
\begin{table}[h]
  \centering
  \begin{tabular}{|p{4cm}|p{4cm}|p{4cm}|}
    \hline
    \cellcolor{tableheaderblue}\color{white}\textbf{Feature} & 
    \cellcolor{tableheadergreen}\color{white}\textbf{Description} & 
    \cellcolor{tableheaderyellow}\color{black}\textbf{Use Case} \\
    \hline
    \rowcolor{codeblue}
    Section & Organize parts & Form UI \\
    \hline
    \rowcolor{tablerowgreen}
    Layers Panel & Collapse/expand & Clean canvas \\
    \hline
    \rowcolor{codeblue}
    Naming & Descriptive names & Team clarity \\
    \hline
  \end{tabular}
  \caption{Section Overview}
  \label{tab:section}
\end{table}

% Extra Tips for Beginners %
\section*{\textbf{\LARGE \textcolor{violet}{Beginners ke liye Extra Tips}}} %
\begin{itemize}
  \item \textbf{Community Filters}: \textcolor{warningred}{Figma Community mein ``Free'' filter laga taaki budget-friendly resources milein.}
  \item \textbf{Icon Editing}: \textcolor{warningred}{Icons import karne ke baad unko ungroup kar aur Pen Tool se tweak kar taaki unique ho.}
  \item \textbf{Fill Shortcuts}: \textcolor{warningred}{Fill colors jaldi change karne ke liye color picker (\texttt{I}) use kar canvas se colors chunne ke liye.}
  \item \textbf{Export Automation}: \textcolor{warningred}{Multiple assets ek saath export karne ke liye \textbf{Batch Export} plugins try kar.}
  \item \textbf{Component Tree Cleanup}: \textcolor{warningred}{Unnecessary layers ya components delete kar taaki Layers Panel clean rahe.}
  \item \textbf{Alignment Practice}: \textcolor{warningred}{Ek dummy form bana aur alignment options try kar taaki muscle memory bane.}
  \item \textbf{Slice Naming}: \textcolor{warningred}{Slices ke naam clear rakho (jaise ``button-slice'') taaki export ke time confusion na ho.}
  \item \textbf{Section Grouping}: \textcolor{warningred}{Ek Section ke andar multiple frames daal taaki bade projects mein clarity rahe.}
  \item \textbf{Practice Project}: Ek chhota project try kar -- jaise ek app ka dashboard bana jisme Community se icons lo, Pen Tool se custom shape bana, Fill se style kar, Section mein organize kar, aur Slice se assets export kar.
\end{itemize}

% Table for Extra Tips %
\begin{table}[h]
  \centering
  \begin{tabular}{|p{4cm}|p{4cm}|p{4cm}|}
    \hline
    \cellcolor{tableheaderblue}\color{white}\textbf{Tip} & 
    \cellcolor{tableheadergreen}\color{white}\textbf{Description} & 
    \cellcolor{tableheaderyellow}\color{black}\textbf{Benefit} \\
    \hline
    \rowcolor{codeblue}
    Community Filters & Free filter & Budget-friendly \\
    \hline
    \rowcolor{tablerowgreen}
    Fill Shortcuts & \texttt{I} picker & Fast styling \\
    \hline
    \rowcolor{codeblue}
    Slice Naming & Clear names & Avoid confusion \\
    \hline
  \end{tabular}
  \caption{Beginner Tips Overview}
  \label{tab:beginner_tips}
\end{table}

% UI/UX Importance %
\section*{\textbf{\LARGE \textcolor{violet}{UI/UX Job ke liye Kyun Zaroori}}} %
UI/UX designer ke roop mein ye skills daily kaam aayengi taaki:
\begin{itemize}
  \item \textcolor{warningred}{High-quality resources jaldi use ho sakein (Community, icons).}
  \item \textcolor{warningred}{Custom aur precise designs ban sakein (Pen Tool, Fill).}
  \item \textcolor{warningred}{Assets developers ke liye clear hon (export, Slice).}
  \item \textcolor{warningred}{Designs organized aur team-friendly hon (component tree, Section).}
  \item \textcolor{warningred}{Alignment aur grouping se UI clean ho (alignment, moving items).}
\end{itemize}
In se tu professional designs bana sakta hai jo clients ko impress kare aur developers ke liye handoff aasan ho.

% Conclusion in tcolorbox %
\begin{notebox}
\textbf{Conclusion:} %
\begin{itemize}
  \item \textcolor{warningred}{Community resources time aur effort bachta hai.}
  \item \textcolor{warningred}{Pen Tool aur Fill se unique designs bante hain.}
  \item \textcolor{warningred}{Export aur Slice developer handoff ko clear karte hain.}
  \item \textcolor{warningred}{Component tree aur Section organization dete hain.}
  \item \textcolor{warningred}{Alignment aur grouping UI ko polished banata hai.}
\end{itemize}
\end{notebox}


===============================
\hrule


% Section formatting (only for unstarred sections)
\titleformat{\section}
  {\Large\bfseries\color{headingblue}}{}{0em}{}

% Configure listings for code blocks (optional)
\lstset{
  language=Python,
  backgroundcolor=\color{codeblue},
  basicstyle=\ttfamily\small,
  frame=single,
  breaklines=true,
  keywordstyle=\color{blue},
  stringstyle=\color{purple},
  commentstyle=\color{gray},
  showstringspaces=false
}

% Configure tcolorbox for examples
\newtcolorbox{examplebox}[1]{
  colback=examplegreen!10,
  colframe=examplegreen!50!black,
  title=#1,
  breakable,
  enhanced
}

% Configure tcolorbox for notes/conclusion
\newtcolorbox{notebox}{
  colback=warningred!5!white,
  colframe=warningred!75!black,
  title=Point To Note,
  breakable,
  enhanced
}

% Begin document
\begin{document}

% Title Page %
\begin{titlepage}
  \centering
  \vspace*{\fill}
  {\huge\bfseries\color{warningred} Figma Notes: Design Systems, Auto Layout,\\ Variables, Accessibility, and More}\par % Line break, reduced size
  \vspace{1cm}
  {\Large A Comprehensive Guide to Advanced Figma Features}\par
  \vspace*{\fill}
\end{titlepage}

% Topic 1: Design Systems %
\section*{\textbf{\LARGE \textcolor{violet}{Design Systems}}} %
\textbf{Explanation:} %
\begin{itemize}
  \item \textbf{Design Systems Kya Hai}: \textcolor{warningred}{Ek design system ek collection hota hai reusable components, styles, aur guidelines ka jo ek consistent aur scalable UI banane ke liye use hota hai. Isme global colors, typography, buttons, inputs, aur rules hote hain (jaise spacing ya accessibility standards).}
  \item \textbf{Kaise Banaye}:
    \begin{enumerate}
      \item \textcolor{warningred}{Ek naya Figma file bana aur naam rakho ``Design System.''}
      \item \textcolor{warningred}{Global styles define kar (colors, text styles, effects) right panel ke Styles section mein.}
      \item \textcolor{warningred}{Reusable components bana (jaise buttons, inputs, cards) aur variants add kar (normal, hover, disabled).}
      \item Guidelines document kar -- ek frame mein spacing rules, font sizes, ya color usage likho.
      \item Team ke saath share karne ke liye Library publish kar (File > Publish Styles and Components).
    \end{enumerate}
  \item \textbf{When to Use}:
    \begin{itemize}
      \item \textcolor{warningred}{Jab tujhe ek bada project ya team ke saath kaam karna ho jahan consistency chahiye, jaise ek app ke multiple screens design karte waqt.}
      \item Example: Ek e-commerce app ke liye buttons, colors, aur fonts ka ek design system bana taaki har screen same style follow kare.
    \end{itemize}
  \item \textbf{Why to Use}:
    \begin{itemize}
      \item \textbf{Consistency}: Saare designs ek jaisa dikhte hain, brand identity strong hoti hai.
      \item \textbf{Efficiency}: Components reuse hone se time bachta hai.
      \item \textbf{Scalability}: Naye features ya screens jaldi add ho sakte hain.
      \item \textbf{Collaboration}: Team members ek hi source use karte hain, confusion nahi hota.
    \end{itemize}
  \item \textbf{Tip}: Design system ko regularly update kar aur team ke feedback lo.
\end{itemize}

\begin{examplebox}{Real-Life Example}
Tu ek startup ke liye food delivery app jaise Zomato design kar raha hai. Ek \textbf{Design System} file banata hai jisme primary color (\#FF5733), secondary color (\#FFFFFF), aur text styles (H1: 24px bold, Body: 16px regular) define karta hai. Buttons ke liye components banata hai (Primary, Secondary, Disabled variants). Jab team ka doosra designer checkout screen banata hai, wo system se components drag karta hai -- sab kuch consistent aur fast ban jata hai. Client bolta hai ``Wow, sab screens ek jaise lagte hain!''
\end{examplebox}

\textbf{Summary:} %
\begin{itemize}
  \item \textcolor{warningred}{Design system reusable styles aur components ka collection hai jo consistency aur speed deta hai.}
  \item \textcolor{warningred}{Global styles aur components bana, guidelines likho, aur Library publish kar. Ye bade projects aur teams ke liye game-changer hai.}
\end{itemize}

% Table for Design Systems %
\begin{table}[h]
  \centering
  \begin{tabular}{|p{4cm}|p{4cm}|p{4cm}|}
    \hline
    \cellcolor{tableheaderblue}\color{white}\textbf{Feature} & 
    \cellcolor{tableheadergreen}\color{white}\textbf{Description} & 
    \cellcolor{tableheaderyellow}\color{black}\textbf{Use Case} \\
    \hline
    \rowcolor{codeblue}
    Design System & Reusable components & App consistency \\
    \hline
    \rowcolor{tablerowgreen}
    Styles & Colors, typography & Brand identity \\
    \hline
    \rowcolor{codeblue}
    Library & Publish to team & Collaboration \\
    \hline
  \end{tabular}
  \caption{Design Systems Overview}
  \label{tab:design_systems}
\end{table}

% Topic 2: Auto Layout Advanced %
\section*{\textbf{\LARGE \textcolor{violet}{Auto Layout Advanced (Spacing Modes aur Resizing)}}} %
\textbf{Explanation:} %
\begin{itemize}
  \item \textbf{Auto Layout Advanced Kya Hai}: \textcolor{warningred}{Tere notes mein Auto Layout cover hua hai, lekin spacing modes (Fixed, Packed, Space Between) aur resizing options (Fill Container, Fixed, Hug Contents) missing hain. Ye advanced features responsive designs ko aur powerful banate hain.}
  \item \textbf{Kaise Use Kare}:
    \begin{enumerate}
      \item \textcolor{warningred}{Ek frame ko Auto Layout mein convert kar (\texttt{Shift+A}).}
      \item \textcolor{warningred}{Right panel mein Auto Layout settings khol:}
        \begin{itemize}
          \item \textbf{Spacing Modes}:
            \begin{itemize}
              \item \textbf{Packed}: Elements ek doosre ke paas chipke rehte hain (jaise list items).
              \item \textbf{Space Between}: Elements frame ke edges ke beech equal space lete hain (jaise buttons toolbar mein).
              \item \textbf{Fixed}: Specific spacing set kar (jaise 16px gap).
            \end{itemize}
          \item \textbf{Resizing Options}:
            \begin{itemize}
              \item \textbf{Fill Container}: Element frame ka poora space bharta hai (jaise full-width button).
              \item \textbf{Fixed}: Element ka size fixed rehta hai (jaise 48px button).
              \item \textbf{Hug Contents}: Element apne content ke size ke hisaab se shrink karta hai (jaise text label).
            \end{itemize}
        \end{itemize}
      \item \textcolor{warningred}{Elements add ya remove kar, Auto Layout apne aap adjust karega.}
    \end{enumerate}
  \item \textbf{When to Use}:
    \begin{itemize}
      \item \textcolor{warningred}{Jab tujhe dynamic layouts banane hon jo content ya screen size ke hisaab se badal sakein, jaise forms, lists, ya navbars.}
      \item Example: Ek chat app ke message bubbles ke liye Packed mode use kar taaki naye messages add hone pe layout adjust ho.
    \end{itemize}
  \item \textbf{Why to Use}:
    \begin{itemize}
      \item \textbf{Responsiveness}: Designs alag-alag devices pe sahi dikhte hain.
      \item \textbf{Flexibility}: Content changes (jaise text length) ke bawajood layout stable rehta hai.
      \item \textbf{Time-saving}: Manual adjustments ki zarurat nahi padti.
    \end{itemize}
  \item \textbf{Tip}: Nested Auto Layouts (frame ke andar frame) use kar complex designs ke liye.
\end{itemize}

\begin{examplebox}{Real-Life Example}
Tu ek to-do app jaise Todoist ke liye task list design kar raha hai. Ek frame ko Auto Layout mein set karta hai aur \textbf{Packed} mode chunta hai taaki tasks ek doosre ke neeche chipke rahein (8px spacing). Har task ka text \textbf{Hug Contents} set karta hai taaki lamba text hone pe bhi layout adjust ho. ``Add Task'' button ko \textbf{Fill Container} deta hai taaki wo full-width rahe. Jab client bolta hai ek task aur add karo, tu bas ek naya task daalta hai aur Auto Layout sab arrange kar deta hai -- zero manual kaam!
\end{examplebox}

\textbf{Summary:} %
\begin{itemize}
  \item \textcolor{warningred}{Auto Layout ke advanced features (Spacing Modes aur Resizing) dynamic aur responsive designs banate hain.}
  \item \textcolor{warningred}{Packed, Space Between, ya Fixed modes chun, aur Fill, Fixed, ya Hug Contents set kar. Ye layouts ko flexible aur fast banata hai.}
\end{itemize}

% Table for Auto Layout Advanced %
\begin{table}[h]
  \centering
  \begin{tabular}{|p{4cm}|p{4cm}|p{4cm}|}
    \hline
    \cellcolor{tableheaderblue}\color{white}\textbf{Feature} & 
    \cellcolor{tableheadergreen}\color{white}\textbf{Description} & 
    \cellcolor{tableheaderyellow}\color{black}\textbf{Use Case} \\
    \hline
    \rowcolor{codeblue}
    Spacing Modes & Packed, Fixed & Task lists \\
    \hline
    \rowcolor{tablerowgreen}
    Resizing & Fill, Hug & Responsive buttons \\
    \hline
    \rowcolor{codeblue}
    Auto Layout & \texttt{Shift+A} & Dynamic forms \\
    \hline
  \end{tabular}
  \caption{Auto Layout Advanced Overview}
  \label{tab:auto_layout_advanced}
\end{table}

% Topic 3: Figma Variables %
\section*{\textbf{\LARGE \textcolor{violet}{Figma Variables}}} %
\textbf{Explanation:} %
\begin{itemize}
  \item \textbf{Figma Variables Kya Hai}: \textcolor{warningred}{Variables ek naya feature hai jahan tu reusable values store kar sakta hai, jaise colors, numbers (spacing, sizes), strings (text), ya booleans (true/false). Ye design systems ko aur flexible banate hain.}
  \item \textbf{Kaise Use Kare}:
    \begin{enumerate}
      \item \textcolor{warningred}{Right panel mein \textbf{Variables} tab khol (Styles ke paas).}
      \item \textcolor{warningred}{\texttt{+} click karke variable bana:}
        \begin{itemize}
          \item \textbf{Color}: Jaise \texttt{Primary = \#FF5733}.
          \item \textbf{Number}: Jaise \texttt{Spacing/Small = 8px}, \texttt{Button/Height = 48px}.
          \item \textbf{String}: Jaise \texttt{Button/Text = Submit}.
        \end{itemize}
      \item \textcolor{warningred}{Variables ko components, styles, ya prototypes mein apply kar.}
      \item Ek variable edit karne se saare uses update ho jate hain.
      \item Modes bana (jaise Light/Dark theme) aur variables switch kar.
    \end{enumerate}
  \item \textbf{When to Use}:
    \begin{itemize}
      \item \textcolor{warningred}{Jab tujhe design mein reusable aur centralized values chahiye, jaise ek app ke colors ya spacing ko manage karna.}
      \item Example: Ek app ke liye Light aur Dark mode ke colors variables mein define kar taaki theme switch karna aasan ho.
    \end{itemize}
  \item \textbf{Why to Use}:
    \begin{itemize}
      \item \textbf{Centralized Control}: Ek jagah change karo, sab jagah update ho.
      \item \textbf{Theming}: Light/Dark modes ya brand variations jaldi bana sakte ho.
      \item \textbf{Scalability}: Bade projects mein values manage karna aasan hota hai.
      \item \textbf{Prototyping}: Interactive states (jaise hover) ke liye variables use ho sakte hain.
    \end{itemize}
  \item \textbf{Tip}: Variable names logical rakho, jaise \texttt{Color/Primary}, \texttt{Spacing/Large}.
\end{itemize}

\begin{examplebox}{Real-Life Example}
Tu ek social media app jaise Instagram ke liye design kar raha hai. \textbf{Variables} mein \texttt{Primary = \#405DE6}, \texttt{Spacing/Small = 8px}, aur \texttt{Button/Height = 48px} define karta hai. Saare buttons mein ye variables use karta hai. Jab client bolta hai primary color ko purple (\#5851DB) karo, tu bas variable edit karta hai aur poora design update ho jata hai. Dark mode ke liye naya mode banata hai aur colors switch karta hai -- ekdum smooth!
\end{examplebox}

\textbf{Summary:} %
\begin{itemize}
  \item \textcolor{warningred}{Figma Variables reusable values (colors, numbers, strings) store karte hain.}
  \item \textcolor{warningred}{Variables tab mein bana, apply kar, aur modes ke saath themes manage kar. Ye design systems ko flexible aur scalable banata hai.}
\end{itemize}

% Table for Figma Variables %
\begin{table}[h]
  \centering
  \begin{tabular}{|p{4cm}|p{4cm}|p{4cm}|}
    \hline
    \cellcolor{tableheaderblue}\color{white}\textbf{Feature} & 
    \cellcolor{tableheadergreen}\color{white}\textbf{Description} & 
    \cellcolor{tableheaderyellow}\color{black}\textbf{Use Case} \\
    \hline
    \rowcolor{codeblue}
    Variables & Colors, numbers & Theming \\
    \hline
    \rowcolor{tablerowgreen}
    Modes & Light/Dark & Theme switching \\
    \hline
    \rowcolor{codeblue}
    Apply & Components & Scalable designs \\
    \hline
  \end{tabular}
  \caption{Figma Variables Overview}
  \label{tab:figma_variables}
\end{table}

% Topic 4: Accessibility in Figma %
\section*{\textbf{\LARGE \textcolor{violet}{Accessibility in Figma}}} %
\textbf{Explanation:} %
\begin{itemize}
  \item \textbf{Accessibility Kya Hai}: \textcolor{warningred}{Accessibility (a11y) ka matlab hai designs ko sab users ke liye usable banana, including disabled users (jaise color blindness, screen readers). Figma mein accessibility check aur implement karna zaroori hai.}
  \item \textbf{Kaise Kare}:
    \begin{enumerate}
      \item \textcolor{warningred}{\textbf{Color Contrast}: Right panel ke Fill section mein colors chunte waqt contrast ratio check kar (WCAG 2.1 recommends 4.5:1 for text).}
      \item \textcolor{warningred}{\textbf{Text Size}: Fonts ko readable rakho (minimum 16px for body text).}
      \item \textcolor{warningred}{\textbf{Focus States}: Components ke variants bana (jaise button ka ``Focused'' state) taaki keyboard navigation clear ho.}
      \item Plugins jaise \textbf{Stark} ya \textbf{A11y -- Color Contrast Checker} use kar contrast aur accessibility issues check karne ke liye.
      \item Layers ko proper order mein rakho taaki screen readers sahi sequence mein padhein.
    \end{enumerate}
  \item \textbf{When to Use}:
    \begin{itemize}
      \item \textcolor{warningred}{Har design mein accessibility consider kar, especially public-facing apps ya websites ke liye.}
      \item Example: Ek form ke labels aur buttons ka contrast high rakho taaki color-blind users bhi easily use kar sakein.
    \end{itemize}
  \item \textbf{Why to Use}:
    \begin{itemize}
      \item \textbf{Inclusivity}: Sab users (disabled ya non-disabled) app use kar sakein.
      \item \textbf{Legal Compliance}: Kai countries mein accessibility laws hote hain (jaise ADA, WCAG).
      \item \textbf{Better UX}: Clear designs sab ke liye better experience dete hain.
      \item \textbf{Professionalism}: Companies accessibility ko seriously leti hain.
    \end{itemize}
  \item \textbf{Tip}: Design phase mein hi accessibility check kar taaki baad mein rework na karna pade.
\end{itemize}

\begin{examplebox}{Real-Life Example}
Tu ek healthcare app ke liye login screen design kar raha hai. \textbf{Stark} plugin se check karta hai ki ``Submit'' button ka green color (\#00FF00) aur white text ka contrast low hai (3:1). Tu green ko darker (\#008000) karta hai taaki contrast 4.5:1 ho aur WCAG compliant rahe. Phir button ka ``Focused'' variant banata hai (yellow outline) taaki keyboard users ko clear dikhe. Ye design ab sab users ke liye accessible hai, aur client bolta hai ``Ye inclusive approach pasand aaya!''
\end{examplebox}

\textbf{Summary:} %
\begin{itemize}
  \item \textcolor{warningred}{Accessibility se designs sab users ke liye usable bante hain.}
  \item \textcolor{warningred}{High contrast, readable fonts, focus states, aur plugins use kar. Ye inclusivity, compliance, aur better UX ke liye must hai.}
\end{itemize}

% Table for Accessibility in Figma %
\begin{table}[h]
  \centering
  \begin{tabular}{|p{4cm}|p{4cm}|p{4cm}|}
    \hline
    \cellcolor{tableheaderblue}\color{white}\textbf{Feature} & 
    \cellcolor{tableheadergreen}\color{white}\textbf{Description} & 
    \cellcolor{tableheaderyellow}\color{black}\textbf{Use Case} \\
    \hline
    \rowcolor{codeblue}
    Accessibility & Inclusive design & Form usability \\
    \hline
    \rowcolor{tablerowgreen}
    Contrast & 4.5:1 ratio & Color blindness \\
    \hline
    \rowcolor{codeblue}
    Plugins & Stark, A11y & Compliance checks \\
    \hline
  \end{tabular}
  \caption{Accessibility in Figma Overview}
  \label{tab:accessibility}
\end{table}

% Topic 5: Developer Handoff %
\section*{\textbf{\LARGE \textcolor{violet}{Developer Handoff}}} %
\textbf{Explanation:} %
\begin{itemize}
  \item \textbf{Developer Handoff Kya Hai}: \textcolor{warningred}{Handoff ka matlab hai apne Figma designs ko developers ke liye clear aur usable format mein dena taaki wo UI code kar sakein. Figma ke features isko aasan banate hain.}
  \item \textbf{Kaise Kare}:
    \begin{enumerate}
      \item \textcolor{warningred}{\textbf{Inspect Tab}: Right panel mein \textbf{Inspect} tab khol -- yahan developers ko CSS, iOS, ya Android code milta hai (jaise margins, padding, colors).}
      \item \textcolor{warningred}{\textbf{Assets Export}: Layers ya frames ko PNG, SVG, ya JPG mein export kar (tere notes mein hai, lekin yahan context ke liye mention).}
      \item \textcolor{warningred}{\textbf{Annotations}: Comments Tool (\texttt{C}) se measurements, spacing, ya specific instructions likho.}
      \item \textbf{Share Link}: Top-right se \textbf{Share} button use kar aur developers ko ``Can View'' ya ``Can Inspect'' access de.
      \item \textbf{Plugins}: \textbf{Zeplin} ya \textbf{Figma to Code} plugins use kar extra details ke liye.
    \end{enumerate}
  \item \textbf{When to Use}:
    \begin{itemize}
      \item \textcolor{warningred}{Jab design finalize ho jaye aur developers ko implementation ke liye dena ho.}
      \item Example: Ek app ka homepage design developers ko dena ho toh Inspect tab aur assets provide kar.
    \end{itemize}
  \item \textbf{Why to Use}:
    \begin{itemize}
      \item \textbf{Clarity}: Developers ko exact specs (sizes, colors, spacing) milte hain.
      \item \textbf{Efficiency}: Manual explanations ki zarurat nahi, direct Figma se code copy ho sakta hai.
      \item \textbf{Accuracy}: Design aur final product mein mismatch nahi hota.
      \item \textbf{Collaboration}: Developers aur designers ke beech smooth communication hoti hai.
    \end{itemize}
  \item \textbf{Tip}: Design file clean rakho -- unnecessary layers delete kar aur components organize kar.
\end{itemize}

\begin{examplebox}{Real-Life Example}
Tu ek payment app jaise PhonePe ke liye dashboard design kiya. Developers ko handoff ke liye tu \textbf{Inspect} tab open karta hai jahan button ka CSS code (width: 160px, height: 48px, background: \#5851DB) milta hai. ``Pay Now'' button ko PNG 2x mein export karta hai aur Comments mein likhta hai ``16px margin top aur bottom.'' Share link developers ko bhejta hai. Wo bolte hain ``Ye specs bohot clear hain, coding start karte hain!'' -- handoff ekdum smooth ho jata hai.
\end{examplebox}

\textbf{Summary:} %
\begin{itemize}
  \item \textcolor{warningred}{Developer handoff se designs code-ready format mein dete hain.}
  \item \textcolor{warningred}{Inspect tab, assets, comments, aur share link use kar. Ye clarity, accuracy, aur collaboration ke liye zaroori hai.}
\end{itemize}

% Table for Developer Handoff %
\begin{table}[h]
  \centering
  \begin{tabular}{|p{4cm}|p{4cm}|p{4cm}|}
    \hline
    \cellcolor{tableheaderblue}\color{white}\textbf{Feature} & 
    \cellcolor{tableheadergreen}\color{white}\textbf{Description} & 
    \cellcolor{tableheaderyellow}\color{black}\textbf{Use Case} \\
    \hline
    \rowcolor{codeblue}
    Inspect Tab & CSS, iOS code & Code specs \\
    \hline
    \rowcolor{tablerowgreen}
    Export & PNG, SVG assets & Button images \\
    \hline
    \rowcolor{codeblue}
    Comments & \texttt{C} annotations & Clear instructions \\
    \hline
  \end{tabular}
  \caption{Developer Handoff Overview}
  \label{tab:developer_handoff}
\end{table}

% Topic 6: Animation in Prototyping %
\section*{\textbf{\LARGE \textcolor{violet}{Animation in Prototyping}}} %
\textbf{Explanation:} %
\begin{itemize}
  \item \textbf{Animation in Prototyping Kya Hai}: \textcolor{warningred}{Figma ke Prototype Tab mein animations add kar sakte hain taaki transitions (jaise screen change ya button hover) real app jaisa feel karein. Ye micro-interactions design ko polished banate hain.}
  \item \textbf{Kaise Kare}:
    \begin{enumerate}
      \item \textcolor{warningred}{\textbf{Prototype Tab} mein switch kar.}
      \item \textcolor{warningred}{Ek element (jaise button) se doosre frame ya state tak arrow drag kar.}
      \item \textcolor{warningred}{Right panel mein \textbf{Interaction} settings khol:}
        \begin{itemize}
          \item Trigger: On Click, On Hover, After Delay, etc.
          \item Action: Navigate To, Open Overlay, Swap Instance.
          \item Animation: \textbf{Smart Animate} (auto-transitions), Slide In, Fade, Push, etc.
        \end{itemize}
      \item Easing (Linear, Ease In/Out) aur duration (ms) set kar taaki animation smooth ho.
      \item Play button (\texttt{\symbol{16}}) se test kar.
    \end{enumerate}
  \item \textbf{When to Use}:
    \begin{itemize}
      \item \textcolor{warningred}{Jab tujhe prototype ko realistic banana ho ya client ko app ka flow dikhana ho.}
      \item Example: Ek button click pe next screen slide-in animation ke saath khule.
    \end{itemize}
  \item \textbf{Why to Use}:
    \begin{itemize}
      \item \textbf{Realism}: Animations real app jaisa experience dete hain.
      \item \textbf{Engagement}: Smooth transitions users ka attention rakhte hain.
      \item \textbf{Clarity}: Developers ko animations ka idea milta hai implementation ke liye.
      \item \textbf{Wow Factor}: Clients ko polished prototype impress karta hai.
    \end{itemize}
  \item \textbf{Tip}: Over-animation avoid kar, simple aur purposeful transitions best hote hain.
\end{itemize}

\begin{examplebox}{Real-Life Example}
Tu ek music app jaise Spotify ke liye prototype bana raha hai. Play button pe click karne se ``Now Playing'' screen khulta hai. Tu Prototype Tab mein \textbf{Smart Animate} set karta hai aur Slide In (300ms, Ease Out) chunta hai. Button ka Hover state bhi banata hai jisme size 10\% badhta hai (Fade animation, 200ms). Jab client prototype dekhta hai, wo bolta hai ``Ye toh bilkul real app jaisa lag raha hai!'' -- animations ne design ko next level pe le gaya.
\end{examplebox}

\textbf{Summary:} %
\begin{itemize}
  \item \textcolor{warningred}{Animation in prototyping se transitions aur micro-interactions real feel dete hain.}
  \item \textcolor{warningred}{Prototype Tab mein interactions aur Smart Animate use kar, easing aur duration set kar. Ye prototypes ko engaging aur professional banata hai.}
\end{itemize}

% Table for Animation in Prototyping %
\begin{table}[h]
  \centering
  \begin{tabular}{|p{4cm}|p{4cm}|p{4cm}|}
    \hline
    \cellcolor{tableheaderblue}\color{white}\textbf{Feature} & 
    \cellcolor{tableheadergreen}\color{white}\textbf{Description} & 
    \cellcolor{tableheaderyellow}\color{black}\textbf{Use Case} \\
    \hline
    \rowcolor{codeblue}
    Animation & Smart Animate & Screen transitions \\
    \hline
    \rowcolor{tablerowgreen}
    Triggers & Click, Hover & Button interactions \\
    \hline
    \rowcolor{codeblue}
    Easing & Smooth duration & Polished feel \\
    \hline
  \end{tabular}
  \caption{Animation in Prototyping Overview}
  \label{tab:animation_prototyping}
\end{table}

% Missing Topics and Importance %
\section*{\textbf{\LARGE \textcolor{violet}{Kya Missing Tha aur Kyun Zaroori Hai}}} %
Tere notes mein Figma ke core features bohot ache se cover hain, lekin ye advanced topics missing the:
\begin{itemize}
  \item \textbf{Design Systems}: \textcolor{warningred}{Bade projects mein consistency aur scalability for must hai, warna har screen alag dikhega.}
  \item \textbf{Auto Layout Advanced}: \textcolor{warningred}{Responsive designs ke liye zaruri hai kyunki modern apps alag-alag devices pe kaam karte hain.}
  \item \textbf{Figma Variables}: \textcolor{warningred}{Theming aur centralized control ke liye future-proof feature hai, jo ab industry standard ban raha hai.}
  \item \textbf{Accessibility}: \textcolor{warningred}{Inclusive designs ab har company demand karti hai, aur legal compliance bhi zaroori hota hai.}
  \item \textbf{Developer Handoff}: \textcolor{warningred}{Real-world mein designers aur developers ke beech gap nahi hona chahiye, isliye handoff key hai.}
  \item \textbf{Animation in Prototyping}: \textcolor{warningred}{Clients aur users ko wow factor dene ke liye animations ek edge dete hain.}
\end{itemize}
Ye sab skills tujhe ek beginner se professional UI/UX designer banane mein help karengi, kyunki companies inhi cheezon ko dekhti hain -- consistency, responsiveness, inclusivity, aur collaboration.

% Table for Missing Topics %
\begin{table}[h]
  \centering
  \begin{tabular}{|p{4cm}|p{4cm}|p{4cm}|}
    \hline
    \cellcolor{tableheaderblue}\color{white}\textbf{Topic} & 
    \cellcolor{tableheadergreen}\color{white}\textbf{Why Missing} & 
    \cellcolor{tableheaderyellow}\color{black}\textbf{Importance} \\
    \hline
    \rowcolor{codeblue}
    Design Systems & Not covered & Consistency \\
    \hline
    \rowcolor{tablerowgreen}
    Variables & New feature & Scalable theming \\
    \hline
    \rowcolor{codeblue}
    Accessibility & Overlooked & Inclusivity \\
    \hline
  \end{tabular}
  \caption{Missing Topics Overview}
  \label{tab:missing_topics}
\end{table}

% Extra Tips for Beginners %
\section*{\textbf{\LARGE \textcolor{violet}{Beginners ke liye Extra Tips}}} %
\begin{itemize}
  \item \textbf{Design System Template}: \textcolor{warningred}{Figma Community se free design system templates lo aur unko customize karke seekho.}
  \item \textbf{Auto Layout Practice}: \textcolor{warningred}{Ek dummy form ya list bana aur Packed/Space Between modes try kar taaki hands-on samajh aaye.}
  \item \textbf{Variable Modes}: \textcolor{warningred}{Light/Dark mode ke liye variables ka mini-project bana taaki theming clear ho.}
  \item \textbf{Accessibility Checklist}: \textcolor{warningred}{Har design ke baad contrast aur font size check karne ki habit daal.}
  \item \textbf{Handoff Mock}: \textcolor{warningred}{Ek chhota design bana aur apne friend ko ``developer'' assume karke handoff kar -- feedback se seekh.}
  \item \textbf{Animation Testing}: \textcolor{warningred}{Simple animations (jaise button hover) bana aur Play button se test kar taaki flow samajh aaye.}
  \item \textbf{Portfolio Project}: \textcolor{warningred}{Ek complete app UI bana (jaise todo app) jisme design system, variables, accessibility, aur handoff include ho -- ye portfolio mein shine karega.}
\end{itemize}

% Table for Extra Tips %
\begin{table}[h]
  \centering
  \begin{tabular}{|p{4cm}|p{4cm}|p{4cm}|}
    \hline
    \cellcolor{tableheaderblue}\color{white}\textbf{Tip} & 
    \cellcolor{tableheadergreen}\color{white}\textbf{Description} & 
    \cellcolor{tableheaderyellow}\color{black}\textbf{Benefit} \\
    \hline
    \rowcolor{codeblue}
    Design System & Community templates & Learn structure \\
    \hline
    \rowcolor{tablerowgreen}
    Accessibility & Checklist habit & Inclusive designs \\
    \hline
    \rowcolor{codeblue}
    Portfolio & Full app UI & Job readiness \\
    \hline
  \end{tabular}
  \caption{Beginner Tips Overview}
  \label{tab:beginner_tips}
\end{table}

% UI/UX Importance %
\section*{\textbf{\LARGE \textcolor{violet}{UI/UX Job ke liye Kyun Zaroori}}} %
In missing topics ke saath tera skillset complete ho jayega:
\begin{itemize}
  \item \textcolor{warningred}{Design systems aur variables se tu consistent aur scalable designs bana sakta hai.}
  \item \textcolor{warningred}{Auto Layout aur accessibility se responsive aur inclusive UIs banenge.}
  \item \textcolor{warningred}{Handoff aur animations se developers aur clients ke saath collaboration aur presentation top-notch hoga.}
\end{itemize}
Ye sab tujhe job interviews mein stand out karayega kyunki companies aise designers chahte hain jo practical aur modern tools samajhte hon.

% Conclusion in tcolorbox %
\begin{notebox}
\textbf{Conclusion:} %
\begin{itemize}
  \item \textcolor{warningred}{Design systems consistency aur scalability dete hain.}
  \item \textcolor{warningred}{Auto Layout aur variables responsive aur flexible designs banate hain.}
  \item \textcolor{warningred}{Accessibility sab users ke liye inclusive UI ensure karta hai.}
  \item \textcolor{warningred}{Handoff aur animations collaboration aur presentation ko enhance karte hain.}
\end{itemize}
\end{notebox}



===============================
\hrule


% Section formatting (only for unstarred sections)
\titleformat{\section}
  {\Large\bfseries\color{headingblue}}{}{0em}{}

% Configure listings for code blocks (optional)
\lstset{
  language=Python,
  backgroundcolor=\color{codeblue},
  basicstyle=\ttfamily\small,
  frame=single,
  breaklines=true,
  keywordstyle=\color{blue},
  stringstyle=\color{purple},
  commentstyle=\color{gray},
  showstringspaces=false
}

% Configure tcolorbox for examples
\newtcolorbox{examplebox}[1]{
  colback=examplegreen!10,
  colframe=examplegreen!50!black,
  title=#1,
  breakable,
  enhanced
}

% Configure tcolorbox for notes/conclusion
\newtcolorbox{notebox}{
  colback=warningred!5!white,
  colframe=warningred!75!black,
  title=Point To Note,
  breakable,
  enhanced
}

% Begin document
\begin{document}

% Title Page %
\begin{titlepage}
  \centering
  \vspace*{\fill}
  {\huge\bfseries\color{warningred} Figma Notes: Prototype Arrows\\ for User Flows}\par % Line break, reduced size
  \vspace{1cm}
  {\Large A Guide to Creating Interactive Prototypes in Figma}\par
  \vspace*{\fill}
\end{titlepage}

% Topic: Prototype Arrows in Figma %
\section*{\textbf{\LARGE \textcolor{violet}{Prototype Arrows in Figma}}} %
\textbf{Explanation:} %
\begin{itemize}
  \item \textbf{Prototype Arrows Kya Hai}: \textcolor{warningred}{Figma ke \textbf{Prototype Tab} mein ye arrows (nishaane) ek element ya frame ko doosre frame se connect karte hain taaki user flow ya interactions define ho sakein. Ye dikhata hai ki user jab kisi button ya element pe click karega, toh kya hoga -- jaise ek screen se doosra screen khulega. Ye prototyping ka core part hai.}
  \item \textbf{Kaise Use Kare}:
    \begin{enumerate}
      \item \textcolor{warningred}{Figma mein apna design khol aur top-right corner se \textbf{Prototype Tab} pe switch kar (Design Tab ke bagal mein).}
      \item \textcolor{warningred}{Ek element ya frame select kar (jaise ek ``Login'' button ya homepage frame).}
      \item \textcolor{warningred}{Element ke edge pe ek chhota blue dot (drag handle) dikhega -- isko click-drag kar aur doosre frame ya element tak le ja.}
        \begin{itemize}
          \item Ek \textbf{arrow} banega jo dono ko connect karta hai.
        \end{itemize}
      \item \textcolor{warningred}{Right panel mein \textbf{Interaction} settings khul jayengi:}
        \begin{itemize}
          \item \textbf{Trigger}: Action ka type chun (jaise \textbf{On Click}, \textbf{On Hover}, \textbf{After Delay}).
          \item \textbf{Action}: Kya hoga (jaise \textbf{Navigate To} doosre frame, \textbf{Open Overlay}, ya \textbf{Swap Instance}).
          \item \textbf{Destination}: Konsa frame ya element khulega (dropdown se select kar).
          \item \textbf{Animation}: Transition type chun (jaise \textbf{Smart Animate}, \textbf{Slide In}, \textbf{Fade}, duration aur easing ke saath).
        \end{itemize}
      \item Arrow ke saath label dikhega (jaise ``On Click > Navigate To'') jo interaction describe karta hai.
      \item Prototype test karne ke liye top-right mein \textbf{Play} button (\texttt{\symbol{16}}) click kar aur flow dekho.
    \end{enumerate}
  \item \textbf{When to Use}:
    \begin{itemize}
      \item \textcolor{warningred}{Jab tujhe app ya website ka user flow dikhana ho, jaise ek button click karne se kya hota hai.}
      \item Example: Ek login screen ke ``Sign In'' button se dashboard frame tak arrow connect kar taaki flow clear ho.
      \item Client presentations ya developer handoff ke liye jab interactive demo chahiye.
    \end{itemize}
  \item \textbf{Why to Use}:
    \begin{itemize}
      \item \textbf{Realistic Flow}: Arrows se prototype real app jaisa lagta hai, user journey clear hota hai.
      \item \textbf{Client Approval}: Clients ko dikhata hai ki app kaam kaise karega, feedback lena aasan hota hai.
      \item \textbf{Developer Clarity}: Developers ko samajh aata hai ki interactions aur transitions kaise implement karne hain.
      \item \textbf{Testing}: Prototype ke saath bugs ya flow issues pehle hi catch ho jate hain.
    \end{itemize}
  \item \textbf{Tip}: Arrows ko clean rakho -- zyada complex connections se prototype confusing ho sakta hai. Labels aur animations simple rakho.
\end{itemize}

\begin{examplebox}{Real-Life Example}
Tu ek e-commerce app jaise Flipkart ke liye prototype bana raha hai. Tere paas teen frames hain: \textbf{Homepage}, \textbf{Product Page}, aur \textbf{Cart}. Prototype Tab mein tu Homepage ke ``View Product'' button pe blue dot drag karta hai aur Product Page frame tak arrow connect karta hai. Interaction set karta hai: \textbf{On Click > Navigate To > Product Page > Slide In (300ms)}. Phir Product Page ke ``Add to Cart'' button se Cart frame tak ek aur arrow banata hai (\textbf{Fade animation, 200ms}). Jab tu Play button dabata hai, prototype ek real app jaisa chalta hai -- client bolta hai, ``Bhai, ye toh bilkul asli lag raha hai!'' Developers bhi isko dekh ke exact transitions code kar dete hain.
\end{examplebox}

\textbf{Summary:} %
\begin{itemize}
  \item \textcolor{warningred}{Figma ke Prototype Arrows ek screen ko doosre se connect karte hain taaki user flow aur interactions define ho sakein.}
  \item \textcolor{warningred}{Prototype Tab mein element select kar, arrow drag kar, interaction set kar, aur Play se test kar. Ye realistic prototypes, client demos, aur developer handoff ke liye must hai.}
\end{itemize}

% Table for Prototype Arrows %
\begin{table}[h]
  \centering
  \begin{tabular}{|p{4cm}|p{4cm}|p{4cm}|}
    \hline
    \cellcolor{tableheaderblue}\color{white}\textbf{Feature} & 
    \cellcolor{tableheadergreen}\color{white}\textbf{Description} & 
    \cellcolor{tableheaderyellow}\color{black}\textbf{Use Case} \\
    \hline
    \rowcolor{codeblue}
    Prototype Arrows & Connect frames & User flows \\
    \hline
    \rowcolor{tablerowgreen}
    Interactions & On Click, Navigate & Screen transitions \\
    \hline
    \rowcolor{codeblue}
    Animations & Smart Animate & Realistic demos \\
    \hline
  \end{tabular}
  \caption{Prototype Arrows Overview}
  \label{tab:prototype_arrows}
\end{table}

% Context in Notes %
\section*{\textbf{\LARGE \textcolor{violet}{Tere Notes ke Context Mein}}} %
Tere notes mein prototyping cover hua tha (Topic: Prototype, Creating Connections), lekin \textbf{arrows} ka specific mention nahi tha, aur ye ek beginner ke liye confusing ho sakta hai kyunki ye Figma ke prototyping ka visual core hai. Maine isko detail mein cover kiya taaki tujhe clear ho ki ye kaise kaam karta hai aur job mein kaise use hota hai. Ye feature tere notes ka hissa banega aur tera prototyping game aur strong karega!

% Extra Tips for Arrows %
\section*{\textbf{\LARGE \textcolor{violet}{Extra Tips for Arrows}}} %
\begin{itemize}
  \item \textbf{Label Clarity}: \textcolor{warningred}{Arrows ke labels descriptive rakho (jaise ``Login to Dashboard'') taaki team ko samajh aaye.}
  \item \textbf{Smart Animate}: \textcolor{warningred}{Complex transitions ke liye Smart Animate use kar -- ye layers ke changes ko automatically smooth karta hai.}
  \item \textbf{Flow Organization}: \textcolor{warningred}{Agar bohot saare arrows hain, toh frames ko canvas pe logically arrange kar taaki wires cross na karein.}
  \item \textbf{Test Often}: \textcolor{warningred}{Har arrow set karne ke baad Play button se test kar taaki galti turant pata chal jaye.}
  \item \textbf{Practice}: \textcolor{warningred}{Ek simple flow bana -- jaise login screen se homepage tak -- aur arrows ke saath khel taaki confidence aaye.}
\end{itemize}

% Table for Extra Tips %
\begin{table}[h]
  \centering
  \begin{tabular}{|p{4cm}|p{4cm}|p{4cm}|}
    \hline
    \cellcolor{tableheaderblue}\color{white}\textbf{Tip} & 
    \cellcolor{tableheadergreen}\color{white}\textbf{Description} & 
    \cellcolor{tableheaderyellow}\color{black}\textbf{Benefit} \\
    \hline
    \rowcolor{codeblue}
    Label Clarity & Descriptive names & Team understanding \\
    \hline
    \rowcolor{tablerowgreen}
    Smart Animate & Smooth transitions & Polished flows \\
    \hline
    \rowcolor{codeblue}
    Test Often & Play button checks & Catch errors early \\
    \hline
  \end{tabular}
  \caption{Extra Tips for Prototype Arrows}
  \label{tab:extra_tips_arrows}
\end{table}

% Job Importance %
\section*{\textbf{\LARGE \textcolor{violet}{Job ke Liye Kyun Zaroori}}} %
UI/UX designer ke roop mein prototype arrows ka use bohot hota hai kyunki:
\begin{itemize}
  \item \textcolor{warningred}{Ye user flow visually communicate karta hai, jo clients aur stakeholders ke liye zaroori hai.}
  \item \textcolor{warningred}{Developers ko exact interactions aur animations ka idea milta hai, jo implementation mein mismatch rokta hai.}
  \item \textcolor{warningred}{Interviews mein jab tu prototype dikhayega, arrows ke saath clean flow impress karega.}
\end{itemize}
Ye skill tujhe ek polished aur professional designer ke roop mein stand out karayegi.

% Conclusion in tcolorbox %
\begin{notebox}
\textbf{Conclusion:} %
\begin{itemize}
  \item \textcolor{warningred}{Prototype Arrows se user flows aur interactions visually define hote hain.}
  \item \textcolor{warningred}{Simple arrows, clear labels, aur Smart Animate se prototypes realistic aur professional bante hain.}
  \item \textcolor{warningred}{Ye client demos, developer handoff, aur job interviews ke liye critical hai.}
\end{itemize}
\end{notebox}


===============================
\hrule

\end{document}



